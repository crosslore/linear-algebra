\subsection{Eigenvalues of Markov matrices}

The following is an important proposition.

\begin{proposition}{Eigenvalues of a migration matrix}{eigenvalue-migration-matrix}
Let $A=\mat{a_{ij}} $ be a migration matrix. Then $1$ is always an
eigenvalue for $A$.
\end{proposition}

\begin{proof} Remember that the determinant of a matrix always equals that of its transpose.
Therefore,
\begin{equation*}
\det (\eigenvar I - A) =\det ((\eigenvar I - A)
^{T}) =\det (\eigenvar I - A^T)
\end{equation*}
because $I^{T}=I$. Thus the characteristic equation for $A$ is the same as
the characteristic equation for $A^{T}$. Consequently, $A$ and $A^{T}$ have the same
eigenvalues. We will show that $1$ is an eigenvalue for $A^{T}$ and then it
will follow that $1$ is an eigenvalue for $A$.

Remember that for a migration matrix, $\sum_{i}a_{ij}=1$. Therefore, if
$A^{T}=\mat{b_{ij}} $ with $b_{ij}=a_{ji}$, it follows that
\begin{equation*}
\sum_{j}b_{ij}=\sum_{j}a_{ji}=1
\end{equation*}

Therefore, from matrix multiplication,
\begin{equation*}
A^{T}\begin{mymatrix}{r}
1 \\
\vdots \\
1
\end{mymatrix} =\begin{mymatrix}{c}
\sum_{j}b_{ij} \\
\vdots \\
\sum_{j}b_{ij}
\end{mymatrix} =\begin{mymatrix}{r}
1 \\
\vdots \\
1
\end{mymatrix}
\end{equation*}

Notice that this shows that $\begin{mymatrix}{r}
1 \\
\vdots \\
1
\end{mymatrix} $ is an eigenvector for $A^{T}$ corresponding to the eigenvalue, $\lambda =1$.
 As explained above, this shows that $\lambda =1$ is an
eigenvalue for $A$ because $A$ and $A^{T}$ have the same eigenvalues.
\end{proof}
