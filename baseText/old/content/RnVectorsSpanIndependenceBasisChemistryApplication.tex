\section{A short application to chemistry}

The following section applies the concepts of spanning and linear independence to the subject of chemistry.

When working with chemical reactions, there are sometimes a large number of reactions and some are in a sense redundant. Suppose you have the following chemical
reactions.
\begin{equation*}
\begin{array}{c}
CO+\vspace{0.05in}\frac{1}{2}O_{2}\rightarrow CO_{2} \\
H_{2}+\vspace{0.05in}\frac{1}{2}O_{2}\rightarrow H_{2}O \\
CH_{4}+\vspace{0.05in}\frac{3}{2}O_{2}\rightarrow CO+2H_{2}O \\
CH_{4}+2O_{2}\rightarrow CO_{2}+2H_{2}O
\end{array}
\end{equation*}
There are four chemical reactions here but they are not independent
reactions. There is some redundancy. What are the independent reactions? Is
there a way to consider a shorter list of reactions? To analyze this
situation, we can write the reactions in a matrix as follows
\begin{equation*}
\begin{mymatrix}{cccccc}
CO & O_{2} & CO_{2} & H_{2} & H_{2}O & CH_{4} \\
1 & 1/2 & -1 & 0 & 0 & 0 \\
0 & 1/2 & 0 & 1 & -1 & 0 \\
-1 & 3/2 & 0 & 0 & -2 & 1 \\
0 & 2 & -1 & 0 & -2 & 1
\end{mymatrix}
\end{equation*}

Each row contains the coefficients of the respective elements in each reaction. For example, the top row of numbers comes from $CO+\frac{1}{2}O_{2}-CO_{2}=0$ which
represents the first of the chemical reactions.

We can write these coefficients in the following matrix
\begin{equation*}
\begin{mymatrix}{rrrrrr}
1 & 1/2 & -1 & 0 & 0 & 0 \\
0 & 1/2 & 0 & 1 & -1 & 0 \\
-1 & 3/2 & 0 & 0 & -2 & 1 \\
0 & 2 & -1 & 0 & -2 & 1
\end{mymatrix}
\end{equation*}
Rather than listing all of the
reactions as above, it would be more efficient to only list those which are independent by throwing out that which is redundant. We can use the concepts of the previous section to accomplish this.

First, take the {\rref} of the above matrix.
\begin{equation*}
\begin{mymatrix}{rrrrrr}
1 & 0 & 0 & 3 & -1 & -1 \\
0 & 1 & 0 & 2 & -2 & 0 \\
0 & 0 & 1 & 4 & -2 & -1 \\
0 & 0 & 0 & 0 & 0 & 0
\end{mymatrix}
\end{equation*}
The top three rows represent \textquotedblleft independent" reactions which
come from the original four reactions. One can obtain each of the original
four rows of the  matrix given above by taking a suitable
linear combination of rows of this {\rref} matrix.

With the redundant reaction removed, we can consider the simplified reactions as the following equations
\begin{equation*}
\begin{array}{c}
CO+3H_{2}-1H_{2}O-1CH_{4}=0 \\
O_{2}+2H_{2}-2H_{2}O=0 \\
CO_{2}+4H_{2}-2H_{2}O-1CH_{4}=0
\end{array}
\end{equation*}
In terms of the original notation, these are the reactions
\begin{equation*}
\begin{array}{c}
CO+3H_{2}\rightarrow H_{2}O+CH_{4} \\
O_{2}+2H_{2}\rightarrow 2H_{2}O \\
CO_{2}+4H_{2}\rightarrow 2H_{2}O+CH_{4}
\end{array}
\end{equation*}

These three reactions provide an equivalent system to the original four equations. The idea is that, in terms of what happens chemically, you
obtain the same information with the shorter list of reactions. Such a simplification is especially useful when dealing with very large lists of reactions which may result from experimental evidence.
