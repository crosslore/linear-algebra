\subsubsection{The $QR$ factorization and eigenvalues}

The $QR$ factorization of a matrix has a very useful application. It turns out that it can be used repeatedly to estimate the eigenvalues of a matrix. Consider the following procedure.

\begin{procedure}{Using the $QR$ factorization to estimate eigenvalues}{qr-eigenvalues}
Let $A$ be an invertible matrix. Define the matrices $A_1, A_2, \ldots$ as follows:
\begin{enumerate}
\item
$A_1 = A$ factored as $A_1 = Q_1R_1$
\item
$A_2 = R_1Q_1$ factored as $A_2 = Q_2R_2$
\item
$A_3 = R_2Q_2$ factored as $A_3 = Q_3R_3$
\end{enumerate}

Continue in this manner, where in general $A_k = Q_kR_k$ and $A_{k+1} = R_kQ_k$.

Then it follows that this sequence of $A_i$ converges to an upper triangular matrix which is similar to $A$. Therefore the eigenvalues of $A$ can be approximated by the entries on the main diagonal of this upper triangular matrix.
\end{procedure}
