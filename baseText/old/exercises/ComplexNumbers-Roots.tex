\section*{Exercises}

\begin{ex} Give the complete solution to $x^{4}+16=0$.
\begin{sol}
 Solution is:
\[
(1-i) \sqrt{2},-(1+i) \sqrt{2},-(1-i)
\sqrt{2},(1+i) \sqrt{2}
\]
\end{sol}
\end{ex}

\begin{ex} \label{cube-roots} Find the complex cube roots of $8$.
\begin{sol}
The cube roots are the solutions to $%
z^{3}+8=0$, Solution is: $i\sqrt{3} +1,1-i\sqrt{3},-2$
\end{sol}
\end{ex}

\begin{ex} \label{cube-roots2} Find the four fourth roots of $16$.
\begin{sol}
The fourth roots are
the solutions to $z^{4}+16=0$, Solution is:
\[
(1-i) \sqrt{2},-(1+i) \sqrt{2},-(1-i)
\sqrt{2},(1+i)\sqrt{2}
\]
\end{sol}
\end{ex}

\begin{ex} \label{exer-complex1}De Moivre's theorem says $\mat{r(\cos
t+i\sin t)} ^{n}=r^{n}(\cos nt+i\sin nt) $ for $n$
a positive integer. Does this formula continue to hold for all integers $n$,
even negative integers? Explain.
\begin{sol}
Yes, it holds for all integers. First of
all, it clearly holds if $n=0$. Suppose now that $n$ is a negative integer.
Then $-n>0$ and so
\[
\mat{r(\cos t+i\sin t)} ^{n}=\frac{1}{\mat{r(
\cos t+i\sin t)} ^{-n}}=\frac{1}{r^{-n}(\cos (
-nt) +i\sin (-nt)) }
\]
\begin{eqnarray*}
&=&\frac{r^{n}}{(\cos (nt) -i\sin (nt))
}=\frac{r^{n}(\cos (nt) +i\sin (nt)) }{
(\cos (nt) -i\sin (nt)) (\cos
(nt) +i\sin (nt)) } \\
&=&r^{n}(\cos (nt) +i\sin (nt))
\end{eqnarray*}
because $(\cos (nt) -i\sin (nt)) (
\cos (nt) +i\sin (nt)) =1$.
\end{sol}
\end{ex}

\begin{ex} Factor $x^{3}+8$ as a product of linear factors. \textbf{Hint:} Use the result of {\eqref{cube-roots}}.
\begin{sol}
Solution
is: $i\sqrt{3}+1,1-i\sqrt{3},-2$ and so this polynomial equals
\[
(x+2) (x-(i\sqrt{3}+1)) (x-(
1-i\sqrt{3}))
\]
\end{sol}
\end{ex}

\begin{ex} Write $x^{3}+27$ in the form $(x+3) (
x^{2}+ax+b) $ where $x^{2}+ax+b$ cannot be factored any more using
only real numbers.
\begin{sol}
$x^{3}+27= (x+3) (
x^{2}-3x+9) $
\end{sol}
\end{ex}

\begin{ex} Completely factor $x^{4}+16$ as a product of linear factors. \textbf{Hint:} Use the result of {\eqref{cube-roots2}}.
\begin{sol}
Solution is:
\[
(1-i) \sqrt{2},-(1+i) \sqrt{2},-(1-i)
\sqrt{2},(1+i) \sqrt{2}.
\]
These are just the fourth roots of $-16$. Then to factor, you get
\begin{eqnarray*}
&&(x-((1-i) \sqrt{2})) (x-(
-(1+i) \sqrt{2})) \cdot \\
&&(x-(-(1-i) \sqrt{2})) (x-(
(1+i) \allowbreak \sqrt{2}))
\end{eqnarray*}
\end{sol}
\end{ex}

\begin{ex} Factor $x^{4}+16$ as the product of two quadratic polynomials each of
which cannot be factored further without using complex numbers.
\begin{sol}
$x^{4}+16=(x^{2}-2\sqrt{2}x+4) (x^{2}+2\sqrt{2}x+4) .
$ You can use the information in the preceding problem. Note that $(
x-z) (x-\conjugate{z}) $ has real coefficients.
\end{sol}
\end{ex}

\begin{ex} If $n$ is an integer, is it always true that $(\cos \theta
-i\sin \theta) ^{n}=\cos (n\theta) -i\sin (n\theta
)$? Explain.
\begin{sol}
Yes, this is true.
\begin{eqnarray*}
(\cos \theta -i\sin \theta) ^{n} &=&(\cos (-\theta
) +i\sin (-\theta)) ^{n} \\
&=&\cos (-n\theta) +i\sin (-n\theta) \\
&=&\cos (n\theta) -i\sin (n\theta)
\end{eqnarray*}
\end{sol}
\end{ex}

\begin{ex} Suppose $p(x) =a_{n}x^{n}+a_{n-1}x^{n-1}+\ldots+a_{1}x+a_{0}$ \ is a polynomial and it has $n$ zeros,
\begin{equation*}
z_{1},z_{2},\ldots,z_{n}
\end{equation*}
listed according to multiplicity. ($z$ is a root of multiplicity $m$ if the
polynomial $f(x) =(x-z) ^{m}$ divides $p(
x) $ but $(x-z) f(x) $ does not.) Show that
\begin{equation*}
p(x) =a_{n}(x-z_{1}) (x-z_{2}) \cdots
(x-z_{n})
\end{equation*}
\begin{sol}
$p(x) =(x-z_{1}) q(x) +r(x) $
where $r(x) $ is a non-zero constant or equal to $0$. However, $r(z_{1}) =0$ and so $r(x) =0$. Now do to $q(
x) $ what was done to $p(x) $ and continue until the
degree of the resulting $q(x) $ equals $0$. Then you have the
above factorization.
\end{sol}
\end{ex}

