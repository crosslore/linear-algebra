\subsection{The \texorpdfstring{$ij^{th}$}{ijth} entry of a product}

In previous sections, we used the entries of a matrix to describe the action of matrix addition and scalar multiplication.
We can also study matrix multiplication using the entries of matrices. 

What is the $ij^{th}$ entry of $AB?$ It is the entry in the $i^{th}$ row
and the $j^{th}$ column of the product $AB$. 
 
Now if $A$ is $m \times n$ and $B$ is $n \times p$, then we know that the product $AB$ has the form 
\begin{equation*}
\begin{mymatrix}{cccc}
a_{11} & a_{12} & \cdots & a_{1n} \\
a_{21} & a_{22} & \cdots & a_{2n} \\
\vdots & \vdots &  \ddots & \vdots \\
a_{m1} & a_{m2} & \cdots & a_{mn}
\end{mymatrix} \begin{mymatrix}{cccccc}
b_{11} & b_{12} & \cdots & b_{1j} & \cdots & b_{1p} \\
b_{21} & b_{22} & \cdots & b_{2j} & \cdots & b_{2p} \\
\vdots & \vdots &  & \vdots & & \vdots\\
b_{n1} & b_{n2} & \cdots & b_{nj} & \cdots & b_{np}
\end{mymatrix} 
\end{equation*}

The $j^{th}$ column of $AB$ is of the form
\begin{equation*}
\begin{mymatrix}{cccc}
a_{11} & a_{12} & \cdots & a_{1n} \\
a_{21} & a_{22} & \cdots & a_{2n} \\
\vdots & \vdots & \ddots & \vdots \\
a_{m1} & a_{m2} & \cdots & a_{mn}
\end{mymatrix} \begin{mymatrix}{c}
b_{1j} \\
b_{2j} \\
\vdots \\
b_{nj}
\end{mymatrix}
\end{equation*}
which is an $m\times 1$ column vector. It is calculated by 
\begin{equation*}
b_{1j}
\begin{mymatrix}{c}
a_{11} \\
a_{21} \\
\vdots \\
a_{m1}
\end{mymatrix} + b_{2j}\begin{mymatrix}{c}
a_{12} \\
a_{22} \\
\vdots \\
a_{m2}
\end{mymatrix} +\cdots + b_{nj}\begin{mymatrix}{c}
a_{1n} \\
a_{2n} \\
\vdots \\
a_{mn}
\end{mymatrix} 
\end{equation*}

Therefore, the $ij^{th}$ entry is the entry in row $i$ of this vector.
This is computed by 
\begin{equation*}
a_{i1}b_{1j}+a_{i2}b_{2j}+\cdots + a_{in}b_{nj}=\sum_{k=1}^{n}a_{ik}b_{kj}
\end{equation*}

The following is the formal definition for the $ij^{th}$ entry of a product of matrices. 

\begin{definition}{The $ij^{th}$ entry of a product}{ijentryofproduct}
Let $A=\mat{a_{ij}} $ be an $m\times n$ matrix and
let $B=\mat{b_{ij}} $ be an $n\times p$ matrix. Then $AB$ is an 
$m\times p$ matrix and the $\tup{i, j }$-entry of $AB$\index{matrix multiplication!ijth entry} is defined as 
\begin{equation*}
(AB)_{ij}=\sum_{k=1}^{n}a_{ik}b_{kj}  
%\label{mat12}
\end{equation*}
Another way to write this is
\begin{equation*}
(AB)_{ij}=\begin{mymatrix}{cccc}
a_{i1} & a_{i2} & \cdots & a_{in}
\end{mymatrix} \begin{mymatrix}{c}
b_{1j} \\
b_{2j} \\
\vdots \\
b_{nj}
\end{mymatrix}
= 
a_{i1}b_{1j} + a_{i2}b_{2j} + \cdots + a_{in}b_{nj}
\end{equation*}
\end{definition}

In other words, to find the $\tup{i, j }$-entry of the product $AB$, or $(AB)_{ij}$,
you multiply the $i^{th}$ row of 
$A,$ on the left by the $j^{th}$ column of $B$. To express $AB$ in terms of its entries, we write $AB = \mat{(AB)_{ij} }$.

Consider the following example. 

\begin{example}{The entries of a product}{entriesofproduct1}
Compute $AB$ if possible. If it is, find the $\tup{3,2 }$-entry of $AB$ using Definition \ref{def:ijentryofproduct}. 
\begin{equation*}
A = \begin{mymatrix}{cc}
1 & 2 \\
3 & 1 \\
2 & 6
\end{mymatrix}, B = \begin{mymatrix}{ccc}
2 & 3 & 1 \\
7 & 6 & 2
\end{mymatrix}
\end{equation*}
\end{example}

\begin{solution} First check if the product is possible. It is of the form $\tup{3\times
2} \tup{2\times 3} $ and since the inside numbers match, it is possible to do the multiplication. The result should be a $3\times 3$ matrix. 
We can first compute $AB$:
\begin{equation*}
\mat{\begin{mymatrix}{rr}
1 & 2 \\
3 & 1 \\
2 & 6
\end{mymatrix} \begin{mymatrix}{r}
2 \\
7
\end{mymatrix} ,\begin{mymatrix}{rr}
1 & 2 \\
3 & 1 \\
2 & 6
\end{mymatrix} \begin{mymatrix}{r}
3 \\
6
\end{mymatrix} ,\begin{mymatrix}{rr}
1 & 2 \\
3 & 1 \\
2 & 6
\end{mymatrix} \begin{mymatrix}{r}
1 \\
2
\end{mymatrix} }
\end{equation*}
where the commas separate the columns in the resulting product. Thus the
above product equals
\begin{equation*}
 \begin{mymatrix}{rrr}
16 & 15 & 5 \\
13 & 15 & 5 \\
46 & 42 & 14
\end{mymatrix} 
\end{equation*}
which is a $3\times 3$ matrix as desired. Thus, the $\tup{3,2 }$-entry equals 42.

Now using Definition
\ref{def:ijentryofproduct}, we can find that the $\tup{3,2 }$-entry equals
\begin{eqnarray*}
\sum_{k=1}^{2}a_{3k}b_{k2} &=&a_{31}b_{12}+a_{32}b_{22} \\
&=&2\times 3+6\times 6=42\\
\end{eqnarray*}
Consulting our result for $AB$ above, this is correct! 

You may wish to use this method to verify that the rest of the entries in $AB$ are correct.
\end{solution}

Here is another example.

\begin{example}{Finding the entries of a product}{entriesofproduct2}
Determine if the product $AB$ is defined. If it is, find the $\tup{2, 1 }$-entry of the product.
\begin{equation*}
A=
\begin{mymatrix}{rrr}
2 & 3 & 1 \\
7 & 6 & 2 \\
0 & 0 & 0
\end{mymatrix},  B =\begin{mymatrix}{rr}
1 & 2 \\
3 & 1 \\
2 & 6
\end{mymatrix} 
\end{equation*}
\end{example}

\begin{solution} This product is of the form $\tup{3\times
3} \tup{3\times 2} $. The middle numbers match so the
matrices are conformable and it is possible to compute the product.  

We want to find the $\tup{2, 1 }$-entry of $AB$, that is, the entry in the second row and first column of the product.
We will use Definition \ref{def:ijentryofproduct}, which states 
\begin{equation*}
(AB)_{ij}=\sum_{k=1}^{n}a_{ik}b_{kj}
\end{equation*}
In this case, $n=3$, $i=2$ and $j=1$. Hence the $\tup{2, 1 }$-entry is found by computing
\begin{equation*}
 (AB)_{21} = \sum_{k=1}^{3}a_{2k}b_{k1} = 
 \begin{mymatrix}{ccc}
a_{21} & a_{22} & a_{23}
\end{mymatrix} \begin{mymatrix}{c}
b_{11} \\
b_{21} \\
b_{31}
\end{mymatrix}  
\end{equation*}
Substituting in the appropriate values, this product becomes
\begin{equation*}
\begin{mymatrix}{ccc}
a_{21} & a_{22} & a_{23}
\end{mymatrix} \begin{mymatrix}{c}
b_{11} \\
b_{21} \\
b_{31}
\end{mymatrix} 
=
\begin{mymatrix}{ccc}
7 & 6 & 2
\end{mymatrix} \begin{mymatrix}{c}
1 \\
3 \\
2
\end{mymatrix}
=
1 \times 7 + 3 \times 6 + 2 \times 2
=
29
\end{equation*}

Hence,  $(AB)_{21} = 29$.

You should take a moment to find a few other entries of $AB$. You can multiply the matrices to check that your answers are correct.
The product $AB$ is given by 
\begin{equation*}
AB = \begin{mymatrix}{cc}
13 & 13 \\
29 & 32 \\
0 & 0
\end{mymatrix} 
\end{equation*}
\end{solution}
