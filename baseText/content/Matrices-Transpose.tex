\section{The transpose}

\begin{outcome}
  \begin{enumerate}
  \item Calculate the transpose of a matrix.
  \item Determine whether a matrix is symmetric, antisymmetric, or neither.
  \item Manipulate algebraic expressions involving the transpose of matrices.
  \end{enumerate}
\end{outcome}

Another important operation on matrices is that of taking the
\textbf{transpose}. The transpose of a matrix is obtained by turning
the rows into columns and vice versa.

\begin{definition}{The transpose of a matrix}{matrix-transpose}
  Let $A$ be an $m\times n$-matrix.  Then the \textbf{transpose}%
  \index{matrix!transpose}\index{transpose of a matrix} of $A$,
  denoted $A^T$, is the $n\times m$-matrix whose $(i,j)$-entry is the
  $(j,i)$-entry of $A$.
  \begin{equation*}
    A = \begin{mymatrix}{ccc}
      a_{11} & \cdots & a_{1n} \\
      a_{21} & \cdots & a_{2n} \\
      \vdots & \ddots & \vdots \\
      a_{m1} & \cdots & a_{mn} \\
    \end{mymatrix},
    \quad
    A^T = \begin{mymatrix}{cccc}
      a_{11} & a_{21} & \cdots & a_{m1} \\
      \vdots & \vdots & \ddots & \vdots \\
      a_{1n} & a_{2n} & \cdots & a_{mn} \\
    \end{mymatrix}.
  \end{equation*}
\end{definition}

\begin{example}{The transpose of a matrix}{transpose-matrix}
  Find the transpose of the following matrix:
  \begin{equation*}
    A = \begin{mymatrix}{rrr}
      1 & 2 & 6 \\
      3 & 5 & 4
    \end{mymatrix}.
  \end{equation*}
\end{example}

\begin{solution}
  The transpose is
  \begin{equation*}
    A^T =
    \begin{mymatrix}{rr}
      1 & 3 \\
      2 & 5 \\
      6 & 4
    \end{mymatrix}.
  \end{equation*}
  Notice that $A$ is a $2\times 3$-matrix, while $A^T$ is a
  $3\times 2$-matrix.
\end{solution}

We have already used a special case of the transpose since
Chapter~\ref{cha:vectors-rn}, when we wrote
$\begin{mymatrix}{rrr}1&2&3\end{mymatrix}^T$ as a space-saving
notation for the column vector
\begin{equation*}
  \begin{mymatrix}{c}
    1 \\
    2 \\
    3 \\
  \end{mymatrix}.
\end{equation*}

The transpose of a matrix satisfies the following
properties%
\index{transpose of a matrix!properties}%
\index{matrix!transpose!properties}%
\index{matrix!properties of transpose}:

\begin{lemma}{Properties of the transpose}{transpose-properties}
  Let $A$ and $B$ be matrices of appropriate sizes, and $r$ a
  scalar. Then the following hold.
  \begin{enumerate}
  \item $(A^T)^T = A$.
  \item $(A+B)^T=A^T+B^T$.\label{matrix-transpose-2}
  \item $(rA)^T=rA^T$.\label{matrix-transpose-3}
  \item $(AB)^T=B^TA^T$.\label{matrix-transpose-4}
  \item $0^T = 0$.
  \item $I^T = I$.
  \item $(A^{-1})^T = (A^T)^{-1}$, if $A$ is invertible.
  \end{enumerate}
\end{lemma}

Recall that a column vector is the same thing as a $n\times
1$-matrix. Using the transpose, we can make precise the connection
between the dot product and the matrix product%
\index{dot product!from matrix product}%
\index{vector!dot product!from matrix product}. Namely, let
\begin{equation*}
  \vect{v}=\begin{mymatrix}{c}v_1\\\vdots\\v_n\end{mymatrix}
  \quad\mbox{and}\quad
  \vect{w}=\begin{mymatrix}{c}w_1\\\vdots\\w_n\end{mymatrix}
\end{equation*}
by column vectors. Then
\begin{equation*}
  \vect{v}\dotprod\vect{w}
  ~=~
  v_1w_1 + \ldots + v_nw_n
  ~=~
  \begin{mymatrix}{ccc} v_1 & \cdots & v_n \end{mymatrix}
  \begin{mymatrix}{c} w_1 \\ \vdots \\ w_n \end{mymatrix}
  ~=~
  \vect{v}^T\vect{w}.
\end{equation*}
In other words, the dot product of column vectors $\vect{v}$ and
$\vect{w}$ is the same thing as the matrix product
$\vect{v}^T\vect{w}$.

We can also use the notion of transpose to define what it means for a
matrix to be \textbf{symmetric} and \textbf{antisymmetric}.

\begin{definition}{Symmetric and antisymmetric matrices}{symmetric-and-antisymmetric}
  An $n\times n$-matrix $A$ is said to be \textbf{symmetric}%
  \index{matrix!symmetric}\index{symmetric matrix} if $A^T=A$. It is
  said to be \textbf{antisymmetric}%
  \index{matrix!antisymmetric}\index{antisymmetric matrix} (sometimes
  also called \textbf{skew symmetric}%
  \index{matrix!skew symmetric}\index{skew symmetric matrix}) if
  $A^T=-A$.
\end{definition}

\begin{example}{Symmetric and antisymmetric matrices}{symmetric-matrix}
  Let
  \begin{equation*}
    A=\begin{mymatrix}{rrr}
      2 & 1 & 3 \\
      1 & 5 & -3 \\
      3 & -3 & 7
    \end{mymatrix},
    \quad
    B=\begin{mymatrix}{rrr}
      0 & 1 & 3 \\
      -1 & 0 & 2 \\
      -3 & -2 & 0
    \end{mymatrix},
    \quad\mbox{and}\quad
    C=\begin{mymatrix}{rrr}
      0 & 1 &  3 \\
      1 & 5 & -3 \\
      1 & 3 &  0
    \end{mymatrix}.
  \end{equation*}
  Then $A$ is symmetric because $A^T=A$, $B$ is antisymmetric
  because $B^T=-B$, and $C$ is neither symmetric nor antisymmetric
  because $C^T$ is equal to neither $C$ nor $-C$.
\end{example}

