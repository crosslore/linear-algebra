\section{Diagonalization of symmetric matrices}

In Chapter~\ref{cha:eigenvalues}, we saw that some matrices are
diagonalizable, and others are not. In this section, we will see that
the theory of diagonalization is much nicer when the matrix to be
diagonalized is symmetric. Recall that a matrix $A$ is
\textbf{symmetric}%
\index{symmetric matrix}%
\index{matrix!symmetric} if $A=A^T$. We begin with some observations
about the eigenvalues and eigenvectors of a symmetric matrix.

\begin{proposition}{Eigenvalues and eigenvectors of symmetric matrices}{eigenvalues-symmetric}
  Let $A$ be a symmetric matrix with real entries. Then
  \begin{enumialphparenastyle}
    \begin{enumerate}
    \item All eigenvalues of $A$ are real (i.e., not complex).
    \item Eigenvectors for distinct eigenvalues of $A$ are orthogonal.
    \end{enumerate}
  \end{enumialphparenastyle}
\end{proposition}

\begin{proof}
  (a) Suppose $\eigenvar$ is a (possibly complex) eigenvalue of $A$,
  with (possibly complex) eigenvector $\vect{v}$. We will show that
  $\eigenvar$ is in fact real. Let $\conjugate{\eigenvar}$ be the
  complex conjugate of $\eigenvar$, and let $\conjugate{\vect{v}}$ be
  the complex conjugate of $\vect{v}$, i.e., the vector obtained from
  $\vect{v}$ by taking the complex conjugate of each of its entries.
  Since $\vect{v}$ is an eigenvector for eigenvalue $\eigenvar$ of
  $A$, we have
  \begin{equation}\label{eqn:eigenvalues-symmetric-1}
    \conjugate{\vect{v}}^T A\vect{v}
    = \eigenvar \conjugate{\vect{v}}^T\vect{v}.
  \end{equation}
  Taking the complex conjugate of both sides of the equation, and
  using the fact that $\conjugate{A}=A$ (since $A$ is
  real), we get
  \begin{equation}
    \vect{v}^T A\conjugate{\vect{v}}
    = \conjugate{\eigenvar} \vect{v}^T\conjugate{\vect{v}}.
  \end{equation}
  Then, taking the transpose of both sides of the equation, and using
  the fact that $A^T=A$ (since $A$ is symmetric), we have
  \begin{equation}\label{eqn:eigenvalues-symmetric-2}
    \conjugate{\vect{v}}^T A\vect{v}
    = \conjugate{\eigenvar} \conjugate{\vect{v}}^T\vect{v}.
  \end{equation}
  Comparing equations {\eqref{eqn:eigenvalues-symmetric-1}} and
  {\eqref{eqn:eigenvalues-symmetric-2}}, we find that
  $\eigenvar \conjugate{\vect{v}}^T\vect{v} = \conjugate{\eigenvar}
  \conjugate{\vect{v}}^T\vect{v}$. Since
  $\conjugate{\vect{v}}^T\vect{v}$ is a non-zero scalar, it follows
  that $\eigenvar = \conjugate{\eigenvar}$, i.e., $\eigenvar$ is real.

  \noindent
  (b) Suppose $\vect{v}$ is an eigenvector for eigenvalue
  $\eigenvarA$, $\vect{w}$ is an eigenvector for eigenvalue
  $\eigenvarB$, and $\eigenvarA\neq\eigenvarB$.  We evaluate
  $\vect{v}^TA\vect{w}$ in two different ways. On the on hand, we have
  \begin{equation*}
    \vect{v}^TA\vect{w}
    = \vect{v}^T(A\vect{w})
    = \vect{v}^T(\eigenvarB\vect{w})
    = \eigenvarB \vect{v}^T\vect{w}.
  \end{equation*}
  On the other hand, we have
  \begin{equation*}
    \vect{v}^TA\vect{w}
    = (\vect{v}^TA)\vect{w}
    = (A^T\vect{v})^T\vect{w}
    = (A\vect{v})^T\vect{w}
    = (\eigenvarA\vect{v})^T\vect{w}
    = \eigenvarA\vect{v}^T\vect{w}.
  \end{equation*}
  Therefore,
  $\eigenvarB\vect{v}^T\vect{w} = \eigenvarA\vect{v}^T\vect{w}$, or
  equivalently $(\eigenvarA - \eigenvarB)\vect{v}^T\vect{w} =
  0$. Since by assumption, $\eigenvarA - \eigenvarB\neq 0$, we must
  have $\vect{v}^T\vect{w} = 0$, i.e., $\vect{v}\orth\vect{w}$.
\end{proof}

Recall that an $n\times n$-matrix $A$ is \textbf{diagonalizable}%
\index{diagonalizable matrix}%
\index{matrix!diagonalizable} if there exists an invertible matrix $P$
and a diagonal matrix $D$ such that $D = P^{-1}AP$. We say that $A$ is
\textbf{orthogonally diagonalizable}%
\index{orthogonally diagonalizable matrix}%
\index{diagonalizable matrix!orthogonally diagonalizable}%
\index{diagonalization!orthogonal diagonalization}%
\index{matrix!orthogonally diagonalizable}%
\index{matrix!diagonalizable!orthogonally} if $P$ can, moreover, be
chosen to be orthogonal. Orthogonal diagonalizability is a convenient
property, because when $P$ is orthogonal, then $P^{-1}=P^T$, and we
can interchangeably write $D = P^{-1}AP$ or $D = P^TAP$.  The
following is the main theorem about the diagonalization of real
symmetric matrices.

\begin{theorem}{Diagonalization of symmetric matrices}{diagonalization-symmetric}
  Every real symmetric matrix $A$ is orthogonally diagonalizable.
\end{theorem}

\begin{proof}
  By induction on the size of the matrix. For a $1\times 1$-matrix,
  there is nothing to show, as it is already diagonal. Now consider a
  real symmetric $n\times n$-matrix $A$ with $n\geq 2$. By the
  fundamental theorem of algebra, the characteristic polynomial has at
  least one root, so that $A$ has at least one (possibly complex)
  eigenvalue $\eigenvar$. By Proposition~\ref{prop:eigenvalues-symmetric}(a),
  $\eigenvar$ is real. Since we can solve the system of equations
  $(A-\eigenvar I)\vect{v}=\vect{0}$ over the real numbers, there
  exists a real eigenvector $\vect{v}$ for the eigenvalue
  $\eigenvar$. We can assume without loss of generality that
  $\vect{v}$ is normalized, because else we could replace $\vect{v}$
  by $\frac{1}{\norm{\vect{v}}}\vect{v}$. By the Gram-Schmidt method,
  we can find an orthonormal basis
  $\set{\vect{u}_1,\ldots,\vect{u}_n}$ of $\R^n$ such that
  $\vect{u}_1=\vect{v}$. Let $Q$ be the orthogonal matrix that has
  $\vect{u}_1,\ldots,\vect{u}_n$ as its columns, and consider
  $B=Q^{-1}AQ$. Since
  $B\vect{e}_1 = Q^{-1}AQ\vect{e}_1 = Q^{-1}A\vect{u}_1 = \eigenvar
  Q^{-1}\vect{u}_1 = \eigenvar\vect{e}_1$, the matrix $B$ is of the
  form
  \begin{equation*}
    B = \begin{mymatrix}{cccc}
      \eigenvar & b_{12} & \cdots & b_{1n} \\
      0 & b_{22} & \cdots & b_{2n} \\
      \vdots & \vdots & \ddots & \vdots \\
      0 & b_{n2} & \cdots & b_{nn} \\
    \end{mymatrix}.
  \end{equation*}
  Moreover, since $B=Q^{-1}AQ = Q^TAQ$, the matrix $B$ is symmetric.
  Therefore $b_{12},\ldots,b_{1n} = 0$, and $B$ is of the form
  \begin{equation*}
    B = \begin{mymatrix}{c|ccc}
      \eigenvar & 0 & \cdots & 0 \\\hline
      0 &  & & \\
      \vdots & & C & \\
      0 &  & & \\
    \end{mymatrix},
  \end{equation*}
  where $C$ is a symmetric matrix of dimension $n-1$. By induction
  hypothesis, $C$ is orthogonally diagonalizable, i.e., there exists
  an orthogonal matrix $R$ such that $R^{-1}CR$ is diagonal. Let
  \begin{equation*}
    S = \begin{mymatrix}{c|ccc}
      1 & 0 & \cdots & 0 \\\hline
      0 &  & & \\
      \vdots & & R & \\
      0 &  & & \\
    \end{mymatrix}.
  \end{equation*}
  Then $S$ is orthogonal and $S^{-1}BS$ is diagonal, say
  $S^{-1}BS = D$. Let $P=QS$. Then $P$ is orthogonal, and we have
  \begin{equation*}
    P^{-1}AP = S^{-1}Q^{-1}AQS = S^{-1}BS = D.
  \end{equation*}
  Therefore, $A$ is orthogonally diagonalizable.
\end{proof}

\begin{example}{Diagonalization of symmetric matrices}{diagonalization-symmetric}
  Orthogonally diagonalize the matrix
  \begin{equation*}
    A = \begin{mymatrix}{rr}
      3 & 2 \\
      2 & 6 \\
    \end{mymatrix},
  \end{equation*}
  i.e., find an orthogonal matrix $P$ and a diagonal matrix $D$ such
  that $D = P^{-1}AP$.
\end{example}

\begin{solution}
  We proceed in much the same way as in Chapter~\ref{cha:eigenvalues},
  except that at a crucial moment, we ensure that $P$ is orthogonal.
  We start by calculating the characteristic polynomial of $A$:
  \begin{equation*}
    \det(A-\eigenvar I)
    ~=~ \begin{absmatrix}{cc}
      3-\eigenvar & 2 \\
      2 & 6-\eigenvar
    \end{absmatrix}
    ~=~ (3-\eigenvar)(6-\eigenvar) - 4
    ~=~ \eigenvar^2 - 9\eigenvar + 14.
  \end{equation*}
  We find the roots using the quadratic formula. The roots of the
  characteristic polynomial, and therefore the eigenvalues of $A$, are
  $\eigenvar=7$ and $\eigenvar=2$. For the eigenvalue $\eigenvar=7$,
  we find the eigenvector
  \begin{equation*}
    \vect{v} = \begin{mymatrix}{r} 1 \\ 2 \end{mymatrix},
  \end{equation*}
  and for the eigenvalue $\eigenvar=2$, we find the eigenvector
  \begin{equation*}
    \vect{w} = \begin{mymatrix}{r} -2 \\ 1 \end{mymatrix}.
  \end{equation*}
  We note that these two eigenvectors are orthogonal to each other,
  exactly as predicted by Proposition~\ref{prop:eigenvalues-symmetric}. So
  the vectors $\set{\vect{v},\vect{w}}$ form an orthogonal set. We
  turn this into an orthonormal set by normalizing each eigenvector,
  i.e., the normalized eigenvectors are
  \begin{equation*}
    \vect{u}_1
    ~=~ \frac{1}{\norm{\vect{v}}}\vect{v}
    ~=~ \frac{1}{\sqrt{5}} \begin{mymatrix}{r} 1 \\ 2 \end{mymatrix}
    \quad\mbox{and}\quad
    \vect{u}_2
    ~=~ \frac{1}{\norm{\vect{w}}}\vect{w}
    ~=~ \frac{1}{\sqrt{5}} \begin{mymatrix}{r} -2 \\ 1 \end{mymatrix}.
  \end{equation*}
  We let $P$ be the matrix that has columns $\vect{u}_1$ and
  $\vect{u}_2$, i.e.,
  \begin{equation*}
    P = \frac{1}{\sqrt{5}} \begin{mymatrix}{rr} 1 & -2 \\ 2 & 1 \end{mymatrix}.
  \end{equation*}
  Note that since $\vect{u}_1$ and $\vect{u}_2$ are orthonormal, $P$
  is automatically orthogonal by
  Proposition~\ref{prop:conditions-orthogonal-matrix}. Moreover, we
  have
  \begin{equation*}
    P^{-1}
    ~=~ P^T
    ~=~ \frac{1}{\sqrt{5}} \begin{mymatrix}{rr} 1 & 2 \\ -2 & 1 \end{mymatrix}.
  \end{equation*}
  Finally,
  \begin{equation*}
    P^{-1}AP
    ~=~ P^TAP
    ~=~ \frac{1}{5}
    \begin{mymatrix}{rr} 1 & 2 \\ -2 & 1 \end{mymatrix}
    \begin{mymatrix}{rr} 3 & 2 \\ 2 & 6 \end{mymatrix}
    \begin{mymatrix}{rr} 1 & -2 \\ 2 & 1 \end{mymatrix}
    ~=~ \begin{mymatrix}{rr} 7 & 0 \\ 0 & 2 \end{mymatrix}.
  \end{equation*}
\end{solution}

\begin{example}{Diagonalization of symmetric matrices}{diagonalization-symmetric2}
  Find an orthogonal matrix $P$ and a diagonal matrix $D$ such $D =
  P^{-1}AP$, where
  \begin{equation*}
    A = \begin{mymatrix}{rrr}
      3  & 1 & -2 \\
      1  & 3 &  2 \\
      -2 & 2 &  0 \\
    \end{mymatrix}.
  \end{equation*}
\end{example}

\begin{solution}
  Again, we start by calculating the characteristic polynomial and its roots:
  \begin{equation*}
    \det(A-\eigenvar I)
    ~=~ \begin{absmatrix}{ccc}
      3-\eigenvar  & 1 & -2 \\
      1  & 3-\eigenvar &  2 \\
      -2 & 2 &  -\eigenvar  \\
    \end{absmatrix}
    ~=~ - \eigenvar^3 + 6\eigenvar^2 -32.
  \end{equation*}
  The roots are $\eigenvar=4$ and $\eigenvar=-2$. To find the
  eigenvectors for $\eigenvar=4$, we solve the system of equations
  \begin{equation*}
    (A-4I)\vect{v}
    =
    \begin{mymatrix}{rrr}
      -1  & 1 & -2 \\
      1  & -1 &  2 \\
      -2 & 2 &  -4 \\
    \end{mymatrix}\vect{v}
    = \vect{0}.
  \end{equation*}
  The solution space is 2-dimensional, with basis
  \begin{equation*}
    \vect{v}_1 = \begin{mymatrix}{r} 1 \\ 1 \\ 0 \end{mymatrix}
    \quad\mbox{and}\quad
    \vect{v}_2 = \begin{mymatrix}{r} 0 \\ 2 \\ 1 \end{mymatrix}.
  \end{equation*}
  To find the eigenvectors for $\eigenvar=-2$, we solve the system of
  equations
  \begin{equation*}
    (A+2I)\vect{v}
    =
    \begin{mymatrix}{rrr}
      5  & 1 & -2 \\
      1  & 5 &  2 \\
      -2 & 2 &  2 \\
    \end{mymatrix}\vect{v}
    = \vect{0}.
  \end{equation*}
  The solution space is 2-dimensional, with basis
  \begin{equation*}
    \vect{v}_3 = \begin{mymatrix}{r} 1 \\ -1 \\ 2 \end{mymatrix}.
  \end{equation*}
  As predicted by Proposition~\ref{prop:eigenvalues-symmetric},
  $\vect{v}_3$ is orthogonal to $\vect{v}_1$ and
  $\vect{v}_2$. However, there is a slight complication: $\vect{v}_1$
  and $\vect{v}_2$ are not orthogonal to each other. This is because
  they are eigenvectors for the {\em same} eigenvalue ($\eigenvar=4$),
  not for {\em distinct} eigenvalues. We found basis
  $\set{\vect{v}_1,\vect{v}_2}$ for the eigenspace for $\eigenvar=4$,
  but it doesn't happen to be an orthogonal basis. However, we can fix
  this by applying the Gram-Schmidt method to
  $\set{\vect{v}_1,\vect{v}_2}$. This yields the orthogonal basis
  $\set{\vect{u}_1,\vect{u}_2}$ for the eigenspace for $\eigenvar=4$,
  where
  \begin{equation*}
    \vect{u}_1 = \begin{mymatrix}{r} 1 \\ 1 \\ 0 \end{mymatrix}
    \quad\mbox{and}\quad
    \vect{u}_2 = \begin{mymatrix}{r} -1 \\ 1 \\ 1 \end{mymatrix}.
  \end{equation*}
  We now have an orthogonal basis
  $\set{\vect{u}_1,\vect{u}_2,\vect{v}_3}$ of eigenvectors of the
  matrix $A$. We normalize the three vectors and use them as the
  columns of $P$:
  \begin{equation*}
    \def\arraystretch{1.4}
    P = \begin{mymatrix}{ccc}
      \frac{1}{\sqrt{2}} & -\frac{1}{\sqrt{3}} & \frac{1}{\sqrt{6}}  \\
      \frac{1}{\sqrt{2}} & \frac{1}{\sqrt{3}}  & -\frac{1}{\sqrt{6}} \\
      0                  & \frac{1}{\sqrt{3}}  & \frac{2}{\sqrt{6}}  \\
    \end{mymatrix}.
  \end{equation*}
  It seems that inverting $P$ will not be all that easy, but in fact,
  since $P$ is orthogonal, the inverse is just the transpose:
  \begin{equation*}
    \def\arraystretch{1.4}
    P^{-1}
    ~=~ P^T
    ~=~ \begin{mymatrix}{ccc}
      \frac{1}{\sqrt{2}}  & \frac{1}{\sqrt{2}}  & 0 \\
      -\frac{1}{\sqrt{3}} & \frac{1}{\sqrt{3}}  & \frac{1}{\sqrt{3}} \\
      \frac{1}{\sqrt{6}}  & -\frac{1}{\sqrt{6}} & \frac{2}{\sqrt{6}} \\
    \end{mymatrix}.
  \end{equation*}
  We have
  \begin{equation*}
    P^{-1}AP
    ~=~ D
    ~=~ \begin{mymatrix}{ccc} 4 & 0 & 0 \\ 0 & 4 & 0 \\ 0 & 0 & -2 \end{mymatrix}.
  \end{equation*}
\end{solution}
