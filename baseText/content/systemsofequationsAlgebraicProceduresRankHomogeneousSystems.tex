\section{Homogeneous systems}
\label{sec:homogeneous-systems}

\begin{outcome}
  \begin{enumerate}
  \item[A.] Determine whether a homogeneous system of equations has
    non-trivial solutions from its rank.
    
  \item[B.] Find the basic solutions of a homogeneous system of
    equations.
    
  \item[C.] Understand the relationship between the general solution of
    a system of equations and that of its its associated homogeneous
    system.
  \end{enumerate}
\end{outcome}

There is a special type of system of linear equations that requires
additional study. This type of system is called a {\em
  homogeneous}\footnote{The word ``homogeneous'' has 5 syllables. In
  scientific usage, it is not the same as the word
  ``\nospellcheck{\homogenous}''.}  system of equations.  Our focus in
this section is to consider what types of solutions are possible for a
homogeneous system of equations, and how the solutions of
non-homogeneous systems are related to those of their homogeneous
counterparts.

\begin{definition}{Homogeneous system of equations}{homogeneous-system}
  A system of equations is called
  \textbf{homogeneous}\index{system of linear equations!homogeneous}\index{homogeneous system}
  if each of the constant terms is equal to $0$. A homogeneous system
  therefore has the form
\begin{equation*}
\begin{array}{c}
a_{11}x_{1}+a_{12}x_{2}+\cdots +a_{1n}x_{n}= 0 \\
a_{21}x_{1}+a_{22}x_{2}+\cdots +a_{2n}x_{n}= 0  \\
\vdots \\
a_{m1}x_{1}+a_{m2}x_{2}+\cdots +a_{mn}x_{n}= 0, 
\end{array}
\end{equation*}
where $a_{ij}$ are coefficients and $x_{i}$ are variables.
\end{definition}

The first thing we note is that a homogeneous system is always
consistent. Indeed, it always has the solution $x_1=0, x_2=0, \ldots,
x_n=0$. This solution is called the
\textbf{trivial solution}\index{trivial solution}\index{solution!trivial}.

If the system has a solution in which not all of the
$x_1, \cdots, x_n$ are equal to zero, then we call this solution
\textbf{nontrivial}\index{nontrivial
  solution}\index{solution!nontrivial}.  When working with homogeneous
systems of equations, since the trivial solution always exists, we are
usually interested in finding whether there are non-trivial solutions.

The following theorem is a special case of
Theorem~\ref{thm:rank-consistent-solutions}. Recall that the
{\em rank}\index{rank}
of a system is the number of pivot variables in its {\ef}.

\begin{theorem}{Rank and solutions of homogeneous system of equations}{rank-homogeneous-solutions}
  Consider a homogeneous system of $m$ equations in $n$ variables, and
  assume that the coefficient matrix has rank $r$. Then the system is
  consistent, and 
\begin{enumerate}
\item if $r=n$, then the system has a unique solution;
\item if $r<n$, then the system has infinitely many solutions.
\end{enumerate}
\end{theorem}

\begin{example}{Homogeneous system with more variables than equations}{homo-more-vars}
  True or false: Suppose a homogeneous system has more variables than
  equations. Then the system has infinitely many solutions.  
\end{example}

\begin{solution}
  This is true. If the system has $m$ equations and $n$ variables,
  then the rank can be at most $m$. Since $m<n$, the system has
  infinitely many solutions. Note that it is not possible for a
  homogeneous system to be inconsistent, since there is always the
  trivial solution.
\end{solution} 

\begin{example}{Homogeneous system with an equal number of variables and equations}{homo-less-vars}
  True or false: Suppose a homogeneous system has the same number of
  variables as equations. Then the system has a unique solution.
\end{example}

\begin{solution}
  This is false in general. While it is possible for such a system to
  have a unique solution, it is also possible for it to have
  infinitely many. Let there be $n$ equations and $n$ variables.  Then
  depending on the {\ef}, the rank $r$ could be either equal to $n$,
  in which case there is a unique solution, or less than $n$, in which
  case there are infinitely many.
\end{solution}

We now consider an example of solving a homogeneous system of equations.

\begin{example}{Solutions to a homogeneous system of equations}{homogeneous-solution}
Find the general solution to the following homogeneous system of
equations. Does the system have non-trivial solutions?
\begin{equation*}
\begin{array}{c}
2x + y - z + w = 0 \\
x + 2y - 2z + 3w = 0
\end{array}
\end{equation*}
\end{example}

\begin{solution}
  Notice that this system has $m = 2$ equations and $n = 4$ variables,
  so $n>m$.  Therefore by our previous discussion, we expect this
  system to have infinitely many solutions. In particular, it will
  have non-trivial solutions.

  The process we use to find the solutions for a homogeneous system of
  equations is the same process we used for non-homogeneous
  equations. We construct the augmented matrix and reduce it to
  {\rref}.
  \begin{equation*}
    \begin{mymatrix}{rrrr|r}
      2 & 1 & 1 & 4 & 0 \\ 
      1 & 2 & -1 & 5 & 0
    \end{mymatrix}
    \sim
    \begin{mymatrix}{rrrr|r}
      1 & 0 &  1 & 1 & 0 \\ 
      0 & 1 & -1 & 2 & 0
    \end{mymatrix}
  \end{equation*}
  The corresponding system of equations is 
  \begin{equation*}
    \begin{array}{r}
      x + z + w = 0 \\
      y - z + 2w = 0. \\
    \end{array}
  \end{equation*}
  The free variables are $z$ and $w$. We set them equal to parameters
  $z=s$ and $w=t$. Then our general solution has the form
  \begin{equation*}
    \begin{array}{c}
      x = -s-t \\
      y = s-2t \\
      z = s \\
      w = t.
    \end{array}
  \end{equation*}
  Hence this system has infinitely many solutions, with two parameters
  $s$ and $t$.
\end{solution}

Let us write the solution of the last example in another form.
Specifically, it can be written as 
\begin{equation}\label{eqn:homogeneous-solution-1}
  \begin{mymatrix}{r}
    x\\
    y\\
    z\\
    w
  \end{mymatrix}
  =
  s
  \begin{mymatrix}{r}
    -1\\
    1\\
    1\\
    0
  \end{mymatrix}
  +
  t
  \begin{mymatrix}{r}
    -1\\
    -2\\
    0\\
    1
  \end{mymatrix}.
\end{equation}
Notice that we have constructed a column from the coefficients of $s$
in each equation, and another column from the coefficients of $t$.  We
will discuss this notation more in later chapters. For now, consider
what happens when we choose the parameters to be $s=1$ and $t=0$. In
this case, we get the solution
\begin{equation}\label{eqn:homogeneous-solution-2}
  \begin{mymatrix}{r}
    -1\\
    1\\
    1\\
    0
  \end{mymatrix},
\end{equation}
which is the same as the column of coefficients for $s$. This is
called a \textbf{basic solution}\index{basic solution}\index{solution!basic} of the
homogeneous system of equations. The other basic solution is obtained
by setting $s=0$ and $t=1$. In this case,
\begin{equation}\label{eqn:homogeneous-solution-3}
  \begin{mymatrix}{r}
    -1\\
    -2\\
    0\\
    1
  \end{mymatrix}.
\end{equation}
The basic solutions of a system are columns constructed from the
coefficients on parameters in the solution. If $X_1$ and $X_2$ are the
basic solutions {\eqref{eqn:homogeneous-solution-2}} and
{\eqref{eqn:homogeneous-solution-3}}, then the general solution
{\eqref{eqn:homogeneous-solution-1}} is of the form $sX_1+tX_2$.  We
say that the general solution of the homogeneous system is a
\textbf{linear combination}\index{linear combination!of basic
  solutions} of its basic solutions.

We explore this further in the following example.

\begin{example}{Basic solutions of a homogeneous system}{basic-solutions}
  Consider the following homogeneous system of equations. 
  \begin{equation}\label{eqn:basic-solutions-1}
    \begin{array}{c}
      x + 4y + 3z = 0 \\
      3x + 12y + 9z = 0.
    \end{array}
  \end{equation}
  Find the basic solutions to this system.
\end{example}

\begin{solution}
  The augmented matrix of this system and the resulting {\rref} are 
  \begin{equation*}
    \begin{mymatrix}{rrr|r}
      1 & 4 & 3 & 0 \\
      3 & 12 & 9 & 0
    \end{mymatrix}
    \sim
    \begin{mymatrix}{rrr|r}
      1 & 4 & 3 & 0 \\
      0 & 0 & 0 & 0
    \end{mymatrix}.
  \end{equation*}
  When written in equations, this system is given by 
  \begin{equation*}
    x + 4y +3z=0.
  \end{equation*}
  Notice that $x$ is the only pivot variable, and $y$ and $z$ are free
  variables. Let $y = s$ and $z=t$ for parameters $s$ and $t$. Then the
  general solution is
  \begin{equation*}
    \begin{array}{c}
      x = -4s - 3t \\
      y = s \\
      z = t,
    \end{array}
  \end{equation*}
  which can be written as 
  \begin{equation*}
    \begin{mymatrix}{r}
      x\\
      y\\
      z
    \end{mymatrix}
    =
    s
    \begin{mymatrix}{r}
      -4 \\
      1 \\
      0
    \end{mymatrix}
    + 
    t
    \begin{mymatrix}{r}
      -3 \\
      0 \\
      1
    \end{mymatrix}.
  \end{equation*}
  You can see here that we have two columns of coefficients
  corresponding to parameters, specifically one for $s$ and one for $t$.
  Therefore, this system has two basic solutions! They are
  \begin{equation*}
    X_1=
    \begin{mymatrix}{r}
      -4 \\
      1 \\
      0
    \end{mymatrix},\quad X_2 = \begin{mymatrix}{r}
      -3 \\
      0 \\
      1
    \end{mymatrix}.
  \end{equation*} 
\end{solution}

We can take any non-homogeneous system of equations and get a new
homogeneous system by keeping the left-hand sides the same and setting
all of the constant terms equal to $0$. This is called the
\textbf{associated homogeneous system}\index{homogeneous system!associated}\index{associated homogeneous system}
of the system of equations. We end this section by investigating how
the solutions of a system of equations are related to the solutions of
its associated homogeneous system.

\begin{example}{Non-homogeneous vs. homogeneous system}{non-homo-vs-homo}
  Solve the system of equations
  \begin{equation}\label{eqn:non-homo-vs-homo-1}
    \begin{array}{c}
      x + 4y + 3z = 2 \\
      3x + 12y + 9z = 6.
    \end{array}
  \end{equation}
  How are the solutions related to those of the associated homogeneous
  system in Example~\ref{exa:basic-solutions}?
\end{example}

\begin{solution}
  We note that the associated homogeneous system of
  {\eqref{eqn:non-homo-vs-homo-1}} is the system we saw in
  Example~\ref{exa:basic-solutions}. We solve the system
  {\eqref{eqn:non-homo-vs-homo-1}} in the usual way by reducing its
  augmented matrix to {\rref}
  \begin{equation*}
    \begin{mymatrix}{rrr|r}
      1 & 4 & 3 & 2 \\
      3 & 12 & 9 & 6
    \end{mymatrix}
    \sim
    \begin{mymatrix}{rrr|r}
      1 & 4 & 3 & 2 \\
      0 & 0 & 0 & 0
    \end{mymatrix}
  \end{equation*}
  and then assigning parameters $y=s$, $z=t$ to the free
  variables. From the equation $x+4y+3z=2$, the general solution is
  \begin{equation*}
    \begin{array}{c}
      x = 2 -4s - 3t \\
      y = s \\
      z = t,
    \end{array}
  \end{equation*}
  which can be written as 
  \begin{equation}\label{eqn:non-homo-vs-homo-2}
    \begin{mymatrix}{r}
      x\\
      y\\
      z
    \end{mymatrix}
    =
    \begin{mymatrix}{r}
      2 \\
      0 \\
      0
    \end{mymatrix}
    + 
    s
    \begin{mymatrix}{r}
      -4 \\
      1 \\
      0
    \end{mymatrix}
    + 
    t
    \begin{mymatrix}{r}
      -3 \\
      0 \\
      1
    \end{mymatrix}.
  \end{equation}
  We see that the general solution is almost exactly the same as that
  of the homogeneous system in Example~\ref{exa:basic-solutions}. The
  only difference is the additional column
  \begin{equation}\label{eqn:non-homo-vs-homo-3}
    \begin{mymatrix}{r}
      2 \\
      0 \\
      0
    \end{mymatrix}
  \end{equation}    
\end{solution}

Note that the column {\eqref{eqn:non-homo-vs-homo-3}}, by itself, is a
solution of the non-homogeneous system. It is not the most general
solution, but rather the particular solution resulting from the
parameters $s=0$ and $t=0$. We can therefore interpret equation
{\eqref{eqn:non-homo-vs-homo-2}} as saying that the general
solution\index{general solution} of the non-homogeneous system is
equal to a particular solution\index{particular solution} of the
non-homogeneous system, plus the general solution of the associated
homogeneous system. The same is true in general, and we summarize it
as a theorem.

\begin{theorem}{Non-homogeneous vs. homogeneous system}{non-homo-vs-homo}
  Let $A$ be a system of equations, and let $B$ be the associated
  homogeneous system. Then
  \begin{equation*}
    \mbox{the general solution of $A$}
    = \mbox{a particular solution of $A$}
    + \mbox{the general solution of $B$}.
  \end{equation*}
\end{theorem}
