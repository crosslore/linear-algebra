\section{Homogeneous Systems}

There is a special type of system which requires additional study. This type of system is called a homogeneous system of
equations, which we defined above in Definition \ref{def:homogeneoussystem}. 
Our focus in this section is to consider what types of solutions are possible for a homogeneous system of equations. 

Consider the following definition. 

\begin{definition}{Trivial Solution}{trivialsolution}
Consider the homogeneous system of equations given by
\begin{equation*}
\begin{array}{c}
a_{11}x_{1}+a_{12}x_{2}+\cdots +a_{1n}x_{n}= 0 \\
a_{21}x_{1}+a_{22}x_{2}+\cdots +a_{2n}x_{n}= 0  \\
\vdots \\
a_{m1}x_{1}+a_{m2}x_{2}+\cdots +a_{mn}x_{n}= 0 
\end{array}
\end{equation*}
Then, $x_{1} = 0, x_{2} = 0, \cdots, x_{n} =0$ is always a 
solution to this system. We call this the \textbf{trivial solution}\index{trivial solution}.
\end{definition}

If the system has a solution in which not all of the $x_1, \cdots, x_n$ are equal to zero,
then we call this solution \textbf{nontrivial}\index{nontrivial solution}. The trivial solution
does not tell us much about the system, as it says that $0=0$! 
Therefore, when working with homogeneous systems of 
equations, we want to know when the system has a nontrivial solution. 

Suppose we have a homogeneous system of $m$ equations, using $n$ variables, and suppose that
$n > m$. In other words, there are more variables than equations. 
Then, it turns out that this system always has a nontrivial solution. Not only will the
system have a nontrivial solution, but it also will have infinitely many solutions.
It is also possible, but not required, to have a nontrivial solution if $n=m$ and $n<m$.

Consider the following example.

\begin{example}{Solutions to a Homogeneous System of Equations}{homogeneoussolution}
Find the nontrivial solutions to the following homogeneous system of equations
\begin{equation*}
\begin{array}{c}
2x + y - z = 0 \\
x + 2y - 2z = 0
\end{array}
\end{equation*}
\end{example}

\begin{solution}
Notice that this system has $m = 2$ equations and $n = 3$ variables, so $n>m$.
Therefore by our previous discussion, we expect this system to have infinitely many solutions. 

The process we use to find the solutions for a homogeneous system of equations
is the same process we used in the previous section. 
First, we construct the augmented matrix, given by 
\begin{equation*}
\leftB
\begin{array}{rrr|r}
2 & 1 & -1 & 0 \\ 
1 & 2 & -2 & 0
\end{array}
\rightB
\end{equation*}
Then, we carry this matrix to its {\rref}, given below. 
\begin{equation*}
\leftB
\begin{array}{rrr|r}
1 & 0 & 0 & 0 \\ 
0 & 1 & -1 & 0
\end{array}
\rightB
\end{equation*}
The corresponding system of equations is 
\begin{equation*}
\begin{array}{c}
x = 0 \\
y - z =0 \\
\end{array}
\end{equation*}
Since $z$ is not restrained by any equation, we know that this variable will become our parameter. 
Let $z=t$ where $t$ is any number. 
Therefore, our solution has the form
\begin{equation*}
\begin{array}{c}
x = 0 \\
y = z = t \\
z = t
\end{array}
\end{equation*}
Hence this system has infinitely many solutions, with one parameter $t$.
\end{solution}

Suppose we were to write the solution to the previous example in another form.
Specifically,
\begin{equation*}
\begin{array}{c}
x = 0 \\
y = 0 + t \\
z = 0 + t
\end{array}
\end{equation*}
can be written as 
\begin{equation*}
\leftB
\begin{array}{r}
x\\
y\\
z
\end{array}
\rightB
 =
\leftB
\begin{array}{r}
0\\
0\\
0
\end{array}
\rightB
 +
t
\leftB
\begin{array}{r}
0\\
1\\
1
\end{array}
\rightB
\end{equation*}
Notice that we have constructed a column from the constants in the solution (all equal to $0$),
 as well as a column corresponding to the 
coefficients on $t$ in each equation. While we will discuss this form of solution more in further chapters, 
for now consider the column of coefficients 
of the parameter $t$. In this case, this is the column
$\leftB
\begin{array}{r}
0\\
1\\
1
\end{array}
\rightB$. 

There is a special name for this column, which is \textbf{basic solution}\index{basic solution}. The basic solutions of a system 
are columns constructed from the coefficients on parameters in the solution. We often denote basic solutions 
by $X_1, X_2$ etc., depending on how many solutions occur. Therefore, Example \ref{exa:homogeneoussolution}
has the basic solution $X_1 = \leftB
\begin{array}{r}
0\\
1\\
1
\end{array}
\rightB$. 

We explore this further in the following example. 

\begin{example}{Basic Solutions of a Homogeneous System}{basicsolutions}
Consider the following homogeneous system of equations. 
\begin{equation*}
\begin{array}{c}
x + 4y + 3z = 0 \\
3x + 12y + 9z = 0
\end{array}
\end{equation*}
Find the basic solutions to this system.
\end{example}

\begin{solution}
The augmented matrix of this system and the resulting {\rref} are 
\begin{equation*}
\leftB
\begin{array}{rrr|r}
1 & 4 & 3 & 0 \\
3 & 12 & 9 & 0
\end{array}
\rightB
\rightarrow \cdots \rightarrow
\leftB
\begin{array}{rrr|r}
1 & 4 & 3 & 0 \\
0 & 0 & 0 & 0
\end{array}
\rightB
\end{equation*}
When written in equations, this system is given by 
\begin{equation*}
x + 4y +3z=0
\end{equation*}
Notice that only $x$ corresponds to a pivot column. In this case, we will have two parameters, 
one for $y$ and one for $z$. Let $y = s$ and $z=t$ for any numbers $s$ and $t$. Then, our solution becomes
\begin{equation*}
\begin{array}{c}
x = -4s - 3t \\
y = s \\
z = t
\end{array}
\end{equation*}
which can be written as 
\begin{equation*}
\leftB
\begin{array}{r}
x\\
y\\
z
\end{array}
\rightB
=
\leftB
\begin{array}{r}
0\\
0\\
0
\end{array}
\rightB
+
s
\leftB
\begin{array}{r}
-4 \\
1 \\
0
\end{array}
\rightB
+ 
t
\leftB
\begin{array}{r}
-3 \\
0 \\
1
\end{array}
\rightB
\end{equation*}
You can see here that we have two columns of coefficients corresponding to parameters, specifically one for $s$ and one for $t$. 
Therefore, this system has two basic solutions! These are
\begin{equation*}
X_1=
\leftB
\begin{array}{r}
-4 \\
1 \\
0
\end{array}
\rightB, X_2 = \leftB
\begin{array}{r}
-3 \\
0 \\
1
\end{array}
\rightB
\end{equation*} 
\end{solution}

We now present a new definition. 

\begin{definition}{Linear Combination}{linearcombination-col-matrices}
Let $X_1,\cdots ,X_n,V$ be column matrices. Then 
$V$ is said to be a \textbf{linear combination}
\index{linear combination} of the columns $X_1,\cdots , X_n $ 
if there exist scalars, $a_{1},\cdots ,a_{n}$ such
that
\begin{equation*}
V = a_1 X_1 + \cdots + a_n X_n
\end{equation*}
\end{definition}

A remarkable result of this section is that a linear combination of the basic solutions is again a solution to the system.
Even more remarkable is that every solution can be written as a linear combination of these solutions. 
Therefore, if we take a linear combination of the two solutions to Example \ref{exa:basicsolutions},
this would also be a solution. 
For example, we could take the following linear combination
\begin{equation*}
3
\leftB
\begin{array}{r}
-4 \\
1 \\
0
\end{array}
\rightB
+
2
\leftB
\begin{array}{r}
-3 \\
0\\
1
\end{array}
\rightB
 =
\leftB
\begin{array}{r}
-18 \\
3 \\
2
\end{array}
\rightB
\end{equation*}
You should take a moment to verify that
\begin{equation*}
\leftB
\begin{array}{r}
x \\
y \\
z
\end{array}
\rightB
=
\leftB
\begin{array}{r}
-18 \\
3 \\
2
\end{array}
\rightB
\end{equation*}
is in fact a solution to the system in Example \ref{exa:basicsolutions}.

