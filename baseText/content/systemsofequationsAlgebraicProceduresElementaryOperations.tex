\section{Elementary Operations}

\begin{outcome}
  \begin{enumerate}
  \item[A.] Use elementary operations to find the solution to a linear
    system of equations.
  \end{enumerate}
\end{outcome}

Our strategy for solving systems of linear equations is to
successively transform a difficult system of equations into a simpler
equivalent system. Here, by an ``equivalent'' system of equations we
mean one that has the same solutions as the original one. We will
perform the process of simplifying a system of equations by applying
certain basic steps called ``elementary operations''.

\begin{definition}{Equivalent Systems}{equivalentsystems}
  Two systems of equations are called
  \textbf{equivalent}\index{system of linear equations!equivalence}
  if they have the same solutions. This means that every solution of
  the first system is also a solution of the second system, and every
  solution of the second system is also a solution of the first system.
\end{definition}

How can we know whether two systems of equations are equivalent? It
turns out that the following basic operations are all that is needed
to transform any system of equations into any equivalent system. In
face, these operations are the {\em key tool} we use in linear algebra
to solve systems of equations.

\begin{definition}{Elementary Operations}{elementaryoperations}
\textbf{Elementary operations} \index{elementary operations} are the
following operations:

\begin{enumerate}
\item Interchange the order in which the equations are listed.

\item Multiply any equation by a nonzero number.

\item Add a multiple of one equation to another equation.
\end{enumerate}
\end{definition}

The most important property of the elementary operations is that they
do not change the solutions to the system of equations. Before proving
that this is true in general, we will first verify it in an example.

\begin{example}{Equivalent Systems}{equivalentsystems}
Show that the systems
\begin{equation*}
\begin{array}{r@{~}c@{~}l}
x+2y&=&7 \\
-2x  &=& -6
\end{array}
\end{equation*}
and 
\begin{equation*}
\begin{array}{r@{~}c@{~}l}
x+2y&=&7 \\
4y &=& 8
\end{array}
\end{equation*}
are equivalent.
\end{example}

\begin{solution}
  We can see that the second system is obtained from the first one by
  applying an elementary operation, namely, adding 2 times the first
  equation to the second equation:
  \begin{equation*}
      -2x + 2(x+2y) = -6 + 2(7)
  \end{equation*}
  By simplifying, we obtain $4y = 8$.

  To verify that the two systems are indeed equivalent, let us first
  solve the first system. From the second equation, we see that
  $x=3$. Substituting $x=3$ into the first equation, the equation
  becomes $3+2y=7$, which we can solve to find $y=2$. Therefore, the
  only solution to the first system of equations is $(x,y) = (3,2)$.

  Now let us solve the second system. From the second equation, we
  find that $y=2$. Substituting $y=2$ into the first equation, we get
  $x+4=7$, which we can solve to find $x=3$. Therefore, the only
  solution to the second system of equations is $(x,y) = (3,2)$.
  Since the two systems have the same solutions, they are equivalent.
\end{solution}

This example illustrates how an elementary operation applied to a
system of two equations in two variables does not affect the set of
solutions. The same is true for any size of system in any number of
variables.  In the following theorem, we use the notation $E_i$ to
represent the left-hand side of an equation, while $b_i$ denotes a
constant term.

\begin{theorem}{Elementary Operations and Solutions}{elementaryoperationsandsolns}
Suppose you have a system of two linear equations in any number of variables
\begin{equation}
 \begin{array}{c}
  E_{1}=b_{1}\\
  E_{2}=b_{2}.
\end{array} \label{system}
\end{equation}
Then the following systems are equivalent to \eqref{system}: 
\begin{enumerate}
\item   \begin{equation}
	\begin{array}{c}
	E_{2}=b_{2}\\
	E_{1}=b_{1}.
	\end{array}
	\label{thm1.9.1}
	\end{equation}
\item  \begin{equation}
	\begin{array}{c}
	E_{1}=b_{1} \\
	kE_{2}=kb_{2}\\        
	\end{array}
	\label{thm1.9.2}
	\end{equation}
  for any real number $k$, provided $k\neq0$.
\item \begin{equation}
      \begin{array}{c}
       E_{1}=b_{1} \\
       E_{2}+kE_{1}=b_{2}+kb_{1}
       \end{array}  
	\label{thm1.9.3}
	\end{equation}
	for any real number  $k$ (including $k=0$).

\end{enumerate}
\end{theorem}

\begin{proof} 
\begin{enumerate}
\item By definition, a solution of \eqref{system} is an assignment of
  real numbers to the variables that is a solution to $E_1=b_1$ and to
  $E_2=b_2$. But that is exactly the same thing as a solution of
  \eqref{thm1.9.1}.

\item To prove that the systems \eqref{system} and \eqref{thm1.9.2}
  have the same solution set, let $\left(x_{1},\cdots,x_{n}\right)$ be
  any solution of \eqref{system}. Then $E_1=b_1$ and $E_2=b_2$ are
  both true. Multiplying both sides of the last equation by $k$, we
  know that $kE_2=kb_2$ is true, and so
  $\left(x_{1},\cdots,x_{n}\right)$ is a solution of
  \eqref{thm1.9.2}. Conversely, let $\left(x_{1},\cdots,x_{n}\right)$
  be any solution of \eqref{thm1.9.2}.  Then $E_1=b_1$ and $kE_2=kb_2$
  are true. Because $k\neq 0$, we are allowed to divide both sides of
  the last equation by $k$, and therefore $E_2=b_2$ is true. Hence,
  $\left(x_{1},\cdots,x_{n}\right)$ is also a solution of
  \eqref{system}. Since we have shown that every solution of
  \eqref{system}  is a solution of \eqref{thm1.9.2} and vice versa,
  the two systems are equivalent.

\item To prove that the systems \eqref{system} and \eqref{thm1.9.3}
  have the same solution set, let $\left(x_{1},\cdots,x_{n}\right)$ be
  any solution of \eqref{system}. Then $E_1=b_1$ and $E_2=b_2$ are
  both true. We multiply both sides of the first equation by $k$ to
  obtain $kE_1=kb_1$. Then $kE_1+E_2 = kb_1+b_2$, and hence
  $\left(x_{1},\cdots,x_{n}\right)$ is a solution of
  \eqref{thm1.9.3}. For the converse direction, assume
  $\left(x_{1},\cdots,x_{n}\right)$ is a solution of $E_1=b_1$ and
  $kE_1+E_2 = kb_1+b_2$. From the first equation, we have $kE_1=kb_1$,
  and subtracting this from the second equation, we get $E_2=b_2$,
  hence $\left(x_{1},\cdots,x_{n}\right)$ is a solution of
  \eqref{system}. Note that unlike in case 2., there was no need to
  divide by $k$, and therefore it was not necessary to require $k\neq 0$.
\end{enumerate}
\end{proof}

We will now use elementary operations to solve a system of three
equations and three variables.

\begin{example}{Solving a System of Equations with Elementary Operations}{solvingasystemwithelementaryops}
Solve the system of equations
\begin{equation*}
\begin{array}{c@{~}c@{~}c@{~}c@{~}c@{~}c@{~}c}
x&+&3y&+&6z&=&25 \\
2x&+&7y&+&14z&=&58 \\
&&2y&+&5z&=&19.
\end{array}
\label{solvingasystem1}
\end{equation*}
\end{example}

\begin{solution}
  By Theorem \ref{thm:elementaryoperationsandsolns}, we can do
  elementary operations on this system without changing the solution
  set. We will therefore use elementary operations to try to simplify
  the system of equations.  First, we add $\left( -2\right)$ times the
  first equation to the second equation. This yields the system
  \begin{equation*}
    \begin{array}{c@{~}c@{~}c@{~}c@{~}c@{~}c@{~}c}
      x&+&3y&+&6z&=&25 \\
       &&y&+&2z&=&8 \\
       &&2y&+&5z&=&19.
    \end{array}
    \label{solvingasystem2}
  \end{equation*}
  Next, we add $\left(-2\right)$ times the second equation to the
  third equation. This yields the system
  \begin{equation}
    \begin{array}{c@{~}c@{~}c@{~}c@{~}c@{~}c@{~}c}
      x&+&3y&+&6z&=&25 \\
      &&y&+&2z&=&8 \\
      &&&&z&=&3.
    \end{array}
    \label{solvingasystem3}
  \end{equation}
  At this point, it is easy to find the solution. The last equation
  tells us that $z=3$. We can substitute this value of $z$ back into
  the second equation to get
  \begin{equation*}
    y+2(3)=8,
  \end{equation*}
  which we can simplify and solve for $y$ to find that $y=2$. Finally,
  we can substitute the values $z=3$ and $y=2$ back into the first
  equation to get
  \begin{equation*}
    x+3(2)+6(3)=25.
  \end{equation*}
  Simplifying and solving for $x$, we find that $x=1$. Hence, the
  solution to the system is $(x,y,z)=(1,2,3)$.

  The process we followed for solving \eqref{solvingasystem3} by first
  computing $z$, then $y$, then $x$ is called \textbf{back
    substitution}\index{back substitution}.  Alternatively, we could
  have continued from \eqref{solvingasystem3} with more elementary
  operations as follows. Add $\left( -2\right) $ times the third
  equation to the second and then add $\left( -6\right) $ times the
  second to the first. This yields
  \begin{equation*}
    \allowbreak
    \begin{array}{c@{~}c@{~}c@{~}c@{~}c@{~}c@{~}c}
      x&+&3y&&&=&7 \\
      &&y&&&=&2 \\
      &&&&z&=&3.
    \end{array}
  \end{equation*}
  Now add $\left( -3\right) $ times the second to the first. This yields
  \begin{equation*}
    \allowbreak
    \begin{array}{c@{~}c@{~}c@{~}c@{~}c@{~}c@{~}c}
      x&&&&&=&1 \\
      &&y&&&=&2 \\
      &&&&z&=&3,
    \end{array}
  \end{equation*}
  a system which has the same solution set as the original
  system. This second method avoided back substitution and led to the
  same solution set. It is your decision which you prefer to use, as
  both methods lead to the correct solution,
  $\left( x,y,z \right) = \left(1,2,3\right)$.

  Note how we have written each system of equations so that ``like''
  variables line up on columns: one column for $x$, one column for
  $y$, and one column for $z$. This makes it easier to perform
  elementary operations.
\end{solution}

We end this section by introducing some short-hand notations for
elementary operations. 

\begin{enumerate}
\item $R_i\leftrightarrow R_j$: switch equations $i$ and $j$.
\item $R_i\leftarrow kR_i$: multiply equation $i$ by $k$.
\item $R_i\leftarrow R_i+kR_j$: add $k$ times equation $j$ to equation $i$.
\end{enumerate}
