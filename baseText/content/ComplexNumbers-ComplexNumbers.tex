\section{The complex numbers}

\begin{outcome}
  \begin{enumerate}
  \item Add, subtract, multiply, and divide complex numbers.
  \item Find the conjugate and the magnitude of a complex number.
  \item Apply algebraic properties of the complex numbers to simplify
    equations.
  \end{enumerate}
\end{outcome}

\begin{definition}{The complex numbers}{complex-numbers}
  Let $i$ be an imaginary number such that $i\,^2=-1$. A \textbf{complex
    number}%
  \index{complex number} is a number of the form
  \begin{equation*}
    z = a + bi,
  \end{equation*}
  where $a$ and $b$ are real numbers. The set of all complex numbers
  is denoted $\C$.
\end{definition}

The form $z = a+bi$ is called the \textbf{standard form}%
\index{complex number!standard form}%
\index{standard form!of a complex number} or \textbf{Cartesian form}%
\index{complex number!Cartesian form}%
\index{Cartesian form!of a complex number} of the complex number $z$.
We refer to $a$ as the \textbf{real part}%
\index{complex number!real part}%
\index{real part!of a complex number} and to $b$ as the
\textbf{imaginary part}%
\index{complex number!imaginary part}%
\index{imaginary part!of a complex number} of $z$.

\textbf{Addition}%
\index{complex number!addition}%
\index{addition!of complex numbers}, \textbf{subtraction}%
\index{complex number!subtraction}%
\index{subtraction!of complex numbers}, and \textbf{multiplication}%
\index{complex number!multiplication}%
\index{multiplication!of complex numbers} of complex numbers are
defined in the obvious way, keeping in mind that $i\,^2=-1$. Namely, we
have
\begin{eqnarray*}
  (a+bi) + (c+di) &=& (a+c) + (b+d)i, \\
  (a+bi) - (c+di) &=& (a-c) + (b-d)i, \\
  (a+bi) (c+di)   &=& ac+adi+bci+bdi\,^2 ~=~ (ac-bd) + (ad + bc)i.
\end{eqnarray*}

\begin{example}{Addition, subtraction, and multiplication of complex numbers}{complex-add-subtract-multiply}
  \begin{itemize}
  \item $(3+5i) + (2-3i) = (3+2) + (5-3)i         = 5 + 2i$.
  \item $(3+5i) - (2-3i) = (3-2) + (5+3)i         = 1 + 8i$.
  \item $(3+5i) (2-3i)   = 6 - 9i + 10i - 15i\,^2 = 21 + i$.
  \end{itemize}
\end{example}

Division of complex numbers is more complicated. We first note that it
is easy to divide a complex number by a {\em real} number. Namely,
\begin{equation*}
  \frac{a+bi}{r} = \frac{a}{r} + \frac{b}{r}i.
\end{equation*}
But how can we divide by a complex number? We use the following
trick. Let $z=a+bi$ be a complex number, and consider the product
$(a+bi)(a-bi)$. It is equal to
\begin{equation*}
  (a+bi)(a-bi) = a^2 - b^2i\,^2 = a^2+b^2.
\end{equation*}
Therefore, $(a+bi)(a-bi)$ is always a {\em real} number, and therefore
easy to divide by. Therefore, we can compute the
\textbf{multiplicative inverse}%
\index{multiplicative inverse!of a complex number}%
\index{inverse!of a complex number}%
\index{complex number!inverse} of a complex number $z=a+bi$ as
follows:
\begin{equation*}
  z^{-1}
  = \frac{1}{z}
  = \frac{1}{a+bi}
  = \frac{1}{a+bi}\,\frac{a-bi}{a-bi}=\frac{a-bi}{a^2+b^2}.
\end{equation*}

\begin{example}{Inverse of a complex number}{complex-inverse}
  \begin{equation*}
    \frac{1}{2+5i}
    = \frac{2-5i}{2^2+5^2}
    = \frac{2}{29} - \frac{5}{29}i.
  \end{equation*}
\end{example}

You should verify that this is indeed the inverse, by multiplying
$2+5i$ by $\frac{2}{29} - \frac{5}{29}i$ and checking that the answer
is indeed $1$.

\textbf{Division}%
\index{complex number!division}%
\index{division!of complex numbers} of complex numbers can then be
defined in terms of the multiplicative inverse, i.e.,
$\frac{z}{w} = zw^{-1}$.

\begin{example}{Division of complex numbers}{complex-division}
  \begin{equation*}
    \frac{5+7i}{3-4i}
    = \frac{(5+7i)(3+4i)}{(3-4i)(3+4i)}
    = \frac{-13+41i}{3^2+4^2}
    = -\frac{13}{25} + \frac{41}{25}i.
  \end{equation*}
\end{example}

As a special case of division, note that
\begin{equation*}
  i\,^{-1} = \frac{1}{i} = \frac{1(-i)}{i(-i)} = -i.
\end{equation*}

The complex numbers form a {\em field}, i.e., they satisfy the nine
field axioms. See Section~\ref{sec:fields} for the definition of a
field.

\begin{proposition}{The complex numbers form a field}{complex-field}
  The complex numbers, with the operations of addition, subtraction,
  multiplication, and division, form a {\em field}%
  \index{complex number!field axioms}%
  \index{field!of complex numbers}%
  \index{properties of addition!complex numbers}%
  \index{properties of multiplication!complex numbers}. Specifically,
  this means that they satisfy the following properties:
  \begin{itemize}
  \item[(A1)] {Commutative law of addition:} $z+w=w+z$;
  \item[(A2)] {Associative law of addition:} $(z+w)+u = z+(w+u)$;
  \item[(A3)] {Unit law of addition:} $0+z = z$;
  \item[(A4)] {Additive inverse:} $z+(-z)=0$;
  \item[(M1)] {Commutative law of multiplication:} $zw=wz$;
  \item[(M2)] {Associative law of multiplication:} $(zw)u=z(wu)$;
  \item[(M3)] {Unit law of multiplication:} $1z=z$;
  \item[(M4)] {Multiplicative inverse:} when $z$ is non-zero: $zz^{-1}=1$;
  \item[(D)] {Distributive law:} $z(w+u)=zw+zu$.
  \end{itemize}
\end{proposition}

Another useful operation on complex numbers is the complex
conjugate. Let $z = a+bi$ be a complex number. Then the
\textbf{conjugate}%
\index{complex number!conjugate}%
\index{conjugate!of a complex number}%
\index{complex conjugate} of $z$, written $\conjugate{z}$, is given by
\begin{equation*}
  \conjugate{z} = a-bi.
\end{equation*}
Note that if $z=a+bi$ is a complex number, then
$z\conjugate{z} = (a+bi)(a-bi) = a^2+b^2$. Therefore, $z\conjugate{z}$
is always a real number and $z\conjugate{z}\geq 0$. We define the
\textbf{magnitude}%
\index{complex number!magnitude}%
\index{magnitude!of a complex number} of $z$ to be
\begin{equation*}
  \abs{z} = \sqrt{z\conjugate{z}} = \sqrt{a^2+b^2}.
\end{equation*}
The magnitude is also sometimes called the \textbf{absolute value}%
\index{complex number!absolute value|see{magnitude}}%
\index{absolute value!of a complex number|see{magnitude}} or the
\textbf{modulus}%
\index{complex number!modulus|see{magnitude}}%
\index{modulus!of a complex number|see{magnitude}} of the complex
number.

\begin{example}{Conjugate and magnitude of a complex number}{complex-conjugate}
  \begin{itemize}
  \item $\conjugate{3+5i} = 3-5i$.
  \item $\conjugate{i} = -i$.
  \item $\conjugate{7} = 7$.
  \item $\abs{3+5i} = \sqrt{3^2+5^2} = \sqrt{34}$.
  \item $\abs{i} = 1$.
  \item $\abs{-6} = 6$.
  \end{itemize}
\end{example}

Note that for a real number $a$, we have $\conjugate{a}=a$. Also, when
$a$ is real, the magnitude $|a| = \sqrt{a^2}$ is just the usual
absolute value of real numbers. The following two propositions
list some basic properties of the conjugate and of the magnitude.

\begin{proposition}{Properties of the conjugate}{properties-complex-conjugate}
  Let $z$ and $w$ be complex numbers. Then, the following properties
  of the conjugate hold.%
  \index{complex number!conjugate!properties}%
  \index{conjugate!of a complex number!properties}%
  \index{complex conjugate!properties}%
  \index{properties of complex conjugate}
  \begin{itemize}
  \item $\conjugate{z\pm w} = \conjugate{z} \pm \conjugate{w}$.
  \item $\conjugate{(zw)} = \conjugate{z}~ \conjugate{w}$.
  \item $\conjugate{z^{-1}} = \conjugate{z}^{-1}$.
  \item $\conjugate{z/w} = \conjugate{z} / \conjugate{w}$.
  \item $\conjugate{\conjugate{z}}=z$.
  \item $z$ is real if and only if $\conjugate{z}=z$.
  \end{itemize}
\end{proposition}

\begin{proposition}{Properties of the magnitude}{properties-complex-magnitude}
  Let $z,w$ be complex numbers.  The following properties hold.%
  \index{complex number!magnitude!properties}%
  \index{magnitude!of a complex number!properties}
  \begin{itemize}
  \item $\abs{z} \geq 0$, and $\abs{z}=0$ if and only if $z=0$.
  \item $\abs{zw} = \abs{z}\abs{w}$.
  \item $\abs{\conjugate{z}} = \abs{z}$.
  \item $\abs{z/w} = \abs{z}/\abs{w}$.
  \item \textbf{Triangle inequality}%
    \index{triangle inequality!in complex numbers}:
    $\abs{z+w} \leq \abs{z} + \abs{w}$.
  \end{itemize}
\end{proposition}
