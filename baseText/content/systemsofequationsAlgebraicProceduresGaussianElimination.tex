\section{Gaussian elimination}

\begin{outcome}
  \begin{enumerate}
  \item Find the {\ef} of a matrix.
  \item Determine whether a system of linear equations has no
    solution, a unique solution or an infinite number of solutions
    from its {\ef}.
  \item Solve a system of linear equations using Gaussian elimination
    and back substitution.
  \item Find the rank of a matrix.
  \item Determine whether a consistent system of linear equations has
    a unique solution or an infinite number of solutions from its
    rank.
  \end{enumerate}
\end{outcome}

In the previous section, we saw examples of how to solve a system of
equations using elementary row operations (and sometimes back
substitution). But it is not clear whether every system of equations
can be solved this way. How do we know which elementary row operation
to apply next? In this section, you will learn a procedure called {\em
  Gaussian elimination} by which every system of linear equations can
be solved systematically.

Before we start, let's figure out what it means to be ``done''. At
what point should we stop performing row operations? The answer is
that we will stop performing row operations when the system of
equations is in a special form called {\em {\ef}}, which we now define.

\begin{definition}{\Ef}{echelon-form}
  An entry of an augmented matrix is called a
  \textbf{leading entry}\index{leading entry} or
  \textbf{pivot entry}\index{pivot entry}
  if it is the leftmost non-zero entry of a row.
  An augmented matrix is in
  \textbf{\ef}\eindex{\ef}
  (also called \textbf{row echelon form}\index{row echelon form|see {\ef}}) if
  \begin{enumerate}
  \item All rows of zeros are below all non-zero rows.

  \item Each leading entry of a row is in a column to the right of the
    leading entry of any row above it.
  \end{enumerate}
  A column containing a pivot entry is also called a
  \textbf{pivot column}\index{pivot column}.
\end{definition}

The word {\em echelon} comes from French {\em \'echelle}, which means
ladder. This is because an {\ef} looks a bit like a ladder or
staircase. Here are some examples of {\ef}s.

\begin{example}{Matrices in {\ef}}{matrices-not-rref}
  The following augmented matrices are in {\ef}. We have circled the
  pivot entries for clarity.
  \begin{equation*}
    \begin{mymatrix}{rrrrr|r}
      0 & \circled{5} & 2 & 1 & 3 & 5 \\
      0 & 0 & 0 & 0 & \circled{1} & 6 \\
      0 & 0 & 0 & 0 & 0 & 0 \\
      0 & 0 & 0 & 0 & 0 & 0
    \end{mymatrix}, \quad \begin{mymatrix}{rrrrr|r}
      \circled{1} & 4 & 0 & 0 & 5 & 0 \\
      0 & 0 & \circled{1} & 0 & 2 & 0 \\
      0 & 0 & 0 & \circled{1} & 4 & 0 \\
      0 & 0 & 0 & 0 & 0 & \circled{1}
    \end{mymatrix}, \quad \begin{mymatrix}{rrr|r}
      \circled{3} & 0 & 6 & 2 \\
      0 & \circled{1} & 4 & 0 \\
      0 & 0 & \circled{2} & 1 \\
      0 &  0 & 0 & 0
    \end{mymatrix}
  \end{equation*}
\end{example}

\begin{example}{Not in {\ef}}{matrices-not-echelon}
The following augmented matrices are not in {\ef}.

\begin{equation*}
\begin{mymatrix}{rrr|r}
0 & 0 & 0 & 0 \\
\circled{1} & 2 & 3 & 3 \\
0 & \circled{1} & 0 & 2 \\
0 & 0 & 0 & \circled{1} \\
0 & 0 & 0 & 0
\end{mymatrix}, \quad \begin{mymatrix}{rr|r}
\circled{1} & 2 & 3 \\
\circled{2} & 4 & -6 \\
\circled{4} & 0 & 7
\end{mymatrix}, \quad \begin{mymatrix}{rrr|r}
0 & \circled{2} & 3 & 3 \\
\circled{1} & 5 & 0 & 2 \\
0 & 0 & \circled{1} & 0 \\
0 & 0 & 0 & 0
\end{mymatrix}
\end{equation*}
In the first matrix, a row of zeros is above a non-zero row. In the
second and third matrix, the leading entries of some rows are not to
the right of the leading entries of previous rows.
\end{example}

An augmented matrix can always be converted to {\ef} by using
elementary row operations. The following algorithm shows how to do
this.

\begin{algorithm}{Gaussian elimination}{gaussian-elimination}
  This algorithm provides a method for using row operations to take a
  matrix to its {\ef} \eindex{\ef!algorithm}\index{Gaussian elimination}.
  We begin with the matrix in its original form.

  \begin{enumerate}
  \item Starting from the left, find the first non-zero column. This is
    the first pivot column, and the position at the top of this column
    will be the position of the first pivot entry. Switch rows if
    necessary to place a non-zero number in the first pivot position.

  \item Use row operations to make the entries below the first pivot
    entry (in the first pivot column) equal to zero.

  \item Ignoring the row containing the first pivot entry, repeat
    steps 1 and 2 with the remaining rows.  Repeat the process until
    there are no more non-zero rows left.

  \end{enumerate}
\end{algorithm}

Most often we will apply this algorithm in order to solve a system of
linear equations. This works by first converting the system to {\ef},
then using back substitution to find the solutions. The next few
examples show how to do this.

\begin{example}{Solving a system of equations: one solution}{system-with-one-solution}
  Solve the following system of equations:
  \begin{equation*}
    \begin{array}{r@{~}c@{~}l}
      x+4y+3z &=& 11\\
      2x+10y+7z &=& 27\\
      x+y+2z &=& 5.
    \end{array}
  \end{equation*}
\end{example}

\begin{solution} The augmented matrix for this system is
  \begin{equation*}
    \begin{mymatrix}{rrr|r}
      1 & 4 & 3 & 11 \\
      2 & 10 & 7 & 27 \\
      1 & 1 & 2 & 5
    \end{mymatrix}.
  \end{equation*}
  In order to find the solution(s) to this system, we first use
  Algorithm~\ref{algo:gaussian-elimination} to carry the augmented
  matrix to {\ef}. Notice that the first column is non-zero, so this is
  our first pivot column. The first entry in the first row, $1$, is
  the first pivot entry. We will use row operations to create zeros in
  the entries below the $1$.
  \begin{equation*}
    \begin{mymatrix}{rrr|r}
      \circled{1} & 4 & 3 & 11 \\
      2 & 10 & 7 & 27 \\
      1 & 1 & 2 & 5
    \end{mymatrix}
    \stackrel{R_2\rowop R_2-2R_1}{\roweq}
    \begin{mymatrix}{rrr|r}
      \circled{1} & 4 & 3 & 11 \\
      0 & 2 & 1 & 5 \\
      1 & 1 & 2 & 5
    \end{mymatrix}
    \stackrel{R_3\rowop R_3-R_1}{\roweq}
    \begin{mymatrix}{rrr|r}
      \circled{1} & 4 & 3 & 11 \\
      0 & 2 & 1 & 5 \\
      0 & -3 & -1 & -6
    \end{mymatrix}
  \end{equation*}
  Now the entries in the first column below the pivot position are
  zeros. We now look for the second pivot column, which in this case
  is column two.  Here, the $2$ in the second row and second column is
  in the pivot entry. We could create a zero below the $2$ with a
  single row operation by adding $\frac{3}{2}$ times the second row
  from the third row. But it is sometimes more convenient not to work
  with fractions, and therefore we start instead by multiplying the
  third row by $2$.
  \begin{equation*}
    \begin{mymatrix}{rrr|r}
      \circled{1} & 4 & 3 & 11 \\
      0 & \circled{2} & 1 & 5 \\
      0 & -3 & -1 & -6
    \end{mymatrix}
    \stackrel{R_3\rowop 2R_3}{\roweq}
    \begin{mymatrix}{rrr|r}
      \circled{1} & 4 & 3 & 11 \\
      0 & \circled{2} & 1 & 5 \\
      0 & -6 & -2 & -12
    \end{mymatrix}
    \stackrel{R_3\rowop R_3+3R_2}{\roweq}
    \begin{mymatrix}{rrr|r}
      \circled{1} & 4 & 3 & 11 \\
      0 & \circled{2} & 1 & 5 \\
      0 & 0 & 1 & 3
    \end{mymatrix}
  \end{equation*}
  The final matrix is our desired {\ef}.
  \begin{equation}\label{eqn:system-with-one-solution-1}
    \begin{mymatrix}{rrr|r}
      \circled{1} & 4 & 3 & 11 \\
      0 & \circled{2} & 1 & 5 \\
      0 & 0 & \circled{1} & 3
    \end{mymatrix}
  \end{equation}
  Now we do back substitution to solve for $z$, $y$, and then $x$. The
  last equation of the {\ef} gives us $z=3$. Substituting this into
  the second equation, we get $2y+3=5$, which we can solve for $y$ to
  get $y=1$. Finally, substituting $y=1$ and $z=3$ into the first
  equation, we have $x+4(1)+3(3)=11$, which we can solve to get
  $x=-2$. Therefore we have the solution $(x,y,z)=(-2,1,3)$.

  At this point, it is a good idea to double-check this solution using
  the original equations
  \begin{equation*}
    \begin{array}{r@{~}c@{~}l}
      x+4y+3z &=& 11\\
      2x+10y+7z &=& 27\\
      x+y+2z &=& 5.
    \end{array}
  \end{equation*}
  For example, we can double-check that $(-2)+4(1)+3(3)$ is indeed
  equal to $11$, and similarly for the other two equations.
  Double-checking the solution against the original equations is an
  excellent way to guard against any errors that might have happened
  during the row operations or back substitution.

  Finally, we note that $(x,y,z)=(-2,1,3)$ is the {\em only} solution
  to this system of equations. There cannot be any other solutions,
  because by Theorem~\ref{thm:elementary-operations-and-solutions}, any
  solution of the original system of equations would also have to be a
  solution of {\eqref{eqn:system-with-one-solution-1}}, and the back
  substitution leaves us no choice except $z=3$, $y=1$, and $x=-2$.
\end{solution}

\begin{example}{Solving a system of equations: no solution}{system-with-no-solution}
  Solve the following system of equations:
  \begin{equation*}
    \begin{array}{r@{~}c@{~}l}
           y  +2z &=& 2\\
      2x  +y  -2z &=& 3\\
      4x  -y -10z &=& 4.
    \end{array}
  \end{equation*}
\end{example}

\begin{solution} The augmented matrix for this system is
  \begin{equation*}
    \begin{mymatrix}{rrr|r}
      0 &  1 &   2 & 2\\
      2 &  1 &  -2 & 3\\
      4 & -1 & -10 & 4
    \end{mymatrix}.
  \end{equation*}
  We use Algorithm~\ref{algo:gaussian-elimination} to carry the
  augmented matrix to {\ef}. The first column is non-zero and will be the
  first pivot column. We switch the first two rows to move a non-zero
  number into the pivot position:
  \begin{equation*}
  \stackrel{R_2\rowswap R_1}{\roweq}
    \begin{mymatrix}{rrr|r}
      \circled{2} &  1 &  -2 & 3\\
      0 &  1 &   2 & 2\\
      4 & -1 & -10 & 4
    \end{mymatrix}.
  \end{equation*}
  To create zeros below the pivot entry, we subtract $2$ times the
  first row from the third row:
  \begin{equation*}
  \stackrel{R_3\rowop R_3-2R_1}{\roweq}
    \begin{mymatrix}{rrr|r}
      \circled{2} &  1 &  -2 & 3\\
      0 &  1 &   2 & 2\\
      0 & -3 & -6 & -2
    \end{mymatrix}.
  \end{equation*}
  This finishes the first column. The second pivot column will be
  column two, with the $1$ in the second row and column as the pivot
  entry. We add $3$ times the second row to the third row to create a
  zero below the pivot:
  \begin{equation*}
  \stackrel{R_3\rowop R_3+3R_2}{\roweq}
    \begin{mymatrix}{rrr|r}
      \circled{2} &  1 &  -2 & 3\\
      0 &  \circled{1} &   2 & 2\\
      0 & 0 & 0 & \circled{4}
    \end{mymatrix}.
  \end{equation*}
  This matrix is in {\ef}. Note that the final pivot entry is on the
  right-hand side. The last row corresponds to the equation
  \begin{equation*}
    0x + 0y + 0z = 4.
  \end{equation*}
  This equation has no solution, because for all $x,y,z$, the
  left-hand size will equal $0$ and not $4$. Therefore, there is no
  solution to the given system of equations. In other words, the
  system is inconsistent.
\end{solution}

\begin{example}{Solving a system of equations: an infinite set of solutions}{infinite-set-of-solution}
  Solve the following system of equations:
  \begin{equation}
    \begin{array}{r@{~}c@{~}l}
      3x  -y  +5z &=& 8 \\
           y -10z &=& 1 \\
      6x  -y      &=& 17.
    \end{array}
  \end{equation}
\end{example}

\begin{solution} The augmented matrix of this system is
  \begin{equation*}
    \begin{mymatrix}{rrr|r}
      3 &  -1  &  5 & 8 \\
      0 &   1 & -10 & 1 \\
      6 &  -1  &  0 & 17
    \end{mymatrix}.
  \end{equation*}
  We use Gaussian elimination to carry the augmented matrix to
  {\ef}. The first column is the first pivot column, and $3$ is the
  pivot entry. We use row operating to create zeros beneath the pivot
  entry. We subtract $2$ times the first row from the third row and get:
  \begin{equation*}
  \stackrel{R_3\rowop R_3-2R_1}{\roweq}
    \begin{mymatrix}{rrr|r}
      \circled{3} &  -1  &  5 & 8 \\
      0 &   1 & -10 & 1 \\
      0 &   1 & -10 & 1
    \end{mymatrix}
  \end{equation*}
  Now, we have created zeros beneath the pivot entry in the first
  column, so we move on to the second pivot column (which is the
  second column) and repeat the procedure. Subtracting the second row
  from the third row, we get:
  \begin{equation*}
  \stackrel{R_3\rowop R_3-R_2}{\roweq}
    \begin{mymatrix}{rrr|r}
      \circled{3} &  -1  &  5 & 8 \\
      0 &   \circled{1} & -10 & 1 \\
      0 &   0 & 0 & 0
    \end{mymatrix}
  \end{equation*}
  This matrix is now in {\ef}. Observe that the first two columns are
  pivot columns, and the third column is not. We call the
  corresponding variables $x$ and $y$
  \textbf{pivot variables}\index{pivot variable}\index{variable!pivot},
  and the variable $z$ is a
  \textbf{free variable}\index{free variable}\index{variable!free}.
  The equations corresponding to
  this {\ef} are
  \begin{equation*}
    \begin{array}{r@{~}c@{~}l}
      3x - y + 5z &=& 8 \\
      y - 10z &=& 1.
    \end{array}
  \end{equation*}
  Observe that the free variable $z$ is not constrained by any
  equation. In fact, $z$ can equal any number. We choose $t$ to be any
  number and let $z = t$.  In this context $t$ is called a
  \textbf{parameter}\index{parameter}. We then use back substitution
  to solve for the pivot variables $y$ and $x$. From the second
  equation, we have $y = 1+10z = 1+10t$. From the first equation, we
  have $3x = 8+y-5z = 2+(1+10t)-5t = 9+5t$, and therefore
  $x=3+\frac{5}{3}t$. Therefore, the general solution of this system
  is
  \begin{equation*}
    \begin{array}{l}
      x=3+\frac{5}{3}t \\
      y=1+10t \\
      z=t,
    \end{array}
  \end{equation*}
  where $t$ is arbitrary. The system has an infinite set of solutions
  which are given by these equations. For any value of the parameter
  $t$ we select, $x$, $y$, and $z$ will be given by the above
  equations. For example, if we choose $t=4$ then the corresponding
  solution would be
  \begin{equation*}
    \begin{array}{l}
      x = 3 + \frac{5}{3}(4) = \frac{29}{3}\\
      y = 1+10(4) = 41 \\
      z = 4.
    \end{array}
  \end{equation*}
\end{solution}

In Example~\ref{exa:infinite-set-of-solution} the solution involved a
parameter. It may happen that the solution to a system involves more
than one parameter, as shown in the following example.

\begin{example}{Solving a system of equations: a two parameter set of solutions}{two-parameter-set-of-solution}
  Solve the following system of equations:
  \begin{equation*}
    \begin{array}{r@{~}c@{~}l}
      x +2y  -2z  +2w &=& 3 \\
      x +2y  -z   +3w &=& 5 \\
      x +2y  -3z  +1w &=& 1.
    \end{array}
  \end{equation*}
\end{example}

\begin{solution} The augmented matrix is
  \begin{equation*}
    \begin{mymatrix}{rrrr|r}
      1 & 2 & -2 & 2 & 3 \\
      1 & 2 & -1 & 3 & 5 \\
      1 & 2 & -3 & 1 & 1
    \end{mymatrix}.
  \end{equation*}
  We carry this matrix to {\ef} using row operations.
  \begin{equation*}
    \begin{mymatrix}{rrrr|r}
      1 & 2 & -2 & 2 & 3 \\
      1 & 2 & -1 & 3 & 5 \\
      1 & 2 & -3 & 1 & 1
    \end{mymatrix}
    \stackrel{R_2\rowop R_2-R_1}{\stackrel{R_3\rowop R_3-R_1}{\roweq}}
    \begin{mymatrix}{rrrr|r}
      \circled{1} & 2 & -2 & 2 & 3 \\
      0 & 0 & 1 & 1 & 2 \\
      0 & 0 & -1 & -1 & -2
    \end{mymatrix}
    \stackrel{R_3\rowop R_3+R_2}{\roweq}
    \begin{mymatrix}{rrrr|r}
      \circled{1} & 2 & -2 & 2 & 3 \\
      0 & 0 & \circled{1} & 1 & 2 \\
      0 & 0 & 0 & 0 & 0
    \end{mymatrix}.
  \end{equation*}
  This matrix is in {\ef} and we can see that the first and third
  columns are pivot columns, whereas the second and fourth columns are
  not. Therefore, $x$ and $z$ are pivot variables and $y$ and $w$ are
  free variables. We assign parameters $y=s$ and $w=t$ to the free
  variables. Then we do back substitution to solve for $z$ and $x$.
  From the second equation, we have $z=2-w=2-t$. From the first
  equation, we have $x=3-2y+2z-2w = 3-2s+2(2-t)-2t =
  7-2s-4t$. Therefore, the general solution is given by
  \begin{equation*}
    \begin{array}{r@{~}c@{~}l}
      x &=& 7-2s-4t \\
      y &=& s \\
      z &=& 2-t \\
      w &=& t.
    \end{array}
  \end{equation*}
  It is customary to write this solution in the form
  \begin{equation}\label{two-parameters-2}
    \begin{mymatrix}{c}
      x \\
      y \\
      z \\
      w
    \end{mymatrix} =\begin{mymatrix}{c}
      7-2s-4t \\
      s \\
      2-t \\
      t
    \end{mymatrix}.
  \end{equation}
\end{solution}

In
Examples~\ref{exa:system-with-one-solution}--\ref{exa:two-parameter-set-of-solution},
we have seem systems of equations with one solution, no solution, and
infinitely many solutions with one parameter as well as two
parameters. Moreover, in each case, we have been able to determine the
number of solutions by looking at the {\ef} of the augmented matrix.
To summarize, we have the following possibilities for a system of equations:

\begin{enumerate}
\item {\em No solution:} If the {\ef} has a row of the form
  \begin{equation*}
    \begin{mymatrix}{rrr|r}
      0 & 0 & 0 & b
    \end{mymatrix},
  \end{equation*}
  where $b\neq 0$, then the system is inconsistent and has no
  solution.

\item {\em One solution:} For a consistent system of equations: If
  every column of the coefficient matrix of the {\ef} is a pivot
  column, the system has exactly one solution. The following is an
  example of an augmented matrix in {\ef} for a system of equations
  with one solution.
  \begin{equation*}
    \begin{mymatrix}{rrr|r}
      \circled{1} & 1 & -2 & 5 \\
      0 & \circled{2} & 3 & 0 \\
      0 & 0 & \circled{1} & 2 \\
      0 & 0 & 0 & 0
    \end{mymatrix}.
  \end{equation*}

\item {\em Infinitely many solutions:} For a consistent system of
  equations: If not all columns of the coefficient matrix of the {\ef}
  are pivot columns, then the system has infinitely many solutions.
  In this case, each variable corresponding to a non-pivot column is a
  {\em free variable}\index{free variable}\index{variable!free} and
  can be assigned a {\em parameter}\index{parameter}. The remaining
  variables are
  {\em pivot variables}\index{pivot variable}\index{variable!pivot}
  and can be expressed in terms of the parameters. Therefore, the
  number of parameters in the general solution is equal to the number
  of non-pivot columns.
  The following are examples of {\ef}s for systems of equations with
  infinitely many solutions.
  \begin{equation*}
    \begin{mymatrix}{rrr|r}
      \circled{1} & 0 & 1 & 5 \\
      0 & \circled{1} & 2 & -3 \\
      0 & 0 & 0 & 0 \\
      0 & 0 & 0 & 0
    \end{mymatrix}
  \end{equation*}
  or
  \begin{equation*}
    \begin{mymatrix}{rrr|r}
      \circled{1} & 2 & 3 & 5 \\
      0 & 0 & \circled{4} & 6
    \end{mymatrix}.
\end{equation*}
\end{enumerate}

There is a special name for the number of pivot variables in a system
of equations. It is called the {\em rank} of the system.

\begin{definition}{Rank of a matrix}{rank}
  Let $A$ be a matrix and consider any {\ef} of $A$.  Then, the number
  $r$ of pivot entries of $A$ does not depend on the {\ef} we choose,
  and is called the \textbf{rank}\index{matrix!rank}\index{rank} of
  $A$. We denote it by $\rank(A)$.

  The rank of a system of linear equations is the rank of its
  coefficient matrix (i.e., the matrix on the left-hand side).
  It is equal to the number of pivot variables.
\end{definition}

\begin{example}{Finding the rank of a matrix}{rank-of-a-matrix}
Consider the matrix
\begin{equation*}
\begin{mymatrix}{rrr}
1 & 2 & 3 \\
1 & 5 & 9 \\
2 & 4 & 6
\end{mymatrix}
\end{equation*}
What is its rank?
\end{example}

\begin{solution}
  First, we need to find an {\ef} of $A$. Through the usual algorithm,
  we find that this is
  \begin{equation*}
    \begin{mymatrix}{rrr}
      \circled{1} & 0 & -1 \\
      0 & \circled{3} & 6 \\
      0 & 0 & 0
    \end{mymatrix}.
  \end{equation*}
  Here we have two pivot entries, and therefore the rank of $A$ is
  $r=2$.
\end{solution}

Suppose we have a system of $m$ equations in $n$ variables, and
suppose that $n>m$. Further assume that the system is consistent. From
our above discussion, we know that this system will have infinitely
many solutions. This is because there can be at most one pivot entry
per row, and therefore at most $m$ variables can be pivot
variables. It follows that there are at least $n-m$ free variables.
Therefore, the general solution of this system has at least $n-m$
parameters.

Notice that if $n=m$ or $n<m$, it is possible for the system to have a
unique solution or infinitely many solutions. In all cases ($n>m$,
$n=m$, or $n<m$), it is also possible for the system to be
inconsistent (have no solutions).

By refining the above argument, we get the following theorem:

\begin{theorem}{Rank and solutions of consistent system of equations}{rank-consistent-solutions}
  Consider a system of $m$ equations in $n$ variables, and assume that
  the coefficient matrix has rank $r$. Assume further that the system
  is consistent.
\begin{enumerate}
\item If $r=n$, then the system has a unique solution.
\item If $r<n$, then the system has infinitely many solutions, with $n-r$ parameters.
\end{enumerate}
\end{theorem}

Here is a final summary of how the rank affects the number of
solutions:

\begin{enumerate}
\item {\em No solution.} If the system of equations is inconsistent,
  then it has no solution, regardless of the rank.

\item {\em Unique solution.} For a consistent system, suppose
  $r=n$. Then there is a pivot position in every column of the
  coefficient matrix of $A$. Hence, there is a unique solution.

\item {\em Infinitely many solutions.} For a consistent system, suppose
  $r<n$. Then there are less pivot positions than columns in the
  coefficient matrix, meaning that not every column is a pivot
  column. The columns which are {\em not} pivot columns correspond to
  parameters. In fact, in this case we have $n-r$ parameters. The
  system has infinitely many solutions.
\end{enumerate}
