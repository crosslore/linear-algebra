\section{Positive semidefinite and positive definite matrices}

\begin{outcome}
  \begin{enumerate}
  \item Determine whether a matrix is positive semidefinite and/or
    positive definite, either directly or by looking at the
    eigenvalues.
  \item Determine whether a matrix is positive semidefinite and/or
    positive definite using Descartes's rule of signs.
  \item Determine whether a matrix defines an inner product.
  \end{enumerate}
\end{outcome}

In Example~\ref{exa:rn-with-inner-product}, we saw that it is
sometimes possible to define an inner product on $\R^n$ from an
$n\times n$-matrix $A$ by the formula
\begin{equation*}
  \iprod{\vect{u},\vect{v}} = \vect{u}^T A \vect{v}.
\end{equation*}
We will now explore in more detail under what conditions this formula
defines an inner product.

\begin{definition}{Positive semidefinite and positive definite matrices}{positive-matrix}
  Let $A$ be an $n\times n$-matrix over the real numbers.
  \begin{itemize}
  \item $A$ is called \textbf{symmetric}%
    \index{symmetric matrix}%
    \index{matrix!symmetric} if $A=A^T$.
  \item $A$ is called \textbf{positive semidefinite}%
    \index{positive semidefinite matrix}%
    \index{matrix!positive semidefinite}%
    \index{semidefinite} if it is symmetric and for all
    $\vect{v}\in\R^n$, we have $\vect{v}^TA\vect{v}\geq 0$.
  \item $A$ is called \textbf{positive definite}%
    \index{positive definite matrix}%
    \index{matrix!positive definite} if it is positive semidefinite
    and $\vect{v}^TA\vect{v}=0$ if and only if $\vect{v}=\vect{0}$.
\end{itemize}
\end{definition}

Equivalently, the positive definite property can also be stated as
follows: $A$ is positive definite if it is symmetric and for all
$\vect{v}\in\R^n$ with $\vect{v}\neq\vect{0}$, we have
$\vect{v}^TA\vect{v}>0$.

\begin{example}{Positive semidefinite and positive definite matrices}{positive-matrix}
  Which of the following matrices are positive semidefinite? Which
  ones are positive definite?
  \begin{equation*}
    A = \begin{mymatrix}{cc} 2 & -1 \\ -1 & 1 \end{mymatrix},\quad
    B = \begin{mymatrix}{cc} 1 & 2 \\ 2 & 1 \end{mymatrix},\quad
    C = \begin{mymatrix}{cc} 1 & -1 \\ -1 & 1 \end{mymatrix},\quad
    D = \begin{mymatrix}{cc} 2 & 0 \\ 2 & 1 \end{mymatrix}.
  \end{equation*}
\end{example}

\begin{solution}
  The matrix $A$ is positive definite. To see why, consider any vector
  $\vect{v}=\begin{mymatrix}{c} x \\ y \end{mymatrix}$. Then
  \begin{equation*}
    \vect{v}^TA\vect{v}
    ~=~
    \begin{mymatrix}{cc} x & y \end{mymatrix}
    \begin{mymatrix}{cc} 2 & -1 \\ -1 & 1 \end{mymatrix}
    \begin{mymatrix}{c} x \\ y \end{mymatrix}
    ~=~ 2x^2 - 2xy + y^2
    ~=~ x^2 + (x-y)^2
    ~\geq~ 0.
  \end{equation*}
  This inequality implies that $A$ is positive semidefinite. Moreover,
  $\vect{v}^TA\vect{v}=0$ if and only if $x=0$ and $x-y=0$, which
  implies $x=y=0$. So $\vect{v}=\vect{0}$ is the only solution of
  $\vect{v}^TA\vect{v}=0$, and $A$ is positive definite.

  The matrix $B$ is not positive semidefinite (and therefore not
  positive definite either). For example, consider
  $\vect{v}=\begin{mymatrix}{c} 1 \\ -1 \end{mymatrix}$. Then
  \begin{equation*}
    \vect{v}^TB\vect{v}
    ~=~
    \begin{mymatrix}{cc} 1 & -1 \end{mymatrix}
    \begin{mymatrix}{cc} 1 & 2 \\ 2 & 1 \end{mymatrix}
    \begin{mymatrix}{c} 1 \\ -1 \end{mymatrix}
    ~=~ 1-2-2+1
    ~=~ -2
    ~<~ 0,
  \end{equation*}
  showing that $B$ is not positive semidefinite.

  The matrix $C$ is positive semidefinite, but not positive definite.
  To see why, consider
  $\vect{v}=\begin{mymatrix}{c} x \\ y \end{mymatrix}$. Then
  \begin{equation*}
    \vect{v}^TC\vect{v}
    ~=~
    \begin{mymatrix}{cc} x & y \end{mymatrix}
    \begin{mymatrix}{cc} 1 & -1 \\ -1 & 1 \end{mymatrix}
    \begin{mymatrix}{c} x \\ y \end{mymatrix}
    ~=~ x^2 - 2xy + y^2
    ~=~ (x-y)^2
    ~\geq~ 0.
  \end{equation*}
  Therefore, $C$ is positive semidefinite. However, it is not positive
  definite because, for example, for
  $\vect{v}=\begin{mymatrix}{c} 1 \\ 1 \end{mymatrix}$, we have
  $\vect{v}\neq\vect{0}$ but $\vect{v}^TC\vect{v}=0$.

  The matrix $D$ is not symmetric, and therefore neither positive
  semidefinite nor positive definite.
\end{solution}

The interest of positive definite matrices lies in the following
proposition.

\begin{proposition}{Positive definite matrices and inner products on $\R^n$}{positive-definite-and-inner-product}
  Let $A$ be an $n\times n$-matrix over the real numbers. Then the
  formula
  \begin{equation*}
    \iprod{\vect{u},\vect{v}} = \vect{u}^T A \vect{v}
  \end{equation*}
  defines an inner product on $\R^n$ if and only if $A$ is positive
  definite. Moreover, every inner product on $\R^n$ arises from some
  positive definite matrix $A$ in this way.
\end{proposition}

\begin{proof}
  Define $\iprod{\vect{u},\vect{v}} = \vect{u}^T A \vect{v}$. We will
  check each of the three properties of an inner product from
  Definition~\ref{def:real-inner-product-space}. Linearity holds
  for all matrices $A$, because
  \begin{equation*}
    \iprod{\vect{u},k\vect{v}}
    ~=~ \vect{u}^T A (k\vect{v})
    ~=~ k(\vect{u}^T A \vect{v})
    ~=~ k\iprod{\vect{u},\vect{v}}
  \end{equation*}
  and
  \begin{equation*}
    \iprod{\vect{u},\vect{v}+\vect{v}'}
    ~=~ \vect{u}^T A (\vect{v}+\vect{v}')
    ~=~ \vect{u}^T A \vect{v} + \vect{u}^T A \vect{v}'
    ~=~ \iprod{\vect{u},\vect{v}} + \iprod{\vect{u},\vect{v}'}.
  \end{equation*}
  For symmetry, first observe that for all $\vect{u},\vect{v}$,
  \begin{equation*}
    \iprod{\vect{v},\vect{u}}
    ~=~ \vect{v}^T A \vect{u}
    ~=~ (\vect{v}^T A \vect{u})^T
    ~=~ \vect{u}^T A^T \vect{v}.
  \end{equation*}
  Therefore $\iprod{\vect{u},\vect{v}}=\iprod{\vect{v},\vect{u}}$ if
  and only if $\vect{u}^T A \vect{v} = \vect{u}^T A^T \vect{v}$. This
  holds for all $\vect{u},\vect{v}$ if and only if $A=A^T$. Therefore,
  symmetry holds if and only if $A$ is symmetric.  Finally, the
  positive definite property of the inner product holds, by
  definition, if and only if $A$ is positive definite.

  To prove the second part, consider any inner product
  $\iprod{\vect{u},\vect{v}}$ on $\R^n$. Let
  $\vect{e}_1,\ldots,\vect{e}_n$ be the standard basis vectors, and
  define $a_{ij} = \iprod{\vect{e}_i,\vect{e}_j}$. Then
  $A=\mat{a_{ij}}$ is a matrix. We claim that
  $\iprod{\vect{u},\vect{v}}=\vect{u}^TA\vect{v}$ for all
  $\vect{u},\vect{v}$. Indeed, let
  $\vect{u}=u_1\vect{e}_1+\ldots+u_n\vect{e}_n$ and
  $\vect{v}=v_1\vect{e}_1+\ldots+v_n\vect{e}_n$. Then
  \begin{eqnarray*}
    \iprod{\vect{u},\vect{v}}
    &=& \iprod{u_1\vect{e}_1+\ldots+u_n\vect{e}_n,~ v_1\vect{e}_1+\ldots+v_n\vect{e}_n} \\
    &=& u_1v_1\iprod{\vect{e}_1, \vect{e}_1}
    +   u_1v_2\iprod{\vect{e}_1, \vect{e}_2}
    + \ldots
    +   u_nv_n\iprod{\vect{e}_n, \vect{e}_n} \\
    &=& u_1v_1a_{11}
    +   u_1v_2a_{12}
    + \ldots
    +   u_nv_na_{nn} \\
    &=& \vect{u}^TA\vect{v}.
  \end{eqnarray*}
  By the first part, $A$ is positive definite.
\end{proof}

Given a symmetric matrix $A$, it is not always an easy task to
determine whether $A$ is positive definite (or semidefinite) by using
the definition directly. This would require checking the condition
$\vect{v}^TA\vect{v}\geq 0$ for {\em all} vectors $\vect{v}$, of which
there are infinitely many. The following proposition gives us a more
practical method for determining whether a matrix is positive
(semi)definite.

\begin{proposition}{Characterization of positive (semi)definite matrices using eigenvalues}{characterize-positive}
  Let $A$ be a symmetric $n\times n$-matrix, and let
  $\eigenvar_1,\ldots,\eigenvar_n$ be its eigenvalues. Then $A$ is
  positive semidefinite if and only if
  $\eigenvar_1,\ldots,\eigenvar_n\geq 0$. Moreover, $A$ is positive
  definite if and only if $\eigenvar_1,\ldots,\eigenvar_n>0$.
\end{proposition}

\begin{proof}
  By Theorem~\ref{thm:diagonalization-symmetric}, we know that $A$ is
  orthogonally diagonalizable, i.e., $D=P^TAP$, where $P$ is an
  orthogonal matrix and
  \begin{equation*}
    D = \begin{mymatrix}{ccc}
      \eigenvar_1 & \cdots & 0 \\
      \vdots & \ddots & \vdots \\
      0 & \cdots & \eigenvar_n
    \end{mymatrix}.
  \end{equation*}
  Let $\vect{v}=\mat{x_1,\ldots,x_n}^T$ be any vector, and let
  $\vect{w}=P\vect{v}$. Then
  $\vect{w}^TA\vect{w} = \vect{v}^TP^TAP\vect{v} =
  \vect{v}^TD\vect{v}$, so that $A$ is positive (semi)definite if and
  only if $D$ is positive (semi)definite. Also, we have
  $\vect{v}^TD\vect{v} = \eigenvar_1x_1^2 + \ldots +
  \eigenvar_nx_n^2$. Therefore $\vect{v}^TD\vect{v}\geq 0$ for all
  $\vect{v}$ if and only if $\eigenvar_1,\ldots,\eigenvar_n\geq
  0$. Moreover, $\vect{v}^TD\vect{v}>0$ for all
  $\vect{v}\neq \vect{0}$ if and only if
  $\eigenvar_1,\ldots,\eigenvar_n>0$, as claimed.
\end{proof}

\begin{example}{Using eigenvalues to check whether a matrix is positive definite}{characterize-positive}
  Use eigenvalues to determine whether the matrix
  \begin{equation*}
    A = \begin{mymatrix}{ccc}
      5 & 0 & 2 \\
      0 & 5 & 1 \\
      2 & 1 & 1 \\
    \end{mymatrix}
  \end{equation*}
  is positive definite, positive semidefinite, or neither.
\end{example}

\begin{solution}
  The characteristic polynomial is
  \begin{equation*}
    \det(A-\eigenvar I)
    ~=~ \begin{absmatrix}{ccc}
      5-\eigenvar & 0 & 2 \\
      0 & 5-\eigenvar & 1 \\
      2 & 1 & 1-\eigenvar \\
    \end{absmatrix}
    ~=~ -\eigenvar^3 + 11\eigenvar^2 - 30\eigenvar
    ~=~ -\eigenvar(\eigenvar-5)(\eigenvar-6).
  \end{equation*}
  The eigenvalues are $\eigenvar_1=0$, $\eigenvar_2=5$, and
  $\eigenvar_3=6$.  Therefore, the matrix $A$ is positive
  semidefinite, but not positive definite.
\end{solution}

While Proposition~\ref{prop:characterize-positive} gives us a method
to determine whether a matrix $A$ is positve definite (or
semidefinite), it still requires finding all of the eigenvalues of
$A$, i.e., to factor the characteristic polynomial. This can be a
difficult calculation, especially if the degree of the polynomial is
large or if the roots aren't integers. Fortunately, there is a better
way to determine whether a matrix is positive definite (or
semidefinite) by directly looking at the characteristic polynomial,
without having to calculate its roots. The method was found by
Ren\'e Descartes%
\index{Descartes, Ren\'e}%
\index{Ren\'e Descartes} in 1637 and is called {\em Descartes's rule
  of signs}. The general form of Descartes's rule of signs is
actually more complicated than what we consider here. We state a
version that has been specialized to characteristic polynomials of
symmetric matrices.

Let $a_0,\ldots,a_n$ be a sequence of real numbers. We say that
$a_0,\ldots,a_n$ have \textbf{strongly alternating signs}%
\index{strongly alternating signs}%
\index{alternating signs!strongly}%
\index{sign!strongly alternating} if $a_0>0$,
$a_1<0$, $a_2>0$, and so on (i.e., $a_i>0$ when $i$ is even, and
$a_i<0$ when $i$ is odd). We say that $a_0,\ldots,a_n$ have
\textbf{weakly alternating signs}%
\index{weakly alternating signs}%
\index{alternating signs!weakly}%
\index{sign!weakly alternating} if $a_0\geq 0$, $a_1\leq 0$,
$a_2\geq 0$, and so on (i.e., $a_i\geq 0$ when $i$ is even, and
$a_i\leq 0$ when $i$ is odd).


\begin{proposition}{Descartes's rule of signs for symmetric matrices}{descartes-rule-of-signs}
  Let $A$ be a symmetric $n\times n$-matrix, and let
  \begin{equation*}
    p(\eigenvar) = a_n\eigenvar^n + a_{n-1}\eigenvar^{n-1} + \ldots + a_1\eigenvar + a_0
  \end{equation*}
  be its characteristic polynomial. Then:
  \begin{itemize}
  \item $A$ is positive definite if and only if $a_0,\ldots,a_n$ have
    strongly alternating signs.
  \item $A$ is positive semidefinite if and only if $a_0,\ldots,a_n$
    have weakly alternating signs.
  \end{itemize}
\end{proposition}

\begin{proof}
  We first prove a general fact about polynomials. Suppose $d_1,\ldots,d_n$
  are real numbers and
  \begin{equation*}
    (x+d_1)(x+d_2)\cdots(x+d_n) = b_nx^n + b_{n-1}x^{n-1} + \ldots +
    b_1x + b_0.
  \end{equation*}
  Then $d_1,\ldots,d_n > 0$ if and only if $b_0,\ldots,b_n > 0$.
  Moreover, $d_1,\ldots,d_n \geq 0$ if and only if $b_0,\ldots,b_n \geq 0$.

  Proof: To prove the first claim, assume $d_1,\ldots,d_n > 0$. It is
  easy to see that by multiplying out $(x+d_1)(x+d_2)\cdots(x+d_n)$,
  we can only obtain positive coefficients, so $b_0,\ldots,b_n >
  0$. Conversely, assume $b_0,\ldots,b_n > 0$. Then clearly for every
  $x\geq 0$, we have $b_nx^n + b_{n-1}x^{n-1} + \ldots + b_1x + b_0 >
  0$, and therefore no such $x\geq 0$ can be a root of this
  polynomial.  In other words, all of the roots must be
  negative. Since the roots are $-d_1,\ldots,-d_n$, it follows that
  $d_1,\ldots,d_n>0$. The proof of the second claim (using ``$\geq$''
  instead of ``$>$'') is similar.
  
  We are now ready to prove
  Proposition~\ref{prop:descartes-rule-of-signs}. By
  Theorem~\ref{thm:diagonalization-symmetric}, we know that $A$ is
  diagonalizable, i.e., $A=PDP^{-1}$ for some real diagonal matrix
  $D$. Note that $A$ and $D$ have the same characteristic
  polynomial. If $d_1,\ldots,d_n$ are the diagonal entries of $D$, the
  characteristic polynomial can therefore be written in two different
  ways:
  \begin{equation*}
    p(\eigenvar)
    = (d_1-\eigenvar)(d_2-\eigenvar)\cdots(d_n-\eigenvar)
    = a_0 + a_1\eigenvar + a_2\eigenvar^2 + a_3\eigenvar^3 + \ldots + a_n\eigenvar^n.
  \end{equation*}
  Now let $\eigenvar=-x$ and consider
  \begin{equation*}
    p(-x) = (d_1+x)(d_2+x)\cdots(d_n+x)
    = a_0 - a_1 x + a_2 x^2 - a_3x^3 + \ldots \pm a_n x^n.
  \end{equation*}
  We have: $A$ is positive definite if and only if $d_1,\ldots,d_n>0$,
  if and only if $a_0,-a_1,a_2,-a_3,\ldots > 0$, if and only if
  $a_0,\ldots,a_n$ are strongly alternating.  Moreover, $A$ is
  positive semidefinite if and only if $d_1,\ldots,d_n\geq 0$, if and
  only if $a_0,-a_1,a_2,-a_3,\ldots \geq 0$, if and only if
  $a_0,\ldots,a_n$ are weakly alternating.
\end{proof}

\begin{example}{Using Descartes's rule of signs to check if a matrix is positive definite}{descartes-rule-of-signs1}
  Use Descartes's rule of signs to check whether the matrix
  \begin{equation*}
    A = \begin{mymatrix}{ccc}
      5 & 0 & 2 \\
      0 & 5 & 1 \\
      2 & 1 & 1 \\
    \end{mymatrix}
  \end{equation*}
  is positive definite, positive semidefinite, or neither.
\end{example}

\begin{solution}
  We already found the characteristic polynomial in
  Example~\ref{exa:characterize-positive}: it is $-\eigenvar^3 +
  11\eigenvar^2 - 30\eigenvar$. The coefficients are $a_0=0$,
  $a_1=-30$, $a_2=11$, and $a_3=-1$. Note that we have included all of
  the coefficients, even ones that are zero. Since the signs are
  weakly alternating, but not strongly alternating, the matrix is
  positive semidefinite, but not positive definite.
\end{solution}

\begin{example}{Using Descartes's rule of signs to check if a matrix is positive definite}{descartes-rule-of-signs2}
  Use Descartes's rule of signs to determine which of the following
  matrices are positive definite and/or positive semidefinite.
  \begin{equation*}
    A = \begin{mymatrix}{ccc}
      2 & 1 & 2 \\
      1 & 1 & 0 \\
      2 & 0 & 2 \\
    \end{mymatrix},
    \quad
    B = \begin{mymatrix}{ccc}
      2  &  1 & -2 \\
      1  &  1 & -1 \\
      -2 & -1 &  2 \\
    \end{mymatrix},
    \quad
    C = \begin{mymatrix}{ccc}
      2  & -1 & 2 \\
      -1 &  2 & 0 \\
      2  &  0 & 3 \\
    \end{mymatrix}.
  \end{equation*}
\end{example}

\begin{solution}
  The characteristic polynomials are:
  \begin{equation*}
    \begin{array}{lcl}
      \det(A-\eigenvar I) &=& -\eigenvar^3 + 5\eigenvar^2 - 3\eigenvar - 2, \\
      \det(B-\eigenvar I) &=& -\eigenvar^3 + 5\eigenvar^2 - 2\eigenvar + 0, \\
      \det(C-\eigenvar I) &=& -\eigenvar^3 + 7\eigenvar^2 - 11\eigenvar + 1. \\
    \end{array}
  \end{equation*}
  For $A$, the coefficients are not weakly alternating, so $A$ is not
  positive semidefinite. For $B$, the coefficients are weakly, but not
  strongly alternating (note that $a_0=0$), and therefore $B$ is
  positive semidefinite, but not positive definite. For $C$, the
  coefficients are strongly alternating, so $C$ is positive definite.
\end{solution}
