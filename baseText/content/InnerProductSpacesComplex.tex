\section{Complex inner product spaces}

So far, in this chapter, the field $K$ was always $\R$, the set of
real numbers.  The reason we have not considered inner products over
other fields $K$ is that the positive definite property requires
$\iprod{\vect{u},\vect{u}}\geq 0$, and the requirement that a scalar
is ``greater or equal to 0'' does not make sense if $K$ is, say, the
field of integers modulo $p$.

In this section, we will consider inner product spaces over the
complex numbers. It turns out that the theory of complex inner product
spaces is similar, but not completely identical, to that of real inner
product spaces. To explain the difference, consider the definition of
the dot product. In $\R^n$, the dot product of two vectors
\begin{equation*}
  \vect{v}=\begin{mymatrix}{c} x_1 \\ \vdots \\ x_n \end{mymatrix}
  \quad\mbox{and}\quad
  \vect{w}=\begin{mymatrix}{c} y_1 \\ \vdots \\ y_n \end{mymatrix}
\end{equation*}
is defined to be
\begin{equation*}
  \vect{v}\dotprod \vect{w} = x_1y_1 + \ldots + x_ny_n.
\end{equation*}
One of the most important properties of the dot product is positivity:
for all $\vect{v}$, we have
\begin{equation*}
  \vect{v}\dotprod\vect{v} = x_1^2 + \ldots + x_n^2\geq 0.
\end{equation*}
The reason positivity holds is that the square of a real number is
always greater or equal to 0. If we blindly replaced $x_1,\ldots,x_n$
by complex numbers and kept the same definition of dot product,
positivity would no longer hold. This is because for a complex number
$z$, it is not in general true that $z^2\geq 0$. In fact, $z^2$ may
not be a real number, and even in cases where $z^2$ is real, it may
not be positive. For example, if $z=i$, then $z^2=-1$.

Fortunately, all is not lost: the complex numbers actually do have a
useful positivity property. Namely, if $z=a+bi$ is a complex number
and $\overline{z}=a-bi$ is its complex conjugate, then
\begin{equation*}
  \overline{z}z = (a-bi)(a+bi) = a^2 + b^2 \geq 0.
\end{equation*}
So instead of squaring a complex number, we should multiply it by its
conjugate. With this in mind, we arrive at the following definition of
dot product on $\C^n$:

\begin{definition}{The dot product on $\C^n$}{complex-dot-product}
  Let
  \begin{equation*}
    \vect{v}=\begin{mymatrix}{c} v_1 \\ \vdots \\ v_n \end{mymatrix}
    \quad\mbox{and}\quad
    \vect{w}=\begin{mymatrix}{c} w_1 \\ \vdots \\ w_n \end{mymatrix}
  \end{equation*}
  be vectors in $\C^n$. Their \textbf{(complex) dot product}%
  \index{dot product!complex}%
  \index{vector!dot product!complex}%
  \index{complex dot product}
  is defined to be
  \begin{equation*}
    \vect{v}\dotprod \vect{w} = \overline{v_1}w_1 + \ldots + \overline{v_n}w_n.
  \end{equation*}
\end{definition}

The complex dot product satisfies properties that are similar to, but
not exactly the same as, the properties satisfied by the real dot
product.

\begin{proposition}{Properties of the complex dot product}{properties-complex-dot-product}
  \index{dot product!complex!properties}%
  \index{properties of dot product!complex}%
  \index{vector!dot product!complex!properties}%
  \index{vector!properties of dot product!complex}%
  \index{complex dot product!properties}%
  The dot product satisfies the following properties, where
  $\vect{u},\vect{v},\vect{w}\in\C^n$ and $k,\ell\in\C$.
  \begin{itemize}
  \item Skew symmetry: $\vect{u}\dotprod\vect{v}=\overline{\vect{v}\dotprod\vect{u}}$.
  \item Linearity on the right: $\vect{u}\dotprod(k\vect{v}+\ell\vect{w})
    =k(\vect{u}\dotprod \vect{v})+\ell(\vect{u}\dotprod\vect{w})$.
  \item Skew linearity on the left: $(k\vect{u}+\ell\vect{v})\dotprod\vect{w}=\overline{k}(\vect{u}\dotprod\vect{w})+\overline{\ell}(\vect{v}\dotprod\vect{w})$.
  \item The positive definite property: $\vect{u}\dotprod\vect{u}\geq 0$, and $\vect{u}\dotprod\vect{u}=0$ if and only if $\vect{u}=\vect{0}$.
  \end{itemize}
\end{proposition}

We note that the complex dot product can be equivalently expressed as
a matrix product, namely
\begin{equation*}
  \vect{v}\dotprod\vect{w}
  ~=~ \begin{mymatrix}{ccc} \overline{v_1} & \cdots & \overline{v_n} \end{mymatrix}
  \begin{mymatrix}{c} w_1 \\ \vdots \\ w_n \end{mymatrix}
  ~=~ \overline{\vect{v}}^T \vect{w}.
\end{equation*}
Here, $\overline{\vect{v}}$ denotes the complex conjugate of a vector
(i.e., taking the complex conjugate of each component of a vector),
and $(-)^T$ denotes the transpose as usual.  As a matter of fact, when
working with complex vectors and matrices, it turns out that we should
almost {\em always} take the complex conjugate at the same time as
taking the transpose. For this reason, we introduce a special name and
notation for the conjugate transpose of a vector or matrix.

\begin{definition}{Adjoint of a matrix}{adjoint}
  Let $A$ be a complex $n\times m$-matrix. The \textbf{adjoint} of
  $A$, denoted $A^*$, is the transpose of the complex conjugate of
  $A$. In symbols:
  \begin{equation*}
    A^{\adjoint} = \overline{A}{}^T.
  \end{equation*}
\end{definition}

With this definition, we can also write the dot product as
\begin{equation*}
  \vect{v}\dotprod\vect{w} = \vect{v}^{\adjoint}\vect{w}.
\end{equation*}

