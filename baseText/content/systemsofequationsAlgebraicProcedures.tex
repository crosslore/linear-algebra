\section{Algebraic View of Systems of Equations}

\begin{outcome}
  \begin{enumerate}
  \item[A.] Recognize the difference between a linear equation and a
    non-linear equation.
  \item[B.] Understand the definition of a system of linear equations.
  \item[C.] Determine whether a tuple of real numbers is a solution
    for a system of linear equations.
  \item[D.] Recognize whether a system of linear equations is homogeneous.
  \item[E.] Understand what it means for a system of linear equations
    to be consistent or inconsistent.
  \end{enumerate}
\end{outcome}

\void{
\begin{outcome}
\begin{enumerate}
\item[B.] Find the \ef\;and \rref\;of a matrix. 

\item[C.] Determine whether a system of linear equations has no solution, a
unique solution or an infinite number of solutions from its \ef.

\item[D.] Solve a system of equations using Gaussian Elimination and Gauss-Jordan Elimination.

\item[E.] Model a physical system with linear equations and then solve. 
\end{enumerate}

\end{outcome}
}

We have taken an in-depth look at graphical representations of systems of equations, as well as how to find possible
solutions graphically. Our attention now turns to working with systems algebraically. 

\begin{definition}{Linear Equation}{linearequation}
A \textbf{linear equation} \index{linear equation} is an equation of
the form
\begin{equation*}
  a_1x_1 + a_2x_2 + \cdots + a_nx_n = b.
\end{equation*}
Here, $a_1,\ldots,a_n$ are real numbers called the
\textbf{coefficients}\index{coefficient} of the equation, $b$ is a
real number called the \textbf{constant term}\index{constant term}
of the equation, and $x_1,\ldots,x_n$ are
\textbf{variables}\index{variable}.
\end{definition}

\begin{example}{Linear vs. Non-Linear Equation}{linear-vs-nonlinear}
  Which of the following equations are linear?
  \begin{equation*}
    \begin{array}{c}
      2x+3y=5\\
      2x^2+3y=5\\
      2\sqrt{x} + 3y = 5\\
      (\sqrt{2}) x + 3y = 5^2
    \end{array}
  \end{equation*}
\end{example}

\begin{solution}
  The equation $2x+3y=5$ is linear. The equation $2x^2+3y=5$ is not
  linear, because it contains the square of a variable instead of a
  variable.  The equation $2\sqrt{x} + 3y = 5$ is also not linear,
  because the square root is applied to one of the variables. On the
  other hand, the equation $(\sqrt{2}) x + 3y = 5^2$ is linear,
  because $\sqrt{2}$ and $5^2$ are real numbers, and can
  therefore be used as coefficients and constant terms.
\end{solution}
  
We also permit minor notational variants of linear equations. The
equation $2x-3y=5$ is linear although
Definition~\ref{def:linearequation} does not mention subtraction,
because it can be regarded as just another notation for
$2x+(-3)y = 5$. Similarly, the equation $2x=5+3y$ can be regarded as
linear, because it can be easily rewritten as $2x-3y=5$ by bringing
all the variables (and their coefficients) to the left-hand side.

A \textbf{solution}\index{linear equation!solution} to a linear
equation is an assignment of real numbers to the variables, making the
equation true. More precisely, if $r_1,\ldots,r_n$ are real numbers,
the assignment $x_1=r_1$, \ldots, $x_n=r_n$ is a solution to the
equation in Definition~\ref{def:linearequation} if the real number
$a_1r_1 + a_2r_2 + \ldots + a_nr_n$ is equal to the real number
$b$. To save space, we often write solutions in \textbf{tuple
  notation}\index{tuple}\footnote{The terminology ``tuple'' arose as
  follows. A collection of two items is called a ``pair'', a
  collection of three items is called a ``triple'', followed by
  ``quadruple'', ``quintuple'', ``sextuple'', and so on. You have to
  know Latin to know what the next ones are called. To avoid these
  Latin terms, mathematicians started saying $4$-tuple, $5$-tuple,
  $6$-tuple and so on, and more generally, $n$-tuple for an ordered
  collection of $n$ items. When $n$ doesn't matter or is clear from
  the context, we often just say ``tuple''.}  as
$(x_1,\ldots,x_n) = (r_1,\ldots,r_n)$. When there is no doubt about
the order of the variables, we also often simply write the solution as
$(r_1,\ldots,r_n)$.

\begin{example}{Solutions of a Linear Equation}{linear-equation-solutions}
  Consider the linear equation $2x+3y-4z=5$. Which of the following
  are solutions? (a) $(x,y,z)=(1,1,0)$, (b) $(x,y,z)=(0,3,1)$, (c)
  $(x,y,z)=(1,1,1)$.
\end{example}

\begin{solution}
  The assignment $(x,y,z) = (1,1,0)$ is a solution because
  $2(1)+3(1)-4(0) = 5$. The assignment $(x,y,z) = (0,3,1)$ is also a
  solution, because $2(0) + 3(3) - 4(1) = 5$. On the other hand,
  $(x,y,z) = (1,1,1)$ is not a solution, because
  $2(1) + 3(1)-4(1) = 1 \neq 5$.
\end{solution}

A system of linear equations is just several linear equations taken together.

\begin{definition}{System of Linear Equations}{systemoflinearequations}
A \textbf{system of linear equations}\index{system of linear equations} is a list of equations
\begin{equation*}
\begin{array}{c@{~}c@{~}c}
a_{11}x_{1}+a_{12}x_{2}+\cdots +a_{1n}x_{n}&=&b_{1} \\
a_{21}x_{1}+a_{22}x_{2}+\cdots +a_{2n}x_{n}&=&b_{2} \\
\vdots \\
a_{m1}x_{1}+a_{m2}x_{2}+\cdots +a_{mn}x_{n}&=&b_{m},
\end{array}
\end{equation*}
where $a_{ij}$ and $b_{j}$ are real numbers. The above is a system
of $m$ equations in the $n$ variables, $x_{1},x_{2}\cdots ,x_{n}$.
As before, the numbers $a_{ij}$ are called the
\textbf{coefficients}\index{coefficient} and the numbers $b_{j}$ are
called the \textbf{constant terms}\index{constant term} of the system
of equations.
\end{definition}

The relative size of $m$ and $n$ is not important here. We may have
more variables than equations, more equations than variables, or an
equal number of equations and variables. 

A \textbf{solution}\index{system of linear equations!solution} to a
system of linear equations is an assignment of real numbers to the
variables that is a solution to {\em all} of the equations in the
system.

\begin{example}{Solutions of a System of Linear Equations}{linear-system-solutions}
  Consider the system of linear equations
  \begin{equation*}
    \begin{array}{c@{~}c@{~}c@{~}c@{~}c@{~}c@{~}c}
      2x&+&3y&-&4z &=& 5\\
      -2x&+&y&+&2z &=& -1.
    \end{array}
  \end{equation*}  
  Which of the following are solutions of the system? (a)
  $(x,y,z)=(1,1,0)$, (b) $(x,y,z)=(6,3,4)$, (c) $(x,y,z)=(0,3,1)$.
\end{example}

\begin{solution}
  The assignment $(x,y,z)=(1,1,0)$ is a solution of this system of
  equations, because it is a solution to the first equation and the
  second equation. Also, $(x,y,z)=(6,3,4)$ is another solution of this
  system of equations (check this!).  On the other hand,
  $(x,y,z)=(0,3,1)$ is not a solution of the system, because although
  it is a solution to the first equation, it is not a solution to the
  second equation.
\end{solution}

Recall from Section~\ref{sec:systems-geometric} that a system of
equations either has a unique solution, infinitely many solutions, or
no solution. It is very important to us whether a system of equations
has solutions or not. For this reason, we introduce the following terminology:

\begin{definition}{Consistent and Inconsistent Systems}{consistentandinconsistent}
A system of linear equations is called
\index{consistent system}\textbf{consistent} if there exists at least one solution. It is
called
\index{inconsistent system}\textbf{inconsistent }if there is no solution.
\end{definition}

If we think of each equation as a condition that must be satisfied by
the variables, consistent means that there is some choice of values
for the variables which can satisfy \textbf{all} of the
conditions. Inconsistent means that there is no such choice of values
for the variables. In the following sections, you will learn a method
for determining whether a system of equations is consistent or not,
and in case it is consistent, to find all of its solutions. 

%%% MOVE THIS TO 1.2.4, but first rewrite 1.2.3

We conclude this section by introducing a special name for a system of
equations where all of the constant terms are $0$. 

\begin{definition}{Homogeneous System of Equations}{homogeneoussystem}
  A system of equations is called
  \textbf{homogeneous}\index{system of linear equations!homogeneous}
  if each of the constant terms is equal to $0$. A homogeneous system
  therefore has the form
\begin{equation*}
\begin{array}{c}
a_{11}x_{1}+a_{12}x_{2}+\cdots +a_{1n}x_{n}= 0 \\
a_{21}x_{1}+a_{22}x_{2}+\cdots +a_{2n}x_{n}= 0  \\
\vdots \\
a_{m1}x_{1}+a_{m2}x_{2}+\cdots +a_{mn}x_{n}= 0, 
\end{array}
\end{equation*}
where $a_{ij}$ are coefficients and $x_{i}$ are variables.
\end{definition}

