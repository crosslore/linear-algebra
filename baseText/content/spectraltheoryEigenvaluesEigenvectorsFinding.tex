\section{Finding eigenvalues}

In the previous section, we saw how to find the eigenvectors
corresponding to a given eigenvalue $\eigenVar$, if $\eigenVar$ is already
known. But we have not yet seen how to find the eigenvalues of a
matrix. However, the calculations in
Examples~\ref{exa:find-eigenvectors-given-eigenvalue} and
{\ref{exa:basis-eigenspace}} suggest a way forward. We can see that
the following are equivalent:
\begin{enumerate}
\item $\eigenVar$ is an eigenvalue of $A$.
\item There exists a non-zero vector $\vect{v}$ such that
  $A\vect{v}=\eigenVar\vect{v}$.
\item The homogeneous system of equations
  $(A-\eigenVar I)\vect{v}=\vect{0}$ has a non-trivial solution.
\end{enumerate}
Indeed, the equivalence between 1 and 2 is just the definition of an
eigenvalue, and the equivalence between 2 and 3 is just algebra.
By Corollary~\ref{cor:determinant-homogeneous}, we know that the
system $(A-\eigenVar I)\vect{v}=\vect{0}$ has a non-trivial solution if
and only if $\det(A-\eigenVar I)=0$. Therefore, we have proved the
following theorem:

\begin{theorem}{Eigenvalues}{eigenvalues}
  Let $A$ be a square matrix, and let $\eigenVar$ be a scalar. Then
  $\eigenVar$ is an eigenvalue of $A$ if and only if
  \begin{equation*}
    \det(A-\eigenVar I)=0.
  \end{equation*}
\end{theorem}

\begin{example}{Finding the eigenvalues}{finding-eigenvalues}
  Find the eigenvalues of the matrix
  \begin{equation*}
    A = \begin{mymatrix}{rr}
      -5 & 2 \\
      -7 & 4 \\
    \end{mymatrix}.
  \end{equation*}
\end{example}

\begin{solution}
  By Theorem~\ref{thm:eigenvalues}, a scalar $\eigenVar$ is an
  eigenvalue of $A$ if and only if $\det(A-\eigenVar I)=0$. We calculate
  the determinant:
  \begin{eqnarray*}
    \det(A-\eigenVar I)
    &=&
    \begin{absmatrix}{cc}
      -5-\eigenVar & 2 \\
      -7 & 4-\eigenVar \\
    \end{absmatrix} \\
    &=& (-5-\eigenVar)(4-\eigenVar) + 14 \\
    &=& \eigenVar^2 + \eigenVar - 6.
  \end{eqnarray*}
  Therefore, $\eigenVar$ is an eigenvalue if and only if
  $\eigenVar^2 + \eigenVar - 6 = 0$. We can find the roots of this
  equation using the quadratic formula, or equivalently, by factoring
  the left-hand side:
  \begin{equation*}
    \eigenVar^2 + \eigenVar - 6 = 0
    \iff
    (\eigenVar+3)(\eigenVar-2) = 0.
  \end{equation*}
  Therefore, the eigenvalues are $\eigenVar=-3$ and $\eigenVar=2$.
\end{solution}

\begin{example}{Finding the eigenvalues}{finding-eigenvalues2}
  Find the eigenvalues of the matrix
  \begin{equation*}
    A=\begin{mymatrix}{rrr}
      5 & -4 & 4 \\
      2 & -1 & 2 \\
      0 &  0 & 2 \\
    \end{mymatrix}.
  \end{equation*}
\end{example}

\begin{solution}
  Once again, we calculate $\det(A-\eigenVar I)$:
  \begin{eqnarray*}
    \det(A-\eigenVar I)
    &=&
        \begin{absmatrix}{ccc}
          5-\eigenVar & -4 & 4 \\
          2 & -1-\eigenVar & 2 \\
          0 &  0 & 2-\eigenVar \\
        \end{absmatrix} \\
    &=&
        (5-\eigenVar)(-1-\eigenVar)(2-\eigenVar) - 2(-4)(2-\eigenVar) \\
    &=& -\eigenVar^3 + 6\eigenVar^2 - 11\eigenVar + 6 \\
    &=& (3-\eigenVar)(1-\eigenVar)(2-\eigenVar).
  \end{eqnarray*}
  The eigenvalues are the roots of this polynomial, i.e., the
  solutions of the equation
  $(\eigenVar-3)(\eigenVar-1)(2-\eigenVar)=0$. Therefore, the eigenvalues of
  $A$ are $\eigenVar=1$, $\eigenVar=2$, and $\eigenVar=3$.
\end{solution}

\begin{example}{No real eigenvalue}{no-real-eigenvalue}
  Find the eigenvalues of the matrix
  \begin{equation*}
    A=\begin{mymatrix}{rr}
      0 & -1 \\
      1 &  0 \\
    \end{mymatrix}.
  \end{equation*}
\end{example}

\begin{solution}
  We have
  \begin{equation*}
    \det(A-\eigenVar I)
    ~=~
        \begin{absmatrix}{cc}
          -\eigenVar & -1 \\
          1 & -\eigenVar \\
        \end{absmatrix} \\
    ~=~
        \eigenVar^2 + 1.
  \end{equation*}
  Since $\eigenVar^2+1 = 0$ does not have any solutions in the real
  numbers, the matrix $A$ has no real eigenvalues. (However, if we
  were working over the field of complex numbers rather than real
  numbers, this matrix would have eigenvalues $\eigenVar=\pm i$).
\end{solution}

As the examples show, the quantity $\det(A-\eigenVar I)$ is always a
polynomial in the variable $\eigenVar$. A \textbf{polynomial}%
\index{polynomial} is an expression of the form
\begin{equation*}
  p(\eigenVar) = a_n\eigenVar^n + a_{n-1}\eigenVar^{n-1} + \ldots + a_1\eigenVar + a_0,
\end{equation*}
where $a_0,\ldots,a_n$ are constants called the \textbf{coefficients}
of the polynomial. The polynomial $\det(A-\eigenVar I)$ has a special
name:

\begin{definition}{Characteristic polynomial}{characteristic-polynomial}
  Let $A$ be a square matrix. The expression
  \begin{equation*}
    p(\eigenVar) = \det(A-\eigenVar I)
  \end{equation*}
  is called the \textbf{characteristic polynomial}%
  \index{characteristic polynomial}%
  \index{polynomial!characteristic}%
  \index{matrix!characteristic polynomial} of $A$.
\end{definition}

\begin{example}{Characteristic polynomial}{characteristic-polynomial}
  Find the characteristic polynomial of the matrix
  \begin{equation}
    A = \begin{mymatrix}{rrr}
      2  & 0 & 3 \\
      2  & 1 & 2 \\
      -6 & 0 & -7 \\
    \end{mymatrix}.
  \end{equation}
\end{example}

\begin{solution}
  The characteristic polynomial is
  \begin{eqnarray*}
    \det(A-\eigenVar I) \\
    &=&
        \begin{absmatrix}{ccc}
          2-\eigenVar & 0 & 3 \\
          2  & 1-\eigenVar & 2 \\
          -6 & 0 & -7-\eigenVar \\
        \end{absmatrix} \\
    &=&
        (2-\eigenVar)\begin{absmatrix}{ccc}
          1-\eigenVar & 2 \\
          0 & -7-\eigenVar \\
        \end{absmatrix}
    + 3 \begin{absmatrix}{ccc}
          2  & 1-\eigenVar \\
          -6 & 0 \\
        \end{absmatrix} \\
    &=& (2-\eigenVar)(1-\eigenVar)(-7-\eigenVar) + 18(1-\eigenVar) \\
    &=& -\eigenVar^3 -4\eigenVar^2 +\eigenVar + 4.
  \end{eqnarray*}
\end{solution}

It is time to summarize the method for finding the eigenvalues and
eigenvectors of a matrix.

\begin{procedure}{Finding eigenvalues and eigenvectors}{find-eigenvalues-vectors}
  \index{eigenvalue!calculating}%
  \index{eigenvector!calculating}%
  \index{matrix!eigenvalue!calculating}%
  \index{matrix!eigenvector!calculating}%
  Let $A$ be an $n\times n$-matrix. To find the eigenvalues and
  eigenvectors of $A$:
  \begin{enumerate}
  \item Calculate the characteristic polynomial $\det(A-\eigenVar I)$.
  \item The eigenvalues are the roots of the characteristic polynomial.
  \item For each eigenvalue $\eigenVar$, find a basis for the
    eigenvectors by solving the homogeneous system
    \begin{equation*}
      (A-\eigenVar I)\vect{v} = \vect{0}.
    \end{equation*}
  \end{enumerate}
  To double-check your work, make sure that $A\vect{v}=\eigenVar\vect{v}$
  for each eigenvalue $\eigenVar$ and associated eigenvector $\vect{v}$.
\end{procedure}

\begin{example}{Finding eigenvalues and eigenvectors}{finding-eigenvalues-eigenvectors}
  Find the eigenvalues and eigenvectors of the matrix
  \begin{equation}
    A = \begin{mymatrix}{rrr}
      2  & 0 & 3 \\
      2  & 1 & 2 \\
      -6 & 0 & -7 \\
    \end{mymatrix}.
  \end{equation}
\end{example}

\begin{solution}
  We already found the characteristic polynomial in
  Example~\ref{exa:characteristic-polynomial}. It is
  \begin{equation*}
    p(\eigenVar) = \det(A-\eigenVar I) = -\eigenVar^3 -4\eigenVar^2 +\eigenVar + 4.
  \end{equation*}
  Finding the roots of a cubic polynomial can be a bit tricky, but
  with some trial and error, we can find that $\eigenVar=1$ is a
  root. We can therefore factor out $(\eigenVar-1)$:
  \begin{equation*}
    p(\eigenVar) = (\eigenVar-1)(-\eigenVar^2-5\eigenVar-4).
  \end{equation*}
  Then we can use the quadratic formula to find the remaining two
  roots:
  \begin{equation*}
    \eigenVar = \frac{5\pm\sqrt{25-16}}{-2},
  \end{equation*}
  which yields the two roots $\eigenVar=-4$ and $\eigenVar=-1$. Therefore,
  we have
  \begin{equation*}
    p(\eigenVar) = -(\eigenVar-1)(\eigenVar+4)(\eigenVar+1),
  \end{equation*}
  and the eigenvalues of $A$ are $\eigenVar=1$, $\eigenVar=-1$, and
  $\eigenVar=-4$. We now find the eigenvectors for each eigenvalue.
  \begin{itemize}
  \item {\bf{\underline{For $\eigenVar=1$:}}} We must solve
    $(A-I)\vect{v}=\vect{0}$, i.e.,
    \begin{equation*}
      \begin{mymatrix}{rrr}
        1  & 0 & 3 \\
        2  & 0 & 2 \\
        -6 & 0 & -8 \\
      \end{mymatrix}\vect{v}=\vect{0}.
    \end{equation*}
    The basic solution is
    \begin{equation*}
      \vect{v}_1 = \begin{mymatrix}{r} 0 \\ 1 \\ 0 \end{mymatrix}.
    \end{equation*}
  \item {\bf{\underline{For $\eigenVar=-1$:}}} We must solve
    $(A+I)\vect{v}=\vect{0}$, i.e.,
    \begin{equation*}
      \begin{mymatrix}{rrr}
        3  & 0 & 3 \\
        2  & 2 & 2 \\
        -6 & 0 & -6 \\
      \end{mymatrix}\vect{v}=\vect{0}.
    \end{equation*}
    The basic solution is
    \begin{equation*}
      \vect{v}_2 = \begin{mymatrix}{r} 1 \\ 0 \\ -1 \end{mymatrix}.
    \end{equation*}
  \item {\bf{\underline{For $\eigenVar=-4$:}}} We must solve
    $(A+4I)\vect{v}=\vect{0}$, i.e.,
    \begin{equation*}
      \begin{mymatrix}{rrr}
        6  & 0 & 3 \\
        2  & 5 & 2 \\
        -6 & 0 & -3 \\
      \end{mymatrix}\vect{v}=\vect{0}.
    \end{equation*}
    The basic solution is
    \begin{equation*}
      \vect{v}_3 = \begin{mymatrix}{r} 5 \\ 2 \\ -10 \end{mymatrix}.
    \end{equation*}
  \end{itemize}
\end{solution}

\begin{example}{A zero eigenvalue}{zero-eigenvalue}
  Let
  \begin{equation*}
    A=\begin{mymatrix}{rrr}
      2 & 2 & -2 \\
      1 & 3 & -1 \\
      -1 & 1 & 1 \\
    \end{mymatrix}.
  \end{equation*}
  Find the eigenvalues and eigenvectors of $A$.
\end{example}

\begin{solution}
  To find the eigenvalues of $A$, we first compute the characteristic
  polynomial.
  \begin{equation*}
    \det(A-\eigenVar I) =
    \begin{absmatrix}{ccc}
      2-\eigenVar & 2 & -2 \\
      1 & 3-\eigenVar & -1 \\
      -1 & 1 & 1-\eigenVar \\
    \end{absmatrix}
    = -\eigenVar^{3}+6 \eigenVar^{2}-8\eigenVar.
  \end{equation*}
  You can verify that the roots of this polynomial are
  $\eigenVar_1 = 0$, $\eigenVar_2 = 2$, $\eigenVar_3 = 4$.  Notice that
  while eigenvectors can never equal $0$, it is possible to have an
  eigenvalue equal to $0$.  Now we will find the basic
  eigenvectors.
  \begin{itemize}
  \item {\bf{\underline{For $\eigenVar_1 =0$:}}} We must solve the
    equation $(A-0I)\vect{v} = \vect{0}$. This equation becomes
    $A\vect{v}=\vect{0}$. We write the augmented matrix for this
    system and reduce to echelon form:
    \begin{equation*}
      \begin{mymatrix}{rrr|r}
        2 & 2 & -2 & 0 \\
        1 & 3 & -1 & 0 \\
        -1 & 1 & 1 & 0 \\
      \end{mymatrix}
      \roweq\ldots\roweq
      \begin{mymatrix}{rrr|r}
        1 & 0 & -1 & 0 \\
        0 & 1 &  0 & 0 \\
        0 & 0 &  0 & 0 \\
      \end{mymatrix}.
    \end{equation*}
    The basic solution is
    \begin{equation*}
      \vect{v}_1
      =
      \begin{mymatrix}{r} 1 \\ 0 \\ 1 \end{mymatrix}.
    \end{equation*}
  \item {\bf{\underline{For $\eigenVar_2=2$:}}} We solve the
    equation $(A-2I)\vect{v} = \vect{0}$:
    \begin{equation*}
      \begin{mymatrix}{rrr|r}
        0  & 2 & -2 & 0 \\
        1  & 1 & -1 & 0 \\
        -1 & 1 & -1 & 0 \\
      \end{mymatrix}
      \roweq\ldots\roweq
      \begin{mymatrix}{rrr|r}
        1  & 0 &  0 & 0 \\
        0  & 1 & -1 & 0 \\
        0  & 0 &  0 & 0 \\
      \end{mymatrix}.
    \end{equation*}
    The basic solution is
    \begin{equation*}
      \vect{v}_2
      =
      \begin{mymatrix}{r} 0 \\ 1 \\ 1 \end{mymatrix}.
    \end{equation*}
  \item {\bf{\underline{For $\eigenVar_3=4$:}}} We solve the
    equation $(A-4I)\vect{v} = \vect{0}$:
    \begin{equation*}
      \begin{mymatrix}{rrr|r}
        -2 & 2  & -2 & 0 \\
        1  & -1 & -1 & 0 \\
        -1 &  1 & -3 & 0 \\
      \end{mymatrix}
      \roweq\ldots\roweq
      \begin{mymatrix}{rrr|r}
        1 & -1 & 0 & 0 \\
        0 &  0 & 1 & 0 \\
        0 &  0 & 0 & 0 \\
      \end{mymatrix}.
    \end{equation*}
    The basic solution is
    \begin{equation*}
      \vect{v}_3
      =
      \begin{mymatrix}{r} 1 \\ 1 \\ 0 \end{mymatrix}.
    \end{equation*}
  \end{itemize}
  Thus we have found the eigenvectors $\vect{v}_1$ for $\eigenVar_1$,
  $\vect{v}_2$ for $\eigenVar_2$, and $\vect{v}_3$ for $\eigenVar_3$.
  We can double-check our answers by checking the equation
  $A\vect{v}=\eigenVar\vect{v}$ in each case:
  \begin{eqnarray*}
    A\vect{v}_1
    &=&
    \begin{mymatrix}{rrr}
      2 & 2 & -2 \\
      1 & 3 & -1 \\
      -1 & 1 & 1 \\
    \end{mymatrix}
    \begin{mymatrix}{r} 1 \\ 0 \\ 1 \end{mymatrix}
    =
    \begin{mymatrix}{r} 0 \\ 0 \\ 0 \end{mymatrix}
    = 0\vect{v}_1,
    \\
    A\vect{v}_2
    &=&
    \begin{mymatrix}{rrr}
      2 & 2 & -2 \\
      1 & 3 & -1 \\
      -1 & 1 & 1 \\
    \end{mymatrix}
    \begin{mymatrix}{r} 0 \\ 1 \\ 1 \end{mymatrix}
    =
    \begin{mymatrix}{r} 0 \\ 2 \\ 2 \end{mymatrix}
    = 2\vect{v}_2,
    \\
    A\vect{v}_3
    &=&
    \begin{mymatrix}{rrr}
      2 & 2 & -2 \\
      1 & 3 & -1 \\
      -1 & 1 & 1 \\
    \end{mymatrix}
    \begin{mymatrix}{r} 1 \\ 1 \\ 0 \end{mymatrix}
    =
    \begin{mymatrix}{r} 4 \\ 4 \\ 0 \end{mymatrix}
    = 4\vect{v}_3.
  \end{eqnarray*}
  Therefore, our eigenvectors and eigenvalues are correct.
\end{solution}

