\subsection{Finding eigenvectors and eigenvalues}

Now that eigenvalues and eigenvectors have been defined, we will study how to find
 them for a matrix $A$. 

First, consider the following definition.

\begin{definition}{Multiplicity of an eigenvalue}{multiplicity}
Let $A$ be an $n \times n$ matrix with characteristic polynomial given by 
$\det \left( \eigenVar I -  A\right)$. Then, the multiplicity\index{multiplicity} of an eigenvalue $\lambda$ of $A$
is the number of times $\lambda$ occurs as a root of that characteristic polynomial.
\end{definition}

For example, suppose the characteristic polynomial of $A$ is given by
$\left( \eigenVar - 2 \right)^2$. Solving for the roots of this
polynomial, we set $\left( \eigenVar - 2 \right)^2 = 0$ and solve for
$\eigenVar$.  We find that $\lambda = 2$ is a root that occurs
twice. Hence, in this case, $\lambda = 2$ is an eigenvalue of $A$ of
multiplicity equal to $2$.

We will now look at how to find the eigenvalues and eigenvectors for a
matrix $A$ in detail.  The steps used are summarized in the following
procedure.

\begin{procedure}{Finding eigenvalues and eigenvectors}{findeigenvaluesvectors}
\index{eigenvalues!calculating}\index{eigenvectors!calculating}%
Let $A$ be an $n \times n$ matrix. 
\begin{enumerate}
\item First, find the eigenvalues $\lambda$ of $A$ by solving the equation $\det \left( \eigenVar I -A \right) = 0$. 

\item For each $\lambda$, find the basic eigenvectors $X \neq 0$ by finding the basic solutions to  $\left( \lambda I - A \right) X = 0$.
\end{enumerate}

To verify your work, make sure that $AX=\lambda X$
for each $\lambda$ and associated eigenvector $X$.
\end{procedure}

We will explore these steps further in the following example.

\begin{example}{Find the eigenvalues and eigenvectors}{findeigenvectors2x2}
Let $A = \leftB \begin{array}{rr}
-5 & 2 \\
-7 & 4 
\end{array} \rightB$. Find its eigenvalues and eigenvectors. 
\end{example}

\begin{solution}
We will use Procedure \ref{proc:findeigenvaluesvectors}. First we find the eigenvalues of $A$ by solving the equation
\[
\det \left( \eigenVar I - A \right) =0
\]

This gives
\begin{eqnarray*}
\det \left( \eigenVar \leftB \begin{array}{rr}
1 & 0 \\
0 & 1 
\end{array} \rightB 
- 
\leftB \begin{array}{rr}
-5 & 2 \\
-7 & 4
\end{array} \rightB \right) &=& 0 \\
\\
\det \leftB \begin{array}{cc}
\eigenVar+5 & -2 \\
7 & \eigenVar-4 
\end{array} \rightB &=& 0 
\end{eqnarray*}

Computing the determinant as usual, the result is
\[
\eigenVar^2 + \eigenVar - 6 = 0
\]

Solving this equation, we find that $\lambda_1 = 2$ and $\lambda_2 = -3$. 

Now we need to find the basic eigenvectors for each $\lambda$. First we will find the eigenvectors for $\lambda_1 = 2$. We wish to find all vectors $X \neq 0$ such that $AX = 2X$. These are the solutions to $(2I - A)X = 0$. 
\begin{eqnarray*}
\left( 
2 \leftB \begin{array}{rr}
1 & 0 \\
0 & 1 
\end{array}\rightB - 
\leftB \begin{array}{rr}
-5 & 2 \\
-7 & 4
\end{array}\rightB
 \right) \leftB \begin{array}{c}
x \\
y 
\end{array}\rightB &=& \leftB \begin{array}{r}
0 \\
0
\end{array} \rightB \\
\\
\leftB \begin{array}{rr}
7 & -2 \\
7 & -2
\end{array}\rightB \leftB \begin{array}{c}
x \\
y 
\end{array}\rightB &=& \leftB \begin{array}{r}
0 \\
0
\end{array} \rightB 
\end{eqnarray*}

The augmented matrix for this system and corresponding {\rref} are given by 
\[
\leftB \begin{array}{rr|r}
7 & -2 & 0 \\
7 & -2 & 0
\end{array}\rightB 
\rightarrow \cdots \rightarrow 
\leftB \begin{array}{rr|r}
1 & -\vspace{0.05in}\frac{2}{7} & 0 \\
0 & 0 & 0 
\end{array} \rightB 
\]

The solution is any vector of the form
\[
\leftB \begin{array}{c}
\vspace{0.05in}\frac{2}{7}s \\
s
\end{array} \rightB
=
s
\leftB \begin{array}{r}
\vspace{0.05in}\frac{2}{7} \\
1
\end{array} \rightB
\]

Multiplying this vector by $7$ we obtain a simpler description for the solution to this system, given by
\[
t \leftB \begin{array}{r}
2 \\
7
\end{array} \rightB
\]

This gives the basic eigenvector for $\lambda_1 = 2$ as 
\[
\leftB \begin{array}{r}
2\\
7
\end{array} \rightB
\]

To check, we verify that $AX = 2X$ for this basic eigenvector. 

\[
\leftB \begin{array}{rr}
-5 & 2 \\
-7 & 4
\end{array}\rightB 
\leftB \begin{array}{r}
2 \\
7
\end{array} \rightB
=
\leftB \begin{array}{r}
4 \\
14
\end{array}\rightB
=
2
\leftB \begin{array}{r}
2\\
7
\end{array}
\rightB
\]

This is what we wanted, so we know this basic eigenvector is correct. 

Next we will repeat this process to find the basic eigenvector for $\lambda_2 = -3$. We wish to find all vectors $X \neq 0$ such that $AX = -3X$. These are the solutions to $((-3)I-A)X = 0$. 
\begin{eqnarray*}
\left( 
(-3) \leftB \begin{array}{rr}
1 & 0 \\
0 & 1 
\end{array}\rightB - 
\leftB \begin{array}{rr}
-5 & 2 \\
-7 & 4
\end{array}\rightB  \right) \leftB \begin{array}{c}
x \\
y 
\end{array}\rightB &=& \leftB \begin{array}{r}
0 \\
0
\end{array} \rightB \\
\leftB \begin{array}{rr}
2 & -2 \\
7 & -7
\end{array}\rightB \leftB \begin{array}{c}
x \\
y 
\end{array}\rightB &=& \leftB \begin{array}{r}
0 \\
0
\end{array} \rightB 
\end{eqnarray*}

The augmented matrix for this system and corresponding {\rref} are given by 
\[
\leftB \begin{array}{rr|r}
2 & -2 & 0 \\
7 & -7 & 0
\end{array}\rightB 
\rightarrow \cdots \rightarrow 
\leftB \begin{array}{rr|r}
1 & -1 & 0 \\
0 & 0 & 0 
\end{array} \rightB 
\]

The solution is any vector of the form
\[
\leftB \begin{array}{c}
s \\
s
\end{array} \rightB
=
s
\leftB \begin{array}{r}
1 \\
1
\end{array} \rightB
\]

This gives the basic eigenvector for $\lambda_2 = -3$ as 
\[
\leftB \begin{array}{r}
1\\
1
\end{array} \rightB
\]

To check, we verify that $AX = -3X$ for this basic eigenvector. 

\[
\leftB \begin{array}{rr}
-5 & 2 \\
-7 & 4
\end{array}\rightB 
\leftB \begin{array}{r}
1 \\
1
\end{array} \rightB
=
\leftB \begin{array}{r}
-3 \\
-3
\end{array}\rightB
=
-3
\leftB \begin{array}{r}
1\\
1
\end{array}
\rightB
\]

This is what we wanted, so we know this basic eigenvector is correct. 
\end{solution}

The following is an example using Procedure \ref{proc:findeigenvaluesvectors} for a $3 \times 3$ matrix. 

\begin{example}{Find the eigenvalues and eigenvectors}{findeigenvectors}
Find the eigenvalues and eigenvectors for the matrix
\begin{equation*}
A=\leftB 
\begin{array}{rrr}
5 & -10 & -5 \\
2 & 14 & 2 \\
-4 & -8 & 6
\end{array}
\rightB 
\end{equation*}
\end{example}

\begin{solution}
We will use Procedure \ref{proc:findeigenvaluesvectors}.
First we need to find the eigenvalues of $A$. Recall that they are the
solutions of the equation
\begin{equation*}
\det \left( \eigenVar I - A \right) =0
\end{equation*}

In this case the equation is
\begin{equation*}
\det \left( 
\eigenVar \leftB
\begin{array}{rrr}
1 & 0 & 0 \\
0 & 1 & 0 \\
0 & 0 & 1
\end{array}
\rightB
-
\leftB
\begin{array}{rrr}
5 & -10 & -5 \\
2 & 14 & 2 \\
-4 & -8 & 6
\end{array}
\rightB  \right) =0
\end{equation*}

which becomes

\begin{equation*}
\det \leftB
\begin{array}{ccc}
\eigenVar - 5 & 10 & 5 \\
-2 & \eigenVar - 14  & -2 \\
4 & 8 & \eigenVar - 6
\end{array}
\rightB = 0
\end{equation*}

Using Laplace Expansion, compute this determinant and simplify.
The result is the following equation.
\begin{equation*}
\left( \eigenVar -5\right) \left( \eigenVar ^{2}-20\eigenVar +100\right) =0
\end{equation*}

Solving this equation, we find that the eigenvalues are $\lambda_1 = 5, \lambda_2=10$ and
$\lambda_3=10$. Notice that $10$ is a root of multiplicity two due to
\begin{equation*}
\eigenVar ^{2}-20\eigenVar +100=\left( \eigenVar -10\right) ^{2}
\end{equation*}
Therefore, $\lambda_2 = 10$ is an eigenvalue of multiplicity two. 

Now that we have found the eigenvalues for $A$, we can compute the eigenvectors.

First we will find the basic eigenvectors for $\lambda_1 =5.$ In other
words, we want to find all non-zero vectors $X$ so that $AX =
5X$. This requires that we solve the equation $\left( 5 I - A 
\right) X = 0$ for $X$ as follows.
\begin{equation*}
\left( 5\leftB
\begin{array}{rrr}
1 & 0 & 0 \\
0 & 1 & 0 \\
0 & 0 & 1
\end{array}
\rightB -  \leftB 
\begin{array}{rrr}
5 & -10 & -5 \\
2 & 14 & 2 \\
-4 & -8 & 6
\end{array}
\rightB  \right) \leftB
\begin{array}{r}
x \\
y \\
z
\end{array}
\rightB =\leftB
\begin{array}{r}
0 \\
0 \\
0
\end{array}
\rightB 
\end{equation*}

That is you need to find the solution to
\begin{equation*}
\allowbreak \leftB
\begin{array}{rrr}
0 & 10 & 5 \\
-2 & -9 & -2 \\
4 & 8 & -1
\end{array}
\rightB \leftB
\begin{array}{r}
x \\
y \\
z
\end{array}
\rightB =\leftB
\begin{array}{r}
0 \\
0 \\
0
\end{array}
\rightB
\end{equation*}

By now this is a familiar problem. You set up the augmented matrix and row
reduce to get the solution. Thus the matrix you must row reduce is
\begin{equation*}
\leftB
\begin{array}{rrr|r}
 0 & 10 & 5 & 0 \\
 -2 &  -9  &  -2 & 0 \\
4 & 8  &  -1 & 0
\end{array}
\rightB  
\end{equation*}
The {\rref} is
\begin{equation*}
\leftB
\begin{array}{rrr|r}
1 & 0 & -
\vspace{0.05in}\frac{5}{4} & 0 \\
0 & 1 & \vspace{0.05in}\frac{1}{2} & 0 \\
0 & 0 & 0 & 0
\end{array}
\rightB
\end{equation*}

and so the solution is any vector of the form
\begin{equation*}
\leftB
\begin{array}{c}
\vspace{0.05in}\frac{5}{4}s \\
-\vspace{0.05in}\frac{1}{2}s \\
s
\end{array}
\rightB =s\leftB
\begin{array}{r}
\vspace{0.05in}\frac{5}{4} \\
-\vspace{0.05in}\frac{1}{2} \\
1
\end{array}
\rightB
\end{equation*}
where $s\in \mathbb{R}$. If we multiply this vector by $4$, we obtain
a simpler description for the solution to this system, as given by
\begin{equation}
 t \leftB
\begin{array}{r}
 5 \\
-2 \\
 4
\end{array}
\rightB  \label{basiceigenvect}
\end{equation}
where $t\in \mathbb{R}$. Here, the basic eigenvector is given by 
\begin{equation*}
X_1 = 
\leftB
\begin{array}{r}
5 \\
-2 \\
4
\end{array}
\rightB
\end{equation*}

Notice that we cannot let $t=0$ here, because this would result in the zero vector and
eigenvectors are never equal to 0!
Other than this value, every other choice of $t$ in \ref{basiceigenvect} results in
an eigenvector.

It is a good idea to check your work! To do so, we will
take the original matrix and multiply by the basic eigenvector $X_1$. We check 
to see if we get $5X_1$.
\begin{equation*}
\leftB
\begin{array}{rrr}
5 & -10 & -5 \\
2 & 14 & 2 \\
-4 & -8 & 6
\end{array}
\rightB \leftB
\begin{array}{r}
 5 \\
-2 \\
 4
\end{array}
\rightB = \leftB
\begin{array}{r}
 25 \\
-10 \\
 20
\end{array}
\rightB =5\leftB
\begin{array}{r}
 5 \\
-2 \\
 4
\end{array}
\rightB
\end{equation*}
This is what we wanted, so we know that our calculations were correct.

Next we will find the basic eigenvectors for $\lambda_2, \lambda_3=10.$ These vectors are the basic 
solutions to the equation,
\begin{equation*}
\left( 10\leftB
\begin{array}{rrr}
1 & 0 & 0 \\
0 & 1 & 0 \\
0 & 0 & 1
\end{array}
\rightB - \leftB
\begin{array}{rrr}
5 & -10 & -5 \\
2 & 14 & 2 \\
-4 & -8 & 6
\end{array}
\rightB  \right) \leftB
\begin{array}{r}
x \\
y \\
z
\end{array}
\rightB =\leftB
\begin{array}{r}
0 \\
0 \\
0
\end{array}
\rightB
\end{equation*}
That is you must find the solutions to
\begin{equation*}
\leftB
\begin{array}{rrr}
5 & 10 & 5 \\
 -2 &  -4  & -2 \\
4 & 8  & 4
\end{array}
\rightB \leftB
\begin{array}{c}
x \\
y \\
z
\end{array}
\rightB =\leftB
\begin{array}{r}
0 \\
0 \\
0
\end{array}
\rightB
\end{equation*}

Consider the augmented matrix 
\begin{equation*}
\leftB
\begin{array}{rrr|r}
5 & 10 & 5 & 0 \\
 -2 &  -4  &  -2 & 0 \\
4 & 8  & 4 & 0
\end{array}
\rightB
\end{equation*}
The {\rref} for this matrix is
\begin{equation*}
\leftB
\begin{array}{rrr|r}
1 & 2 & 1 & 0 \\
0 & 0 & 0 & 0 \\
0 & 0 & 0 & 0
\end{array}
\rightB
\end{equation*}
and so the eigenvectors are of the form
\begin{equation*}
\leftB
\begin{array}{c}
-2s-t \\
s \\
t
\end{array}
\rightB =s\leftB
\begin{array}{r}
-2 \\
1 \\
0
\end{array}
\rightB +t\leftB
\begin{array}{r}
-1 \\
0 \\
1
\end{array}
\rightB 
\end{equation*}
Note that you can't pick $t$ and $s$ both equal to zero because this would result in
the zero vector and eigenvectors are never equal to zero. 

Here, there are two basic eigenvectors, given by 
\begin{equation*}
X_2
=
\leftB
\begin{array}{r}
-2 \\
1\\
0
\end{array}
\rightB ,
X_3
=
\leftB
\begin{array}{r}
-1 \\
0 \\
1
\end{array}
\rightB
\end{equation*}

Taking any (nonzero) linear combination of $X_2$ and $X_3$ will also result in an eigenvector for
the eigenvalue $\lambda =10.$ As in the case for $\lambda =5$, always check your work! 
For the first basic eigenvector,  we can check $AX_2 = 10 X_2$ as follows. 
\begin{equation*}
\leftB 
\begin{array}{rrr}
5 & -10 & -5 \\
2 & 14 & 2 \\
-4 & -8 & 6
\end{array}
\rightB \leftB
\begin{array}{r}
-1 \\
0 \\
1
\end{array}
\rightB = \leftB
\begin{array}{r}
-10 \\
0 \\
10
\end{array}
\rightB =10\leftB
\begin{array}{r}
-1 \\
0 \\
1
\end{array}
\rightB
\end{equation*}
This is what we wanted. Checking the second basic eigenvector, $X_3$, is left as an exercise. 
\end{solution}

It is important to remember that for any eigenvector $X$, $X \neq 0$. However, it is possible 
to have eigenvalues equal to zero. This is illustrated in the following example. 

\begin{example}{A zero eigenvalue}{zeroeigenvalue}
Let
\begin{equation*}
A=\leftB
\begin{array}{rrr}
2 & 2 & -2 \\
1 & 3 & -1 \\
-1 & 1 & 1
\end{array}
\rightB
\end{equation*}
Find the eigenvalues and eigenvectors of $A$.
\end{example}

\begin{solution}
First we find the eigenvalues of $A$. We will do so using Definition
\ref{def:eigenvaluesandeigenvectors}. 

In order to find the eigenvalues of $A$, we solve the following equation.
\begin{equation*}
\det \left(\eigenVar I -A \right) =
\det  \leftB 
\begin{array}{ccc}
\eigenVar -2 & -2 & 2 \\
 -1 & \eigenVar - 3  & 1 \\
1 & -1 &  \eigenVar  -1 
\end{array}
\rightB
 =0
\end{equation*}

This reduces to $ \eigenVar^{3}-6 \eigenVar^{2}+8\eigenVar =0$. You can verify that the
solutions are $ \lambda_1 = 0, \lambda_2 = 2, \lambda_3 = 4$.
Notice that while eigenvectors can never equal $0$, it is possible to have an eigenvalue equal to $0$. 

Now we will find the basic eigenvectors. For $\lambda_1 =0$, we need to solve the equation
$\left( 0 I - A \right) X = 0$. This equation becomes $-AX=0$, and so the augmented matrix for finding
the solutions is given by 
\begin{equation*}
\leftB 
\begin{array}{rrr|r}
-2 & -2 & 2 & 0 \\
-1 & -3 & 1 & 0 \\
1 & -1 & -1 & 0
\end{array}
\rightB
\end{equation*}
The {\rref} is
\begin{equation*}
\leftB
\begin{array}{rrr|r}
1 & 0 & -1 & 0 \\
0 & 1 &  0 & 0 \\
0 & 0 &  0 & 0
\end{array}
\rightB
\end{equation*}
Therefore, the eigenvectors are of the form $ t\leftB
\begin{array}{r}
1 \\
0 \\
1
\end{array}
\rightB$ where $t\neq 0$ and the basic eigenvector is given by
\begin{equation*}
X_1
=
\leftB
\begin{array}{r}
1 \\
0 \\
1
\end{array}
\rightB
\end{equation*}

We can verify that this eigenvector is correct by checking that the equation $AX_1 = 0 X_1$ holds.
The product $AX_1$ is given by
\begin{equation*}
AX_1=\leftB
\begin{array}{rrr}
2 & 2 & -2 \\
1 & 3 & -1 \\
-1 & 1 & 1
\end{array}
\rightB
\leftB
\begin{array}{r}
1 \\
0 \\
1
\end{array}
\rightB
=
\leftB
\begin{array}{r}
0 \\
0 \\
0
\end{array}
\rightB
\end{equation*}

This clearly equals $0X_1$, so the equation holds. Hence, $AX_1 = 0X_1$ and so $0$ is an eigenvalue of $A$.

Computing the other basic eigenvectors is left as an exercise. 
\end{solution}

In the following sections, we examine ways to simplify this process of finding eigenvalues and eigenvectors by using 
properties of special types of matrices.
