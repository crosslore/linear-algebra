\subsection{Markov Matrices}

There are applications of great importance which feature a special type of matrix.  Matrices whose columns consist of non-negative numbers that sum to one are
\index{Markov matrices} called \textbf{Markov matrices}. An important application of Markov matrices is
in population migration, as illustrated in the following definition. 

\begin{definition}{Migration Matrices}{migrationmatrix}
Let $m$ locations be denoted by the numbers $1,2,\cdots ,m.$ Suppose
it is the case that each year the proportion of residents in location
$j$ which move to location $i$ is $a_{ij}$. Also suppose no one
escapes or emigrates from without these $m$ locations. This last
assumption requires $\sum_{i}a_{ij}=1$, and means that the matrix $A$,
such that $A = \leftB a_{ij} \rightB$, is a Markov matrix. In this context,
$A$ is also called a \textbf{migration matrix}.
\index{migration matrix}
\end{definition}

Consider the following example which demonstrates this situation.

\begin{example}{Migration Matrix}{migrationmatrix}
Let $A$ be a Markov matrix given by 
\begin{equation*}
A = 
\leftB
\begin{array}{rr}
.4 & .2 \\
.6 & .8
\end{array}
\rightB
\end{equation*}
Verify that $A$ is a Markov matrix and describe the entries of $A$ in terms of population migration.
\end{example}

\begin{solution}
The columns of $A$ are comprised of non-negative numbers which sum to $1$. Hence, $A$ is a Markov matrix. 

Now, consider the entries $a_{ij}$ of $A$ in terms of population. The
entry $a_{11} = .4$ is the proportion of residents in location one
which stay in location one in a given time period.  Entry $a_{21} =
.6$ is the proportion of residents in location 1 which move to
location 2 in the same time period. Entry $a_{12} = .2$ is the
proportion of residents in location 2 which move to location
1. Finally, entry $a_{22} = .8$ is the proportion of residents in
location 2 which stay in location 2 in this time period.  

Considered as a Markov matrix, these numbers are usually identified
with probabilities. Hence, we can say that the probability that a
resident of location one will stay in location one in the time period
is $.4$.
\end{solution}

Observe that in Example \ref{exa:migrationmatrix} if there was initially say 15
thousand people in location 1 and 10 thousands in location 2, then
after one year there would be $.4 \times 15 + .2 \times 10 = 8$
thousands people in location 1 the following year, and similarly 
there would be $.6 \times 15 + .8 \times 10 = 17$
thousands people in location 2 the following year.

More generally let $X_n=\leftB x_{1n} \cdots x_{mn}\rightB ^{T}$ where $x_{in}$ is the
population of location $i$ at time period $n$. We call $X_n$ the \textbf{state vector at period $n$}\index{state vector}. In particular, we call $X_0$ the initial state vector. Letting $A$ be the migration matrix, we compute the population in each location $i$ one time period later by $AX_n$. In order to find the population of location $i$ after $k$
years, we compute the $i^{th}$ component of $A^{k}X.$ This discussion is summarized in the following theorem. 

\begin{theorem}{State Vector}{statevector}
Let $A$ be the migration matrix of a population and let $X_n$ be the vector whose entries give the population of each location at time period $n$. Then $X_n$ is the state vector at period $n$ and it follows that 
\[
X_{n+1} = A X_n
\]
\end{theorem}

The sum of the entries of $X_n$ will equal the sum of the entries of the initial
vector $X_{0}$. Since the columns of $A$ sum to $1$, this sum is preserved for every
multiplication by $A$ as demonstrated below. 
\begin{equation*}
\sum_{i}\sum_{j}a_{ij}x_{j}=\sum_{j}x_{j}\left( \sum_{i}a_{ij}\right)
=\sum_{j}x_{j}
\end{equation*}

Consider the following example.

\begin{example}{Using a Migration Matrix}{usingmigrationmatrix}
Consider the migration matrix 
\begin{equation*}
A = 
\leftB
\begin{array}{rrr}
.6 & 0 & .1 \\
.2 & .8 & 0 \\
.2 & .2 & .9
\end{array}
\rightB 
\end{equation*}
 for locations $1,2,$ and $3.$ Suppose initially there are $100$
residents in location $1$, $200$ in location $2$ and $400$ in location $3$. Find the
population in the three locations after $1,2,$ and $10$ units of time.
\end{example}

\begin{solution}
Using Theorem \ref{thm:statevector} we can find the population in each location using the equation $X_{n+1} = AX_n$. For the population after $1$ unit, we calculate $X_1 = AX_0$ as follows.
\begin{eqnarray*}
X_1 &=& AX_0 \\
\leftB \begin{array}{r}
x_{11} \\
x_{21} \\
x_{31}
\end{array}\rightB 
&=& 
\leftB
\begin{array}{rrr}
.6 & 0 & .1 \\
.2 & .8 & 0 \\
.2 & .2 & .9
\end{array}
\rightB 
\leftB
\begin{array}{r}
100 \\
200 \\
400
\end{array}
\rightB \\
&=& 
\leftB
\begin{array}{r}
100 \\
180 \\
420
\end{array}\rightB
\end{eqnarray*}
Therefore after one time period, location $1$ has $100$ residents, location $2$ has $180$, and location $3$ has $420$. Notice that the \textbf{total} population is unchanged, it simply migrates within the given locations.
We find the locations after two time periods in the same way. 
\begin{eqnarray*}
X_2 &=& AX_1 \\
\leftB \begin{array}{r}
x_{12} \\
x_{22} \\
x_{32}
\end{array}\rightB 
&=& 
\leftB
\begin{array}{rrr}
.6 & 0 & .1 \\
.2 & .8 & 0 \\
.2 & .2 & .9
\end{array}
\rightB 
\leftB
\begin{array}{r}
100 \\
180 \\
420
\end{array}
\rightB \\
&=& 
\leftB
\begin{array}{r}
102 \\
164 \\
434
\end{array}\rightB
\end{eqnarray*}

We could progress in this manner to find the populations after $10$ time periods. However from our above discussion, we can simply calculate $\left( A^{n}X_0\right) _{i}$, 
where $n$ denotes the number of time periods which have passed. Therefore, we compute the populations in each location after $10$ units of time as follows. 
\begin{eqnarray*}
X_{10} &=& A^{10}X_0 \\
\leftB \begin{array}{r}
x_{1 10} \\
x_{2 10} \\
x_{3 10} 
\end{array}
\rightB 
&=& 
\leftB
\begin{array}{rrr}
.6 & 0 & .1 \\
.2 & .8 & 0 \\
.2 & .2 & .9
\end{array}
\rightB ^{10}\leftB
\begin{array}{r}
100 \\
200 \\
400
\end{array}
\rightB \\
&=&  \leftB
\begin{array}{c}
115.\,\allowbreak 085\,829\,22 \\
120.\,\allowbreak 130\,672\,44 \\
464.\,\allowbreak 783\,498\,34
\end{array}
\rightB
\end{eqnarray*}
Since we are speaking about populations, we would need to round these numbers to provide a logical 
answer. Therefore, we can say that after $10$ units of time, there will be $115$ residents in location one, $120$ in location two,
and $465$ in location three.
\end{solution}

A second important application of Markov matrices is the concept of \index{random walk} random walks. Suppose a walker has $m$ locations to choose from, denoted $1, 2, \cdots, m$. Let $a_{ij}$ refer to the probability that the person will travel \textbf{to}  location $i$ \textbf{from} location $j$. Again, this requires that
\[
\sum_{i=1}^{k}a_{ij}=1
\]
In this context, the vector $X_n=\leftB x_{1n} \cdots x_{mn}\rightB ^{T}$ contains the probabilities $x_{in}$ the walker ends up in location $i, 1\leq i \leq m$ at time $n$. 

\begin{example}{Random Walks}{randomwalks}
Suppose three locations exist, referred to as locations $1, 2$ and $3$. The Markov matrix of probabilities $A = [a_{ij}]$ is given by
\[
\leftB
\begin{array}{rrr}
0.4 & 0.1 & 0.5 \\
0.4 & 0.6 & 0.1 \\
0.2 & 0.3 & 0.4
\end{array}
\rightB
\]
If the walker starts in location $1$, calculate the probability that he ends up in location $3$ at time $n = 2$. 
\end{example}

\begin{solution}
Since the walker begins in location $1$, we have 
\[
X_{0} = \leftB
\begin{array}{c}
1 \\
0 \\
0
\end{array}
\rightB
\]
The goal is to calculate $x_{32}$. To do this we calculate $X_{2}$, using $X_{n+1} = AX_{n}$.
\begin{eqnarray*}
X_{1} &=& A X_{0} \\
&=& \leftB
\begin{array}{rrr}
0.4 & 0.1 & 0.5 \\
0.4 & 0.6 & 0.1 \\
0.2 & 0.3 & 0.4
\end{array}
\rightB \leftB
\begin{array}{c}
1 \\
0 \\
0
\end{array}
\rightB \\
&=& 
\leftB 
\begin{array}{r}
0.4 \\
0.4 \\
0.2
\end{array}
\rightB \\
\end{eqnarray*}
\begin{eqnarray*}
X_{2} &=& A X_{1} \\
&=& \leftB
\begin{array}{rrr}
0.4 & 0.1 & 0.5 \\
0.4 & 0.6 & 0.1 \\
0.2 & 0.3 & 0.4
\end{array}
\rightB \leftB
\begin{array}{c}
0.4 \\
0.4 \\
0.2
\end{array}
\rightB \\
&=& 
\leftB 
\begin{array}{r}
0.3 \\
0.42 \\
0.28
\end{array}
\rightB \\
\end{eqnarray*}
This gives the probabilities that our walker ends up in locations 1, 2, and 3. For this example we are interested in location 3, with a probability on $0.28$.
\end{solution}

Returning to the context of migration, suppose we wish to know how many residents will be in a certain location
after a very long time. It turns out that if some power of the
migration matrix has all positive entries, then there is a vector $X_s$ such that $A^{n}X_{0}$ approaches $X_s$ as $n$ becomes very large. Hence as more time passes and $n$ increases, $A^{n}X_{0}$ will become closer to the vector $X_s$. 

Consider Theorem \ref{thm:statevector}. Let $n$ increase so that $X_n$ approaches $X_s$. As $X_n$ becomes closer to $X_s$, so too does $X_{n+1}$. For sufficiently large $n$, the statement $X_{n+1} = AX_n$ can be written as $X_s = AX_s$. 

This discussion motivates the following theorem.

\begin{theorem}{Steady State Vector}{steadystate}
Let $A$ be a migration matrix. Then there exists a \textbf{steady state vector} written $X_s$ such that 
\[
X_s = AX_s
\]
where $X_s$ has positive entries which have the same sum as the entries of $X_0$. 

As $n$ increases, the state vectors $X_n$ will approach $X_s$. 
\end{theorem}

Note that the condition in Theorem \ref{thm:steadystate} can be written as $(I - A)X_s=0$, representing a homogeneous system of equations. 

Consider the following example. Notice that it is the same example as the Example \ref{exa:usingmigrationmatrix} but here it will
involve a longer time frame.

\begin{example}{Populations over the Long Run}{longrunpopulation}
Consider the migration matrix 
\begin{equation*}
A
=
\leftB
\begin{array}{rrr}
.6 & 0 & .1 \\
.2 & .8 & 0 \\
.2 & .2 & .9
\end{array}
\rightB 
\end{equation*}
 for locations $1,2,$ and $3.$ Suppose initially there are 100
residents in location 1, 200 in location 2 and 400 in location 4. Find the
population in the three locations after a long time.
\end{example}

\begin{solution}
By Theorem \ref{thm:steadystate} the steady state vector $X_s$ can be found by solving the system $(I-A)X_s = 0$. 

Thus we need to find a solution to
\begin{equation*}
\left( \leftB
\begin{array}{rrr}
1 & 0 & 0 \\
0 & 1 & 0 \\
0 & 0 & 1
\end{array}
\rightB -\leftB
\begin{array}{rrr}
.6 & 0 & .1 \\
.2 & .8 & 0 \\
.2 & .2 & .9
\end{array}
\rightB \right) \leftB
\begin{array}{c}
x_{1s} \\
x_{2s}\\
x_{3s}
\end{array}
\rightB =\leftB
\begin{array}{c}
0 \\
0 \\
0
\end{array}
\rightB
\end{equation*}
The augmented matrix and the resulting {\rref} are given by 
\begin{equation*}
\leftB
\begin{array}{rrr|r}
0.4 & 0 & -0.1 & 0 \\
-0.2 & 0.2 & 0 & 0 \\
-0.2 & -0.2 & 0.1 & 0
\end{array}
\rightB
\rightarrow \cdots \rightarrow
\leftB
\begin{array}{rrr|r}
1 & 0 & -0.25 & 0 \\
0 & 1 & -0.25 & 0 \\
0 & 0 & 0 & 0
\end{array}
\rightB
\end{equation*}
Therefore, the eigenvectors are
\begin{equation*}
t\leftB
\begin{array}{c}
0.25 \\
0.25 \\
1
\end{array}
\rightB
\end{equation*}

The initial vector $X_0$ is given by 
\begin{equation*}
\leftB
\begin{array}{r}
100 \\
200 \\
400
\end{array}
\rightB
\end{equation*}

Now all that remains is to choose the value of $t$ such that
\begin{equation*}
0.25t+0.25t+t=100+200+400
\end{equation*}
Solving this equation for $t$ yields $t=\vspace{.05in} \frac{1400}{3}$. Therefore the population in the long run
is given by
\begin{equation*}
\vspace{.05in} \frac{1400}{3}\leftB
\begin{array}{c}
0.25 \\
0.25 \\
1
\end{array}
\rightB = \leftB
\begin{array}{c}
116. 666\,666\,666\, 666\,7 \\
116. 666\,666\,666\, 666\,7 \\
466. 666\,666\,666\, 666\,7
\end{array}
\rightB 
\end{equation*}

Again, because we are working with populations, these values need to be rounded. The steady state vector $X_s$ is given by
\[
\leftB
\begin{array}{c}
117 \\
117 \\
466
\end{array}
\rightB 
\]
\end{solution}

We can see that the numbers we calculated in Example \ref{exa:usingmigrationmatrix} for the populations
after the $10^{th}$ unit of time are not far from the long term values. 

Consider another example.

\begin{example}{Populations After a Long Time}{populationafterlongtime}
Suppose a migration matrix is given by 
\begin{equation*}
A = \leftB
\begin{array}{ccc}
\vspace{.05in} \frac{1}{5} & \vspace{.05in} \frac{1}{2} & \vspace{.05in}
\frac{1}{5} \\
\vspace{.05in} \frac{1}{4} & \vspace{.05in} \frac{1}{4} & \vspace{.05in}
\frac{1}{2} \\
\vspace{.05in} \frac{11}{20} & \vspace{.05in} \frac{1}{4} & \vspace{.05in}
\frac{3}{10}
\end{array}
\rightB 
\end{equation*}
 Find the comparison between the populations in the three
locations after a long time.
\end{example}

\begin{solution}
In order to compare the populations in the long term, we want to find the steady state vector $X_s$. 
Solve
\begin{equation*}
\left( \leftB
\begin{array}{rrr}
1 & 0 & 0 \\
0 & 1 & 0 \\
0 & 0 & 1
\end{array}
\rightB -\leftB
\begin{array}{ccc}
\vspace{.05in} \frac{1}{5} & \vspace{.05in} \frac{1}{2} & \vspace{.05in}
\frac{1}{5} \\
\vspace{.05in} \frac{1}{4} & \vspace{.05in} \frac{1}{4} & \vspace{.05in}
\frac{1}{2} \\
\vspace{.05in} \frac{11}{20} & \vspace{.05in} \frac{1}{4} & \vspace{.05in}
\frac{3}{10}
\end{array}
\rightB \right) \leftB
\begin{array}{c}
x_{1s} \\
x_{2s} \\
x_{3s}
\end{array}
\rightB =\leftB
\begin{array}{c}
0 \\
0 \\
0
\end{array}
\rightB 
\end{equation*}
The augmented matrix and the resulting {\rref} are given by
\begin{equation*}
\leftB
\begin{array}{rrr|r}
\vspace{.05in} \frac{4}{5} & -\vspace{.05in} \frac{1}{2} & -\vspace{.05in}
\frac{1}{5} &  0 \\
-\vspace{.05in} \frac{1}{4} & \vspace{.05in} \frac{3}{4} & -\vspace{.05in}
\frac{1}{2} &  0 \\
-\vspace{.05in} \frac{11}{20} & -\vspace{.05in} \frac{1}{4} & \vspace{.05in}
\frac{7}{10} &  0
\end{array}
\rightB
\rightarrow \cdots \rightarrow
\leftB
\begin{array}{rrr|r}
1 & 0 & -\vspace{.05in} \frac{16}{19} & 0 \\
0 & 1 & -\vspace{.05in} \frac{18}{19} & 0 \\
0 & 0 & 0 & 0
\end{array}
\rightB
\end{equation*}
and so an eigenvector is
\begin{equation*}
\leftB
\begin{array}{c}
16 \\
18 \\
19
\end{array}
\rightB 
\end{equation*}

Therefore, the proportion of population in location 2 to location 1 is given by $\vspace{.05in} \frac{18}{16}$.
The proportion of population 3 to location 2 is given by $\vspace{.05in} \frac{19}{18}$.
\end{solution}
