\section{Spanning set of vectors}

We begin this section with a definition.

\begin{definition}{Span of a set of vectors}{span}
The collection of all linear combinations of a set of vectors $\{\vect{u}_1,
\cdots ,\vect{u}_k\}$ in $\R^{n}$ is known as the span\index{span}\index{vectors!span} of these
vectors and is written as $\sspan \{\vect{u}_1, \cdots , \vect{u}_k\}$.
\end{definition}

Consider the following example.

\begin{example}{Span of vectors}{span-vectors}
Describe the span of the vectors $\vect{u}=\begin{mymatrix}{rrr}
1  & 1 & 0
\end{mymatrix}^T$ and
$\vect{v}=\begin{mymatrix}{rrr}
3  & 2 & 0
\end{mymatrix}^T \in \R^{3}$.
\end{example}

\begin{solution}
You can see that any linear combination of the vectors $\vect{u}$ and $\vect{v}$ yields a vector of the form 
$\begin{mymatrix}{rrr}
x  & y & 0
\end{mymatrix}^T$ in the $XY$-plane. 

Moreover every vector in the $XY$-plane is in fact such a linear
combination of the vectors $\vect{u}$ and $\vect{v}$. That's because
\[ \begin{mymatrix}{r}
x \\
y \\
 0
\end{mymatrix} 
=
(-2x+3y) \begin{mymatrix}{r}
1  \\
1 \\
0
\end{mymatrix}
+
(x-y)\begin{mymatrix}{r}
3 \\
2 \\
0
\end{mymatrix} 
\]

Thus  $\sspan\{\vect{u},\vect{v}\}$ is precisely the $XY$-plane.
\end{solution}

You can convince yourself that no single vector can span the
$XY$-plane. In fact, take a moment to consider what is meant by the span of a single vector.

However you can make the set larger if you wish. For example consider
the larger set of vectors $\{\vect{u}, \vect{v},
\vect{w}\}$ where $ \vect{w}=\begin{mymatrix}{rrr}
4 & 5 & 0
\end{mymatrix}^T$. 
Since
the first two vectors already span the entire $XY$-plane, the span is
once again precisely the $XY$-plane and nothing has been gained. Of
course if you add a new vector such as
$ \vect{w}=\begin{mymatrix}{rrr}
0 & 0 & 1
\end{mymatrix}^T$ then it does span a different space. What is the span of $\vect{u}, \vect{v}, \vect{w}$ in this case?   

The distinction between the sets $\{\vect{u}, \vect{v}\}$ and $\{
\vect{u}, \vect{v}, \vect{w}\}$ will be made using the concept of linear independence. 

Consider the vectors $\vect{u}, \vect{v}$, and $\vect{w}$ discussed above. In the next example, we will show how to formally demonstrate that $\vect{w}$ is in the span of $\vect{u}$ and $\vect{v}$. 

\begin{example}{Vector in a span}{vector-in-span}
Let $\vect{u}=\begin{mymatrix}{rrr}
1  & 1 & 0
\end{mymatrix}^T$ and
$\vect{v}=\begin{mymatrix}{rrr}
3  & 2 & 0
\end{mymatrix}^T \in \R^{3}$. Show that $\vect{w} = \begin{mymatrix}{rrr}
4 & 5 & 0 
\end{mymatrix}^{T}$ is in $\sspan \set{\vect{u}, \vect{v} }$.
\end{example}

\begin{solution}
For a vector to be in $\sspan \set{\vect{u}, \vect{v} }$, it must be a linear combination of these vectors. If $\vect{w} \in \sspan \set{\vect{u}, \vect{v} }$, we must be able to find scalars $a,b$ such that\[
\vect{w} = a \vect{u} +b \vect{v}
\]

We proceed as follows.
\[
\begin{mymatrix}{r}
4 \\
5 \\
0
\end{mymatrix}
=
a 
\begin{mymatrix}{r}
1 \\
1 \\
0
\end{mymatrix}
+
b
\begin{mymatrix}{r}
3 \\
2 \\
0
\end{mymatrix}
\]
This is equivalent to the following system of equations
\begin{eqnarray*}
a + 3b &=& 4 \\
a + 2b &=& 5
\end{eqnarray*}

We solving this system the usual way, constructing the augmented matrix and row reducing to find the {\rref}.
\[
\begin{mymatrix}{rr|r}
1 & 3 & 4 \\
1 & 2 & 5 
\end{mymatrix}
\rightarrow \cdots \rightarrow
\begin{mymatrix}{rr|r}
1 & 0 & 7 \\
0 & 1 & -1
\end{mymatrix}
\]
The solution is $a=7, b=-1$. This means that 
\[
\vect{w} = 7 \vect{u} - \vect{v}
\] 
Therefore we can say that $\vect{w}$ is in $\sspan \set{\vect{u}, \vect{v} }$. 
\end{solution}
