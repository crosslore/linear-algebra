\section{Application: Cryptography}

% local definitions for this section
\begingroup
\newcommand{\qq}[1]{\mbox{#1}}
\newcommand{\q}[1]{\qq{``#1''}}

Cryptography is about encoding a message so that it is hard for a
third party to read. The original message is called the
\textbf{plaintext}\index{plaintext}\index{cipher!plaintext} and the
encrypted message is called the
\textbf{ciphertext}\index{ciphertext}\index{cipher!ciphertext}. The
process of turning a plaintext into the corresponding ciphertext is
called \textbf{encryption}\index{encryption}\index{cipher!encryption},
and the process of turning a ciphertext into the corresponding
plaintext is called \textbf{decryption}\index{decryption}%
\index{cipher!decryption}. An encryption and decryption method is also
called a \textbf{cipher}\index{cipher}.  Modern ciphers are designed
in such a way that the cipher itself is not secret, but the encryption
depends on a secret \textbf{key}\index{key!cryptography}%
\index{cipher!key}. A cipher should be designed so that decryption is
easy for a person who knows the key, but difficult for everybody
else. The art of designing ciphers is called
\textbf{cryptography}\index{cryptography}, and the art of breaking
ciphers is called \textbf{cryptanalysis}\index{cryptanalysis}.

In order to be able to define ciphers using algebraic operations, we
start by encoding strings as sequences of numbers. To that end, we
assign a number to each letter of the alphabet, as well as the special
symbols ``space'', ``comma'', and ``period'', according to the
following scheme.
\begin{center}
  \tabcolsep=2.5ex
  \begin{tabular}{|c|c|c|c|c|c|c|c|c|}
    \hline
    Space & A & B & C & D & \ldots & Z & Comma & Period \\\hline
    0 & 1 & 2 & 3 & 4 & \ldots & 26 & 27 & 28 \\\hline
  \end{tabular}
\end{center}
In practical applications, one would probably use a larger set of
symbols and a standard encoding such as ASCII or UTF-8. But the above
29 symbols will be sufficient for our purposes.

\begin{example}{Representing strings as sequences of numbers}{string-encoding}
  Convert the string ``Attack at dawn'' to a sequence of
  numbers. Convert the sequence of numbers
  $9,0,12,9,11,5,0,3,15,4,5,19,28$ to a string.
\end{example}

\begin{solution}
  We have $\q{A}=1$, $\q{T}=20$, $\q{T}=20$, $\q{A}=1$, $\q{C}=3$,
  $\q{K}=11$, $\qq{Space}=0$, and so on. Continuing in this way, the
  encoding of ``Attack at dawn'' is
  $1,20,20,1,3,11,0,1,20,0,4,1,23,14$.  Conversely, we have $9=\q{I}$,
  $0=\qq{Space}$, $12=\q{L}$, $9=\q{I}$, $11=\q{K}$, $5=\q{E}$, and so
  on. We find that the decoded string is ``I like codes.''
\end{solution}

There are many different ways to define ciphers. Some of the oldest
known ciphers date back thousands of years.  An example of such a
``classic'' cipher is a \textbf{substitution cipher}%
\index{substitution cipher}\index{cipher!substitution cipher}, where
each letter of the alphabet is replaced by a different letter, for
example $\q{A}\mapsto\q{D}$, $\q{B}\mapsto\q{E}$, and so
on. Substitution ciphers have the property that changing one letter of
the plaintext always changes exactly one letter of the
ciphertext. This is not a desirable property, because it makes the
cipher easy to break. Therefore, modern ciphers are designed to
satisfy a property called \textbf{diffusion}\index{diffusion}%
\index{cipher!diffusion}: changing one letter of the plaintext should
change many letters of the ciphertext.

In a \textbf{block cipher}%
\index{block cipher}\index{cipher!block cipher}, the plaintext is
first divided into blocks of equal size, and then each block is
encrypted separately. The \textbf{block size}\index{block size} is the
number of plaintext symbols in each block. If the length of the
plaintext is not divisible by the block size, we pad the final block
with additional spaces. In the context of a block cipher, the
diffusion property means that changing one symbols of a plaintext
block potentially affects every symbol of the ciphertext block. The
following is an example of a block cipher.

\begin{definition}{Hill cipher}{hill-cipher}
  The \textbf{Hill cipher}%
  \index{Hill cipher}\index{cipher!Hill cipher} of block size $n$ has
  as its key an invertible $n\times n$-matrix $A$ with scalars from
  $\Z_{29}$. Each ciphertext block $b_1,\ldots,b_n$  is computed from
  the corresponding plaintext block $a_1,\ldots,a_n$ by matrix
  multiplication modulo $\Z_{29}$:
  \begin{equation*}
    \begin{mymatrix}{c}b_1\\\vdots\\b_n\end{mymatrix}
    ~=~ A\begin{mymatrix}{c}a_1\\\vdots\\a_n\end{mymatrix}.
  \end{equation*}
  The matrix $A$ is called the \textbf{encryption matrix}%
  \index{encryption matrix}\index{matrix!encryption} of the cipher. It
  inverse $A^{-1}$ is called the \textbf{decryption matrix}%
  \index{decryption matrix}\index{matrix!decryption}.
\end{definition}

\begin{example}{Hill cipher: encryption}{hill-cipher-encryption}
  Encrypt the message ``Meet me at noon'' using the Hill cipher with
  block size $3$ and encryption matrix
  \begin{equation*}
    A ~=~ \begin{mymatrix}{ccc}
      2 & 4 & 1 \\
      3 & 1 & 5 \\
      1 & 3 & 2 \\
    \end{mymatrix}.
  \end{equation*}
\end{example}

\begin{solution}
  We start by converting the message ``Meet me tomorrow'' to a sequence
  of scalars. We have $\q{M}=13$, $\q{E}=5$, and so on. The encoded
  plaintext is $13,5,5,20,0,13,5,0,20,15,13,15,18,18,15,23$. Next, we
  divide the plaintext into blocks of length 3. Since the length of
  the plaintext is not a multiple of three, we pad the final block
  with spaces, i.e., with zeros.
  \begin{equation*}
    \mbox{Plaintext blocks:}\quad
    (13,5,5),\
    (20,0,13),\
    (5,0,20),\
    (15,13,15),\
    (18,18,15),\
    (23,0,0).
  \end{equation*}
  To compute the ciphertext, we regard each plaintext block as a
  3-dimensional column vector and multiply by the encryption matrix
  $A$. All calculations are done modulo $29$. For example, for the
  first block, we have
  \begin{equation*}
    A \begin{mymatrix}{c} 13 \\ 5 \\ 5 \end{mymatrix}
    ~=~ \begin{mymatrix}{ccc}
      2 & 4 & 1 \\
      3 & 1 & 5 \\
      1 & 3 & 2 \\
    \end{mymatrix}
    \begin{mymatrix}{c} 13 \\ 5 \\ 5 \end{mymatrix}
    ~=~ \begin{mymatrix}{c} 22 \\ 11 \\ 9 \end{mymatrix},
  \end{equation*}
  so the first ciphertext block is $(22,11,9)$. We repeat the same
  with the remaining plaintext blocks.
  \begin{align*}
    &A \begin{mymatrix}{c} 20 \\ 0 \\ 13 \end{mymatrix}
    = \begin{mymatrix}{c} 24 \\ 9 \\ 17 \end{mymatrix},
    \quad
    A \begin{mymatrix}{c} 5 \\ 0 \\ 20 \end{mymatrix}
    = \begin{mymatrix}{c} 1 \\ 28 \\ 16 \end{mymatrix},
    \quad
    A \begin{mymatrix}{c} 15 \\ 13 \\ 15 \end{mymatrix}
    = \begin{mymatrix}{c} 10 \\ 17 \\ 26 \end{mymatrix},
    \\\\[-2ex]
    &A \begin{mymatrix}{c} 18 \\ 18 \\ 15 \end{mymatrix}
    = \begin{mymatrix}{c} 7 \\ 2 \\ 15 \end{mymatrix},
    \quad
    A \begin{mymatrix}{c} 23 \\ 0 \\ 0 \end{mymatrix}
    = \begin{mymatrix}{c} 17 \\ 11 \\ 23 \end{mymatrix}.
  \end{align*}
  Therefore, we have found the following ciphertext blocks:
  \begin{equation*}
    \mbox{Ciphertext blocks:}\quad
    (22,11,9),\
    (24,9,17),\
    (1,28,16),\
    (10,17,26),\
    (7,2,15),\
    (17,11,23).
  \end{equation*}
  Finally, we can convert the ciphertext to a list of symbols:
  ``VKIXIQA.PJQZGBOQKW''.
\end{solution}

\begin{example}{Hill cipher: decryption}{hill-cipher-decryption}
  Decrypt the message ``RNOLFPHHCIGH DE'' using the Hill cipher with
  block size $3$ and encryption matrix
  \begin{equation*}
    A ~=~ \begin{mymatrix}{ccc}
      2 & 4 & 1 \\
      3 & 1 & 5 \\
      1 & 3 & 2 \\
    \end{mymatrix}.
  \end{equation*}
\end{example}

\begin{solution}
  The process is analogous to encryption, except that we need to use
  the decryption matrix $A^{-1}$ instead of $A$. We first calculate
  $A^{-1}$, keeping in mind that scalars are from the field $\Z_{29}$.
  The method is the same as in Example~\ref{exa:matrix-inverse-z7}; we
  skip the individual steps in the interest of brevity.
  \begin{equation*}
    \mat{A\mid I}
    ~=~
    \begin{mymatrix}{ccc|ccc}
      2 & 4 & 1  &  1 & 0 & 0 \\
      3 & 1 & 5  &  0 & 1 & 0 \\
      1 & 3 & 2  &  0 & 0 & 1 \\
    \end{mymatrix}
    ~\sim~\ldots~\sim~
    \begin{mymatrix}{ccc|ccc}
      1 & 0 & 0  &  23 & 20 & 11 \\
      0 & 1 & 0  &   4 & 17 & 28 \\
      0 & 0 & 1  &  26 &  8 & 11 \\
    \end{mymatrix}
    ~=~
    \mat{I\mid A^{-1}}.
  \end{equation*}
  Next, we convert the ciphertext ``RNOLFPHHCIGH DE'' to a sequence
  of scalars and divide it into blocks of length 3:
  \begin{equation*}
    \mbox{Ciphertext blocks:}\quad
    (18,14,15),\
    (12,6,16),\
    (8,8,3),\
    (9,7,8),\
    (0,4,5).
  \end{equation*}
  Now we decrypt each ciphertext block by a matrix multiplication
  with $A^{-1}$.
  \begin{align*}
    &A^{-1} \begin{mymatrix}{c} 18 \\ 14 \\ 15 \end{mymatrix}
    = \begin{mymatrix}{c} 18 \\ 5 \\ 20 \end{mymatrix},
    \quad
    A^{-1} \begin{mymatrix}{c} 12 \\ 6 \\ 16 \end{mymatrix}
    = \begin{mymatrix}{c} 21 \\ 18 \\ 14 \end{mymatrix},
    \quad
    A^{-1} \begin{mymatrix}{c} 8 \\ 8 \\ 3 \end{mymatrix}
    = \begin{mymatrix}{c} 0 \\ 20 \\ 15 \end{mymatrix},
    \\\\[-2ex]
    &A^{-1} \begin{mymatrix}{c} 9 \\ 7 \\ 8 \end{mymatrix}
    = \begin{mymatrix}{c} 0 \\ 2 \\ 1 \end{mymatrix},
    \quad
    A^{-1} \begin{mymatrix}{c} 0 \\ 4 \\ 5 \end{mymatrix}
    = \begin{mymatrix}{c} 19 \\ 5 \\ 0 \end{mymatrix}.
  \end{align*}
  This yields the following plaintext blocks:
    \begin{equation*}
    \mbox{Plaintext blocks:}\quad
    (18,5,20),\
    (21,18,14),\
    (0,20,15),\
    (0,2,1),\
    (19,5,0).
  \end{equation*}
  Converting these back to letters, and omitting the trailing space,
  we find that the plaintext is ``return to base''.
\end{solution}

It is important to note that, despite its good diffusion properties,
the Hill cipher is not secure. The cipher has many weaknesses. For
one, because $A\vect{0}=\vect{0}$, a block of blanks in the plaintext
will always be encrypted as a block of blanks in the ciphertext. More
importantly, the cipher is subject to a so-called \textbf{known
  plaintext attack}\index{known plaintext attack}%
\index{cipher!known plaintext attack}.  If an eavesdropper intercepts
some ciphertext for which a small amount of the corresponding
plaintext happens to be known, it is immediately possible to recover
the key and therefore decrypt the rest of the ciphertext. Carrying out
this attack only requires some basic knowledge of linear algebra. The
following example illustrates how this is done.

\begin{example}{Cryptanalysis of the Hill cipher: known plaintext attack}{hill-cipher-cryptanalysis}
  Eve intercepts the following encrypted message sent by Alice:
  \begin{center}
    ``EFNOR.AHIFNEPL.TSZS,RSKT.ZBBRFVUPFVZLFHNTV''.
  \end{center}
  Eve knows that Alice uses a Hill cipher, but she does not know the
  secret encryption matrix. Eve also knows that Alice begins all of
  her correspondence with ``My dear love''. Decrypt the message.
\end{example}

\begin{solution}
  The first three blocks of the ciphertext are ``EFNOR.AHI'', i.e.,
  \begin{equation*}
    \mbox{Ciphertext blocks:}\quad
    (5,6,14),\
    (15,18,28),\
    (1,8,9).
  \end{equation*}
  Eve also knows that the first three blocks of the plaintext are ``MY
  DEAR L'', i.e.,
  \begin{equation*}
    \mbox{Plaintext blocks:}\quad
    (13,25,0),\
    (4,5,1),\
    (18,0,12).
  \end{equation*}
  From these facts, Eve deduces the following information about the
  unknown decryption matrix $A^{-1}$:
  \begin{equation*}
    A^{-1} \begin{mymatrix}{c} 5 \\ 6 \\ 14 \end{mymatrix}
    = \begin{mymatrix}{c} 13 \\ 25 \\ 0 \end{mymatrix},\quad
    A^{-1} \begin{mymatrix}{c} 15 \\ 18 \\ 28 \end{mymatrix}
    = \begin{mymatrix}{c} 4 \\ 5 \\ 1 \end{mymatrix},\quad
    A^{-1} \begin{mymatrix}{c} 1 \\ 8 \\ 9 \end{mymatrix}
    = \begin{mymatrix}{c} 18 \\ 0 \\ 12 \end{mymatrix}.
  \end{equation*}
  Since Eve remembers the column method of matrix multiplication, she
  knows that these three equations can be written as a single equation
  in matrix form:
  \begin{equation*}
    A^{-1} \begin{mymatrix}{ccc}
      5 & 15 & 1 \\
      6 & 18 & 8 \\
      14 & 28 & 9 \\
    \end{mymatrix}
    = \begin{mymatrix}{ccc}
      13 & 4 & 18 \\
      25 & 5 & 0 \\
      0 & 1 & 12 \\
    \end{mymatrix}.\quad
  \end{equation*}
  Note that this equation is of the form $A^{-1}C=P$. (Here, $C$
  stands for ``ciphertext'' and $P$ for ``plaintext''). Multiplying
  both sides of the equation by $C^{-1}$ on the right, we get
  $A^{-1} = PC^{-1}$. Thus, assuming that $C$ is invertible, Eve can
  easily compute the decryption matrix $A^{-1}$. Eve computes:
  \begin{equation*}
    C^{-1}
    =
    \begin{mymatrix}{ccc}
      5 & 15 & 1 \\
      6 & 18 & 8 \\
      14 & 28 & 9 \\
    \end{mymatrix}^{-1}
    =
    \begin{mymatrix}{ccc}
      19 & 8 & 23 \\
      0 & 5 & 2 \\
      22 & 1 & 0 \\
    \end{mymatrix}
  \end{equation*}
  and then
  \begin{equation*}
    A^{-1}
    =
    PC^{-1}
    =
    \begin{mymatrix}{ccc}
      13 & 4 & 18 \\
      25 & 5 & 0 \\
      0 & 1 & 12 \\
    \end{mymatrix}
    \begin{mymatrix}{ccc}
      19 & 8 & 23 \\
      0 & 5 & 2 \\
      22 & 1 & 0 \\
    \end{mymatrix}
    =
    \begin{mymatrix}{ccc}
      5  & 26 & 17 \\
      11 & 22 & 5 \\
      3  & 17 & 2 \\
    \end{mymatrix}.
  \end{equation*}
  Armed with the decryption matrix $A^{-1}$, Eve can now decrypt
  Alice's entire message, using the same method as in
  Example~\ref{exa:hill-cipher-decryption}. The plaintext is ``My dear
  love, run away with me at midnight''.
\end{solution}


\void{
S-boxes.
}

\endgroup
