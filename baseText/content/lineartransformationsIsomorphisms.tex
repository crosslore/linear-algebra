\section{Isomorphisms}

\begin{outcome}
  \begin{enumerate}
  \item Determine if a linear transformation is an isomorphism.
  \item Determine if two subspaces of $\R^n$ are isomorphic. 
  \end{enumerate}
\end{outcome}

Recall the definition of a linear transformation. Let $V$ and $W$ be two subspaces of $\R^{n}$ and $\R^{m}$
respectively. A mapping $T:V\rightarrow W$ is called a\textbf{\ linear
transformation} or \textbf{linear map} if it preserves the algebraic
operations of addition and
\index{linear map}
\index{linear transformation} scalar multiplication. Specifically, if $a,b$
are scalars and $
\vect{x},\vect{y}$ are vectors, 
\begin{equation*}
T\tup{a\vect{x}+b\vect{y}} =aT(\vect{x})+bT(\vect{y})
\end{equation*}

Consider the following important definition.

\begin{definition}{Isomorphism}{isomorphism}
A linear map $T$ is called an \textbf{isomorphism} 
\index{isomorphism}if the following two conditions are satisfied.

\begin{itemize}
\item $T$ is one to one. That is, if $T(\vect{x})=T(\vect{y})$, then $\vect{x}=\vect{y}$.

\item $T$ is onto. That is, if $\vect{w}\in W$, there exists $
\vect{v}\in V$ such that $T(\vect{v})=\vect{w}$.
\end{itemize}

Two such subspaces which have an isomorphism as described above are said to
be \textbf{isomorphic.}
\index{isomorphic}
\end{definition}

Consider the following example of an isomorphism.

\begin{example}{Isomorphism}{isomorphism}
Let $T: \R^2 \mapsto \R^2$ be defined by 
\[
T \begin{mymatrix}{c}
x \\
y
\end{mymatrix} = \begin{mymatrix}{c}
x + y \\
x - y 
\end{mymatrix}
\]
Show that $T$ is an isomorphism.
\end{example}

\begin{solution}
To prove that $T$ is an isomorphism we must show
\begin{enumerate}
\item $T$ is a linear transformation;
\item $T$ is one to one;
\item $T$ is onto.
\end{enumerate}

We proceed as follows.

\begin{enumerate}
\item $T$ is a linear transformation:

Let $k, p$ be scalars. 
\begin{eqnarray*}
T \tup{k \begin{mymatrix}{c}
x_1 \\
y_1
\end{mymatrix} + p \begin{mymatrix}{c}
x_2 \\
y_2
\end{mymatrix} } &=& 
T \tup{\begin{mymatrix}{c}
kx_1 \\
ky_1
\end{mymatrix} + \begin{mymatrix}{c}
px_2 \\
py_2
\end{mymatrix} } \\
&=&
T \tup{\begin{mymatrix}{c}
kx_1 + px_2 \\
ky_1 + py_2
\end{mymatrix}  } \\
&=& 
\begin{mymatrix}{c}
(kx_1 + px_2) + (ky_1 + py_2) \\
(kx_1 + px_2) - (ky_1 + py_2)
\end{mymatrix} \\
&=& 
\begin{mymatrix}{c}
(kx_1 + ky_1) + (px_2 + py_2) \\
(kx_1  - ky_1) + (px_2 - py_2)
\end{mymatrix} \\
&=&
\begin{mymatrix}{c}
kx_1 + ky_1  \\
kx_1  - ky_1
\end{mymatrix} + 
\begin{mymatrix}{c}
px_2 + py_2 \\
px_2 - py_2
\end{mymatrix} \\
&=& k \begin{mymatrix}{c}
x_1 + y_1  \\
x_1  - y_1
\end{mymatrix} + 
p \begin{mymatrix}{c}
x_2 + y_2 \\
x_2 - y_2
\end{mymatrix} \\
&=& 
k T \tup{\begin{mymatrix}{c}
x_1 \\
y_1
\end{mymatrix} } + p T \tup{\begin{mymatrix}{c}
x_2 \\
y_2
\end{mymatrix} }
\end{eqnarray*}

Therefore $T$ is linear. 

\item $T$ is one to one:

We need to show that if $T (\vect{x}) = \vect{0}$ for a vector $\vect{x} \in \R^2$, then it follows that $\vect{x} = \vect{0}$.  Let $\vect{x} = \begin{mymatrix}{c}
x \\
y
\end{mymatrix}$. 

\[
T  \tup{\begin{mymatrix}{c}
x \\
y
\end{mymatrix} } = \begin{mymatrix}{c}
x + y\\
x - y 
\end{mymatrix} = \begin{mymatrix}{c}
0 \\
0
\end{mymatrix}
\]
This provides a system of equations given by 
\begin{eqnarray*}
x + y &=& 0\\
x - y &=& 0
\end{eqnarray*}
You can verify that the solution to this system if $x = y =0$. Therefore 
\[
\vect{x} = \begin{mymatrix}{c}
x \\
y
\end{mymatrix}
 = \begin{mymatrix}{c}
0 \\
0
\end{mymatrix}
\]
and $T$ is one to one.

\item $T$ is onto:

Let $a,b$ be scalars. We want to check if there is always a solution to 
\[
T  \tup{\begin{mymatrix}{c}
x \\
y
\end{mymatrix} } = \begin{mymatrix}{c}
x + y\\
x - y 
\end{mymatrix} = \begin{mymatrix}{c}
a \\
b
\end{mymatrix}
\]

This can be represented as the system of equations 
\begin{eqnarray*}
x + y &=& a\\
x - y &=& b
\end{eqnarray*}

Setting up the augmented matrix and row reducing gives
\[
\begin{mymatrix}{cc|c}
1 & 1 & a \\
1 & -1 & b
\end{mymatrix} \rightarrow \cdots \rightarrow
\begin{mymatrix}{cc|c}
1 & 0 & \vspace{0.05in}\frac{a+b}{2} \\
0 & 1 & \vspace{0.05in}\frac{a-b}{2}
\end{mymatrix}
\]
This has a solution for all $a,b$ and therefore $T$ is onto. 
\end{enumerate}

Therefore $T$ is an isomorphism.
\end{solution}

An important property of isomorphisms is that its inverse is also an isomorphism. 

\begin{proposition}{Inverse of an isomorphism}{inverse-isomorphism}
Let $T:V\rightarrow W$ be an isomorphism and $V,W$ be subspaces of $\R^n$. Then $T^{-1}:W\rightarrow V$ is
also an isomorphism.
\index{isomorphism!inverse}
\end{proposition}

\begin{proof} Let $T$ be an isomorphism.  Since $T$ is onto, a typical
vector in $W$ is of the form $T(\vect{v})$ where $\vect{v} \in V$. Consider then for $a,b$
scalars, 
\begin{equation*}
T^{-1}\tup{aT(\vect{v}_{1})+bT(\vect{v}_{2})}
\end{equation*}
where $\vect{v}_{1}, \vect{v}_2 \in V$. Is this equal to 
\begin{equation*}
aT^{-1}\tup{T (\vect{v}_{1})} +bT^{-1}\tup{T(\vect{v}_{2})} =a\vect{v}_{1}+b\vect{v}_{2}?
\end{equation*}
Since $T$ is one to one, this will be so if 
\begin{equation*}
T\tup{a\vect{v}_{1}+b\vect{v}_{2}} =T\tup{T^{-1}\tup{aT(\vect{v}_{1})+bT(\vect{v}_{2})}
} =aT(\vect{v}_{1})+bT(\vect{v}_{2}).
\end{equation*}
However, the above statement is just the condition that $T$ is a linear map.
Thus $T^{-1}$ is indeed a linear map. If $\vect{v} \in V$ is given, then $\vect{v}=T^{-1}\tup{T(\vect{v})} $ and so $T^{-1}$ is onto. If $T^{-1} (\vect{v})=0$, then 
\begin{equation*}
\vect{v}=T\tup{T^{-1}(\vect{v})} =T(\vect{0})=\vect{0}
\end{equation*}
and so $T^{-1}$ is one to one.
\end{proof}

Another important result is that the composition of multiple isomorphisms is also an isomorphism.

\begin{proposition}{Composition of isomorphisms}{composition-isomorphisms}
Let $T:V\rightarrow W$ and  $S:W\rightarrow Z$ be isomorphisms where $V,W,Z$ are subspaces of $\R^n$. Then $S\circ
T $ defined by $\tup{S\circ T} \tup{\vect{v} } = S\tup{
T\tup{\vect{v} } } $ is also an isomorphism.
\index{isomorphism!composition}
\end{proposition}

\begin{proof}
Suppose $T:V\rightarrow W$ and  $S:W\rightarrow Z$ are isomorphisms. Why is $S\circ T$ a linear map?
For $a,b$ scalars,
\begin{eqnarray*}
S\circ T\tup{a\vect{v}_{1}+b(\vect{v}_{2})} 
&=& S\tup{T\tup{a\vect{v}_{1}+b\vect{v}_{2}} } =S\tup{aT\vect{v}_{1}+bT\vect{v}_{2}} \\
&=&aS\tup{T\vect{v}_{1}} +bS\tup{T\vect{v}_{2}} = a\tup{S\circ
T} \tup{\vect{v}_{1}} +b\tup{S\circ T} \tup{\vect{v}_{2}}
\end{eqnarray*}
Hence $S\circ T$ is a linear map. If $\tup{S\circ T} \tup{\vect{v} }
=0$, then $S\tup{T\tup{\vect{v} } } =0$ and it follows that $T(\vect{v})=\vect{0}$ and hence by this lemma again, $\vect{v}=\vect{0}$. Thus $S\circ
T $ is one to one. It remains to verify that it is onto. Let $\vect{z} \in Z$. Then
since $S$ is onto, there exists $\vect{w} \in W$ such that $S(\vect{w})=\vect{z}$. Also, since $T$
is onto, there exists $\vect{v}\in V$ such that $T(\vect{v})=\vect{w}$. It follows that $S\tup{
T\tup{\vect{v}} } =\vect{z}$ and so $S\circ T$ is also onto.
\end{proof}

Consider two subspaces $V$ and $W$, and suppose there exists an isomorphism mapping one to the other. In this way the two subspaces are related, which we can write as $V \sim W$. Then the previous two propositions together claim that $\sim $ is an equivalence relation. That is: $\sim $
satisfies the following conditions:
\index{isomorphic!equivalence relation}

\begin{itemize}
\item $V\sim V$

\item If $V\sim W$, it follows that $W\sim V$

\item If $V\sim W$ and $W\sim Z$, then $V\sim Z$
\end{itemize}

We leave the verification of these conditions as an exercise.

Consider the following example. 

\begin{example}{Matrix isomorphism}{matrix-isomorphism}
Let $T:\R^{n}\rightarrow \R^{n}$ be defined by $T(\vect{x}) = A(\vect{x})$ where $A$ is an invertible $n\times n$-matrix. Then $T$ is
an isomorphism.
\end{example}

\begin{solution}
The reason for this is that, since $A$ is invertible, the only vector it
sends to $\vect{0}$ is the zero vector. Hence if $A(\vect{x})=A(\vect{y})$, then $A\tup{\vect{x}-\vect{y}} =\vect{0}$ and so $\vect{x}=\vect{y}$. It is onto
because if $\vect{y}\in \R^{n},A\tup{A^{-1} (\vect{y})} =\tup{
AA^{-1}} (\vect{y})$ $=\vect{y}$. 
\end{solution}

In fact, all isomorphisms from $\R^{n}$ to $\R^{n}$ can be expressed as $T(\vect{x}) = A(\vect{x})$ where $A$ is an invertible $n \times n$-matrix. One
simply considers the matrix whose $i^{th}$ column is $T\vect{e}_{i}$.

Recall that a basis of a subspace $V$ is a set of linearly independent vectors which span $V$. The following fundamental lemma describes the relation between bases and
isomorphisms.

\begin{lemma}{Mapping bases}{mapping-bases}
Let $T:V\rightarrow W$ be a linear transformation where $V,W$ are
subspaces of $\R^n$. If $T$ is one to one, then it has the property that if $\set{\vect{u}_{1},\cdots ,\vect{u}_{k}} $ is linearly independent, so is $\set{T(\vect{u}_{1}),\cdots ,T(\vect{u}_{k})}$.

More generally, $T$ is an isomorphism if and only if whenever $\set{
\vect{v}_{1},\cdots ,\vect{v}_{n}} $ is a basis for $V$, it follows that $\set{T
(\vect{v}_{1}),\cdots ,T(\vect{v}_{n})} $ is a basis for $W$. 
\index{isomorphism!bases}
\index{one to one!linear independence}
\end{lemma}

\begin{proof}First suppose that $T$ is a linear transformation and is one to one
and $\set{\vect{u}_{1},\cdots ,\vect{u}_{k}} $ is linearly
independent. It is required to show that $\set{T(\vect{u}_{1}),\cdots ,T(\vect{
u}_{k})} $ is also linearly independent. Suppose then that 
\begin{equation*}
\sum_{i=1}^{k}c_{i}T(\vect{u}_{i})=\vect{0}
\end{equation*}
Then, since $T$ is linear, 
\begin{equation*}
T\tup{\sum_{i=1}^{n}c_{i}\vect{u}_{i}} =\vect{0}
\end{equation*}
Since $T$ is one to one, it follows that 
\begin{equation*}
\sum_{i=1}^{n}c_{i}\vect{u}_{i}=0
\end{equation*}
Now the fact that $\set{\vect{u}_{1},\cdots ,\vect{u}_{n}} $ is
linearly independent implies that each $c_{i}=0$. Hence $\set{T(\vect{u}
_{1}),\cdots ,T(\vect{u}_{n})} $ is linearly independent.

Now suppose that $T$ is an isomorphism and $\set{\vect{v}_{1},\cdots ,\vect{
v}_{n}} $ is a basis for $V$. It was just shown that $\set{T(\vect{v}
_{1}),\cdots ,T(\vect{v}_{n})} $ is linearly independent. It remains to
verify that span$\set{T(\vect{v}_{1}),\cdots ,T(\vect{v}_{n})}=W$. If $\vect{w}\in W$, then since $T$ is onto there
exists $\vect{v}\in V$ such that $T(\vect{v})=\vect{w}$. Since $\set{\vect{v}
_{1},\cdots ,\vect{v}_{n}} $ is a basis, it follows that there exists
scalars $\set{c_{i}} _{i=1}^{n}$ such that 
\begin{equation*}
\sum_{i=1}^{n}c_{i}\vect{v}_{i}=\vect{v}.
\end{equation*}
Hence, 
\begin{equation*}
\vect{w}=T(\vect{v})=T\tup{\sum_{i=1}^{n}c_{i}\vect{v}_{i}}
=\sum_{i=1}^{n}c_{i}T(\vect{v}_{i})
\end{equation*}
It follows that span$\set{T(\vect{v}_{1}),\cdots , T(\vect{v}_{n})} =W$ showing that this set of vectors is a
basis for $W$.

Next suppose that $T$ is a linear transformation which takes a basis to a basis. This means that if $\set{\vect{v}_{1},\cdots ,\vect{v}_{n}} $ is a basis for $V$, it
follows $\set{T(\vect{v}_{1}),\cdots ,T(\vect{v}_{n})} $ is a basis for $
W$. Then if $w\in W$, there exist scalars $c_{i}$ such that $
w=\sum_{i=1}^{n}c_{i}T(\vect{v}_{i})=T\tup{\sum_{i=1}^{n}c_{i}\vect{v}_{i}} $
showing that $T$ is onto. If $T\tup{\sum_{i=1}^{n}c_{i}\vect{v}_{i}} =\vect{0}$
then $\sum_{i=1}^{n}c_{i}T(\vect{v}_{i})=\vect{0}$ and since the vectors $\set{T(\vect{v}
_{1}),\cdots ,T(\vect{v}_{n})} $ are linearly independent, it follows
that each $c_{i}=0$. Since $\sum_{i=1}^{n}c_{i}\vect{v}_{i}$ is a typical vector in 
$V$, this has shown that if $T(\vect{v})=\vect{0}$ then $\vect{v}=\vect{0}$ and so $T$ is also one to one.
Thus $T$ is an isomorphism. 
\end{proof}

The following theorem illustrates a very useful idea for defining an
isomorphism. Basically, if you know what it does to a basis, then you can
construct the isomorphism.

\begin{theorem}{Isomorphic subspaces}{isomorphic-subspaces}
Suppose $V$ and $W$ are two subspaces of $\R^n$. Then the two
subspaces are isomorphic if and only if they have the same dimension. In the
case that the two subspaces have the same dimension, then for
\index{isomorphism!equivalence} a linear map $T:V\rightarrow W$, the
following are equivalent.

\begin{enumerate}
\item $T$ is one to one.

\item $T$ is onto.

\item $T$ is an isomorphism.
\end{enumerate}
\end{theorem}

\begin{proof} Suppose first that these two subspaces have the same
dimension. Let a basis for $V$ be $\set{
\vect{v}_{1},\cdots ,\vect{v}_{n}} $ and let a basis for $W$ be $
\set{\vect{w}_{1},\cdots ,\vect{w}_{n}}$. Now define $T$ as
follows. 
\begin{equation*}
T(\vect{v}_{i})=\vect{w}_{i}
\end{equation*}
for $\sum_{i=1}^{n}c_{i}\vect{v}_{i}$ an arbitrary vector of $V$,
\begin{equation*}
T\tup{\sum_{i=1}^{n}c_{i}\vect{v}_{i}} = \sum_{i=1}^{n}c_{i}T
\vect{v}_{i}=\sum_{i=1}^{n}c_{i}\vect{w}_{i}.
\end{equation*}
It is necessary to verify that this is well defined. Suppose then that 
\begin{equation*}
\sum_{i=1}^{n}c_{i}\vect{v}_{i}=\sum_{i=1}^{n}\hat{c}_{i}\vect{v}_{i}
\end{equation*}
Then 
\begin{equation*}
\sum_{i=1}^{n}\tup{c_{i}-\hat{c}_{i}} \vect{v}_{i}=\vect{0}
\end{equation*}
and since $\set{\vect{v}_{1},\cdots ,\vect{v}_{n}} $ is a basis, $
c_{i}=\hat{c}_{i}$ for each $i$. Hence 
\begin{equation*}
\sum_{i=1}^{n}c_{i}\vect{w}_{i}=\sum_{i=1}^{n}\hat{c}_{i}\vect{w}_{i}
\end{equation*}
and so the mapping is well defined. Also if $a,b$ are scalars, 
\begin{eqnarray*}
T\tup{a\sum_{i=1}^{n}c_{i}\vect{v}_{i}+b\sum_{i=1}^{n}\hat{c}_{i}\vect{v}%
_{i}} &=&T\tup{\sum_{i=1}^{n}\tup{ac_{i}+b\hat{c}_{i}} \vect{v%
}_{i}} =\sum_{i=1}^{n}\tup{ac_{i}+b\hat{c}_{i}} \vect{w}_{i} \\
&=&a\sum_{i=1}^{n}c_{i}\vect{w}_{i}+b\sum_{i=1}^{n}\hat{c}_{i}\vect{w}_{i} \\
&=&aT\tup{\sum_{i=1}^{n}c_{i}\vect{v}_{i}} +bT\tup{\sum_{i=1}^{n}%
\hat{c}_{i}\vect{v}_{i}}
\end{eqnarray*}
Thus $T$ is a linear transformation. 

Now if 
\begin{equation*}
T\tup{\sum_{i=1}^{n}c_{i}\vect{v}_{i}} =\sum_{i=1}^{n}c_{i}\vect{w}%
_{i}=\vect{0},
\end{equation*}
then since the $\set{\vect{w}_{1},\cdots ,\vect{w}_{n}} $ are
independent, each $c_{i}=0$ and so $\sum_{i=1}^{n}c_{i}\vect{v}_{i}=\vect{0}$
also. Hence $T$ is one to one. If $\sum_{i=1}^{n}c_{i}\vect{w}_{i}$ is a
vector in $W$, then it equals 
\begin{equation*}
\sum_{i=1}^{n}c_{i}T(\vect{v}_{i})=T\tup{\sum_{i=1}^{n}c_{i}\vect{v}_{i}}
\end{equation*}
showing that $T$ is also onto. Hence $T$ is an isomorphism and so $V$ and $W$
are isomorphic.

Next suppose $T:V \mapsto W$ is an isomorphism, so these two subspaces are isomorphic. Then for $\set{\vect{v}_{1},\cdots ,\vect{v}_{n}} $ a
basis for $V$, it follows that a basis for $W$
is $\set{T(\vect{v}_{1}),\cdots ,T(\vect{v}_{n})} $ showing that the two
subspaces have the same dimension.

Now suppose the two subspaces have the same dimension. Consider the three
claimed equivalences.

First consider the claim that $1.)\Rightarrow 2.)$. If $T$ is one to one and if $\set{\vect{v}_{1},\cdots ,\vect{v}
_{n}} $ is a basis for $V$, then $\set{T(\vect{v}_{1}),\cdots ,T(\vect{v
}_{n})} $ is linearly independent. If it is not a basis, then it must
fail to span $W$. But then there would exist $\vect{w}\notin \sspan
\set{T(\vect{v}_{1}),\cdots ,T(\vect{v}_{n})} $ and it follows that $\set{T(\vect{v}_{1}),\cdots ,T(\vect{v}_{n}),\vect{w}
} $ would be linearly independent which is impossible because there exists a basis for $W$ of $n$ vectors.

Hence $\sspan\set{T(\vect{v}_{1}),\cdots ,T(\vect{v}_{n})} =W$ and
so $\set{T(\vect{v}_{1}),\cdots ,T(\vect{v}_{n})} $ is a basis. If $\vect{w}\in W$, there exist scalars $c_{i}$ such that 
\begin{equation*}
\vect{w}=\sum_{i=1}^{n}c_{i}T(\vect{v}_{i})=T\tup{\sum_{i=1}^{n}c_{i}\vect{v}
_{i}}
\end{equation*}
showing that $T$ is onto. This shows that $1.)\Rightarrow 2.)$.

Next consider the claim that $2.)\Rightarrow 3.)$. Since $2.)$ holds, it
follows that $T$ is onto. It remains to verify that $T$ is one to one. Since 
$T$ is onto, there exists a basis of the form $\set{T(\vect{v}_{i}),\cdots ,T(\vect{v}_{n})}$. Then it follows that $\set{\vect{v}_{1},\cdots ,
\vect{v}_{n}} $ is linearly independent. Suppose 
\begin{equation*}
\sum_{i=1}^{n}c_{i}\vect{v}_{i}=\vect{0}
\end{equation*}
Then 
\begin{equation*}
\sum_{i=1}^{n}c_{i}T(\vect{v}_{i})=\vect{0}
\end{equation*}
Hence each $c_{i}=0$ and so, $\set{\vect{v}_{1},\cdots ,\vect{v}
_{n}} $ is a basis for $V$. Now it follows that a typical vector in $
V $ is of the form $\sum_{i=1}^{n}c_{i}\vect{v}_{i}$. If $T\tup{
\sum_{i=1}^{n}c_{i}\vect{v}_{i}} =\vect{0}$, it follows that 
\begin{equation*}
\sum_{i=1}^{n}c_{i}T(\vect{v}_{i})=\vect{0}
\end{equation*}
and so, since $\set{T(\vect{v}_{i}),\cdots ,T(\vect{v}_{n})} $ is
independent, it follows each $c_{i}=0$ and hence $\sum_{i=1}^{n}c_{i}\vect{v}
_{i}=\vect{0}$. Thus $T$ is one to one as well as onto and so it is an
isomorphism.

If $T$ is an isomorphism, it is both one to one and onto by definition so $
3.)$ implies both $1.)$ and $2.)$.
\end{proof}

Note the interesting way of defining a linear transformation in the first
part of the argument by describing what it does to a basis and then
``extending it linearly'' to the entire subspace.
\index{linear map!defining on a basis}

\begin{example}{Isomorphic subspaces}{}
Let $V=\R^{3}$ and let $W$ denote 
\begin{equation*}
\sspan\set{\begin{mymatrix}{r}
1 \\ 
2 \\ 
1 \\ 
1
\end{mymatrix} ,\begin{mymatrix}{r}
0 \\ 
1 \\ 
0 \\ 
1
\end{mymatrix} ,\begin{mymatrix}{r}
1 \\ 
1 \\ 
2 \\ 
0
\end{mymatrix} }
\end{equation*}
Show that $V$ and $W$ are isomorphic. 
\end{example}

\begin{solution}
First observe that these subspaces are both of dimension 3 and so they are isomorphic by Theorem~\ref{thm:isomorphic-subspaces}. The
three vectors which span $W$ are easily seen to be linearly independent by
making them the columns of a matrix and row reducing to the {\rref}.

You can exhibit an isomorphism of these two spaces as follows. 
\begin{equation*}
T(\vect{e}_{1})=\begin{mymatrix}{c}
1 \\ 
2 \\ 
1 \\ 
1
\end{mymatrix}, T(\vect{e}_{2})=\begin{mymatrix}{c}
0 \\ 
1 \\ 
0 \\ 
1
\end{mymatrix}, T(\vect{e}_{3})=\begin{mymatrix}{c}
1 \\ 
1 \\ 
2 \\ 
0
\end{mymatrix}
\end{equation*}
and extend linearly. Recall that the matrix of this linear transformation is
just the matrix having these vectors as columns. Thus the matrix of this
isomorphism is 
\begin{equation*}
\begin{mymatrix}{rrr}
1 & 0 & 1 \\ 
2 & 1 & 1 \\ 
1 & 0 & 2 \\ 
1 & 1 & 0
\end{mymatrix}
\end{equation*}
You should check that multiplication on the left by this matrix does
reproduce the claimed effect resulting from an application by $T$.
\end{solution}

Consider the following example. 

\begin{example}{Finding the matrix of an isomorphism}{matrix-of-isomorphism}
Let $V=\R^{3}$ and let $W$ denote 
\begin{equation*}
\sspan\set{\begin{mymatrix}{c}
1 \\ 
2 \\ 
1 \\ 
1
\end{mymatrix} ,\begin{mymatrix}{c}
0 \\ 
1 \\ 
0 \\ 
1
\end{mymatrix} ,\begin{mymatrix}{c}
1 \\ 
1 \\ 
2 \\ 
0
\end{mymatrix} }
\end{equation*}

Let $T: V \mapsto W$ be defined as follows. 
\begin{equation*}
T\begin{mymatrix}{c}
1 \\ 
1 \\ 
0
\end{mymatrix} =\begin{mymatrix}{c}
1 \\ 
2 \\ 
1 \\ 
1
\end{mymatrix} ,T\begin{mymatrix}{c}
0 \\ 
1 \\ 
1
\end{mymatrix} =\begin{mymatrix}{c}
0 \\ 
1 \\ 
0 \\ 
1
\end{mymatrix} ,T\begin{mymatrix}{c}
1 \\ 
1 \\ 
1
\end{mymatrix} =\begin{mymatrix}{c}
1 \\ 
1 \\ 
2 \\ 
0
\end{mymatrix}
\end{equation*}
Find the matrix of this isomorphism $T$.
\end{example}

\begin{solution}
 First note that the vectors 
\begin{equation*}
\begin{mymatrix}{c}
1 \\ 
1 \\ 
0
\end{mymatrix} ,\begin{mymatrix}{c}
0 \\ 
1 \\ 
1
\end{mymatrix} ,\begin{mymatrix}{c}
1 \\ 
1 \\ 
1
\end{mymatrix}
\end{equation*}
are indeed a basis for $\R^{3}$ as can be seen by making them the
columns of a matrix and using the {\rref}.

Now recall the matrix of $T$ is a $4\times 3$-matrix $A$ which gives the same
effect as $T$. Thus, from the way we multiply matrices, 
\begin{equation*}
A\begin{mymatrix}{rrr}
1 & 0 & 1 \\ 
1 & 1 & 1 \\ 
0 & 1 & 1
\end{mymatrix} =\begin{mymatrix}{rrr}
1 & 0 & 1 \\ 
2 & 1 & 1 \\ 
1 & 0 & 2 \\ 
1 & 1 & 0
\end{mymatrix}
\end{equation*}
Hence, 
\begin{equation*}
A=\begin{mymatrix}{rrr}
1 & 0 & 1 \\ 
2 & 1 & 1 \\ 
1 & 0 & 2 \\ 
1 & 1 & 0
\end{mymatrix} \begin{mymatrix}{rrr}
1 & 0 & 1 \\ 
1 & 1 & 1 \\ 
0 & 1 & 1
\end{mymatrix} ^{-1}=\begin{mymatrix}{rrr}
1 & 0 & 0 \\ 
0 & 2 & -1 \\ 
2 & -1 & 1 \\ 
-1 & 2 & -1
\end{mymatrix}
\end{equation*}
Note how the span of the columns of this new matrix must be the same as the
span of the vectors defining $W$.
\end{solution}

This idea of defining a linear transformation by what it does on a basis
works for linear maps which are not necessarily isomorphisms.

\begin{example}{Finding the matrix of an isomorphism}{}
Let $V=\R^{3}$ and let $W$ denote 
\begin{equation*}
\sspan\set{\begin{mymatrix}{c}
1 \\ 
0 \\ 
1 \\ 
1
\end{mymatrix} ,\begin{mymatrix}{c}
0 \\ 
1 \\ 
0 \\ 
1
\end{mymatrix} ,\begin{mymatrix}{c}
1 \\ 
1 \\ 
1 \\ 
2
\end{mymatrix} }
\end{equation*}
Let $T: V \mapsto W$ be defined as follows. 
\begin{equation*}
T\begin{mymatrix}{c}
1 \\ 
1 \\ 
0
\end{mymatrix} = \begin{mymatrix}{c}
1 \\ 
0 \\ 
1 \\ 
1
\end{mymatrix} ,T\begin{mymatrix}{c}
0 \\ 
1 \\ 
1
\end{mymatrix} =\begin{mymatrix}{c}
0 \\ 
1 \\ 
0 \\ 
1
\end{mymatrix} ,T\begin{mymatrix}{c}
1 \\ 
1 \\ 
1
\end{mymatrix} =\begin{mymatrix}{c}
1 \\ 
1 \\ 
1 \\ 
2
\end{mymatrix}
\end{equation*}
 Find the matrix of this linear transformation.
\end{example}

\begin{solution}
Note that in this case, the three vectors which span $W$ are not linearly independent. Nevertheless the above procedure will still work.
The reasoning is the same as before. If $A$ is this matrix, then 
\begin{equation*}
A\begin{mymatrix}{rrr}
1 & 0 & 1 \\ 
1 & 1 & 1 \\ 
0 & 1 & 1
\end{mymatrix} =\begin{mymatrix}{rrr}
1 & 0 & 1 \\ 
0 & 1 & 1 \\ 
1 & 0 & 1 \\ 
1 & 1 & 2
\end{mymatrix}
\end{equation*}
and so 
\begin{equation*}
A=\begin{mymatrix}{rrr}
1 & 0 & 1 \\ 
0 & 1 & 1 \\ 
1 & 0 & 1 \\ 
1 & 1 & 2
\end{mymatrix} \begin{mymatrix}{rrr}
1 & 0 & 1 \\ 
1 & 1 & 1 \\ 
0 & 1 & 1
\end{mymatrix} ^{-1}=\begin{mymatrix}{rrr}
1 & 0 & 0 \\ 
0 & 0 & 1 \\ 
1 & 0 & 0 \\ 
1 & 0 & 1
\end{mymatrix}
\end{equation*}
The columns of this last matrix are obviously not linearly independent.
\end{solution}
