\subsection{Eigenvalues and Eigenvectors for Special Types of Matrices}

There are three special kinds of matrices which we can use to simplify the process of finding eigenvalues and eigenvectors. 
Throughout this section, we will discuss similar matrices, elementary matrices, as well as triangular matrices. 

We begin with a definition.

\begin{definition}{Similar matrices}{similarmatrices}
Let $A$ and $B$ be $n \times n$ matrices. Suppose there exists an invertible matrix $P$ such that 
\begin{equation*}
A = P^{-1}BP
\end{equation*}
Then $A$ and $B$ are called \textbf{similar matrices}\index{similar matrix}.
\end{definition}

It turns out that we can use the concept of similar matrices to help us find the eigenvalues
of matrices. Consider the following lemma.

\begin{lemma}{Similar matrices and eigenvalues}{similarmatrices}
Let $A$ and $B$ be similar matrices, so that $A=P^{-1}BP$ where $A,B$ are $n\times n$ matrices and $P$ is invertible. Then $A,B$ have the
same eigenvalues.
\end{lemma}

\begin{proof}
We need to show two things. First, we need to show that if
$A=P^{-1}BP$, then $A$ and $B$ have the same eigenvalues.  Secondly,
we show that if $A$ and $B$ have the same eigenvalues, then
$A=P^{-1}BP$.

Here is the proof of the first statement. 
Suppose $A = P^{-1}BP$ and $\lambda$ is an eigenvalue of $A$, that is $AX=\lambda X$ for some $X\neq 0.$ Then
\begin{equation*}
P^{-1}BPX=\lambda X
\end{equation*}
and so
\begin{equation*}
BPX=\lambda PX
\end{equation*}

Since $P$ is one to one and $X \neq 0$, it follows that $PX \neq
0$. Here, $PX$ plays the role of the eigenvector in this equation.
Thus $\lambda$ is also an eigenvalue of $B$. One can similarly verify
that any eigenvalue of $B$ is also an eigenvalue of $A$, and thus both
matrices have the same eigenvalues as desired.

Proving the second statement is similar and is left as an exercise. 
\end{proof}

Note that this proof also demonstrates that the eigenvectors of $A$ and $B$ will (generally) be {\em different\em}.
We see in the proof that $AX = \lambda X$, while $B \left(PX\right)=\lambda \left(PX\right)$. Therefore,
for an eigenvalue $\lambda$, $A$ will have the eigenvector $X$ while $B$ will have the eigenvector $PX$. 


The second special type of matrices we discuss in this section is elementary matrices.  
Recall from Definition \ref{def:elementarymatricesandrowops} that an elementary matrix $E$ is obtained by applying
one row operation to the identity matrix. 

It is possible to use elementary matrices to simplify a matrix before searching for its
eigenvalues and eigenvectors. This is illustrated in the following
example.

\begin{example}{Simplify using elementary matrices}{simplifyusingelementarymatrices}
Find the eigenvalues for the matrix
\begin{equation*}
A = \leftB
\begin{array}{rrr}
 33 & 105 & 105 \\
 10 &  28 & 30 \\
-20 & -60 & -62
\end{array}
\rightB
\end{equation*}
\end{example}

\begin{solution} This matrix has big numbers and therefore we would like
to simplify as much as possible before computing the eigenvalues.

We will do so using row operations. First, add $2$ times the second row
to the third row. To do so, left multiply $A$ by $E \left(2,2\right)$. 
Then right multiply $A$ by the inverse of $E \left(2,2\right)$ as illustrated.
\begin{equation*}
\leftB
\begin{array}{rrr}
1 & 0 & 0 \\
0 & 1 & 0 \\
0 & 2 & 1
\end{array}
\rightB \leftB
\begin{array}{rrr}
33 & 105 & 105 \\
10 & 28 & 30 \\
-20 & -60 & -62
\end{array}
\rightB \leftB
\begin{array}{rrr}
1 & 0 & 0 \\
0 & 1 & 0 \\
0 & -2 & 1
\end{array}
\rightB =\leftB
\begin{array}{rrr}
33 & -105 & 105 \\
10 & -32 & 30 \\
0 & 0 & -2
\end{array}
\rightB
\end{equation*}
By Lemma \ref{lem:similarmatrices}, the resulting matrix has the same eigenvalues as $A$ where here, the matrix $E \left(2,2\right)$ plays the role of $P$.

We do this step again, as follows. In this step, we use the elementary matrix obtained by adding $-3$
times the second row to the first row. 
\begin{equation}
\leftB
\begin{array}{rrr}
1 & -3 & 0 \\
0 &  1 & 0 \\
0 &  0 & 1
\end{array}
\rightB \leftB
\begin{array}{rrr}
33 & -105 & 105 \\
10 & -32  & 30 \\
0  &   0  & -2
\end{array} 
\rightB \leftB
\begin{array}{rrr}
1 & 3 & 0 \\
0 & 1 & 0 \\
0 & 0 & 1
\end{array}
\rightB =\leftB
\begin{array}{rrr}
3  & 0  & 15 \\
10 & -2 & 30 \\
0  & 0  & -2
\end{array}
\rightB  \label{elemeigenvalue}
\end{equation}
Again by Lemma \ref{lem:similarmatrices}, this resulting matrix has the same eigenvalues as $A$. 
At this point, we can easily find the eigenvalues.
Let 
\begin{equation*}
B = \leftB
\begin{array}{rrr}
3  & 0  & 15 \\
10 & -2 & 30 \\
0  & 0  & -2
\end{array}
\rightB 
\end{equation*}
Then, we find the eigenvalues of $B$ (and therefore of $A$) by solving the equation 
$\det \left( \eigenVar I - B  \right) = 0$.
You should verify that this equation becomes
\begin{equation*}
\left(\eigenVar  +2 \right) \left( \eigenVar  +2 \right) \left( \eigenVar  - 3 \right)
=0
\end{equation*}
Solving this equation results in eigenvalues of $\lambda_1 = -2, \lambda_2 = -2$, and $\lambda_3 = 3$.
Therefore, these are also the eigenvalues of $A$. 

\end{solution}

Through using elementary matrices, we were able to create a matrix for
which finding the eigenvalues was easier than for $A$. At this point,
you could go back to the original matrix $A$ and solve $\left(
\lambda I - A \right) X = 0$ to obtain the eigenvectors of $A$.

Notice that when you multiply on the right by an elementary matrix,
you are doing the column operation defined by the elementary
matrix. In \ref{elemeigenvalue} multiplication by the elementary matrix on
the right merely involves taking three times the first column and
adding to the second. Thus, without referring to the elementary
matrices, the transition to the new matrix in \ref{elemeigenvalue} can be
illustrated by
\begin{equation*}
\leftB
\begin{array}{rrr}
33 & -105 & 105 \\
10 & -32 & 30 \\
0 & 0 & -2
\end{array}
\rightB \rightarrow \leftB
\begin{array}{rrr}
3 & -9 & 15 \\
10 & -32 & 30 \\
0 & 0 & -2
\end{array}
\rightB \rightarrow \leftB
\begin{array}{rrr}
3 & 0 & 15 \\
10 & -2 & 30 \\
0 & 0 & -2
\end{array}
\rightB
\end{equation*}

The third special type of matrix we will consider in this section is
the triangular matrix.  Recall Definition \ref{def:triangularmatrices}
which states that an upper (lower) triangular matrix contains all
zeros below (above) the main diagonal. Remember that finding the
determinant of a triangular matrix is a simple procedure of taking the product of the entries on the main diagonal.. It turns out
that there is also a simple way to find the eigenvalues of a
triangular matrix.

In the next example we will demonstrate that the eigenvalues of a 
triangular matrix are the entries on the main diagonal. 

\begin{example}{Eigenvalues for a triangular matrix}{eigenvaluestriangularmatrix}
Let $A=\leftB
\begin{array}{rrr}
1 & 2 & 4 \\
0 & 4 & 7 \\
0 & 0 & 6
\end{array}
\rightB .$ Find the eigenvalues of $A$.
\end{example}

\begin{solution}
We need to solve the equation $\det \left( \eigenVar I - A \right) = 0$ as follows
\begin{eqnarray*}
\det \left( \eigenVar I - A \right) =
\det \leftB
\begin{array}{ccc}
\eigenVar -1 & -2 & -4 \\
0 & \eigenVar-4 & -7 \\
0 & 0 & \eigenVar-6
\end{array}
\rightB =\left( \eigenVar-1 \right) \left( \eigenVar-4 \right) \left( \eigenVar-6 \right) =0
\end{eqnarray*}

Solving the equation $\left( \eigenVar-1 \right) \left( \eigenVar-4
\right) \left( \eigenVar-6 \right) = 0$ for $\eigenVar$ results in the eigenvalues 
$\lambda_1 = 1, \lambda_2 = 4$ and $\lambda_3 = 6$.  Thus the
eigenvalues are the entries on the main diagonal of the original
matrix.
\end{solution}

The same result is true for lower triangular matrices. For any triangular matrix,
the eigenvalues are equal to the entries on the main diagonal. To find the 
eigenvectors of a triangular matrix, we use the usual procedure. 

In the next section, we explore an important process involving the eigenvalues and eigenvectors of a matrix. 