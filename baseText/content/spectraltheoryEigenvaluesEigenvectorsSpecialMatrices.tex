\section{Eigenvalues of triangular matrices}

Recall from Definition~\ref{def:triangular-matrices} that a matrix is
\textbf{upper triangular} if all entries below the main diagonal zero
zero, and \textbf{lower triangular} if all entries above the main
diagonal are zero. 

\begin{example}{Eigenvalues of a triangular matrix}{eigenvalues-triangular-matrix}
  Find the eigenvalues of
  \begin{equation*}
    A=\begin{mymatrix}{rrr}
      1 & 2 & 4 \\
      0 & 4 & 7 \\
      0 & 0 & 6 \\
    \end{mymatrix}.
  \end{equation*}
\end{example}

\begin{solution}
  We calculate $\det(A - \eigenVar I) = 0$ as follows:
  \begin{eqnarray*}
    \det (\eigenVar I - A) =
    \det \begin{mymatrix}{ccc}
      1-\eigenVar & 2 & 4 \\
      0 & 4-\eigenVar & 7 \\
      0 & 0 & 6-\eigenVar \\
    \end{mymatrix} = (1-\eigenVar)(4-\eigenVar)(6-\eigenVar).
  \end{eqnarray*}
  Solving the equation $(1-\eigenVar)(4-\eigenVar)(6-\eigenVar) = 0$
  results in the eigenvalues $\eigenVar_1 = 1$, $\eigenVar_2 = 4$, and
  $\eigenVar_3 = 6$.  Thus the eigenvalues are the entries on the main
  diagonal of $A$.
\end{solution}

Clearly, the same is true for any (upper or lower) triangular
matrix. We therefore have the following proposition:

\begin{proposition}{Eigenvalues of a triangular matrix}{eigenvalues-triangular-matrix}
  Let $A$ be an upper or lower triangular matrix. Then the eigenvalues
  of $A$ are the entries on the main diagonal.
\end{proposition}
