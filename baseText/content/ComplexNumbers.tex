\chapter{Complex numbers}
\label{app:complex}

Throughout history, mankind has invented more and more complicated
number systems in an effort to make algebra easier.
\begin{itemize}
\item In the beginning, there were the \textbf{natural numbers}%
  \index{natural number}%
  \index{number!natural} $\N = \set{1, 2, 3, \ldots}$. However,
  after a while, it became a problem that certain equations, such as
  $x+5=3$, do not have a solution in the natural numbers.
\item To solve this problem, {\em zero} and {\em negative numbers}
  were invented, resulting in the set of \textbf{integers}%
  \index{integer}%
  \index{number!integer} $\Z=\set{\ldots,-3,-2,-1,0,1,2,3,\ldots}$. In
  the integers, the equation $x+5=3$ has a solution, namely
  $x=-2$. But some other equations, such as $2x=1$, still do not have
  a solution in the integers.
\item To solve this problem, the \textbf{rational numbers}%
  \index{rational number}%
  \index{number!rational} $\Q$ were invented. In the rational
  numbers, the equation $2x=1$ has a solution, namely
  $x=\frac{1}{2}$. However, some other equations, such as $x^2=2$,
  still do not have a solution in the rational numbers.
\item To solve this problem, the \textbf{real numbers}%
  \index{real number}%
  \index{number!real} $\R$ were invented. In the real numbers,
  the equation $x^2=2$ has a solution, namely $x=\sqrt{2}$. However,
  some other equations, such as $x^2=-1$, still don't have a solution
  in the real numbers.
\item To solve this problem, the \textbf{complex numbers}%
  \index{complex number}%
  \index{number!complex|see{complex number}} were invented.
\end{itemize}
The purpose of this section is to summarize the most important facts
about the complex numbers. We will also see that the above process
does not continue. In the complex numbers, all non-trivial polynomial
equations have a solution, and therefore, no additional numbers are
``missing''.  This property is called the \textbf{fundamental theorem
  of algebra}%
\index{fundamental theorem of algebra}%
\index{algebra!fundamental theorem of}. Gauss is usually credited with
giving a proof of this theorem in 1797 but many others worked on it
and the first completely correct proof was due to Argand in 1806.

