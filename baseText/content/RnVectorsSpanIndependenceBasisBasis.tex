% ----------------------------------------------------------------------
\section{Basis and dimension}

% ----------------------------------------------------------------------
\subsection{Definition of basis}

We saw in Proposition~\ref{prop:span-subspace} that spans are
subspaces of\/ $\R^n$. Interestingly, the converse is also true: every
subspace of\/ $\R^n$ is the span of some finite set of vectors.

\begin{theorem}{Subspaces are spans}{subspaces-are-spans}
  \index{subspace!is a span}%
  Let $V$ be a subspace of\/ $\R^n$. Then there exist linearly
  independent vectors $\set{\vect{u}_{1},\ldots,\vect{u}_{k}}$ in $V$
  such that
  \begin{equation*}
    V= \sspan\set{\vect{u}_{1},\ldots ,\vect{u}_{k}}. 
  \end{equation*}
  \vspace{-3ex}
\end{theorem}

\begin{proof}
  We proceed as follows.
  \begin{enumerate}
  \item[0.] If $V=\set{\vect{0}}$, then $V$ is the empty span, and we
    are done.
  \item[1.] Otherwise, $V$ contains some non-zero vector.  Pick a
    non-zero vector $\vect{u}_{1}$ in $V$. If
    $V=\sspan\set{\vect{u}_{1}}$, we are done.
  \item[2.] Otherwise, pick a vector $\vect{u}_{2}$ in $V$ that is not
    in $\sspan\set{\vect{u}_{1}}$. If
    $V=\sspan\set{\vect{u}_{1},\vect{u}_{2}}$, we are done.
  \item[3.] Otherwise, pick a vector $\vect{u}_{3}$ in $V$ that is not
    in $\sspan\set{\vect{u}_{1},\vect{u}_{2}}$. If
    $V=\sspan\set{\vect{u}_{1},\vect{u}_{2},\vect{u}_{3}}$, we are done.
  \item[4.] Otherwise, pick a vector $\vect{u}_{4}$ in $V$ that is not
    in $\sspan\set{\vect{u}_{1},\vect{u}_{2},\vect{u}_{4}}$, and so on.
  \end{enumerate}
  Continue in this way. Note that after the $j\th$ step of this
  process, the vectors $\vect{u}_1,\ldots,\vect{u}_j$ are linearly
  independent. This is because, by construction, no vector is in the
  span of the previous vectors, and therefore no vector is redundant.
  By
  Theorem~\ref{thm:properties-linear-independence}(\ref{properties-linear-independence-c}),
  there can be at most $n$ linearly independent vectors in $\R^n$.
  Therefore the process must stop after $k$ steps for some $k\leq
  n$. But then $V=\sspan\set{\vect{u}_{1},\ldots ,\vect{u}_{k}}$, as
  desired.
\end{proof}

In summary, every subspace of\/ $\R^{n}$ is spanned by a finite,
linearly independent collection of vectors.  Such a collection of
vectors is called a \textbf{basis} of the subspace.

\begin{definition}{Basis of a subspace}{subspace-basis}
  Let $V$ be a subspace of\/ $\R^{n}$. Then
  $\set{\vect{u}_{1},\ldots ,\vect{u}_{k}}$ is a \textbf{basis} for
  $V$ if the following two conditions hold:%
  \index{basis}%
  \index{subspace!basis|see{basis}}%
  \index{vector!basis|see{basis}}%
  \begin{enumerate}
  \item $\sspan\set{\vect{u}_{1},\ldots,\vect{u}_{k}}=V$, and
  \item $\vect{u}_{1},\ldots,\vect{u}_{k}$ are linearly independent.
  \end{enumerate}
\end{definition}

Note that the plural of basis is \textbf{bases}.

% ----------------------------------------------------------------------
\subsection{Examples of bases}

\begin{proposition}{Standard basis of\/ $\R^n$}{standard-basis}
  Let $\vect{e}_i$ be the vector in $\R^n$ whose $i\th$ entry is $1$
  and all of whose other entries are $0$. In other words, $\vect{e}_i$
  is the $i\th$ column of the identity matrix.
  \begin{equation*}
    \vect{e}_1 = \begin{mymatrix}{c} 1 \\ 0 \\ 0 \\ \vdots \\ 0 \end{mymatrix},\quad
    \vect{e}_2 = \begin{mymatrix}{c} 0 \\ 1 \\ 0 \\ \vdots \\ 0 \end{mymatrix},\quad
    \vect{e}_3 = \begin{mymatrix}{c} 0 \\ 0 \\ 1 \\ \vdots \\ 0 \end{mymatrix},\quad
    \ldots~,\quad
    \vect{e}_n = \begin{mymatrix}{c} 0 \\ 0 \\ 0 \\ \vdots \\ 1\end{mymatrix}.
  \end{equation*}
  Then $\set{\vect{e}_1,\vect{e}_2,\ldots,\vect{e}_n}$ is a basis for
  $\R^n$. It is called the \textbf{standard basis}%
  \index{standard basis}%
  \index{basis!standard}
  of\/ $\R^n$.
\end{proposition}

\begin{proof}
  To see that it is a basis of\/ $\R^n$, first notice that the vectors
  $\vect{e}_1,\vect{e}_2,\ldots,\vect{e}_n$ span $\R^n$. Indeed, every
  vector $\vect{v} = \mat{x_1,\ldots,x_n}^T\in\R^n$ can be written as
  $\vect{v} = x_1\vect{e}_1+\ldots+x_n\vect{e}_n$. Second, the vectors
  $\vect{e}_1,\vect{e}_2,\ldots,\vect{e}_n$ are evidently linearly
  independent, because none of these vectors can be written as a
  linear combination of previous vectors. Since the vectors span
  $\R^n$ and are linearly independent, they form a basis of\/ $\R^n$.
\end{proof}

\begin{example}{A non-standard basis of\/ $\R^3$}{non-standard-basis}
  \index{basis!of Rn@of\/ $\R^n$}%
  Check that the vectors
  \begin{equation*}
    \vect{u}_1 = \begin{mymatrix}{r} 1 \\ 2 \\ 1 \end{mymatrix},\quad
    \vect{u}_2 = \begin{mymatrix}{r} 0 \\ 1 \\ 0 \end{mymatrix},\quad
    \mbox{and}\quad
    \vect{u}_3 = \begin{mymatrix}{r} -1 \\ 0 \\ 1 \end{mymatrix}
  \end{equation*}
  form a basis of\/ $\R^3$.
\end{example}

\begin{solution}
  We must check that the vectors $\vect{u}_1,\vect{u}_2,\vect{u}_3$
  are linearly independent and span $\R^3$. To check linear
  independence, we use the casting-out algorithm.
  \begin{equation*}
    \begin{mymatrix}{rrr}
      1 & 0 & -1 \\
      2 & 1 & 0 \\
      1 & 0 & 1 \\
    \end{mymatrix}
    \quad\sim\quad
    \begin{mymatrix}{rrr}
      \circled{1} & 0 & -1 \\
      0 & \circled{1} & 2 \\
      0 & 0 & \circled{2} \\
    \end{mymatrix}.
  \end{equation*}
  Since all columns are pivot columns, there are no redundant vectors,
  so $\vect{u}_1,\vect{u}_2,\vect{u}_3$ are linearly independent.
  To check that they span all of\/ $\R^3$, let $\vect{w}=\mat{x,y,z}^T$
  be an arbitrary element of\/ $\R^3$. We must show that $\vect{w}$ is a linear
  combination of $\vect{u}_1,\vect{u}_2,\vect{u}_3$. This amounts to
  solving the system of equations
  \begin{equation*}
    a_1\,\vect{u}_1+a_2\,\vect{u}_2+a_3\,\vect{u}_3 = \vect{w},
  \end{equation*}
  or in augmented matrix form,
  \begin{equation*}
    \begin{mymatrix}{rrr|c}
      1 & 0 & -1 & x \\
      2 & 1 & 0  & y \\
      1 & 0 & 1  & z \\
    \end{mymatrix}
    \quad\sim\quad
    \begin{mymatrix}{rrr|c}
      1 & 0 & -1 & x    \\
      0 & 1 & 2  & y-2x \\
      0 & 0 & 2  & z-x  \\
    \end{mymatrix}.
  \end{equation*}
  The system is clearly consistent, so it has a solution, and
  therefore $\vect{w}$ is indeed a linear combination of
  $\vect{u}_1,\vect{u}_2,\vect{u}_3$. Since $\vect{w}$ was an
  arbitrary vector of\/ $\R^3$, it follows that
  $\vect{u}_1,\vect{u}_2,\vect{u}_3$ span $\R^3$.
\end{solution}

Generalizing the last example, we find that a set of $n$ vectors forms
a basis of\/ $\R^n$ if and only if the matrix having those vectors as
its columns is invertible. This is the content of the following
proposition.

\begin{proposition}{Invertible matrices and bases of\/ $\R^n$}{invertible-matrices}
  \index{basis!of Rn@of\/ $\R^n$}%
  \index{basis!and invertible matrix}%
  Let $A$ be an $n\times n$-matrix. Then the columns of $A$ form a
  basis of\/ $\R^n$ if and only if $A$ is invertible.
\end{proposition}

We turn to the question of finding bases for subspaces of\/ $\R^n$.

\begin{example}{Basis of a span}{basis-of-span}
  Let
  \begin{equation*}
    \vect{u}_1 = \begin{mymatrix}{r} 2 \\ 0 \\ -2 \end{mymatrix},
    \quad
    \vect{u}_2 = \begin{mymatrix}{r} -1 \\ 0 \\ 1 \end{mymatrix},
    \quad
    \vect{u}_3 = \begin{mymatrix}{r} 1 \\ 3 \\ 5 \end{mymatrix},
    \quad
    \vect{u}_4 = \begin{mymatrix}{r} 3 \\ 5 \\ 7 \end{mymatrix},
    \quad
    \vect{u}_5 = \begin{mymatrix}{r} -1 \\ 1 \\ 3 \end{mymatrix}.
  \end{equation*}
  Find a basis of $\sspan\set{\vect{u}_1,\ldots,\vect{u}_5}$%
  \index{basis!of a span}.
\end{example}

\begin{solution}
  Let $S=\sspan\set{\vect{u}_1,\ldots,\vect{u}_5}$.  By
  Theorem~\ref{thm:linearly-independent-subset}, we know that if we
  remove the redundant vectors from
  $\set{\vect{u}_1,\ldots,\vect{u}_5}$, then the remaining vectors
  will be linearly independent and will still span $S$. In other
  words, the remaining vectors will be a basis for $S$. We use the
  casting-out algorithm to identity the redundant vectors:
  \begin{equation*}
    \begin{mymatrix}{rrrrr}
      2 & -1 & 1 & 3 & -1 \\
      0 & 0 & 3 & 5 & 1 \\
      -2 & 1 & 5 & 7 & 3 \\
    \end{mymatrix}
    ~\sim~
    \begin{mymatrix}{rrrrr}
      2 & -1 & 1 & 3 & -1 \\
      0 & 0 & 3 & 5 & 1 \\
      0 & 0 & 6 & 10 & 2 \\
    \end{mymatrix}
    ~\sim~
    \begin{mymatrix}{rrrrr}
      \circled{2} & -1 & 1 & 3 & -1 \\
      0 & 0 & \circled{3} & 5 & 1 \\
      0 & 0 & 0 & 0 & 0 \\
    \end{mymatrix}.
  \end{equation*}
  Since columns $2$, $4$, and $5$ are the non-pivot columns, it
  follows that the vectors $\vect{u}_2$, $\vect{u}_4$, and
  $\vect{u}_5$ are redundant. Therefore, the desired basis is
  $\set{\vect{u}_1,\vect{u}_3}$. 
\end{solution}

\begin{example}{Basis of the solution space of a homogeneous system of equations}{basis-solution-space}
  Find a basis for the solution space of the system of equations%
  \index{basis!of a solution space}
  \begin{equation*}
    \begin{array}{r@{~}r@{~}r@{~}r@{~}r@{~}r@{~}r@{~}r@{~}r@{~}r@{~}r}
     x &+& y  &-& z &+& 3w &-& 2v &=& 0, \\
     x &+& y &+& z &-& 11w &+& 8v &=& 0, \\
    4x &+& 4y &-& 3z &+& 5w &-& 3v &=& 0. \\
    \end{array}
  \end{equation*}
\end{example}

\begin{solution}
  We solve the system of equations in the usual way:
  \begin{equation*}
    \begin{mymatrix}{rrrrr|r}
      1 & 1 & -1 & 3 & -2 & 0 \\
      1 & 1 &  1 & -11 & 8 & 0 \\
      4 & 4 & -3 & 5 & -3 & 0 \\
    \end{mymatrix}
    ~\sim~
    \begin{mymatrix}{rrrrr|r}
      1 & 1 & -1 & 3 & -2 & 0 \\
      0 & 0 & 2 & -14 & 10 & 0 \\
      0 & 0 & 1 & -7 & 5 & 0 \\
    \end{mymatrix}
    ~\sim~
    \begin{mymatrix}{rrrrr|r}
      \circled{1} & 1 & 0 & -4 & 3 & 0 \\
      0 & 0 & \circled{1} & -7 & 5 & 0 \\
      0 & 0 & 0 &  0 & 0 & 0 \\
    \end{mymatrix}.
  \end{equation*}
  From the {\rref}, we see that $y$, $w$, and $v$ are free
  variables. The general solution is:
  \begin{equation*}
    \begin{mymatrix}{c} x \\ y \\ z \\ w \\ v \end{mymatrix}
    \quad=\quad
    t \begin{mymatrix}{r} -3 \\ 0 \\ -5 \\ 0 \\ 1 \end{mymatrix}
    ~+~ s \begin{mymatrix}{r} 4 \\ 0 \\ 7 \\ 1 \\ 0 \end{mymatrix}
    ~+~ r \begin{mymatrix}{r} -1 \\ 1 \\ 0 \\ 0 \\ 0 \end{mymatrix}.
  \end{equation*}
  Thus, the solution space is spanned by the vectors
  \begin{equation*}
    \set{\begin{mymatrix}{r} -3 \\ 0 \\ -5 \\ 0 \\ 1 \end{mymatrix},
    \begin{mymatrix}{r} 4 \\ 0 \\ 7 \\ 1 \\ 0 \end{mymatrix},
    \begin{mymatrix}{r} -1 \\ 1 \\ 0 \\ 0 \\ 0 \end{mymatrix}
    }.
  \end{equation*}
  Moreover, these vectors are evidently linearly independent, because
  each vector contains a $1$ in a position where all the previous
  vectors have $0$ (and therefore, none of the vectors can be written
  as a linear combination of previous vectors). It follows that the
  above three vectors form a basis of the solution space.
\end{solution}

Note that the basis vectors of the solution space are exactly what we
called the \textbf{basic solutions}%
\index{basic solution}%
\index{solution!basic}%
\index{solution!basis of}
in Section~\ref{sec:homogeneous-systems}. 

% ----------------------------------------------------------------------
\subsection{Bases and coordinate systems}

Let $V$ be a subspace of\/ $\R^n$. A basis of $V$ is essentially the
same thing as a coordinate system for $V$. To see why, let
$B=\set{\vect{u}_1,\ldots,\vect{u}_k}$ be some basis of $V$. This
means that the vectors $\vect{u}_1,\ldots,\vect{u}_k$ are linearly
independent and span $V$. Because the basis vectors are spanning,
every vector $\vect{v}\in V$ can be written as a linear combination of
basis vectors
\begin{equation*}
  \vect{v} = a_1\,\vect{u}_1 + \ldots + a_k\,\vect{u}_k.
\end{equation*}
Moreover, because the basis vectors are linearly independent, it
follows by Theorem~\ref{thm:unique-linear-combination} that the
coefficients $a_1,\ldots,a_k$ are unique. We say that $a_1,\ldots,a_k$
are the \textbf{coordinates of $\vect{v}$ with respect to the basis
$B$}%
\index{coordinate!with respect to basis}%
\index{coordinate system!and basis}, and we write
\begin{equation*}
  \coord{\vect{v}}_B = \begin{mymatrix}{c} a_1 \\ a_2 \\ a_3 \end{mymatrix}.
\end{equation*}
\begin{center}
  \begin{tikzpicture}
    \begin{scope}[scale=2.5,x={(1.2cm,-0.2cm)},y={(0.5cm,0.4cm)},z={(0cm,0.8cm)}]
      \draw(-2.5,0,0) -- (2.5,0,0);
      \draw(0,-2.3,0) -- (0,2.3,0);
      \draw(0,0,-1.3) -- (0,0,1.3);
      \draw[->, thick, blue](0,0,0) -- node[below] {$\vect{u}_1$} +(1,0,0);
      \draw[->, thick, blue](0,0,0) -- node[above left=-1ex] {$\vect{u}_2$} +(0,1,0);
      \draw[->, thick, blue](0,0,0) -- node[left] {$\vect{u}_3$} +(0,0,1);
      \draw(-2,0,0) -- +(0,0,-0.1) node[below] {$-2$};
      \draw(-1,0,0) -- +(0,0,-0.1) node[below] {$-1$};
      \draw(0,0,0) -- +(0,0,-0.1) node[below] {$0$};
      \draw(1,0,0) -- +(0,0,-0.1) node[below] {$1$};
      \draw(2,0,0) -- +(0,0,-0.1) node[below] {$2$};
      \draw(0,-2,0) -- +(0,0,-0.1) node[below] {$-2$};
      \draw(0,-1,0) -- +(0,0,-0.1) node[below] {$-1$};
      \draw(0,1,0) -- +(0,0,-0.1) node[below] {$1$};
      \draw(0,2,0) -- +(0,0,-0.1) node[below] {$2$};
      \draw(0,0,-1) -- +(-0.1,0,0) node[left] {$-1$};
      \draw(0,0,1) -- +(-0.1,0,0) node[left] {$1$};
    \end{scope}
    \path(0,-3.2) node {Basis as coordinate system};
  \end{tikzpicture}
\end{center}

\begin{example}{Find a vector from its coordinates in a basis}{vector-from-coordinates}
  Find the vector $\vect{v}$ that has coordinates
  \begin{equation*}
    \coord{\vect{v}}_B = \begin{mymatrix}{r} 1 \\ -1 \\ 2 \end{mymatrix}
  \end{equation*}
  with respect to the basis $B=\set{\vect{u}_1,\vect{u}_2,\vect{u}_3}$
  of\/ $\R^3$, where
  \begin{equation*}
    \vect{u}_1 = \begin{mymatrix}{r} 1 \\ 2 \\ 1 \end{mymatrix},\quad
    \vect{u}_2 = \begin{mymatrix}{r} 0 \\ 1 \\ 0 \end{mymatrix},\quad
    \mbox{and}\quad
    \vect{u}_3 = \begin{mymatrix}{r} -1 \\ 0 \\ 1\end{mymatrix}.
  \end{equation*}
\end{example}

\begin{solution}
  This simply means that $\vect{v} = 1\vect{u}_1 - 1\vect{u}_2 +
  2\vect{u}_3$. We calculate
  \begin{equation*}
    \vect{v} =
    1\begin{mymatrix}{r} 1 \\ 2 \\ 1 \end{mymatrix}
    - 1\begin{mymatrix}{r} 0 \\ 1 \\ 0 \end{mymatrix}
    + 2\begin{mymatrix}{r} -1 \\ 0 \\ 1 \end{mymatrix}
    = \begin{mymatrix}{r} -1 \\ 1 \\ 3  \end{mymatrix}.
  \end{equation*}
\end{solution}

In case the basis is the standard basis, the coordinates are just the
usual ones, as the following example illustrates:

\begin{example}{Find a vector from its coordinates in the standard basis}{vector-from-coordinates-standard}
  Find the vector $\vect{v}$ that has coordinates
  \begin{equation*}
    \coord{\vect{v}}_B = \begin{mymatrix}{r} 1 \\ -1 \\ 2 \end{mymatrix},
  \end{equation*}
  where $B$ is the standard basis of\/ $\R^3$.
\end{example}

\begin{solution}
  The standard basis is
  \begin{equation*}
    \vect{e}_1 = \begin{mymatrix}{r} 1 \\ 0 \\ 0 \end{mymatrix},\quad
    \vect{e}_2 = \begin{mymatrix}{r} 0 \\ 1 \\ 0 \end{mymatrix},\quad
    \mbox{and}\quad
    \vect{e}_3 = \begin{mymatrix}{r} 0 \\ 0 \\ 1\end{mymatrix},
  \end{equation*}
  and in this case, we simply have
  \begin{equation*}
    \vect{v} = 1\vect{e}_1 - 1\vect{e}_2 + 2\vect{e}_3
        = \begin{mymatrix}{r} 1 \\ -1 \\ 2  \end{mymatrix}.
  \end{equation*}
\end{solution}

We can also ask to find the coordinates of a given vector in a given
basis.

\begin{example}{Find the coordinates of a vector with respect to a basis}{coordinates-from-vector}
  Find the coordinates of the vector
  \begin{equation*}
    \vect{v} = \begin{mymatrix}{r} 1 \\ 2 \\ 3 \end{mymatrix}
  \end{equation*}
  with respect to the basis $B=\set{\vect{u}_1,\vect{u}_2,\vect{u}_3}$
  of\/ $\R^3$, where
  \begin{equation*}
    \vect{u}_1 = \begin{mymatrix}{r} 1 \\ 2 \\ 1 \end{mymatrix},\quad
    \vect{u}_2 = \begin{mymatrix}{r} 0 \\ 1 \\ 0 \end{mymatrix},\quad
    \mbox{and}\quad
    \vect{u}_3 = \begin{mymatrix}{r} -1 \\ 0 \\ 1\end{mymatrix}.
  \end{equation*}
\end{example}

\begin{solution}
  To find the coordinates, we must solve the system of equations
  $\vect{v} = a_1\,\vect{u}_1+a_2\,\vect{u}_2+a_3\,\vect{u}_3$. We
  solve:
  \begin{equation*}
    \begin{mymatrix}{rrr|r}
      1 & 0 & -1 & 1 \\
      2 & 1 & 0 & 2 \\
      1 & 0 & 1 & 3 \\
    \end{mymatrix}
    \quad\sim\quad
    \begin{mymatrix}{rrr|r}
      1 & 0 & -1 & 1 \\
      0 & 1 & 2 & 0 \\
      0 & 0 & 2 & 2 \\
    \end{mymatrix}
    \quad\sim\quad
    \begin{mymatrix}{rrr|r}
      1 & 0 & 0 & 2 \\
      0 & 1 & 0 & -2 \\
      0 & 0 & 1 & 1 \\
    \end{mymatrix}.
  \end{equation*}
  Therefore, the unique solution is $(a_1,a_2,a_3) = (2,-2,1)$. The
  coordinates of $\vect{v}$ with respect to the basis $B$ are
  \begin{equation*}
    \coord{\vect{v}}_B
    = \begin{mymatrix}{r} a_1 \\ a_2 \\ a_3 \end{mymatrix}
    = \begin{mymatrix}{r} 2 \\ -2 \\ 1 \end{mymatrix}.
  \end{equation*}
\end{solution}

% ----------------------------------------------------------------------
\subsection{Dimension}

One of the most important properties of bases is that any two bases
for the same space must be of the same size. To show this, we will
need the the following fundamental result, called the Exchange
Lemma. This theorem states that spanning sets have at least as many
vectors as linearly independent sets.

\begin{lemma}{Exchange Lemma}{exchange-lemma}
  \index{exchange lemma}%
  Suppose $\vect{u}_{1},\ldots,\vect{u}_{r}$ are linearly independent
  elements of $\sspan\set{\vect{v}_{1},\ldots,\vect{v}_{s}}$. Then
  $r\leq s$.
\end{lemma}

\begin{proof}
  Since each $\vect{u}_j$ is an element of
  $\sspan\set{\vect{v}_{1},\ldots,\vect{v}_{s}}$, there exist
  scalars $a_{ij}$ such that
  \begin{equation*}
    \vect{u}_{j} = a_{1j}\,\vect{v}_1 + \ldots + a_{sj}\,\vect{v}_s.
  \end{equation*}
  Let $A = \mat{a_{ij}}$. Note that this matrix has $s$ rows and $r$
  columns, i.e., it is an $s\times r$-matrix. Now suppose, for the
  sake of obtaining a contradiction, that $r>s$. Then by
  Theorem~\ref{thm:rank-homogeneous-solutions}, the system
  $A\vect{x}=\vect{0}$ has a non-trivial solution $\vect{x}$, i.e.,
  there exists $\vect{x}\neq\vect{0}$ such that $A\vect{x}=\vect{0}$.
  In other words, for all $i=1,\ldots,s$, 
  \begin{equation*}
    a_{i1}x_1 + \ldots + a_{ir}x_r = 0.
  \end{equation*}
  Therefore,
  \begin{eqnarray*}
    x_1\,\vect{u}_1 + \ldots + x_r\,\vect{u}_r
    &=&
        x_1(a_{11}\,\vect{v}_1 + \ldots + a_{s1}\,\vect{v}_s)
        + \ldots
        + x_r(a_{1r}\,\vect{v}_1 + \ldots + a_{sr}\,\vect{v}_s)
    \\
    &=&
        (a_{11}x_1 + \ldots + a_{1r}x_r)\vect{v}_1
        + \ldots
        + (a_{s1}x_1 + \ldots + a_{sr}x_r)\vect{v}_s
    \\
    &=& 0\,\vect{v}_1 + \ldots + 0\,\vect{v}_s
    \\
    &=& 0.
  \end{eqnarray*}
  This contradicts the assumption that
  $\vect{u}_{1},\ldots,\vect{u}_{r}$ are linearly independent. Since
  we assumed $r>s$ and obtained a contradiction, it follows that
  $r\leq s$, as desired.
\end{proof}

Armed with the Exchange Lemma, we are now ready to show that any two
bases of a space are of the same size.

\begin{theorem}{Bases are of the same size}{bases-same-size}
  \index{basis!size of}%
  Let $V$ be a subspace of\/ $\R^{n}$, and let $B_1$ and $B_2$ be
  bases of $V$. Suppose $B_1$ contains $s$ vectors and $B_2$ contains
  $r$ vectors. Then $s=r$.
\end{theorem}

\begin{proof}
  This follows right away from the Exchange Lemma. Indeed, observe
  that $B_1 = \set{ \vect{u}_{1},\ldots ,\vect{u}_{s}} $ is a spanning
  set for $V$ while $ B_2 = \set{\vect{v}_{1},\ldots ,\vect{v}_{r}} $
  is linearly independent, so $s \geq r$. Similarly
  $B_2 = \set{\vect{v}_{1},\ldots ,\vect{v} _{r}} $ is a spanning set
  for $V$ while $B_1 = \set{\vect{u}_{1},\ldots , \vect{u}_{s}} $ is
  linearly independent, so $r\geq s$.
\end{proof}

Because every basis of $V$ has the same number of vectors, we give
this number a special name. It is called the \textbf{dimension} of
$V$.

\begin{definition}{Dimension of a subspace}{dimension}
  Let $V$ be a subspace of\/ $\R^{n}$. Then the \textbf{dimension}%
  \index{dimension}%
  \index{subspace!dimension} of $V$, written $\dim(V)$, is
  defined to be the number of vectors in a basis.
\end{definition}

\begin{example}{Dimension of\/ $\R^n$}{dimension-Rn}
  What is the dimension of $\R^{n}$?
\end{example}

\begin{solution}
  The standard basis of $\R^{n}$ is
  $\set{\vect{e}_{1},\ldots,\vect{e}_{n}}$. Since it has $n$ vectors,
  so $\dim(\R^n) = n$.
\end{solution}

\begin{example}{Dimension of a subspace}{dimension-subspace}
  \index{dimension!of solution space}%
  \index{homogeneous system!dimension of solution space}%
  \index{system of linear equations!homogeneous!dimension of solution space}%
  Let 
  \begin{equation*}
    V=\set{\left.
      \begin{mymatrix}{c} x\\ y\\ z\end{mymatrix}\in\R^3 ~\right\vert~
      x-y+2z = 0
    }.
  \end{equation*}
  What is the dimension of\/ $V$?
\end{example}

\begin{solution}
  We know that $V$ is a subspace of $\R^3$, because it is the solution
  space of a system of a homogeneous system of equations (in this
  case, one equation in three variables). We can take $y=t$ and $z=s$
  as the free variables and solve for $x=y-2z=t-2s$. Therefore, a
  general element of $V$ is of the form
  \begin{equation*}
    \begin{mymatrix}{c} x\\ y\\ z\end{mymatrix}
    = \begin{mymatrix}{c} t-2s \\ t \\ s \end{mymatrix}
    = t \begin{mymatrix}{c} 1 \\ 1 \\ 0 \end{mymatrix}
    + s \begin{mymatrix}{c} -2 \\ 0 \\ 1 \end{mymatrix}.
  \end{equation*}
  Thus,
  \begin{equation*}
    V = \sspan\set{
      \begin{mymatrix}{c} 1 \\ 1 \\ 0 \end{mymatrix},~
      \begin{mymatrix}{c} -2 \\ 0 \\ 1 \end{mymatrix}
    }.
  \end{equation*}
  Since the two spanning vectors are linearly independent, they form a
  basis of $V$, and thus $\dim(V)=2$. 
\end{solution}

Note that the dimension of the solution space of a system of equations
is equal to the number of parameters in the general solution, which is
equal to the number of free variables. For this reason, the dimension
is also sometimes called the number of \textbf{degrees of freedom}%
\index{degree of freedom}%
\index{freedom!degree of}.

\begin{example}{Dimension of a span}{dimension-basis}
  \index{dimension!of a span}%
  \index{span!dimension of}%
  Let 
  \begin{equation*}
    W = \sspan\set{
      \begin{mymatrix}{r} 1 \\ 2 \\ -1 \\ 1 \end{mymatrix},
      \begin{mymatrix}{r} 1 \\ 3 \\ -1 \\ 1 \end{mymatrix},
      \begin{mymatrix}{r} 8 \\ 19 \\ -8 \\ 8 \end{mymatrix},
      \begin{mymatrix}{r} -6 \\ -15 \\ 6 \\ -6 \end{mymatrix},
      \begin{mymatrix}{r} 1 \\ 3 \\ 0 \\ 1 \end{mymatrix},
      \begin{mymatrix}{r} 1 \\ 5 \\ 0 \\ 1 \end{mymatrix}
    }.
  \end{equation*}
  What is the dimension of $W$?
\end{example}

\begin{solution}
  Let
  \begin{equation*}
    \vect{u}_1 = \begin{mymatrix}{r} 1 \\ 2 \\ -1 \\ 1 \end{mymatrix},~~
    \vect{u}_2 = \begin{mymatrix}{r} 1 \\ 3 \\ -1 \\ 1 \end{mymatrix},~~
    \vect{u}_3 = \begin{mymatrix}{r} 8 \\ 19 \\ -8 \\ 8 \end{mymatrix},~~
    \vect{u}_4 = \begin{mymatrix}{r} -6 \\ -15 \\ 6 \\ -6 \end{mymatrix},~~
    \vect{u}_5 = \begin{mymatrix}{r} 1 \\ 3 \\ 0 \\ 1 \end{mymatrix},~~
    \vect{u}_6 = \begin{mymatrix}{r} 1 \\ 5 \\ 0 \\ 1 \end{mymatrix},
  \end{equation*}
  so that $W=\sspan\set{\vect{u}_1,\ldots,\vect{u}_6}$.  We use the
  casting-out algorithm to remove any redundant vectors from
  $\vect{u}_1,\ldots,\vect{u}_6$. The remaining vectors will be
  linearly independent, and therefore a basis of the span.
  \begin{equation*}
    \begin{mymatrix}{rrrrrr}
      1 & 1 & 8 & -6 & 1 & 1 \\ 
      2 & 3 & 19 & -15 & 3 & 5 \\ 
      -1 & -1 & -8 & 6 & 0 & 0 \\ 
      1 & 1 & 8 & -6 & 1 & 1
    \end{mymatrix}
    \quad\sim\quad
    \begin{mymatrix}{rrrrrr}
      \circled{1} & 0 & 5 & -3 & 0 & -2 \\ 
      0 & \circled{1} & 3 & -3 & 0 & 2 \\ 
      0 & 0 & 0 & 0 & \circled{1} & 1 \\ 
      0 & 0 & 0 & 0 & 0 & 0
    \end{mymatrix}.
  \end{equation*}
  Therefore, the vectors $\vect{u}_3$, $\vect{u}_4$, and $\vect{u}_6$
  are redundant, and $\set{\vect{u}_1,\vect{u}_2,\vect{u}_5}$ is a
  basis of $W$. It follows that $\dim(W)=3$. 
\end{solution}

% ----------------------------------------------------------------------
\subsection{More properties of bases}

We begin by noting that every subspace $V$ of $\R^n$ has a basis.

\begin{theorem}{Existence of bases}{subspaces-have-bases}
  \index{basis!existence}%
  Every subspace of $\R^n$ has a basis. 
\end{theorem}

\begin{proof}
  This is just a restatement of Theorem~\ref{thm:subspaces-are-spans}.
\end{proof}

Of course, the theorem does not mean that the basis is
unique. Usually, a subspace of $\R^n$ will have many different
bases. The theorem just states that there exists at least one.

Sometimes, when we are looking for a basis of a space, we may already
have a number of linearly independent vectors. We would like to obtain
a basis by adding some {\em additional} linearly independent vectors
to the ones we already have. The following lemma guarantees that this
can always be done.

\begin{lemma}{Linearly independent set can be extended to a basis}{extend-to-basis}
  \index{linear independence!extending to basis}%
  \index{basis!by extending linearly independent set}%
  Let $V$ be a subspace of $\R^n$, and let
  $\vect{u}_1,\ldots,\vect{u}_{\ell}$ be linearly independent elements
  of $V$. Then it is possible to extend
  $\set{\vect{u}_1,\ldots,\vect{u}_{\ell}}$ to a basis of $V$. In other
  words, there exist zero or more vectors
  $\vect{w}_1,\ldots,\vect{w}_s$ such that
  \begin{equation*}
    \set{\vect{u}_1,\ldots,\vect{u}_{\ell},\vect{w}_1,\ldots,\vect{w}_s}
  \end{equation*}
  is a basis of $V$.
\end{lemma}

\begin{proof}
  By Theorem~\ref{thm:subspaces-have-bases}, we know that $V$ has some
  basis, say $\set{\vect{v}_1,\ldots,\vect{v}_k}$. However, this may
  not be the basis we are looking for, because maybe it does not
  contain the vectors $\vect{u}_1,\ldots,\vect{u}_{\ell}$. Consider
  the sequence of $\ell+k$ vectors
  \begin{equation*}
    \vect{u}_1,\ldots,\vect{u}_{\ell},\vect{v}_1,\ldots,\vect{v}_k.
  \end{equation*}
  Since $V$ is spanned by the vectors $\vect{v}_1,\ldots,\vect{v}_k$,
  it is certainly also spanned by the larger set of vectors
  $\vect{u}_1,\ldots,\vect{u}_{\ell},\vect{v}_1,\ldots,\vect{v}_k$.
  From Theorem~\ref{thm:linearly-independent-subset}, we know that we
  can obtain a basis of $V$ by removing the redundant vectors from
  $\vect{u}_1,\ldots,\vect{u}_{\ell},\vect{v}_1,\ldots,\vect{v}_k$. On
  the other hand, $\vect{u}_1,\ldots,\vect{u}_{\ell}$ are linearly
  independent, so none of them can be redundant. It follows that the
  resulting basis of $V$ contains all of the vectors
  $\vect{u}_1,\ldots,\vect{u}_{\ell}$. In other words, we have found a
  basis of $V$ that is an extension of
  $\vect{u}_1,\ldots,\vect{u}_{\ell}$, which is what had to be shown.
\end{proof}

\begin{example}{Extending a linearly independent set to a basis}{extend-to-basis}
  Extend $\set{\vect{u}_1,\vect{u}_2}$ to a basis of $\R^4$, where
  \begin{equation*}
    \vect{u}_1 = \begin{mymatrix}{r} 1 \\ 1 \\ -1 \\ 2 \end{mymatrix},\quad
    \vect{u}_2 = \begin{mymatrix}{r} 1 \\ 2 \\ -2 \\ 4 \end{mymatrix}.
  \end{equation*}
\end{example}

\begin{solution}
  Let $\set{\vect{e}_1,\ldots,\vect{e}_4}$ be the standard basis of
  $\R^4$. We obtain the desired basis by applying the casting-out
  algorithm to $\vect{u}_1,\vect{u}_2,\vect{e}_1,\vect{e}_2,\vect{e}_3,\vect{e}_4$:
  \begin{equation*}
    \begin{mymatrix}{rrrrrr}
      1  & 1  & 1 & 0 & 0 & 0 \\
      1  & 2  & 0 & 1 & 0 & 0 \\
      -1 & -2 & 0 & 0 & 1 & 0 \\
      2  & 4  & 0 & 0 & 0 & 1 \\
    \end{mymatrix}
    \quad\sim\quad
    \begin{mymatrix}{rrrrrr}
      \circled{1} &  1 &  1 & 0  & 0 & 0 \\
      0 &  \circled{1} & -1 & 1  & 0 & 0 \\
      0 &  0 &  0 & \circled{1}  & 1 & 0 \\
      0 &  0 &  0 & 0  & \circled{2} & 1 \\
    \end{mymatrix}.
  \end{equation*}
  Therefore, we cast out the vectors $\vect{e}_1$ and $\vect{e}_4$ and
  keep the rest. The resulting basis is
  \begin{equation*}
    \set{\vect{u}_1,\vect{u}_2,\vect{e}_2,\vect{e}_3}
    = \set{
      \begin{mymatrix}{r} 1 \\ 1 \\ -1 \\ 2 \end{mymatrix},~
      \begin{mymatrix}{r} 1 \\ 2 \\ -2 \\ 4 \end{mymatrix},~
      \begin{mymatrix}{r} 0 \\ 1 \\  0 \\ 0 \end{mymatrix},~
      \begin{mymatrix}{r} 0 \\ 0 \\  1 \\ 0 \end{mymatrix}
    }.
  \end{equation*}
\end{solution}

The next proposition often comes in handy when we need to check that
some set of vectors is a basis for a subspace $V$, where the dimension
of $V$ is already known. If $\dim(V)=k$, we know that any basis has to
have size $k$. Interestingly, to check that a set of $k$ vectors is a
basis of $V$, it is sufficient to check {\em either} that it is
linearly independent {\em or} that it is spanning. This can save half
the work in checking that some set of vectors is a basis (but it only
works if the number of vectors is exactly $k$, the dimension of $V$).

\begin{proposition}{Basis test for $k$ vectors in $k$-dimensional space}{basis-test-k-vectors}
  \index{basis test}%
  \index{basis!basis test}%
  Let $V$ be a $k$-dimensional subspace of $\R^n$, and consider $k$
  vectors $\vect{u}_{1},\ldots,\vect{u}_{k}$ in $V$.
  \begin{itemize}
  \item If $\vect{u}_{1},\ldots,\vect{u}_{k}$ are linearly
    independent, then they form a basis for $V$.
  \item If $\vect{u}_{1},\ldots,\vect{u}_{k}$ span $V$, then they
    form a basis for $V$.
  \end{itemize}
\end{proposition}

\begin{proof}
  The first claim is an easy consequence of
  Lemma~\ref{lem:extend-to-basis}. Assume that
  $\vect{u}_{1},\ldots,\vect{u}_{k}$ are linearly independent. By
  Lemma~\ref{lem:extend-to-basis}, we can add zero or more vectors to
  $\vect{u}_{1},\ldots,\vect{u}_{k}$ to obtain a basis of $V$. On the
  other hand, since $V$ is $k$-dimensional, every basis must have
  exactly $k$ elements, so that the only possibility is that we have
  added zero vectors. Therefore,
  $\set{\vect{u}_{1},\ldots,\vect{u}_{k}}$ is already a basis of $V$,
  as claimed.

  To prove the second claim, assume that
  $V=\sspan\set{\vect{u}_{1},\ldots,\vect{u}_{k}}$. By
  Theorem~\ref{thm:subspaces-are-spans}, there exists a linearly
  independent subset of $\set{\vect{u}_{1},\ldots,\vect{u}_{k}}$ that
  spans $V$, i.e., that is a basis for $V$. But since $\dim(V)=k$,
  every basis must have exactly $k$ elements, so that the only
  possible such subset is $\set{\vect{u}_{1},\ldots,\vect{u}_{k}}$
  itself. Therefore, $\set{\vect{u}_{1},\ldots,\vect{u}_{k}}$ is a
  basis of $V$, as claimed.
\end{proof}

It is important to note that
Proposition~\ref{prop:basis-test-k-vectors} does {\em not} say that
every linearly independent set of vectors in $V$ is a basis. For
example, a set of $k-1$ or fewer linearly independent vectors will not
be spanning. Also, the proposition does {\em not} say that every
spanning set of vectors in $V$ is a basis. For example, a set of $k+1$
or more spanning vectors will not be linearly independent. Rather,
what the proposition is saying is that if we have exactly $k$ vectors
in a $k$-dimensional space, then linear independence implies spanning
and vice versa. 

\begin{example}{Basis test for $3$ vectors in $3$-dimensional space}{check-basis}
  Do the vectors
  \begin{equation*}
    \vect{u}_1 = \begin{mymatrix}{r} 1 \\ 2 \\ 3 \end{mymatrix},\quad
    \vect{u}_2 = \begin{mymatrix}{r} 1 \\ 1 \\ 1 \end{mymatrix},\quad
    \mbox{and}\quad
    \vect{u}_3 = \begin{mymatrix}{r} 1 \\ 1 \\ 2 \end{mymatrix}
  \end{equation*}
  form a basis of\/ $\R^3$?
\end{example}

\begin{solution}
  This is similar to Example~\ref{exa:non-standard-basis}. But because
  we know that $\R^3$ is a $3$-dimensional space, and because we have
  exactly $3$ vectors, by Proposition~\ref{prop:basis-test-k-vectors},
  we only need to check {\em either} whether
  $\vect{u}_1,\vect{u}_2,\vect{u}_3$ are linearly independent {\em or}
  whether they are spanning. We check whether they are linearly
  independent by using the casting-out algorithm.
  \begin{equation*}
    \begin{mymatrix}{rrr}
      1 & 1 & 1 \\
      2 & 1 & 1 \\
      3 & 1 & 2 \\
    \end{mymatrix}
    \quad\sim\quad
    \begin{mymatrix}{rrr}
      1 & 1 & 1 \\
      0 & -1 & -1 \\
      0 & 0 & -1 \\
    \end{mymatrix}.
  \end{equation*}
  Since the matrix has rank 3, the vectors
  $\vect{u}_1,\vect{u}_2,\vect{u}_3$ are linearly
  independent. Therefore, by
  Proposition~\ref{prop:basis-test-k-vectors}, they form a basis of
  $\R^3$.
\end{solution}

The following proposition is also a consequence of
Lemma~\ref{lem:extend-to-basis}. It says that smaller subspaces have
smaller dimension.

\begin{proposition}{Subspace of a subspace}{subset-dimension}
  Let $V$ and $W$ be subspaces of\/ $\R^n$, and suppose that
  $V\subseteq W$.  Then $\dim(V)\leq\dim(W)$, with equality only when
  $V=W$.
\end{proposition}

\begin{proof}
  Consider any basis $\vect{u}_1,\ldots,\vect{u}_k$ of $V$. Because
  $V\subseteq W$, the vectors $\vect{u}_1,\ldots,\vect{u}_k$ are
  linearly independent elements of $W$, and therefore can be extended
  to a basis of $W$ by Lemma~\ref{lem:extend-to-basis}. The resulting
  basis of $W$ has at least $k$ elements, i.e., $\dim(V)\leq\dim(W)$.
  To prove the last claim, assume moreover that $\dim(V)=\dim(W)$. In
  that case, $\dim(W)=k$, so that the $k$ linearly independent vectors
  $\vect{u}_1,\ldots,\vect{u}_k$ form a basis of $W$ by
  Proposition~\ref{prop:basis-test-k-vectors}. Since both $V$ and $W$
  are spanned by $\vect{u}_1,\ldots,\vect{u}_k$, we must have $V=W$,
  as claimed.
\end{proof}
