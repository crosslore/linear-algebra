\section{Properties of matrix multiplication}

As pointed out above, it is sometimes possible
to multiply matrices in one order
but not in the other order. However, even if both $AB$ and $BA$ are defined,
they may not be equal. 

\begin{example}{Matrix multiplication is not commutative}{non-commutative-multiplication}
Compare the products $AB$ and $BA$, for matrices $ A = \begin{mymatrix}{rr}
1 & 2 \\
3 & 4
\end{mymatrix}, B= \begin{mymatrix}{rr}
0 & 1 \\
1 & 0
\end{mymatrix} $ 
\end{example}

\begin{solution} 
First, notice that $A$ and $B$ are both of size $2 \times 2$. Therefore, both 
products $AB$ and $BA$ are defined. 
The first product, $AB$ is
\begin{equation*}
AB = \begin{mymatrix}{rr}
1 & 2 \\
3 & 4
\end{mymatrix} \begin{mymatrix}{rr}
0 & 1 \\
1 & 0
\end{mymatrix} = \begin{mymatrix}{rr}
2 & 1 \\
4 & 3
\end{mymatrix} 
\end{equation*}
The second product, $BA$ is
\begin{equation*}
\begin{mymatrix}{rr}
0 & 1 \\
1 & 0
\end{mymatrix} \begin{mymatrix}{rr}
1 & 2 \\
3 & 4
\end{mymatrix} = \begin{mymatrix}{rr}
3 & 4 \\
1 & 2
\end{mymatrix} 
\end{equation*}
Therefore, $AB \neq BA$. 
\end{solution}

This example illustrates that you cannot assume $AB=BA$ even when
multiplication is defined in both orders. If for some matrices $A$ and
$B$ it is true that $AB=BA$, then we say that $A$ and $B$ \textbf{commute}\index{matrix!commutative}. This is one 
important property of matrix multiplication.

The following are other important properties of matrix multiplication.
Notice that these properties hold only when the size of matrices are such that the products are defined. 

\begin{proposition}{Properties of matrix multiplication}{properties-of-matrix-multiplication}
The following hold\index{matrix multiplication!properties} for matrices $A,B,$ and $C$ and for scalars $r$ and $s$,

\begin{equation}
A\tup{rB+sC} =r\tup{AB} +s\tup{AC}  \label{matrix-properties1}
\end{equation}

\begin{equation}
\tup{B+C} A=BA+CA  \label{matrix-properties2}
\end{equation}

\begin{equation}
A\tup{BC} =\tup{AB} C  \label{matrix-properties3}
\end{equation}
\end{proposition}

\begin{proof}
 First we will prove \ref{matrix-properties1}. We will use Definition \ref{def:ij-entry-of-product} 
and prove this statement using the $ij^{th}$ entries of a matrix. 
Therefore, 
\begin{equation*}
\tup{A\tup{rB+sC} } _{ij}=\sum_{k}a_{ik}\tup{rB+sC}
_{kj}=\sum_{k}a_{ik}\tup{rb_{kj}+sc_{kj}}
\end{equation*}
\begin{equation*}
=r\sum_{k}a_{ik}b_{kj}+s\sum_{k}a_{ik}c_{kj}=r\tup{AB} _{ij}+s\tup{
AC} _{ij}
\end{equation*}
\begin{equation*}
=\tup{r\tup{AB} +s\tup{AC} } _{ij}
\end{equation*}
Thus $A\tup{rB+sC} =r(AB)+s(AC)$ as claimed. 

The proof of \ref{matrix-properties2} follows the same pattern and is left as an exercise. 

Statement \ref{matrix-properties3} is the associative law of multiplication. Using
Definition \ref{def:ij-entry-of-product},
\begin{equation*}
\tup{A\tup{BC} } _{ij}=\sum_{k}a_{ik}\tup{BC}
_{kj}=\sum_{k}a_{ik}\sum_{l}b_{kl}c_{lj}
\end{equation*}
\begin{equation*}
=\sum_{l}\tup{AB} _{il}c_{lj}=\tup{\tup{AB} C} _{ij}.
\end{equation*}
This proves \ref{matrix-properties3}.
\end{proof}
