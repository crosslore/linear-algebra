\section{Matrix Arithmetic}

\begin{outcome}
\begin{enumerate}
\item[A.] Perform the matrix operations of matrix addition, scalar
multiplication, transposition and matrix multiplication. Identify when
these operations are not defined. Represent these operations in terms
of the entries of a matrix.

\item[B.] Prove algebraic properties for matrix addition, scalar
multiplication, transposition, and matrix multiplication. Apply these
properties to manipulate an algebraic expression involving matrices.

\item[C.] Compute the inverse of a matrix using row operations, and prove identities involving
matrix inverses.

\item[E.] Solve a linear system using matrix algebra.

\item[F.] Use multiplication by an elementary matrix to apply row operations. 

\item[G.] Write a matrix as a product of elementary matrices.

\end{enumerate}
\end{outcome}

You have now solved systems of equations by writing them in terms of an
augmented matrix and then doing row operations on this augmented matrix. It
turns out that matrices are important not only for systems of equations but also in many applications.

Recall that a  \textbf{matrix}
\index{matrix}is a rectangular array of numbers. Several of them are
referred to as \textbf{matrices}. For example, here is a matrix.
\begin{equation}
\leftB
\begin{array}{rrrr}
1 & 2 & 3 & 4 \\
5 & 2 & 8 & 7 \\
6 & -9 & 1 & 2
\end{array}
\rightB
\label{matrix}
\end{equation}
Recall that the size or dimension of a matrix is defined as $m\times n$ where $m$ is the
number of rows and $n$ is the number of columns. The above matrix is a 
$3\times 4$ matrix because there are three rows and four columns.  You can remember the columns are
like columns in a Greek temple. They stand upright while the rows lay
flat like rows made by a tractor in a plowed field.

When specifying the size of a matrix, you always list the number
of rows before the number of columns.You
might remember that you always list the rows before the columns by using the
phrase \textbf{Row}man \textbf{C}atholic. 

Consider the following definition.

\begin{definition}{Square Matrix}{squarematrix}
A matrix $A$ which has size $n \times n$ is called a \textbf{square matrix} \index{matrix!square}.
In other words, $A$ is a square matrix if it has the same number of rows
and columns.
\end{definition}

There is some notation specific to matrices which we now introduce. We denote the columns of a matrix $A$ 
by $A_{j}$ as follows
\begin{equation*}
A = 
\leftB
\begin{array}{rrrr}
A_{1} & A_{2} & \cdots & A_{n}
\end{array}
\rightB
\end{equation*}
Therefore, $A_{j}$ is the $j^{th}$ column of $A$, when counted from left to right. 

The individual elements of the matrix are called \textbf{entries} \index{matrix!entries of a matrix} or \textbf{components}  \index{matrix!components of a matrix}of $A$. Elements of the matrix
are identified according to their position. The $\mathbf{\left( i, j \right)}$\textbf{-entry} of a matrix is the entry 
in the $i^{th}$ row and $j^{th}$ column. For example, in the matrix \ref{matrix} above,  $8$ is in
position $\left(2,3 \right)$ (and is called the $\left(2,3 \right)$-entry) because it is in the second row and the third column. 

In order to remember which matrix we are speaking of, we 
will denote the entry in the $i^{th}$ row  and the $j^{th}$ column of matrix $A$ by $a_{ij}$. Then, we can write $A$ in terms of its entries,
as $A= \leftB a_{ij} \rightB$. Using this notation on the matrix in \ref{matrix},
$a_{23}=8, a_{32}=-9, a_{12}=2,$ etc.

There are various operations which are done on matrices of appropriate
sizes. Matrices can be added to and subtracted from other matrices,
multiplied by a scalar, and multiplied by other matrices. We will
never divide a matrix by another matrix, but we will see later how matrix inverses play a similar role. 

In doing arithmetic with matrices, we often define the action by what
happens in terms of the entries (or components) of the
matrices. Before looking at these operations in depth, consider a few
general definitions.

\begin{definition}{The Zero Matrix}{zeromatrix}
The \textbf{$m\timess n$ zero matrix} is the $m\times n$ matrix
having every entry equal to zero. It is
\index{zero matrix} denoted by $0.$
\end{definition}

One possible zero matrix is shown in the following example.

\begin{example}{The Zero Matrix}{zeromatrix}
The $2\times 3$ zero matrix is $0= \leftB
\begin{array}{ccc}
0 & 0 & 0 \\
0 & 0 & 0
\end{array}
\rightB $.
\end{example}

Note there is a $2\times 3$ zero matrix, a $3\times 4$ zero matrix, etc. In
fact there is a zero matrix for every size! 

\begin{definition}{Equality of Matrices}{equalityofmatrices}
 Let $A$ and $B$ be two $m \times n$ matrices. Then $A=B$ \index{matrix!equality} means
that for $A=\leftB a_{ij}\rightB $
and $B=\leftB b_{ij}\rightB ,$ $a_{ij}=b_{ij}$ for all $1\leq i\leq m$ and 
$1\leq j\leq n.$
\end{definition}

In other words, two matrices are equal exactly when they are the same size and the
corresponding entries are identical. Thus
\begin{equation*}
\leftB
\begin{array}{rr}
0 & 0 \\
0 & 0 \\
0 & 0
\end{array}
\rightB \neq \leftB
\begin{array}{rr}
0 & 0 \\
0 & 0
\end{array}
\rightB
\end{equation*}
because they are different sizes. 
Also,
\begin{equation*}
\leftB
\begin{array}{rr}
0 & 1 \\
3 & 2 
\end{array}
\rightB \neq \leftB
\begin{array}{rr}
1 & 0 \\
2 & 3
\end{array}
\rightB
\end{equation*}
because, although they are the same size, their corresponding entries are not identical.

In the following section, we explore addition of matrices. 
