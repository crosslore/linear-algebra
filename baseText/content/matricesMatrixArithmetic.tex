\section{Definition and equality}

\begin{outcome}
  \begin{enumerate}
  \item Identify the dimension and entries of a matrix.
  \item Check equality of matrices.
  \end{enumerate}
\end{outcome}

We have solved systems of equations by writing them in terms of an
augmented matrix and then doing row operations. It turns out that
matrices are important not only for systems of equations but also for
many other purposes.

\begin{definition}{Matrix}{matrix}
  A \textbf{matrix}\index{matrix} is a rectangular array of numbers
  \begin{equation*}
    A = \begin{mymatrix}{cccc}
      a_{11} & a_{12} & \cdots & a_{1n} \\
      a_{21} & a_{22} & \cdots & a_{2n} \\
      \vdots & \vdots & \ddots & \vdots \\
      a_{m1} & a_{m2} & \cdots & a_{mn} \\
    \end{mymatrix},
  \end{equation*}
  where the $a_{ij}$ are scalars, called the
  \textbf{entries}\index{matrix!entry of}%
  \index{entry of a matrix} or
  \textbf{components}\index{matrix!component of}%
  \index{component!of a matrix} of $A$.  The
  \textbf{size}\index{matrix!size of}\index{size of a matrix} or
  \textbf{dimension}\index{matrix!dimension of}%
  \index{dimension!of a matrix} of a matrix is defined as $m\times n$,
  where $m$ is the number of rows and $n$ is the number of columns.
\end{definition}

For example, here is a $3\times 4$-matrix (pronounced ``three-by-four
matrix''):
\begin{equation*}
  \begin{mymatrix}{rrrr}
    1 & 2 & 3 & 4 \\
    5 & 2 & 8 & 7 \\
    6 & -9 & 1 & 2
  \end{mymatrix}.
\end{equation*}
This is a $3\times 4$-matrix because there are three rows and four
columns. When specifying the size of a matrix, we always list the
number of rows before the number of columns.

Entries of the matrix are identified according to their position. The
\textbf{$(i,j)$-entry}\index{matrix!ij-entry@$(i,j)$-entry}%
\index{ij-entry of a matrix@$(i,j)$-entry of a matrix} of a matrix is
the entry in the $i\th$ row and $j\th$ column, and is often denoted
$a_{ij}$. For example, in the above matrix, the $(2,3)$-entry is the
entry in the second row and the third column, and is equal to $8$. We
sometimes use $A=\mat{a_{ij}}$ as a short-hand notation for the entire
$m\times n$-matrix whose $(i,j)$-entry is equal to $a_{ij}$ for all
$i=1,\ldots m$ and $j=1,\ldots,n$.

There are various operations which are done on matrices of appropriate
sizes. Matrices can be added and subtracted, multiplied by a scalar,
and multiplied by other matrices. We will never divide a matrix by
another matrix, but we will see later how matrix inverses play a
similar role.

\begin{definition}{Equality of matrices}{matrix-equality}
  Two matrices are \textbf{equal}\index{matrix!equality}%
  \index{equality!of matrices} if they have the same size and the same
  corresponding entries. More precisely, if $A=\mat{a_{ij}}$ and
  $B=\mat{b_{ij}}$ are two $m\times n$-matrices, then $A=B$ means that
  $a_{ij}=b_{ij}$ for all $i=1,\ldots m$ and $j=1,\ldots,n$.
\end{definition}

For example,
\begin{equation*}
\begin{mymatrix}{rr}
0 & 0 \\
0 & 0 \\
0 & 0
\end{mymatrix} \neq \begin{mymatrix}{rr}
0 & 0 \\
0 & 0
\end{mymatrix}
\end{equation*}
because they are different sizes.
Also,
\begin{equation*}
\begin{mymatrix}{rr}
0 & 1 \\
3 & 2
\end{mymatrix} \neq \begin{mymatrix}{rr}
1 & 0 \\
2 & 3
\end{mymatrix}
\end{equation*}
because, although they are the same size, their corresponding entries are not identical.

There are special names for matrices of certain dimensions: some
matrices are called square matrices, columns vectors, or row vectors.

\begin{definition}{Square matrix}{square-matrix}
  A matrix of size $n\times n$ is called a \textbf{square
    matrix}\index{matrix!square}\index{square matrix}.  In other
  words, $A$ is a square matrix if it has the same number of rows and
  columns.
\end{definition}

\begin{definition}{Column vectors and row vectors}{row-and-column-vectors}
  A matrix of size $n\times 1$ is called a \textbf{column vector}%
  \index{vectors!column vector}\index{column vector}. A matrix of size
  $1\times n$ is called a \textbf{row vector}%
  \index{vectors!row vector}\index{row vector}.
  Here is an example of a column vector $X$ and a row vector $Y$:
  \begin{equation*}
    X=\begin{mymatrix}{c}
      x_{1} \\
      \vdots \\
      x_{n}
    \end{mymatrix},
    \quad
    Y = \begin{mymatrix}{ccc}
      y_{1} & \cdots & y_{n}
    \end{mymatrix}.
  \end{equation*}
\end{definition}

We have already encountered column vectors in
Chapter~\ref{cha:vectors-rn}.  When we use the term \textbf{vector}
without further qualification, we always mean a column vector. Also
recall from Definition~\ref{def:column-vector} that the set of
$n$-dimensional column vectors is called $\R^n$.
