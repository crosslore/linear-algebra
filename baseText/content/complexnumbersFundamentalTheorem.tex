\section{The fundamental theorem of algebra}

\begin{outcome}
  \begin{enumerate}
  \item Find the complex roots of a quadratic polynomial.
  \item In special cases, find the complex roots of a polynomial of
    degree 3 or more.
  \item Factor a polynomial into linear factors. 
  \end{enumerate}
\end{outcome}

The complex numbers were invented so that equations such as $z^2+1=0$
would have solutions. In fact, this equation has two complex
solutions, namely $z=i$ and $z=-i$. However, something much more
general (and surprising) is true: {\em every} non-trivial polynomial
equation has a solution in the complex numbers. To understand this
statement, recall that a \textbf{polynomial}%
\index{polynomial} is an expression of the form
\begin{equation*}
  p(z) = a_nz^n + a_{n-1}z^{n-1} + \ldots + a_1z + a_0.
\end{equation*}
The constants $a_0,\ldots,a_n$ are called the \textbf{coefficients}%
\index{coefficient!of a polynomial}%
\index{polynomial!coefficient} of the polynomial. If $a_n$ is the
largest non-zero coefficient, we say that the polynomial has
\textbf{degree}%
\index{degree!of a polynomial}%
\index{polynomial!degree} $n$. A polynomial of degree $0$ is of the
form $p(z) = a_0$, and is also called a \textbf{constant polynomial}%
\index{polynomial!constant}%
\index{constant polynomial}. Recall that a \textbf{root}%
\index{root!of a polynomial}%
\index{polynomial!root} of a polynomial is a number $z$ such that
$p(z)=0$.  The fundamental theorem of algebra is the following:

\begin{theorem}{Fundamental theorem of algebra}{fundamental-algebra}
  Every non-constant polynomial $p(z)$ with real or complex
  coefficients has a complex root.
\end{theorem}

The proof of this theorem is beyond the scope of this book. Note that
the theorem does not say that the roots are always easy to find. To
find the roots of a polynomial of degree 2, we can use the quadratic
formula. However, if the degree is greater than 2, we may sometimes
have to use fancier methods, such as Newton's method from calculus, or
even a computer algebra system, to locate the roots. We give some
examples.

\begin{example}{Roots of a quadratic polynomial}{complex-root}
  Find the roots of the polynomial $p(z) = z^2 - 2z + 2$.
\end{example}

\begin{solution}
  The quadratic formula gives
  \begin{equation*}
    z = \frac{2 \pm \sqrt{-4}}{2}.
  \end{equation*}
  Of course, in the real numbers, the square root of $-4$ does not
  exist, so $p(z)$ has no roots in the real numbers. However, in the
  complex numbers, the square root of $-4$ exists and is equal to
  $\pm2i$. Thus, the roots of $p(z)$ are:
  \begin{equation*}
    z = \frac{2 \pm 2i}{2} = 1\pm i.
  \end{equation*}
  Indeed, we can double-check that $1+i$ and $1-i$ are in fact roots:
  \begin{equation*}
    \begin{array}{ll}
      p(1+i) = (1+i)^2 - 2(1+i) + 2 = (1 + 2i + (-1)) - 2 - 2i + 2 = 0, \\
      p(1-i) = (1-i)^2 - 2(1-i) + 2 = (1 - 2i + (-1)) - 2 + 2i + 2 = 0. \\
    \end{array}
  \end{equation*}
  \vspace{-2ex}
\end{solution}

\begin{example}{Roots of a cubic polynomial}{complex-root2}
  Find the roots of the polynomial $p(z) = z^3 - 4z^2 + 9z - 10$.
\end{example}

\begin{solution}
  By the intermediate value theorem of calculus, we know that a cubic
  polynomial with real coefficients always has at least one real
  root. This is because $p(z)$ goes to $-\infty$ when $z\to-\infty$
  and to $\infty$ when $z\to\infty$. By trial and error, we find that
  $z=2$ is a root of this polynomial. We can therefore factor out
  $(z-2)$ from this polynomial:
  \begin{equation*}
    p(z) = z^3 - 4z^2 + 9z - 10 = (z-2)(z^2 - 2z + 5).
  \end{equation*}
  Now we can use the quadratic formula to find the roots of $z^2 - 2z
  + 5$. We find
  \begin{equation*}
    z = \frac{2\pm\sqrt{-16}}{2} = \frac{2\pm 4i}{2} = 1\pm 2i.
  \end{equation*}
  Thus, the three complex roots of $p(z)$ are $z=2$, $z=1+2i$, and $z=1-2i$.
\end{solution}

The following proposition is an important and useful consequence of
the fundamental theorem of algebra:

\begin{proposition}{Factoring a polynomial}{complex-factoring}
  Let $p(z)$ be a polynomial of degree $n$ with real or complex
  coefficients. Then $p(z)$ can be factored into $n$ linear factors
  over the complex numbers, i.e., $p(z)$ can be written in the form
  \begin{equation*}
    p(z) = a(z-b_1)(z-b_2)\cdots(z-b_n),
  \end{equation*}
  where $b_1,\ldots,b_n$ are (not necessarily distinct) roots of
  $p(z)$.
\end{proposition}

\begin{proof}
  If $n=0$, then $p(z)=a$ and there is nothing to show. Otherwise, by
  the fundamental theorem of algebra, $p(z)$ has at least one complex
  root, say $b_1$. From calculus, we know that we can factor out
  $(z-b_1)$ from $p(z)$, i.e., we can find a polynomial $q(z)$ of
  degree $n-1$ such that 
  \begin{equation*}
    p(z) = (z-b_1) q(z),
  \end{equation*}
  We can repeatedly apply the same procedure to $q(z)$ until $p(z)$
  has been factored into linear factors.
\end{proof}

\begin{example}{Factoring a polynomial}
  Factor $p(z) = z^3 - 4z^2 + 9z - 10$ into linear factors.
\end{example}

\begin{solution}
  From Example~\ref{exa:complex-root2}, we know that $p(z)$ has three
  distinct roots $b_1=2$, $b_2=1+2i$, and $b_3=1-2i$. We can therefore
  write
  \begin{equation*}
    p(z) = a(z-b_1)(z-b_2)(z-b_3).
  \end{equation*}
  Since the leading term is $z^3$, we find that $a=1$. Therefore
  \begin{equation*}
    p(z) = (z-2)\,(z-1-2i)\,(z-1+2i).
  \end{equation*}
  \vspace{-2ex}
\end{solution}
