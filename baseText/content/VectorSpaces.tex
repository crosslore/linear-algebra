\chapter{Vector spaces}

In Chapter~\ref{cha:vectors-rn}, we considered $\R^n$, the set of
$n$-dimensional column vectors. We now introduce a more general
concept of ``vector'', as an element of an abstract vector
space. Basically, vectors are entities that can be added and
scaled. While some vectors look like lists of numbers (for example,
column vectors, row vectors), other kinds of vectors don't look like
lists of numbers at all (for example, functions, polynomials). Part of
the power of linear algebra comes from our ability to find vector
spaces in many unexpected places.

Much of the content of this chapter will be a repetition of things we
have already seen in Chapter~\ref{cha:vectors-rn} in the context of
$\R^n$. For example, we will be talking about linear combinations,
linear independence, spanning sets, bases, subspaces, linear
transformations, and so on. We initially introduced these concepts in
the context of the vector space $\R^n$, so that they would be easier
to understand. We will now see that they in fact apply to \textit{all}
vector spaces.

