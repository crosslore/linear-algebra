\section{The determinant of a triangular matrix}

\begin{outcome}
  \begin{enumerate}
  \item Calculate the determinant of an upper or lower triangular
    matrix.
  \end{enumerate}
\end{outcome}

There is a certain type of matrix for which finding the determinant is
a very simple procedure: a triangular matrix.

\begin{definition}{Triangular matrices}{triangular-matrices}
  An square matrix $A$ is \textbf{upper triangular}%
  \index{matrix!upper triangular}%
  \index{matrix!triangular}%
  \index{upper triangular matrix}%
  \index{triangular matrix} if $a_{ij}=0$ whenever $i>j$. In other
  words, a matrix is upper triangular if the entries below the main
  diagonal are $0$. Thus, an upper triangular matrix looks as follows,
  where $\ast$ refers to any non-zero number:
  \begin{equation*}
    \begin{mymatrix}{ccccc}
      \ast & \ast & \cdots & \ast & \ast \\
      0 & \ast & \cdots & \ast & \ast \\
      \vdots & \vdots & \ddots & \vdots & \vdots \\
      0 & 0 & \cdots & \ast & \ast \\
      0 & 0 & \cdots & 0 & \ast \\
    \end{mymatrix}.
  \end{equation*}
  Similarly, a square matrix is \textbf{lower triangular}%
  \index{matrix!lower triangular}%
  \index{lower triangular matrix} if all entries above the main
  diagonal%
  \index{diagonal of a matrix}%
  \index{matrix!diagonal of} are $0$.
\end{definition}

The following theorem provides a useful way to calculate the
determinant of a triangular matrix.

\begin{theorem}{Determinant of a triangular matrix}{determinant-of-triangular-matrix}
  Let $A$ be an upper or lower triangular matrix. Then $\det(A)$ is
  equal to the product of the entries on the main diagonal. Written as
  a formula, we have
  \begin{equation*}
    \det(A) = a_{11}\,a_{22}\,\cdots\,a_{nn}.
  \end{equation*}
\end{theorem}

\begin{example}{Determinant of a triangular matrix}{determinant-of-triangular-matrix}
  Compute $\det(A)$, where
  \begin{equation*}
    A=\begin{mymatrix}{rrrr}
      1 & 2 & 3 & 16 \\
      0 & 2 & 6 & -7 \\
      0 & 0 & 3 & 33 \\
      0 & 0 & 0 & -1
    \end{mymatrix}.
  \end{equation*}
\end{example}

\begin{solution}
  By Theorem~\ref{thm:determinant-of-triangular-matrix}, it suffices
  to take the product of the elements on the main diagonal. Thus
  \begin{equation*}
    \det (A) ~=~ 1\cdot 2\cdot 3\cdot (-1) ~=~ -6.
  \end{equation*}
\end{solution}

For comparison, let us compute the determinant without
Theorem~\ref{thm:determinant-of-triangular-matrix}, i.e., by using
cofactor expansion. If we expand the determinant along the first
column, we get:
\begin{equation*}
  \det (A)
  ~=~
  1 \begin{absmatrix}{rrr}
      2 & 6 & -7 \\
      0 & 3 & 33 \\
      0 & 0 & -1
    \end{absmatrix}
  - 0 \begin{absmatrix}{rrr}
      2 & 3 & 16 \\
      0 & 3 & 33 \\
      0 & 0 & -1
    \end{absmatrix}
  + 0 \begin{absmatrix}{rrr}
      2 & 3 & 16 \\
      2 & 6 & -7 \\
      0 & 0 & -1
    \end{absmatrix}
  - 0 \begin{absmatrix}{rrr}
      2 & 3 & 16 \\
      2 & 6 & -7 \\
      0 & 3 & 33 \\
    \end{absmatrix}.
\end{equation*}
The only non-zero term in the expansion is
\begin{equation*}
  1\begin{absmatrix}{rrr}
    2 & 6 & -7 \\
    0 & 3 & 33 \\
    0 & 0 & -1
  \end{absmatrix}.
\end{equation*}
We can in turns expand this $3\times 3$ determinant by the cofactor
method along the first column:
\begin{equation*}
  \begin{absmatrix}{rrr}
    2 & 6 & -7 \\
    0 & 3 & 33 \\
    0 & 0 & -1
  \end{absmatrix}
  ~=~ 2 \begin{absmatrix}{rr}
    3 & 33 \\
    0 & -1
  \end{absmatrix}
  - 0 \begin{absmatrix}{rr}
    6 & -7 \\
    0 & -1
  \end{absmatrix}
  + 0 \begin{absmatrix}{rr}
    6 & -7 \\
    3 & 33 \\
  \end{absmatrix}.
\end{equation*}
Again, the only non-zero term is the first term. In summary,
\begin{equation*}
  \det (A)
  ~=~
  \begin{absmatrix}{rrrr}
    1 & 2 & 3 & 16 \\
    0 & 2 & 6 & -7 \\
    0 & 0 & 3 & 33 \\
    0 & 0 & 0 & -1
  \end{absmatrix}
  ~=~ 1 \begin{absmatrix}{rrr}
    2 & 6 & -7 \\
    0 & 3 & 33 \\
    0 & 0 & -1
  \end{absmatrix}
  ~=~ 1\cdot 2
  \begin{absmatrix}{rr}
    3 & 33 \\
    0 & -1
  \end{absmatrix}
  ~=~ 1\cdot 2\cdot 3\cdot (-1)
  ~=~ -6.
\end{equation*}
Of course this is just the same as the product of the diagonal entries
of $A$, which is the point of
Theorem~\ref{thm:determinant-of-triangular-matrix}.
