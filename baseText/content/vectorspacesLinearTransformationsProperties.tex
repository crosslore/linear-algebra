
% ----------------------------------------------------------------------
\subsection{CONTINUE HERE...}

$T(v) = S(v)$: subspace

\begin{example}{Recurrences as linear equations}{recurrence-as-equation}
  Find a basis for the solution space of the following linear
  equation:
  
\end{example}

Several important examples of linear transformations include the zero
transformation, the identity transformation, and the scalar
transformation.

\begin{example}{Linear transformations}{linear-transformations}
  Let $V$ and $W$ be vector spaces.

  \begin{enumerate}
  \item \textbf{The zero transformation}\index{zero transformation}

    $0:V\to W$ is defined by $0(\vect{v})=\vect{0}$ for all
    $\vect{v}\in V$.

  \item \textbf{The identity transformation}\index{identity transformation}

    $1_V:V\to V$ is defined by $1_V(\vect{v})=\vect{v}$ for all
    $\vect{v}\in V$.

  \item \textbf{The scalar transformation}\index{scalar
      transformation} Let $a\in\R$.

    $s_a:V\to V$ is defined by $s_a(\vect{v})=a\vect{v}$ for all
    $\vect{v}\in V$.
  \end{enumerate}
\end{example}

\begin{solution}
  We will show that the scalar transformation $s_a$ is linear, the
  rest are left as an exercise.

  By Definition~\ref{def:linear-transformation} we must show that for
  all scalars $k ,p $ and vectors $\vect{v}_1$ and $\vect{v}_2$ in
  $V$,
  $s_a(k \vect{v}_1 + p \vect{v}_2) = k s_a(\vect{v}_1)+ p
  s_a(\vect{v}_{2})$. Assume that $a$ is also a scalar.
  \begin{eqnarray*}
    s_a(k \vect{v}_1 + p \vect{v}_2)
    &=& a (k \vect{v}_1 + p \vect{v}_2) \\
    &=&  ak \vect{v}_1 + ap \vect{v}_2  \\
    &=&  k (a \vect{v}_1) + p(a \vect{v}_2)  \\
    &=& k s_a(\vect{v}_1)  + p s_a (\vect{v}_2)
  \end{eqnarray*}
  Therefore $s_a$ is a linear transformation.
\end{solution}

Consider the following important theorem.

\begin{theorem}{Properties of linear transformations}{properties}
  Let $V$ and $W$ be vector spaces, and $T:V \to W$ a linear
  transformation.  Then
  \begin{enumerate}
  \item $T$ preserves the zero vector.
    \begin{equation*}
      T(\vect{0})=\vect{0}
    \end{equation*}
  \item $T$ preserves additive inverses.
    For all $\vect{v}\in V$,
    \begin{equation*}
      T(-\vect{v})= -T(\vect{v})
    \end{equation*}
  \item $T$ preserves linear combinations.
    For all $\vect{v}_1, \vect{v}_2, \ldots, \vect{v}_m \in V$ and
    all $k_1, k_2, \ldots, k_m\in\R$,
    \begin{equation*}
      T(k_1\vect{v}_1 + k_2\vect{v}_2 + \ldots + k_m\vect{v}_m)
      = k_1T(\vect{v}_1) + k_2T(\vect{v}_2) + \ldots + k_mT(\vect{v}_m).
    \end{equation*}
  \end{enumerate}
\end{theorem}

\begin{proof}
  \begin{enumerate}
  \item Let $\vect{0}_V$ denote the zero vector of $V$ and let
    $\vect{0}_W$ denote the zero vector of $W$.  We want to prove that
    $T(\vect{0}_V)=\vect{0}_W$.  Let $\vect{v}\in V$.  Then
    $0\vect{v}=\vect{0}_V$ and
    \begin{equation*}
      T(\vect{0}_V)=T(0\vect{v})=0T(\vect{v})=\vect{0}_W.
    \end{equation*}
  \item Let $\vect{v}\in V$; then $-\vect{v}\in V$ is the additive
    inverse of $\vect{v}$, so $\vect{v} + (-\vect{v})=\vect{0}_V$.
    Thus
    \begin{eqnarray*}
      T(\vect{v} + (-\vect{v})) & = & T(\vect{0}_V) \\
      T(\vect{v}) + T(-\vect{v})) & = & \vect{0}_W \\
      T(-\vect{v}) & = & \vect{0}_W - T(\vect{v}) =  - T(\vect{v}).
    \end{eqnarray*}
  \item This result follows from preservation of addition and
    preservation of scalar multiplication.  A formal proof would be by
    induction on $m$.
  \end{enumerate}
\end{proof}

Consider the following example using the above theorem.

\begin{example}{Linear combination}{linear-transformation-combination2}
  Let $T:\Poly_2 \to \R$ be a linear transformation such that
  \begin{equation*}
    T(x^2+x)=-1; T(x^2-x)=1; T(x^2+1)=3.
  \end{equation*}
  Find $T(4x^2+5x-3)$.
\end{example}

\begin{solution}
  We provide two solutions to this problem.

  \textbf{Solution 1:}
  Suppose $a(x^2+x) + b(x^2-x) + c(x^2+1) = 4x^2+5x-3$.  Then
  \begin{equation*} (a+b+c)x^2 + (a-b)x + c = 4x^2+5x-3.\end{equation*}
  Solving for $a$, $b$, and $c$ results in the unique solution
  $a=6$, $b=1$, $c=-3$.

  Thus
  \begin{eqnarray*}
    T(4x^2+5x-3)
    & = & T(6(x^2+x) + (x^2-x) -3(x^2+1)) \\
    & = & 6T(x^2+x) + T(x^2-x) -3T(x^2+1) \\
    & = & 6(-1) + 1 -3(3) = -14.
  \end{eqnarray*}

  \textbf{Solution 2:}
  Notice that $S=\set{x^2+x, x^2-x, x^2+1}$ is a basis of $\Poly_2$,
  and thus $x^2$, $x$, and $1$ can each be written as a linear
  combination of elements of $S$.

  \begin{eqnarray*}
    x^2 & = & \textstyle \frac{1}{2}(x^2+x) + \frac{1}{2}(x^2-x) \\
    x & = & \textstyle \frac{1}{2}(x^2+x) - \frac{1}{2}(x^2-x) \\
    1 & = & (x^2+1)-\textstyle \frac{1}{2}(x^2+x) - \frac{1}{2}(x^2-x).
  \end{eqnarray*}
  Then
  \begin{eqnarray*}
    T(x^2)
    & = & \textstyle T\paren{\frac{1}{2}(x^2+x) + \frac{1}{2}(x^2-x)}
          =\frac{1}{2}T(x^2+x) + \frac{1}{2}T(x^2-x)\\
    & = & \textstyle \frac{1}{2}(-1) + \frac{1}{2}(1) = 0.  \\
    T(x)
    & = & \textstyle T\paren{\frac{1}{2}(x^2+x) - \frac{1}{2}(x^2-x)}
          = \frac{1}{2}T(x^2+x) - \frac{1}{2}T(x^2-x) \\
    & = & \textstyle \frac{1}{2}(-1) - \frac{1}{2}(1) = -1.\\
    T(1)
    & = & \textstyle T\paren{(x^2+1)-\frac{1}{2}(x^2+x) -
          \frac{1}{2}(x^2-x)}\\
    & = & \textstyle T(x^2+1)-\frac{1}{2}T(x^2+x) - \frac{1}{2}T(x^2-x) \\
    & = & \textstyle 3-\frac{1}{2}(-1) - \frac{1}{2}(1) = 3.
  \end{eqnarray*}
  Therefore,
  \begin{eqnarray*}
    T(4x^2+5x-3) & = & 4T(x^2) + 5T(x) -3T(1) \\
                 & = & 4(0) + 5(-1) - 3(3)=-14.
  \end{eqnarray*}
  The advantage of \textbf{Solution 2} over \textbf{Solution 1} is
  that if you were now asked to find $T(-6x^2-13x+9)$, it is easy to
  use $T(x^2)=0$, $T(x)=-1$ and $T(1)= 3$:
  \begin{eqnarray*}
    T(-6x^2-13x+9) & = & -6T(x^2)-13T(x)+9T(1) \\
                   & = & -6(0)-13(-1)+9(3)=13+27=40.
  \end{eqnarray*}
  More generally,
  \begin{eqnarray*}
    T(ax^2+bx+c) & = & aT(x^2)+bT(x)+cT(1) \\
                 & = & a(0)+b(-1)+c(3)=-b+3c.
  \end{eqnarray*}
\end{solution}

Suppose two linear transformations act in the same way on $\vect{v}$
for all vectors. Then we say that these transformations are equal.

\begin{definition}{Equal transformations}{equal-transformations}
  Let $S$ and $T$ be linear transformations from $V$ to $W$. Then
  $S = T$ if and only if for every $\vect{v} \in V$,
  \begin{equation*}
    S (\vect{v}) = T (\vect{v})
  \end{equation*}
\end{definition}

The definition above requires that two transformations have the same
action on every vector in order for them to be equal. The next theorem
argues that it is only necessary to check the action of the
transformations on basis vectors.

\begin{theorem}{Transformation of a spanning set}{transformation-spanning-set}
  Let $V$ and $W$ be vector spaces and suppose that $S$ and $T$ are
  linear transformations from $V$ to $W$. Then in order for $S$ and
  $T$ to be equal, it suffices that $S(\vect{v}_i) = T(\vect{v}_i)$
  where $V=\sspan \set{\vect{v}_1, \vect{v}_2, \ldots, \vect{v}_n}$.
\end{theorem}

This theorem tells us that a linear transformation is completely
determined by its actions on a spanning set. We can also examine the
effect of a linear transformation on a basis.

\begin{theorem}{Transformation of a basis}{transformation-basis}
  Suppose $V$ and $W$ are vector spaces and let
  $\set{\vect{w}_1, \vect{w}_2, \ldots, \vect{w}_n}$ be any given
  vectors in $W$ that may not be distinct. Then there exists a basis
  $\set{\vect{v}_1, \vect{v}_2, \ldots, \vect{v}_n}$ of $V$ and a
  unique linear transformation $T: V \to W$ with
  $T (\vect{v}_i) = \vect{w}_i$.

  Furthermore, if
  \begin{equation*}
    \vect{v} = k_1\vect{v}_1+k_2\vect{v}_2+\ldots+ k_n\vect{v}_n
  \end{equation*}
  is a vector of $V$, then
  \begin{equation*}
    T(\vect{v}) = k_1\vect{w}_1+k_2\vect{w}_2+\ldots+ k_n\vect{w}_n.
  \end{equation*}
\end{theorem}
