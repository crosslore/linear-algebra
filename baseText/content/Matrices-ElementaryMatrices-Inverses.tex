\subsection{Inverses of elementary matrices}

Suppose we have applied a row operation to a matrix $A$. Consider the
row operation required to return $A$ to its original form, i.e., to
undo the row operation. It turns out that this action is described by
the inverse of an elementary matrix. The following theorem ensures
that the inverse of each elementary matrix is itself an elementary
matrix.

\begin{theorem}{Inverses of elementary matrices}{inverse-elementary-matrix}
  Every elementary matrix is invertible and its inverse is also an
  elementary matrix%
  \index{elementary matrix!inverse}%
  \index{matrix!elementary matrix!inverse}.
\end{theorem}

In fact, the inverse of an elementary matrix is constructed by doing
the {\em reverse }row operation on $I$. $E^{-1}$ is obtained by
performing the row operation which would carry $E$ back to $I$.

\begin{itemize}
\item If $E$ is obtained by switching rows $i$ and $j$, then $E^{-1}$
  is also obtained by switching rows $i$ and $j$.
\item If $E$ is obtained by multiplying row $i$ by the scalar $k$,
  then $E^{-1}$ is obtained by multiplying row $i$ by the scalar
  $\frac{1}{k}$.
\item If $E$ is obtained by adding $k$ times row $i$ to row $j$, then
  $E^{-1}$ is obtained by subtracting $k$ times row $i$ from row $j$.
\end{itemize}

\begin{example}{Inverse of an elementary matrix}{inverse-elementary-matrix}
  Find $E^{-1}$, where $E$ is the elementary matrix
  \begin{equation*}
    E
    =
    \begin{mymatrix}{rr}
      1 & 0 \\
      0 & 2
    \end{mymatrix}
  \end{equation*}
\end{example}

\begin{solution}
  $E$ is obtained from the $2\times 2$ identity matrix by multiplying
  the second row by $2$. In order to carry $E$ back to the identity,
  we need to multiply the second row of $E$ by $\frac{1}{2}$.  Hence,
  $E^{-1}$ is given by
  \begin{equation*}
    E^{-1}
    =
    \begin{mymatrix}{rr}
      1 & 0 \\
      0 & \frac{1}{2}
    \end{mymatrix}.
  \end{equation*}
\end{solution}

