\section{Determinants and row operations}

\begin{outcome}
  \begin{enumerate}
  \item Determine the effect of a row operation on the determinant of
    a matrix.
  \item Use row operations to calculate a determinant.
  \end{enumerate}
\end{outcome}

Recall that there are three kinds of elementary row operations%
\index{matrix!row operation}%
\index{matrix!elementary row operation}%
\index{row operation}%
\index{elementary row operation} on matrices:
\begin{enumerate}
\item Switch two rows.
\item Multiply a row by a non-zero number.
\item Add a multiple of one row to another row.
\end{enumerate}
The following theorem examines the effect of these row operations on
the determinant of a matrix.

\begin{theorem}{Effect of row operations on the determinant}{determinant-row-operations}
  Let $A$ be an $n\times n$-matrix.
  \begin{enumerate}
  \item If $B$ is obtained from $A$ by switching two rows, then
    \begin{equation*}
      \det(B) = -\det(A).
    \end{equation*}
  \item If $B$ is obtained from $A$ by multiplying one row by a
    non-zero scalar $k$, then
    \begin{equation*}
      \det(B) = k\det(A).
    \end{equation*}
  \item If $B$ is obtained from $A$ by adding a multiple of one row to
    another row, then
    \begin{equation*}
      \det(B) = \det(A).
    \end{equation*}
  \end{enumerate}
\end{theorem}

Notice that the second part of this theorem is true when we multiply
{\em one} row of the matrix by $k$.  If we were to multiply {\em two}
rows of $A$ by $k$ to obtain $B$, we would have
$\det(B) = k^2 \det(A)$.

\begin{example}{Using row operations to calculate a determinant}{determinant-row-operations1}
  Use row operations to calculate the following determinant:
  \begin{equation*}
    \begin{absmatrix}{rrr}
      1 & 5 & 5 \\
      0 & 0 & -3 \\
      0 & 2 & 7 \\
    \end{absmatrix}.
  \end{equation*}
\end{example}

\begin{solution}
  If we switch the second and third rows, we obtain a triangular
  matrix, of which the determinant is easy to compute. By
  Theorem~\ref{thm:determinant-row-operations}, switching two rows
  negates the determinant. We therefore have:
  \begin{equation*}
    \begin{absmatrix}{rrr}
      1 & 5 & 5 \\
      0 & 0 & -3 \\
      0 & 2 & 7 \\
    \end{absmatrix}
    ~=~
    -\begin{absmatrix}{rrr}
      1 & 5 & 5 \\
      0 & 2 & 7 \\
      0 & 0 & -3 \\
    \end{absmatrix}
    ~=~ -(1\cdot 2\cdot(-3)) ~=~ 6.
  \end{equation*}
\end{solution}

\begin{example}{Using row operations to calculate a determinant}{determinant-row-operations2}
  Use row operations to calculate the following determinant:
  \begin{equation*}
    \begin{absmatrix}{rrr}
      1 & 4 & -2 \\
      1 & 8 & 1 \\
      2 & 4 & -9 \\
    \end{absmatrix}.
  \end{equation*}
\end{example}

\begin{solution}
  We can use elementary row operations to reduce this matrix to
  triangular form:
  \begin{equation*}
    \begin{mymatrix}{rrr}
      1 & 4 & -2 \\
      1 & 8 & 1 \\
      2 & 4 & -9 \\
    \end{mymatrix}
    \stackrel{R_2\rowop R_2-R_1}{\roweq}
    \begin{mymatrix}{rrr}
      1 & 4 & -2 \\
      0 & 4 & 3 \\
      2 & 4 & -9 \\
    \end{mymatrix}
    \stackrel{R_3\rowop R_3-2R_1}{\roweq}
    \begin{mymatrix}{rrr}
      1 & 4 & -2 \\
      0 & 4 & 3 \\
      0 & -4 & -5 \\
    \end{mymatrix}
    \stackrel{R_3\rowop R_3+R_2}{\roweq}
    \begin{mymatrix}{rrr}
      1 & 4 & -2 \\
      0 & 4 & 3 \\
      0 & 0 & -2 \\
    \end{mymatrix}.
  \end{equation*}
  Each of the row operations is of the form ``add a multiple of one
  row to another row'', and therefore does not change the
  determinant. We therefore have:
  \begin{equation*}
    \begin{absmatrix}{rrr}
      1 & 4 & -2 \\
      1 & 8 & 1 \\
      2 & 4 & -9 \\
    \end{absmatrix}
    ~=~
    \begin{absmatrix}{rrr}
      1 & 4 & -2 \\
      0 & 4 & 3 \\
      2 & 4 & -9 \\
    \end{absmatrix}
    ~=~
    \begin{absmatrix}{rrr}
      1 & 4 & -2 \\
      0 & 4 & 3 \\
      0 & -4 & -5 \\
    \end{absmatrix}
    ~=~
    \begin{absmatrix}{rrr}
      1 & 4 & -2 \\
      0 & 4 & 3 \\
      0 & 0 & -2 \\
    \end{absmatrix}
    ~=~ 1\cdot 4\cdot(-2) = -8.
  \end{equation*}
\end{solution}

In general, we can convert any square matrix to triangular form using
elementary row operations. In fact, it is always possible to do so
using only elementary operations of the first and third kind (swap two
rows or add a multiple of one row to another). This gives us a very
efficient way to compute determinants. If the matrices are large, this
method is much more efficient than the cofactor method.

\begin{example}{Using row operations to calculate a determinant}{determinant-row-operations3}
  Use elementary row operations of the first and third kind to calculate the
  following determinant:
  \begin{equation*}
    \begin{absmatrix}{rrrr}
      0 & 2 & 1 & 4 \\
      2 & 2 & -4 & -1 \\
      1 & 1 & -2 & -1 \\
      1 & 3 & 2 & 5 \\
    \end{absmatrix}.
  \end{equation*}
\end{example}

\begin{solution}
  We use elementary row operations to reduce the matrix to triangular
  form:
  \begin{equation*}
    \begin{array}{ccccc}
      \begin{mymatrix}{rrrr}
        0 & 2 & 1 & 4 \\
        2 & 2 & -4 & -1 \\
        1 & 1 & -2 & -1 \\
        1 & 3 & 2 & 5 \\
      \end{mymatrix}
      &\stackrel{R_1\rowswap R_3}{\roweq}&
      \begin{mymatrix}{rrrr}
        1 & 1 & -2 & -1 \\
        2 & 2 & -4 & -1 \\
        0 & 2 & 1 & 4 \\
        1 & 3 & 2 & 5 \\
      \end{mymatrix}
      &\stackrel{R_2\rowop R_2-2R_1}{\roweq}&
      \begin{mymatrix}{rrrr}
        1 & 1 & -2 & -1 \\
        0 & 0 & 0 & 1 \\
        0 & 2 & 1 & 4 \\
        1 & 3 & 2 & 5 \\
      \end{mymatrix}
      \\\\[-1ex]
      &\stackrel{R_4\rowop R_4-R_1}{\roweq}&
      \begin{mymatrix}{rrrr}
        1 & 1 & -2 & -1 \\
        0 & 0 & 0 & 1 \\
        0 & 2 & 1 & 4 \\
        0 & 2 & 4 & 6 \\
      \end{mymatrix}
      &\stackrel{R_4\rowop R_4-R_3}{\roweq}&
      \begin{mymatrix}{rrrr}
        1 & 1 & -2 & -1 \\
        0 & 0 & 0 & 1 \\
        0 & 2 & 1 & 4 \\
        0 & 0 & 3 & 2 \\
      \end{mymatrix}
      \\\\[-1ex]
      &\stackrel{R_2\rowswap R_3}{\roweq}&
      \begin{mymatrix}{rrrr}
        1 & 1 & -2 & -1 \\
        0 & 2 & 1 & 4 \\
        0 & 0 & 0 & 1 \\
        0 & 0 & 3 & 2 \\
      \end{mymatrix}
      &\stackrel{R_3\rowswap R_4}{\roweq}&
      \begin{mymatrix}{rrrr}
        1 & 1 & -2 & -1 \\
        0 & 2 & 1 & 4 \\
        0 & 0 & 3 & 2 \\
        0 & 0 & 0 & 1 \\
      \end{mymatrix}
    \end{array}
  \end{equation*}
  By Theorem~\ref{thm:determinant-row-operations}, the determinant
  changes signs each time we swap two rows. The determinant is
  unchanged when we add a multiple of one row to another. Therefore,
  we have
  \begin{equation*}
    \begin{array}{ccccccc}
      \begin{absmatrix}{rrrr}
        0 & 2 & 1 & 4 \\
        2 & 2 & -4 & -1 \\
        1 & 1 & -2 & -1 \\
        1 & 3 & 2 & 5 \\
      \end{absmatrix}
      &=&
      -\begin{absmatrix}{rrrr}
        1 & 1 & -2 & -1 \\
        2 & 2 & -4 & -1 \\
        0 & 2 & 1 & 4 \\
        1 & 3 & 2 & 5 \\
      \end{absmatrix}
      &=&
      -\begin{absmatrix}{rrrr}
        1 & 1 & -2 & -1 \\
        0 & 0 & 0 & 1 \\
        0 & 2 & 1 & 4 \\
        1 & 3 & 2 & 5 \\
      \end{absmatrix}
      \\\\[-1ex]
      &=&
      -\begin{absmatrix}{rrrr}
        1 & 1 & -2 & -1 \\
        0 & 0 & 0 & 1 \\
        0 & 2 & 1 & 4 \\
        0 & 2 & 4 & 6 \\
      \end{absmatrix}
      &=&
      -\begin{absmatrix}{rrrr}
        1 & 1 & -2 & -1 \\
        0 & 0 & 0 & 1 \\
        0 & 2 & 1 & 4 \\
        0 & 0 & 3 & 2 \\
      \end{absmatrix}
      \\\\[-1ex]
      &=&
      +\begin{absmatrix}{rrrr}
        1 & 1 & -2 & -1 \\
        0 & 2 & 1 & 4 \\
        0 & 0 & 0 & 1 \\
        0 & 0 & 3 & 2 \\
      \end{absmatrix}
      &=&
      -\begin{absmatrix}{rrrr}
        1 & 1 & -2 & -1 \\
        0 & 2 & 1 & 4 \\
        0 & 0 & 3 & 2 \\
        0 & 0 & 0 & 1 \\
      \end{absmatrix}
      &=& -6.
    \end{array}
  \end{equation*}
  In practice, the last calculation could have been done in a single
  step. All we had to do is count the number of swap operations we
  performed during the row operations. If there is an odd number of
  swap operations, the sign of the determinant changes; otherwise, it
  stays the same.
\end{solution}

