\section{Linear independence}

\begin{outcome}
  \begin{enumerate}
  \item Find the redundant vectors in a set of vectors.
  \item Determine whether a set of vectors is linearly independent.
  \item Find a linearly independent subset of a set of spanning vectors.
  \item Write a vector as a unique linear combination of a set of
    linearly independent vectors.
  \end{enumerate}
\end{outcome}

% ----------------------------------------------------------------------
\subsection{Redundant vectors and linear independence}

In Example~\ref{exa:redundant-span}, we encountered three vectors
$\vect{u}$, $\vect{v}$, and $\vect{w}$ such that
$\sspan\set{\vect{u},\vect{v},\vect{w}} =
\sspan\set{\vect{u},\vect{v}}$.  If this happens, then the vector
$\vect{w}$ does not contribute anything to the span of
$\set{\vect{u},\vect{v},\vect{w}}$, and we say that $\vect{w}$ is
\textbf{redundant}%
\index{redundant vector}\index{vector!redundant}. The following
definition generalizes this notion.

\begin{definition}{Redundant vectors, linear dependence, and linear independence}{redundant-vectors}
  Consider a sequence of $k$ vectors $\vect{u}_1,\ldots,\vect{u}_k$.  We
  say that the vector $\vect{u}_j$ is \textbf{redundant}%
  \index{redundant vector}\index{vector!redundant} if it can be
  written as a linear combination of earlier vectors in the sequence,
  i.e., if
  \begin{equation*}
    \vect{u}_j = a_1\,\vect{u}_1 + a_2\,\vect{u}_2 + \ldots + a_{j-1}\,\vect{u}_{j-1}
  \end{equation*}
  for some scalars $a_1,\ldots,a_{j-1}$. We say that the sequence of
  vectors $\vect{u}_1,\ldots,\vect{u}_k$ is \textbf{linearly
    dependent}%
  \index{linear dependence}\index{vector!linearly dependent} if it
  contains one or more redundant vectors. Otherwise, we say that the
  vectors are \textbf{linearly independent}%
  \index{linear independence}\index{vector!linearly independent}.  
\end{definition}

\begin{example}{Redundant vectors}{redundant-vectors}
  Find the redundant vectors in the following sequence of vectors. Are
  the vectors linearly independent?
  \begin{equation*}
    \vect{u}_1 = \begin{mymatrix}{c} 0 \\ 0 \\ 0 \\ 0\end{mymatrix},
    \quad
    \vect{u}_2 = \begin{mymatrix}{c} 1 \\ 2 \\ 2 \\ 3\end{mymatrix},
    \quad
    \vect{u}_3 = \begin{mymatrix}{c} 1 \\ 1 \\ 1 \\ 1\end{mymatrix},
    \quad
    \vect{u}_4 = \begin{mymatrix}{c} 2 \\ 3 \\ 3 \\ 4\end{mymatrix},
    \quad
    \vect{u}_5 = \begin{mymatrix}{c} 0 \\ 1 \\ 2 \\ 3\end{mymatrix},
    \quad
    \vect{u}_6 = \begin{mymatrix}{c} 3 \\ 3 \\ 2 \\ 2\end{mymatrix}.
  \end{equation*}
\end{example}

\begin{solution}
  \begin{itemize}
  \item The vector $\vect{u}_1$ is redundant, because it is a linear
    combination of earlier vectors. (Although there are no earlier
    vectors, recall from Example~\ref{exa:span-empty-set} that the empty
    sum of vectors is equal to the zero vector $\vect{0}$. Therefore,
    $\vect{u}_1$ is indeed an (empty) linear combination of earlier
    vectors.)
  \item The vector $\vect{u}_2$ is not redundant, because it cannot be
    written as a linear combination of $\vect{u}_1$. This is because
    the system of equations
    \begin{equation*}
      \begin{mymatrix}{l|l}
        0 & 1 \\
        0 & 2 \\
        0 & 2 \\
        0 & 3 \\
      \end{mymatrix}
    \end{equation*}
    has no solution.
  \item The vector $\vect{u}_3$ is not redundant, because it cannot be
    written as a linear combination of $\vect{u}_1$ an $\vect{u}_2$.
    This is because the system of equations
    \begin{equation*}
      \begin{mymatrix}{ll|l}
        0 & 1 & 1 \\
        0 & 2 & 1\\
        0 & 2 & 1\\
        0 & 3 & 1\\
      \end{mymatrix}
    \end{equation*}
    has no solution.
  \item The vector $\vect{u}_4$ is redundant, because $\vect{u}_4 =
    \vect{u}_2 + \vect{u}_3$.
  \item The vector $\vect{u}_5$ is not redundant, because 
    This is because the system of equations
    \begin{equation*}
      \begin{mymatrix}{llll|l}
        0 & 1 & 1 & 2 & 0 \\
        0 & 2 & 1 & 3 & 1 \\
        0 & 2 & 1 & 3 & 2 \\
        0 & 3 & 1 & 4 & 3 \\
      \end{mymatrix}
    \end{equation*}
    has no solution.
  \item The vector $\vect{u}_6$ is redundant, because $\vect{u}_6 =
    \vect{u}_2 + 2\vect{u}_3-\vect{u}_5$.
  \end{itemize}
  In summary, the vectors $\vect{u}_1$, $\vect{u}_4$, and $\vect{u}_6$
  are redundant, and the vectors $\vect{u}_2$, $\vect{u}_3$, and
  $\vect{u}_5$ are not. It follows that the vectors
  $\vect{u}_1,\ldots,\vect{u}_6$ are linearly dependent.
\end{solution}

% ----------------------------------------------------------------------
\subsection{The casting-out algorithm}

The last example shows that it can be a lot of work to find the
redundant vectors in a sequence of $k$ vectors. Doing so in the naive way
require us to solve up to $k$ systems of linear equations!
Fortunately, there is a much faster and easier method, the so-called
{\em casting-out algorithm}.

\begin{algorithm}{Casting-out algorithm}{casting-out}
  \index{casting-out algorithm}%
  \index{linear independence!casting-out algorithm}%
  \index{vector!casting-out algorithm}%
  \textbf{Input:} a list of $k$ vectors
  $\vect{u}_1,\ldots,\vect{u}_k\in\R^n$.  \smallskip

  \textbf{Output:} the set of indices $j$ such that $\vect{u}_j$ is
  redundant.
  \smallskip
  
  \textbf{Algorithm:} Write the vectors $\vect{u}_1,\ldots,\vect{u}_k$
  as the columns of an $n\times k$-matrix, and reduce to {\ef}. Every
  {\em non-pivot} column, if any, corresponds to a redundant vector.
\end{algorithm}

\begin{example}{Casting-out algorithm}{casting-out}
  Use the casting-out algorithm to find the redundant vectors among
  the vectors from Example~\ref{exa:redundant-vectors}.
\end{example}

\begin{solution}
  Following the casting-out algorithm, we write the vectors
  $\vect{u}_1,\ldots,\vect{u}_6$ as the columns of a matrix and reduce
  to {\ef}.
  \begin{equation*}
    \begin{mymatrix}{rrrrrr}
      0 & 1 & 1 & 2 & 0 & 3 \\
      0 & 2 & 1 & 3 & 1 & 3 \\
      0 & 2 & 1 & 3 & 2 & 2 \\
      0 & 3 & 1 & 4 & 3 & 2 \\
    \end{mymatrix}
    \roweq\ldots\roweq
    \begin{mymatrix}{rrrrrr}
      0 & \circled{1} & 1 & 2 & 0 & 3 \\
      0 & 0 & \circled{1} & 1 & -1 & 3 \\
      0 & 0 & 0 & 0 & \circled{1} & -1 \\
      0 & 0 & 0 & 0 & 0 & 0 \\
    \end{mymatrix}.
  \end{equation*}
  The pivot columns are columns $2$, $3$, and $5$. The non-pivot
  columns are columns $1$, $4$, and $6$. Therefore, the vectors
  $\vect{u}_1$, $\vect{u}_4$, and $\vect{u}_6$ are redundant. Note
  that this is the same answer we got in
  Example~\ref{exa:redundant-vectors}.
\end{solution}

The above version of the casting-out algorithm only tells us which of
the vectors (if any) are redundant, but it does not give us a specific
way to write the redundant vectors as linear combinations of previous
vectors. However, we can easily get this additional information if we
reduce the matrix all the way to {\rref}. We call this version of the
algorithm the {\em extended casting-out algorithm}.

\begin{algorithm}{Extended casting-out algorithm}{extended-casting-out}
  \index{extended casting-out algorithm}%
  \index{casting-out algorithm!extended}%
  \index{linear independence!casting-out algorithm!extended}%
  \index{vector!casting-out algorithm!extended}%
  \textbf{Input:} a list of $k$ vectors
  $\vect{u}_1,\ldots,\vect{u}_k\in\R^n$.
  \smallskip

  \textbf{Output:} the set of indices $j$ such that $\vect{u}_j$ is
  redundant, and a set of coefficients for writing each redundant
  vector as a linear combination of previous vectors.
  \smallskip
  
  \textbf{Algorithm:} Write the vectors $\vect{u}_1,\ldots,\vect{u}_k$
  as the columns of an $n\times k$-matrix, and reduce to
  {\rref}. Every {\em non-pivot} column, if any, corresponds to a
  redundant vector. If $\vect{u}_j$ is a redundant vector, then the
  entries in the $j\th$ column of the {\rref} are coefficients for
  writing $\vect{u}_j$ as a linear combination of previous
  non-redundant vectors.
\end{algorithm}

\begin{example}{Extended casting-out algorithm}{extended-casting-out}
  Use the casting-out algorithm to find the redundant vectors among
  the vectors from Example~\ref{exa:redundant-vectors}, and write each
  redundant vector as a linear combination of previous non-redundant
  vectors.
\end{example}

\begin{solution}
  Once again, we write the vectors $\vect{u}_1,\ldots,\vect{u}_6$ as
  the columns of a matrix. This time we use the extended casting-out
  algorithm, which means we reduce the matrix to {\rref} instead of
  {\ef}.
  \begin{equation*}
    \begin{mymatrix}{rrrrrr}
      0 & 1 & 1 & 2 & 0 & 3 \\
      0 & 2 & 1 & 3 & 1 & 3 \\
      0 & 2 & 1 & 3 & 2 & 2 \\
      0 & 3 & 1 & 4 & 3 & 2 \\
    \end{mymatrix}
    \roweq\ldots\roweq
    \begin{mymatrix}{rrrrrr}
      0 & \circled{1} & 0 & 1 & 0 & 1 \\
      0 & 0 & \circled{1} & 1 & 0 & 2 \\
      0 & 0 & 0 & 0 & \circled{1} & -1 \\
      0 & 0 & 0 & 0 & 0 & 0 \\
    \end{mymatrix}.
  \end{equation*}
  As before, the non-pivot columns are columns $1$, $4$, and $6$, and
  therefore, the vectors $\vect{u}_1$, $\vect{u}_4$, and $\vect{u}_6$
  are redundant. The non-redundant vectors are $\vect{u}_2$,
  $\vect{u}_3$, and $\vect{u}_5$. Moreover, the entries in the sixth
  column are $1$, $2$, and $-1$.  Note that this means that the sixth
  column can be written as $1$ times the second column plus $2$ times
  the third column plus $(-1)$ times the fourth column. The same
  coefficients can be used to write $\vect{u}_6$ as a linear
  combination of previous {\em non-redundant} columns, namely:
  \begin{equation*}
    \vect{u}_6 = 1\,\vect{u}_2 + 2\,\vect{u}_3 - 1\,\vect{u}_5.
  \end{equation*}
  Also, the entries in the fourth column are $1$ and $1$, which are
  the coefficients for writing $\vect{u}_4$ as a linear combination of
  previous non-redundant columns, namely:
  \begin{equation*}
    \vect{u}_4 = 1\,\vect{u}_2 + 1\,\vect{u}_3.
  \end{equation*}
  Finally, there are no non-zero entries in the first column. This
  means that $\vect{u}_1$ is the empty linear combination
  \begin{equation*}
    \vect{u}_1 = \vect{0}.
  \end{equation*}
\end{solution}

% ----------------------------------------------------------------------
\subsection{Alternative characterization of linear independence}

Our definition of redundant vectors depends on the order in which the
vectors are written. This is because each redundant vector must be a
linear combination of {\em earlier} vectors in the sequence. For example,
in the sequence of vectors
\begin{equation*}
  \vect{u}=\begin{mymatrix}{r} 1 \\ 1 \\ 1 \end{mymatrix},\quad
  \vect{v}=\begin{mymatrix}{r} 3 \\ 2 \\ 1 \end{mymatrix},\quad
  \vect{w}=\begin{mymatrix}{r} 11 \\ 8 \\ 5 \end{mymatrix},
\end{equation*}
the vector $\vect{w}$ is redundant, because it is a linear combination
of earlier vectors $\vect{w} = 2\,\vect{u}+3\,\vect{v}$. Neither
$\vect{u}$ nor $\vect{v}$ are redundant. On the other hand, in the sequence
of vectors
\begin{equation*}
  \vect{u}=\begin{mymatrix}{r} 1 \\ 1 \\ 1 \end{mymatrix},\quad
  \vect{w}=\begin{mymatrix}{r} 11 \\ 8 \\ 5 \end{mymatrix},\quad
  \vect{v}=\begin{mymatrix}{r} 3 \\ 2 \\ 1 \end{mymatrix},
\end{equation*}
$\vect{v}$ is redundant because
$\vect{v} = \frac{1}{3}\vect{w} - \frac{2}{3}\vect{u}$, but neither
$\vect{u}$ nor $\vect{w}$ are redundant. Note that none of the vectors
have changed; only the order in which they are written is
different. Yet $\vect{w}$ is the redundant vector in the first sequence,
and $\vect{v}$ is the redundant vector in the second sequence.

Because we defined linear independence in terms of the absence of
redundant vectors, you may suspect that the concept of linear
independence also depends on the order in which the vectors are
written. However, this is not the case. The following theorem gives an
alternative characterization of linear independence that is more
symmetric (it does not depend on the order of the vectors).

\begin{theorem}{Characterization of linear independence}{characterization-linear-independence}
  Let $\vect{u}_1,\ldots,\vect{u}_k$ be vectors. Then
  $\vect{u}_1,\ldots,\vect{u}_k$ are linearly independent%
  \index{linear independence!alternative characterization}%
  \index{vector!linearly independent!alternative characterization}%
  if and only if the homogeneous equation
  \begin{equation*}
    a_1\,\vect{u}_1 + \ldots + a_k\,\vect{u}_k = \vect{0}
  \end{equation*}
  has only the trivial solution%
  \index{trivial solution}\index{solution!trivial}%
  \index{system of linear equations!trivial solution}.
\end{theorem}

\begin{proof}
  Let $A$ be the $n\times k$-matrix whose columns are
  $\vect{u}_1,\ldots,\vect{u}_k$. We know from the theory of
  homogeneous systems that the system
  $a_1\,\vect{u}_1 + \ldots + a_k\,\vect{u}_k = \vect{0}$ has no
  non-trivial solution if and only if every column of the {\ef} of $A$
  is a pivot column. By the casting-out algorithm, this is the case if
  and only if none of the vectors $\vect{u}_1,\ldots,\vect{u}_k$ are
  redundant, i.e., if and only if the vectors are linearly independent.
\end{proof}

\begin{example}{Characterization of linear independence}{characterization-linear-independence}
  Use the method of
  Theorem~\ref{thm:characterization-linear-independence} to determine
  whether the following vectors are linearly independent in $\R^4$.
  \begin{equation*}
    \vect{u}_1 = \begin{mymatrix}{r} 1 \\ 1 \\ 2 \\ 0 \end{mymatrix},\quad
    \vect{u}_2 = \begin{mymatrix}{r} 0 \\ 1 \\ 1 \\ 1 \end{mymatrix},\quad
    \vect{u}_3 = \begin{mymatrix}{r} 1 \\ 2 \\ 3 \\ 2 \end{mymatrix},\quad
    \vect{u}_4 = \begin{mymatrix}{r} 2 \\ 3 \\ 3 \\ 1 \end{mymatrix}.
  \end{equation*}
\end{example}

\begin{solution}
  We must check whether the equation
  \begin{equation*}
    a_1 \begin{mymatrix}{r} 1 \\ 1 \\ 2 \\ 0 \end{mymatrix}
    + a_2 \begin{mymatrix}{r} 0 \\ 1 \\ 1 \\ 1 \end{mymatrix}
    + a_3 \begin{mymatrix}{r} 1 \\ 2 \\ 3 \\ 2 \end{mymatrix}
    + a_4 \begin{mymatrix}{r} 2 \\ 3 \\ 3 \\ 1 \end{mymatrix}
    = \begin{mymatrix}{r} 0 \\ 0 \\ 0 \\ 0 \end{mymatrix}
  \end{equation*}
  has a non-trivial solution. If it does, the vectors are linearly
  dependent. On the other hand, if there is only the trivial solution,
  the vectors are linearly independent. We write the augmented matrix
  and solve:
  \begin{equation*}
    \begin{mymatrix}{rrrr|r}
      1 & 0 & 1 & 2 & 0 \\
      1 & 1 & 2 & 3 & 0 \\
      2 & 1 & 3 & 7 & 0 \\
      0 & 1 & 2 & 1 & 0 \\
    \end{mymatrix}
    \roweq
    \ldots
    \roweq
    \begin{mymatrix}{rrrr|r}
      \circled{1} & 0 & 1 & 2 & 0 \\
      0 & \circled{1} & 1 & 1 & 0 \\
      0 & 0 & \circled{1} & 0 & 0 \\
      0 & 0 & 0 & \circled{2} & 0 \\
    \end{mymatrix}.
  \end{equation*}
  Since every column is a pivot column, there are no free variables;
  the system of equations has a unique solution, which is
  $a_1=a_2=a_3=a_4=0$, i.e., the trivial solution. Therefore, the
  vectors $\vect{u}_1,\ldots,\vect{u}_4$ are linearly independent.
\end{solution}

\begin{example}{Characterization of linear independence}{characterization-linear-independence2}
  Use the method of
  Theorem~\ref{thm:characterization-linear-independence} to determine
  whether the following vectors are linearly independent in $\R^3$.
  \begin{equation*}
    \vect{u}_1 = \begin{mymatrix}{r} 1 \\ 1 \\ 0 \end{mymatrix},\quad
    \vect{u}_2 = \begin{mymatrix}{r} 1 \\ 3 \\ 1 \end{mymatrix},\quad
    \vect{u}_3 = \begin{mymatrix}{r} 0 \\ 4 \\ 2 \end{mymatrix},\quad
  \end{equation*}
\end{example}

\begin{solution}
  As in the previous example, we must check whether the equation
  $a_1\,\vect{u}_1 + a_2\,\vect{u}_2 + a_3\,\vect{u}_3 = \vect{0}$ has
  a non-trivial solution. Once again, we write the augmented matrix
  and solve:
  \begin{equation*}
    \begin{mymatrix}{rrr|r}
      1 & 1 & 0 & 0 \\
      1 & 3 & 4 & 0 \\
      0 & 1 & 2 & 0 \\
    \end{mymatrix}
    \roweq
    \ldots
    \roweq
    \begin{mymatrix}{rrr|r}
      \circled{1} & 1 & 0 & 0 \\
      0 & \circled{2} & 4 & 0 \\
      0 & 0 & 0 & 0 \\
    \end{mymatrix}.
  \end{equation*}
  Since column $3$ is not a pivot column, $a_3$ is a free
  variable. Therefore, the system has a non-trivial solution, and the
  vectors are linearly dependent.

  With a small amount of extra work, we can find an actual non-trivial
  solution of
  $a_1\,\vect{u}_1 + a_2\,\vect{u}_2 + a_3\,\vect{u}_3 =
  \vect{0}$. All we have to do is set $a_3=1$ and do a back
  substitution. We find that $(a_1,a_2,a_3)=(2,-2,1)$ is a
  solution. In other words,
  \begin{equation*}
    2\vect{u}_1 - 2\vect{u}_2 + \vect{u}_3 = \vect{0}.
  \end{equation*}
  We can also use this information to write $\vect{u}_3$ as a linear
  combination of previous vectors, namely,
  $\vect{u}_3 = -2\vect{u}_1 + 2\vect{u}_2$.
\end{solution}

The characterization of linear independence in
Theorem~\ref{thm:characterization-linear-independence} is mostly
useful for theoretical reasons. However, it can also help in solving
problems such as the following.

\begin{example}{Related sets of vectors}{related-linear-independence}
  Let $\vect{u},\vect{v},\vect{w}$ be linearly independent vectors in
  $\R^n$. Are the vectors $\vect{u}+\vect{v}$, $2\vect{u}+\vect{w}$,
  and $\vect{v}-5\vect{w}$ linearly independent?
\end{example}

\begin{solution}
  By Theorem~\ref{thm:characterization-linear-independence}, to check
  whether the vectors are linearly independent, we must check whether
  the equation
  \begin{equation}\label{eqn:related-linear-independence1}
    a(\vect{u}+\vect{v}) + b(2\vect{u}+\vect{w}) +
    c(\vect{v}-5\vect{w})=\vect{0}
  \end{equation}
  has non-trivial solutions.  If it does, the vectors are linearly
  dependent, if it does not, they are linearly independent. We can
  simplify the equation as follows:
  \begin{equation}\label{eqn:related-linear-independence2}
    (a+2b)\vect{u} + (a+c)\vect{v} + (b-5c)\vect{w}=\vect{0}.
  \end{equation}
  Since $\vect{u}$, $\vect{v}$, and $\vect{w}$ are linearly
  independent, we know, again by
  Theorem~\ref{thm:characterization-linear-independence}, that
  equation {\eqref{eqn:related-linear-independence2}} only has the
  trivial solution. Therefore,
  \begin{eqnarray*}
    a + 2b & = & 0, \\
    a + c & = & 0, \\
    b - 5c & = & 0. 
  \end{eqnarray*}
  We can solve this system of three equations in three variables, and
  we find that it has the unique solution $a=b=c=0$. Therefore,
  $a=b=c=0$ is the only solution to equation
  {\eqref{eqn:related-linear-independence1}}, which means that the
  vectors $\vect{u}+\vect{v}$, $2\vect{u}+\vect{w}$, and
  $\vect{v}-5\vect{w}$ are linearly independent.
\end{solution}

% ----------------------------------------------------------------------
\subsection{Properties of linear independence}

The following are some properties of linearly independent sets.

\begin{theorem}{Properties of linear independence}{properties-linear-independence}
  \index{linear independence!properties}%
  \index{vector!linearly independent!properties}%
  \index{properties of linear independence}%
  \begin{enumerate}
  \item \textbf{Linear independence and reordering.} If a sequence
    $\vect{u}_1,\ldots,\vect{u}_k$ of $k$ vectors is linearly
    independent, then so is any reordering of the sequence (i.e.,
    whether or not the vectors are linearly independent does not
    depend on the order in which the vectors are written down).
  \item \textbf{Linear independence of a subset.} If
    $\vect{u}_1,\ldots,\vect{u}_k$ are linearly independent, then so
    are $\vect{u}_1,\ldots,\vect{u}_j$ for any $j<k$.
  \item \textbf{Linear independence and dimension.}
    \label{properties-linear-independence-c}
    Let $\vect{u}_1,\ldots,\vect{u}_k$ be a sequence of $k$ vectors in
    $\R^n$. If $k>n$, then the vectors are linearly dependent (i.e.,
    not linearly independent).
  \end{enumerate}
\end{theorem}

\begin{proof}
  \begin{enumerate}
  \item This follows from
    Theorem~\ref{thm:characterization-linear-independence}, because
    whether or not the equation
    $a_1\vect{u}_1 + \ldots + a_k\vect{u}_k = \vect{0}$ has a
    non-trivial solution does not depend on the order in which the
    vectors are written.
  \item If one of the vectors in the sequence
    $\vect{u}_1,\ldots,\vect{u}_j$ were redundant, then it would be
    redundant in the longer sequence $\vect{u}_1,\ldots,\vect{u}_k$ as
    well.
  \item Let $A$ be the $n\times k$-matrix that has the vectors
    $\vect{u}_{1},\ldots,\vect{u}_{k}$ as its columns and suppose that
    $k>n$. Then the rank of $A$ is at most $n$, so the {\ef} of $A$
    has some non-pivot columns. Therefore, the system
    $a_1\vect{u}_1 + \ldots + a_k\vect{u}_k = \vect{0}$ has
    non-trivial solutions, and the vectors are linearly dependent by
    Theorem~\ref{thm:characterization-linear-independence}.
  \end{enumerate}
\end{proof}

\begin{example}{Linear dependence}{linear-dependence}
  Are the following vectors linearly independent?
  \begin{equation*}
    \begin{mymatrix}{r}
        1 \\
        4 
      \end{mymatrix},\quad 
      \begin{mymatrix}{r}
        2 \\
        3
      \end{mymatrix},\quad
      \begin{mymatrix}{r}
        3 \\
        2
      \end{mymatrix}.
  \end{equation*}
\end{example}

\begin{solution}
  Since these are $3$ vectors in $\R^2$, they are linearly dependent
  by Theorem~\ref{thm:properties-linear-independence}. No calculation
  is necessary.
\end{solution}

% ----------------------------------------------------------------------
\subsection{Linear independence and linear combinations}

In general, there is more than one way of writing a given vector as a
linear combination of some spanning vectors. For example, consider
\begin{equation*}
  \vect{u}_1 = \begin{mymatrix}{r} 1 \\ 1 \\ 0 \end{mymatrix},\quad 
  \vect{u}_2 = \begin{mymatrix}{r} 0 \\ 1 \\ 1 \end{mymatrix},\quad 
  \vect{u}_3 = \begin{mymatrix}{r} 1 \\ 2 \\ 1 \end{mymatrix},\quad 
  \vect{v} = \begin{mymatrix}{r} 1 \\ 3 \\ 2 \end{mymatrix}.
\end{equation*}
We can write $\vect{v}$ in many different ways as a linear
combination of $\vect{u}_1,\ldots,\vect{u}_3$, for example
\begin{equation*}
  \begin{array}{r@{~}c@{~}c}
    \vect{v} &=& -\vect{u}_1 + 2\vect{u}_3, \\    
    \vect{v} &=& \vect{u}_2 + \vect{u}_3, \\
    \vect{v} &=& \vect{u}_1 + 2\vect{u}_2, \\
    \vect{v} &=& 2\vect{u}_1 + 3\vect{u}_2 - \vect{u}_3. \\
  \end{array}
\end{equation*}
However, when the vectors $\vect{u}_1,\ldots,\vect{u}_k$ are linearly
independent, this does not happen. In this case, the linear
combination is always unique, as the following theorem shows.

\begin{theorem}{Unique linear combination}{unique-linear-combination}
  Assume $\vect{u}_1,\ldots,\vect{u}_k$ are linearly independent. Then
  every vector $\vect{v}\in\sspan\set{\vect{u}_1,\ldots,\vect{u}_k}$
  can be written as a linear combination of
  $\vect{u}_1,\ldots,\vect{u}_k$ in a unique way.
\end{theorem}

\begin{proof}
  We already know that every vector
  $\vect{v}\in\sspan\set{\vect{u}_1,\ldots,\vect{u}_k}$ can be written
  as a linear combination of $\vect{u}_1,\ldots,\vect{u}_k$, because
  that is the definition of span. So what must be proved is the
  uniqueness. Suppose, therefore, that there are two ways of writing
  $\vect{v}$ as such a linear combination, i.e., that
  \begin{equation*}
    \begin{array}{l}
    \vect{v} = a_1\,\vect{u}_1 + a_2\,\vect{u}_2 + \ldots + a_k\,\vect{u}_k\quad\mbox{and}\\
    \vect{v} = b_1\,\vect{u}_1 + b_2\,\vect{u}_2 + \ldots + b_k\,\vect{u}_k.\\
    \end{array}
  \end{equation*}
  Subtracting one equation from the other, we get
  \begin{equation*}
    \vect{0} = (a_1-b_1)\vect{u}_1 + (a_2-b_2)\vect{u}_2 + \ldots + (a_k-b_k)\vect{u}_k.
  \end{equation*}
  Since $\vect{u}_1,\ldots,\vect{u}_k$ are linearly independent, we
  know by Theorem~\ref{thm:characterization-linear-independence} that
  the last equation only has the trivial solution, i.e., $a_1-b_1=0$,
  $a_2-b_2=0$, \ldots, $a_k-b_k=0$. It follows that $a_1=b_1$,
  $a_2=b_2$, \ldots, $a_k=b_k$. We have shown that any two ways of
  writing $\vect{v}$ as a linear combination of
  $\vect{u}_1,\ldots,\vect{u}_k$ are equal. Therefore, there is only
  one way of doing so.
\end{proof}

% ----------------------------------------------------------------------
\subsection{Removing redundant vectors}

Consider the span of some vectors $\vect{u}_1,\ldots,\vect{u}_k$. As
we just saw in the previous subsection, the span is especially nice
when the vectors $\vect{u}_1,\ldots,\vect{u}_k$ are linearly
independent, because in that case, every element $\vect{v}$ of the
span can be {\em uniquely} written in the form
$\vect{v} = a_1\,\vect{u}_1 + \ldots + a_k\,\vect{u}_k$.

But what if we have a span of some vectors
$\vect{u}_1,\ldots,\vect{u}_k$ that are not linearly independent? It
turns out that we can always find some linearly independent vectors
that span the same set.  In fact, this can be done by simply removing
the redundant vectors from $\vect{u}_1,\ldots,\vect{u}_k$. This is the
subject of the following theorem.

\begin{theorem}{Removing redundant vectors}{linearly-independent-subset}
  \index{redundant vector!removing}%
  \index{vector!redundant!removing}%
  Let $\vect{u}_1,\ldots,\vect{u}_k$ be a sequence of vectors, and
  suppose that $\vect{u}_{j_1},\ldots,\vect{u}_{j_\ell}$ is the
  subsequence of vectors that is obtained by removing all of the
  redundant vectors. Then $\vect{u}_{j_1},\ldots,\vect{u}_{j_\ell}$
  are linearly independent and
  \begin{equation*}
    \sspan\set{\vect{u}_{j_1},\ldots,\vect{u}_{j_\ell}}
    =
    \sspan\set{\vect{u}_1,\ldots,\vect{u}_k}.    
  \end{equation*}
\end{theorem}

\begin{proof}
  Remove the redundant vectors one by one, from right to left. Each
  time a redundant vector is removed, the span does not change; the
  proof of this is similar to Example~\ref{exa:redundant-span}.
  Moreover, the resulting sequence of vectors
  $\sspan\set{\vect{u}_{j_1},\ldots,\vect{u}_{j_\ell}}$ is linearly
  independent, because if any of these vectors were a linear
  combination of earlier ones, then it would have been redundant in
  the original sequence of vectors, and would have therefore been removed.
\end{proof}

\begin{example}{Finding a linearly independent set of spanning vectors}{linearly-independent-subset}
  Find a subset of $\set{\vect{u}_1,\ldots,\vect{u}_4}$ that is
  linearly independent and has the same span as
  $\set{\vect{u}_1,\ldots,\vect{u}_4}$.
  \begin{equation*}
    \vect{u}_1 = \begin{mymatrix}{r} 1 \\ 0 \\ -2 \\ 3 \end{mymatrix},
    \quad
    \vect{u}_2 = \begin{mymatrix}{r} -2 \\ 0 \\ 4 \\ -6 \end{mymatrix},
    \quad
    \vect{u}_3 = \begin{mymatrix}{r} 1 \\ 2 \\ 2 \\ 1 \end{mymatrix},
    \quad
    \vect{u}_4 = \begin{mymatrix}{r} 3 \\ 4 \\ 2 \\ 5 \end{mymatrix}.
  \end{equation*}
\end{example}

\begin{solution}
  We use the casting-out algorithm to find the redundant vectors:
  \begin{equation*}
    \begin{mymatrix}{rrrr}
      1  & -2 & 1 & 3 \\
      0  &  0 & 2 & 4 \\
      -2 &  4 & 2 & 2 \\
      3  & -6 & 1 & 5 \\
    \end{mymatrix}
    \roweq\ldots\roweq
    \begin{mymatrix}{rrrr}
      \circled{1}  & -2 & 1 & 3 \\
      0  &  0 & \circled{2} & 4 \\
      0  &  0 & 0 & 0 \\
      0  &  0 & 0 & 0 \\
    \end{mymatrix}.
  \end{equation*}
  Therefore, the redundant vectors are $\vect{u}_2$ and
  $\vect{u}_4$. We remove them (``cast them out'') and are left with
  $\vect{u}_1$ and $\vect{u}_3$. Therefore, by
  Theorem~\ref{thm:linearly-independent-subset},
  $\set{\vect{u}_1,\vect{u}_3}$ is linearly independent and
  $\sspan\set{\vect{u}_1,\vect{u}_3} =
  \sspan\set{\vect{u}_1,\ldots,\vect{u}_4}$.
\end{solution}

