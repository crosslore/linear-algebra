\section{Geometric interpretation of linear transformations}

\begin{outcome}
  \begin{enumerate}
  \item Find the matrix of rotations, reflections, scalings, and
    shearings in $\R^2$ and $\R^3$.
  \item Determine the action of a rotation or reflection on a vector.
  \end{enumerate}
\end{outcome}

In this section, we will examine some special examples of linear
transformations in $\R^2$ and $\R^3$ including rotations and
reflections.

\begin{example}{Rotation by $90^{\circ}$ in $\R^2$}{rotation-90-R2}
  Consider the linear transformation $T:\R^2\to\R^2$ that is given by
  a counterclockwise rotation by 90 degrees. Find the matrix%
  \index{matrix!of a rotation} $A$ corresponding to this linear
  transformation. Find a formula for $T$.
\end{example}

\begin{solution}
  To visualize a vector function on $\R^2$, it is often useful to
  consider a pair of before-and-after pictures such as the following:
  \begin{center}
    \begin{tikzpicture}
      \begin{scope}[scale=0.5]
        \draw[red,thick,fill=red!10]
        (0,0) -- (0,5) -- (3,5) -- (3,4) -- (1,4) --
        (1,3) -- (2,3) -- (2,2) -- (1,2) -- (1,0) -- cycle;
        \draw[step=1cm, gray!50, very thin] (-5.8,-5.8) grid (5.8,5.8);
        \draw[thick,->] (-6.5,0) -- (6.5,0);
        \draw[thick,->] (0,-6.5) -- (0,6.5);
        \draw[red,thick]
        (0,0) -- (0,5) -- (3,5) -- (3,4) -- (1,4) --
        (1,3) -- (2,3) -- (2,2) -- (1,2) -- (1,0) -- cycle;
        \draw[blue,thick,->] (0,0) -- node[below] {$\vect{e}_1$} (5,0);
        \draw[blue,thick,->] (0,0) -- node[left] {$\vect{e}_2$} (0,5);
        \path (0,-7.2) node {``before''};
      \end{scope}
      \begin{scope}[xshift=4.5cm]
        \path (0,0) node {$\stackrel{T}{\longmapsto}$};
      \end{scope}
      \begin{scope}[xshift=9cm,scale=0.5]
        \draw[red,thick,fill=red!15,rotate=90]
        (0,0) -- (0,5) -- (3,5) -- (3,4) -- (1,4) --
        (1,3) -- (2,3) -- (2,2) -- (1,2) -- (1,0) -- cycle;
        \draw[step=1cm, gray!50, very thin] (-5.8,-5.8) grid (5.8,5.8);
        \draw[thick,->] (-6.5,0) -- (6.5,0);
        \draw[thick,->] (0,-6.5) -- (0,6.5);
        \draw[red,thick,rotate=90]
        (0,0) -- (0,5) -- (3,5) -- (3,4) -- (1,4) --
        (1,3) -- (2,3) -- (2,2) -- (1,2) -- (1,0) -- cycle;
        \draw[blue,thick,->,rotate=90] (0,0) -- node[right]
        {$T(\vect{e}_1) = \begin{mysmallmatrix}{c}0\\1\end{mysmallmatrix}$}
        (5,0);
        \draw[blue,thick,->,rotate=90] (0,0) --
        node[below,xshift=-0.1cm] {$T(\vect{e}_2) = \begin{mysmallmatrix}{c}-1\\0\end{mysmallmatrix}$}
        (0,5);
        \path (0,-7) node {``after''};
      \end{scope}
    \end{tikzpicture}
  \end{center}
  The picture illustrates how the function $T$ rotates the entire
  plane (including the pink letter ``F'') by 90 degrees
  counterclockwise. The picture also illustrates that when we apply
  the rotation $T$ to the first and second standard basis vectors
  $\vect{e}_1$ and $\vect{e}_2$, we obtain the vectors
  \begin{equation*}
    T(\vect{e}_1) = \begin{mymatrix}{c}0\\1\end{mymatrix}
    \quad\mbox{and}\quad
    T(\vect{e}_2) = \begin{mymatrix}{c}-1\\0\end{mymatrix}.
  \end{equation*}
  The matrix of $T$ has these vectors as its columns. Therefore, the
  matrix of $T$ is
  \begin{equation*}
    A = \begin{mymatrix}{cc}
      0 & -1 \\
      1 & 0 \\
    \end{mymatrix}.
  \end{equation*}
  Finally, we can use this to find a formula for the counterclockwise
  90 degree rotation $T$:
  \begin{equation*}
    T\tup{\begin{mymatrix}{c} x \\ y \end{mymatrix}}
    = \begin{mymatrix}{cc}
      0 & -1 \\
      1 & 0 \\
    \end{mymatrix}
    \begin{mymatrix}{c} x \\ y \end{mymatrix}
    = \begin{mymatrix}{c} -y \\ x \end{mymatrix}.
  \end{equation*}
  To illustrate how this works, consider the top right corner of the
  letter ``F''. It has the coordinates $(0.6,1)$. Applying the
  function $T$ to the coordinate vector, we get
  \begin{equation*}
    T\tup{\begin{mymatrix}{c} 0.6 \\ 1 \end{mymatrix}}
    = \begin{mymatrix}{cc}
      0 & -1 \\
      1 & 0 \\
    \end{mymatrix}
    \begin{mymatrix}{c} 0.6 \\ 1 \end{mymatrix}
    = \begin{mymatrix}{c} -1 \\ 0.6 \end{mymatrix}.
  \end{equation*}
  These are precisely the coordinates of the corresponding point on
  the letter ``F'' after the rotation.
\end{solution}

\begin{example}{Reflection about the $y$-axis in $\R^2$}{reflection-y-R2}
  Let $T:\R^2\to\R^2$ be a reflection about the $y$-axis. Find the
  matrix%
  \index{matrix!of a reflection} $A$ corresponding to this linear
  transformation, and a formula for $T$.
\end{example}

\begin{solution}
  The before-and-after picture for a reflection about the $y$-axis
  looks like this:
  \begin{center}
    \begin{tikzpicture}
      \begin{scope}[scale=0.5]
        \draw[red,thick,fill=red!10]
        (0,0) -- (0,5) -- (3,5) -- (3,4) -- (1,4) --
        (1,3) -- (2,3) -- (2,2) -- (1,2) -- (1,0) -- cycle;
        \draw[step=1cm, gray!50, very thin] (-5.8,-5.8) grid (5.8,5.8);
        \draw[thick,->] (-6.5,0) -- (6.5,0);
        \draw[thick,->] (0,-6.5) -- (0,6.5);
        \draw[red,thick]
        (0,0) -- (0,5) -- (3,5) -- (3,4) -- (1,4) --
        (1,3) -- (2,3) -- (2,2) -- (1,2) -- (1,0) -- cycle;
        \draw[blue,thick,->] (0,0) -- node[below] {$\vect{e}_1$} (5,0);
        \draw[blue,thick,->] (0,0) -- node[left] {$\vect{e}_2$} (0,5);
      \end{scope}
      \begin{scope}[xshift=4.5cm]
        \path (0,0) node {$\stackrel{T}{\longmapsto}$};
      \end{scope}
      \begin{scope}[xshift=9cm,scale=0.5]
        \draw[red,thick,fill=red!15,xscale=-1]
        (0,0) -- (0,5) -- (3,5) -- (3,4) -- (1,4) --
        (1,3) -- (2,3) -- (2,2) -- (1,2) -- (1,0) -- cycle;
        \draw[step=1cm, gray!50, very thin] (-5.8,-5.8) grid (5.8,5.8);
        \draw[thick,->] (-6.5,0) -- (6.5,0);
        \draw[thick,->] (0,-6.5) -- (0,6.5);
        \draw[red,thick,xscale=-1]
        (0,0) -- (0,5) -- (3,5) -- (3,4) -- (1,4) --
        (1,3) -- (2,3) -- (2,2) -- (1,2) -- (1,0) -- cycle;
        \draw[blue,thick,->,xscale=-1] (0,0) -- node[below]
        {$T(\vect{e}_1) = \begin{mysmallmatrix}{c}-1\\0\end{mysmallmatrix}$}
        (5,0);
        \draw[blue,thick,->,xscale=-1] (0,0) --
        node[right] {$T(\vect{e}_2) = \begin{mysmallmatrix}{c}0\\1\end{mysmallmatrix}$}
        (0,5);
      \end{scope}
    \end{tikzpicture}
  \end{center}
  We see that
  \begin{equation*}
    T(\vect{e}_1) = -\vect{e}_1 = \begin{mymatrix}{c}-1\\0\end{mymatrix}
    \quad\mbox{and}\quad
    T(\vect{e}_2) = \vect{e}_2 = \begin{mymatrix}{c}0\\1\end{mymatrix}.
  \end{equation*}
  Therefore, the matrix of $T$ is
  \begin{equation*}
    A = \begin{mymatrix}{cc}
      -1 & 0 \\
      0  & 1 \\
    \end{mymatrix}.
  \end{equation*}
  The formula for a reflection about the $y$-axis is:
  \begin{equation*}
    T\tup{\begin{mymatrix}{c} x \\ y \end{mymatrix}}
    = \begin{mymatrix}{cc}
      -1 & 0 \\
      0  & 1 \\
    \end{mymatrix}
    \begin{mymatrix}{c} x \\ y \end{mymatrix}
    = \begin{mymatrix}{c} -x \\ y \end{mymatrix}.
  \end{equation*}
\end{solution}

\begin{example}{Rotation by an arbitrary angle in $\R^2$}{rotation-theta-R2}
  Find the matrix%
  \index{matrix!of a rotation} $A$ for a counterclockwise rotation by
  angle $\theta$ in $\R^2$.
\end{example}

\begin{solution}
  The before-and-after picture is as follows:
  \begin{center}
    \begin{tikzpicture}
      \begin{scope}[scale=0.5]
        \draw[red,thick,fill=red!10]
        (0,0) -- (0,5) -- (3,5) -- (3,4) -- (1,4) --
        (1,3) -- (2,3) -- (2,2) -- (1,2) -- (1,0) -- cycle;
        \draw[step=1cm, gray!50, very thin] (-5.8,-5.8) grid (5.8,5.8);
        \draw[red,thick]
        (0,0) -- (0,5) -- (3,5) -- (3,4) -- (1,4) --
        (1,3) -- (2,3) -- (2,2) -- (1,2) -- (1,0) -- cycle;
        \draw[thick,->] (-6.5,0) -- (6.5,0);
        \draw[thick,->] (0,-6.5) -- (0,6.5);
        \draw[blue,thick,->] (0,0) -- node[below] {$\vect{e}_1$} (5,0);
        \draw[blue,thick,->] (0,0) -- node[left] {$\vect{e}_2$} (0,5);
      \end{scope}
      \begin{scope}[xshift=4.5cm]
        \path (0,0) node {$\stackrel{T}{\longmapsto}$};
      \end{scope}
      \begin{scope}[xshift=9cm,scale=0.5]
        \draw[red,thick,fill=red!15,rotate=30]
        (0,0) -- (0,5) -- (3,5) -- (3,4) -- (1,4) --
        (1,3) -- (2,3) -- (2,2) -- (1,2) -- (1,0) -- cycle;
        \draw[step=1cm, gray!50, very thin] (-5.8,-5.8) grid (5.8,5.8);
        \filldraw[fill=green!20,draw=green!50!black] (0,0) -- (0:20mm) arc (0:30:20mm) -- cycle;
        \node at (15:14mm){$\theta$};
        \draw[red,thick,rotate=30]
        (0,0) -- (0,5) -- (3,5) -- (3,4) -- (1,4) --
        (1,3) -- (2,3) -- (2,2) -- (1,2) -- (1,0) -- cycle;
        \draw[thick,->] (-6.5,0) -- (6.5,0);
        \draw[thick,->] (0,-6.5) -- (0,6.5);
        \draw[blue,thick,->,rotate=30] (0,0) -- (5,0) node[right]
        {$T(\vect{e}_1) = \begin{mysmallmatrix}{c}\cos\theta\\\sin\theta\end{mysmallmatrix}$};
        \draw[blue,thick,->,rotate=30] (0,0) -- (0,5)
        node[left] {$T(\vect{e}_2) = \begin{mysmallmatrix}{c}-\sin\theta\\\cos\theta\end{mysmallmatrix}$};
      \end{scope}
    \end{tikzpicture}
  \end{center}
  Thus the matrix of $T$ is
  \begin{equation*}
    A = \begin{mymatrix}{cc}
      \cos\theta & -\sin\theta \\
      \sin\theta & \cos\theta \\
    \end{mymatrix}.
  \end{equation*}
\end{solution}

\begin{example}{Two rotations}{two-rotations}
  Find the matrix for a counterclockwise rotation by angle
  $\theta+\phi$ in two different ways, and compare.
\end{example}

\begin{solution}
  Let us write $A_{\theta}$ for the matrix of a rotation by angle
  $\theta$, and similarly $A_{\phi}$ for a rotation by angle $\phi$
  and $A_{\theta+\phi}$ for a rotation by angle $\theta+\phi$. One way
  to rotate a vector by angle $\theta+\phi$ is to first multiply it by
  $A_{\phi}$ and then by $A_{\theta}$. Performing one rotation after
  the other amounts to multiplying the two matrices:
  \begin{eqnarray*}
    A_{\theta}A_{\phi}
    &=& \begin{mymatrix}{cc}
      \cos\theta & -\sin\theta \\
      \sin\theta & \cos\theta \\
    \end{mymatrix}
    \begin{mymatrix}{cc}
      \cos\phi & -\sin\phi \\
      \sin\phi & \cos\phi \\
    \end{mymatrix}
    \\
    &=& \begin{mymatrix}{cc}
      \cos\theta \cos\phi - \sin\theta \sin\phi &
      -\cos\theta \sin\phi - \sin\theta \cos\phi \\
      \sin\theta \cos\phi + \cos\theta \sin\phi &
      \cos\theta \cos\phi - \sin\theta \sin\phi
    \end{mymatrix}.
  \end{eqnarray*}
  Another way is to perform the entire rotation in one step:
  \begin{eqnarray*}
    A_{\theta+\phi}
    &=& \begin{mymatrix}{cc}
      \cos(\theta+\phi) & -\sin(\theta+\phi) \\
      \sin(\theta+\phi) & \cos(\theta+\phi) \\
    \end{mymatrix}.
  \end{eqnarray*}
  The fact that these two matrices are equal amounts to the famous
  trigonometric identities for the sum of two angles%
  \index{trigonometry!sum of two angles}%
  \index{sum!of two angles}, which we have here derived using linear
  algebra concepts:
  \begin{eqnarray*}
    \sin(\theta+\phi) &=& \sin\theta \cos\phi + \cos\theta \sin\phi, \\
    \cos(\theta+\phi) &=& \cos\theta \cos\phi - \sin\theta \sin\phi.
  \end{eqnarray*}
\end{solution}

\begin{example}{Rotation in $\R^3$}{rotation-R3}
  Find the matrix of a rotation by angle $\theta$ about the $z$-axis
  in $3$-dimensional space, counterclockwise when viewed from above.
\end{example}

\begin{solution}
  Here is the before-and-after picture. A rotation in $3$-dimensional
  space is usually harder to visualize than in the plane, but
  fortunately, the rotation is about the $z$-axis, so all the
  ``action'' is taking place in the $xy$-plane.
  \begin{center}
    \begin{tikzpicture}
      \begin{scope}[scale=0.5]
        \begin{scope}[cm={-0.4,-0.5,1,0,(0,0)}]
          \draw[red,thick,fill=red!10]
          (0,0) -- (0,5) -- (3,5) -- (3,4) -- (1,4) --
          (1,3) -- (2,3) -- (2,2) -- (1,2) -- (1,0) -- cycle;
          \draw[gray!50, very thin] (0,0) circle [radius=5cm];
          \draw[red,thick]
          (0,0) -- (0,5) -- (3,5) -- (3,4) -- (1,4) --
          (1,3) -- (2,3) -- (2,2) -- (1,2) -- (1,0) -- cycle;
          \draw[thick,->] (-6.5,0) -- (6.5,0);
          \draw[thick,->] (0,-6.5) -- (0,6.5);
          \draw[blue,thick,->] (0,0) -- node[left] {$\vect{e}_1$} (5,0);
          \draw[blue,thick,->] (0,0) -- node[above] {$\vect{e}_2$} (0,5);
        \end{scope}
        \draw[thick,->] (0,0) -- (0,6.5);
        \draw[blue,thick,->] (0,0) -- node[left] {$\vect{e}_3$} (0,5);
      \end{scope}
      \begin{scope}[xshift=4.5cm]
        \path (0,0) node {$\stackrel{T}{\longmapsto}$};
      \end{scope}
      \begin{scope}[xshift=9cm,scale=0.5]
        \begin{scope}[cm={-0.4,-0.5,1,0,(0,0)}]
          \draw[red,thick,fill=red!15,rotate=30]
          (0,0) -- (0,5) -- (3,5) -- (3,4) -- (1,4) --
          (1,3) -- (2,3) -- (2,2) -- (1,2) -- (1,0) -- cycle;
          \filldraw[fill=green!20,draw=green!50!black] (0,0) -- (0:30mm) arc (0:30:30mm) -- cycle;
          \node at (15:20mm){$\theta$};
          \draw[gray!50, very thin] (0,0) circle [radius=5cm];
          \draw[red,thick,rotate=30]
          (0,0) -- (0,5) -- (3,5) -- (3,4) -- (1,4) --
          (1,3) -- (2,3) -- (2,2) -- (1,2) -- (1,0) -- cycle;
          \draw[thick,->] (-6.5,0) -- (6.5,0);
          \draw[thick,->] (0,-6.5) -- (0,6.5);
          \draw[blue,thick,->,rotate=30] (0,0) -- (5,0) node[below,xshift=1cm]
          {$T(\vect{e}_1) = \begin{mysmallmatrix}{c}\cos\theta\\\sin\theta\\0\end{mysmallmatrix}$};
          \draw[blue,thick,->,rotate=30] (0,0) -- (0,5)
          node[above,xshift=1cm] {$T(\vect{e}_2) = \begin{mysmallmatrix}{c}-\sin\theta\\\cos\theta\\0\end{mysmallmatrix}$};
        \end{scope}
        \draw[thick,->] (0,0) -- (0,6.5);
        \draw[blue,thick,->] (0,0) -- node[left] {$T(\vect{e}_3)=\begin{mysmallmatrix}{c}0\\0\\1\end{mysmallmatrix}$} (0,5);
      \end{scope}
    \end{tikzpicture}
  \end{center}
  Therefore, the matrix of the rotation is
  \begin{equation*}
    A = \begin{mymatrix}{ccc}
      \cos\theta & -\sin\theta & 0 \\
      \sin\theta & \cos\theta & 0 \\
      0 & 0 & 1
    \end{mymatrix}.
  \end{equation*}
\end{solution}

\begin{example}{Multiple rotations in $\R^3$}{multiple-rotations}
  Find the matrix of the linear transformation $T:\R^3\to\R^3$ that is
  given as follows: a rotation by $30$ degrees about the $z$-axis,
  followed by a rotation by $45$ degrees about the $x$-axis.
\end{example}

\begin{solution}
  It would be quite difficult to picture the transformation $T$ in one
  step. Fortunately, we don't have to do this. All we have to do is
  find the matrix for each rotation separately, then multiply the two
  matrices. We have the be careful to multiply the matrices in the
  correct order.

  Let $B$ be the matrix for a $30$-degree rotation about the
  $z$-axis. It is given exactly as in
  Example~\ref{exa:rotation-R3}:
  \begin{equation*}
    \def\arraystretch{1.4}
    B = \begin{mymatrix}{ccc}
      \cos 30^{\circ} & -\sin 30^{\circ} & 0 \\
      \sin 30^{\circ} & \cos 30^{\circ} & 0 \\
      0 & 0 & 1 \\
    \end{mymatrix}
    = \begin{mymatrix}{ccc}
      \frac{\sqrt3}{2} & -\frac{1}{2} & 0 \\
      \frac{1}{2} & \frac{\sqrt3}{2} & 0 \\
      0 & 0 & 1 \\
    \end{mymatrix}.
  \end{equation*}
  Let $C$ be the matrix for a $45$-degree rotation about the $x$-axis.
  It is analogous to Example~\ref{exa:rotation-R3}, except that the
  rotation takes place in the $yz$-plane instead of the $xy$-plane.
  \begin{equation*}
    \def\arraystretch{1.4}
    C = \begin{mymatrix}{ccc}
      1 & 0 & 0 \\
      0 & \cos 45^{\circ} & -\sin 45^{\circ} \\
      0 & \sin 45^{\circ} & \cos 45^{\circ} \\
    \end{mymatrix}
    = \begin{mymatrix}{ccc}
      1 & 0 & 0 \\
      0 & \frac{1}{\sqrt2} & -\frac{1}{\sqrt2} \\
      0 & \frac{1}{\sqrt2} & \frac{1}{\sqrt2} \\
    \end{mymatrix}.
  \end{equation*}
  Finally, to apply the linear transformation $T$ to a vector
  $\vect{v}$, we must first apply $B$ and then $C$. This means that
  $T(\vect{v}) = C(B\vect{v})$. Therefore, the matrix corresponding to
  $T$ is $CB$. Note that it is important that we multiply the matrices
  corresponding to each subsequent rotation {\em from right to left}.
  \begin{equation*}
    \def\arraystretch{1.4}
    A ~=~ CB
    ~=~ \begin{mymatrix}{ccc}
      1 & 0 & 0 \\
      0 & \frac{1}{\sqrt2} & -\frac{1}{\sqrt2} \\
      0 & \frac{1}{\sqrt2} & \frac{1}{\sqrt2} \\
    \end{mymatrix}
    \begin{mymatrix}{ccc}
      \frac{\sqrt3}{2} & -\frac{1}{2} & 0 \\
      \frac{1}{2} & \frac{\sqrt3}{2} & 0 \\
      0 & 0 & 1 \\
    \end{mymatrix}
    ~=~ \begin{mymatrix}{ccc}
      \frac{\sqrt3}{2} & -\frac{1}{2} & 0 \\
      \frac{1}{2\sqrt2} & \frac{\sqrt3}{2\sqrt2} & -\frac{1}{\sqrt2} \\
      \frac{1}{2\sqrt2} & \frac{\sqrt3}{2\sqrt2} & \frac{1}{\sqrt2} \\
    \end{mymatrix}.
  \end{equation*}
\end{solution}
