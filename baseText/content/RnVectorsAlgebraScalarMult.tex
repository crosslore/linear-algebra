\subsection{Scalar Multiplication of Vectors in \texorpdfstring{$\mathbb{R}^n$}{Rn}}

Scalar multiplication of vectors in $\mathbb{R}^n$ is defined as \index{vector!scalar multiplication}
follows.

\begin{definition}{Scalar Multiplication of Vectors in $\mathbb{R}^n$}{vectorscalarmultiplication}
If $\vect{u}\in \mathbb{R}^{n}$ and $k\in \mathbb{R}$ is a
scalar,
\index{scalars} then $k\vect{u}\in \mathbb{R}^{n}$ is defined by
\begin{equation*}
k\vect{u}=k\leftB \begin{array}{c}
u_{1} \\
\vdots \\
u_{n}
\end{array}
\rightB = \leftB \begin{array}{c}
ku_{1} \\
\vdots \\
ku_{n}
\end{array}
\rightB
\end{equation*}
\end{definition}

For example 
$3 \leftB
\begin{array}{rrr}
1 & 2 & 3
\end{array}
\rightB^T =
\leftB
\begin{array}{rrr}
3 & 6 & 9
\end{array}
\rightB^T$
 and 
$-2\leftB
\begin{array}{rrr}
1 & 2 & 3
\end{array}
\rightB^T
=
\leftB
\begin{array}{rrr}
-2 & -4 & -6
\end{array}
\rightB^T$.

Just as with addition, scalar multiplication of vectors satisfies several important properties. These are 
outlined in the following theorem. 

\begin{theorem}{Properties of Scalar Multiplication}{vectorscalarmult}
The following properties hold for vectors $\vect{u},\vect{v}\in \mathbb{R}^{n}$ and $k,p $
scalars.
\begin{itemize}
\item The Distributive Law over Vector Addition
\begin{equation*}
k \left( \vect{u}+\vect{v}\right) = k\vect{u}+ k\vect{v}
\end{equation*}
\item The Distributive Law over Scalar Addition
\begin{equation*}
\left( k + p  \right)\vect{u} = k \vect{u}+p \vect{u}
\end{equation*}
\item The Associative Law for Scalar Multiplication
\begin{equation*}
k \left( p \vect{u}\right) = \left(k p \right)\vect{u}
\end{equation*}
\item Rule for Multiplication by $1$
\begin{equation*}
1\vect{u}=\vect{u}  
\end{equation*}
\end{itemize}
\end{theorem}

\begin{proof}
We will show the proof of: 
\begin{equation*}
k \left( \vect{u}+\vect{v}\right) = k \vect{u}+ k \vect{v}
\end{equation*}
Note that:
\begin{equation*}
\begin{array}{ll}
k \left( \vect{u}+\vect{v}\right) & =k \leftB u_{1}+v_{1} \cdots u_{n}+v_{n}\rightB^T \\
& = \leftB k \left( u_{1}+v_{1}\right) \cdots k \left( u_{n}+v_{n}\right) \rightB^T \\
& = \leftB k u_{1}+ k  v_{1} \cdots k u_{n}+ k v_{n}\rightB^T \\
& = \leftB k u_{1} \cdots k u_{n} \rightB^T + \leftB k v_{1} \cdots k v_{n} \rightB^T \\
& = k \vect{u}+k \vect{v} \\
\end{array}
\end{equation*}
\end{proof}

We now present a useful notion you may have seen earlier combining vector addition and scalar multiplication

\begin{definition}{Linear Combination}{linearcombination}
A vector $\vect{v}$ is said to be a \textbf{linear combination }
\index{linear combination} of the vectors $\vect{u}_1,\cdots , \vect{u}_n $ 
if there exist scalars, $a_{1},\cdots ,a_{n}$ such
that
\begin{equation*}
\vect{v} = a_1 \vect{u}_1 + \cdots + a_n \vect{u}_n
\end{equation*}
\end{definition}

For example, 
\begin{equation*}
3
\leftB
\begin{array}{r}
-4 \\
1 \\
0
\end{array}
\rightB
+
2
\leftB
\begin{array}{r}
-3 \\
0\\
1
\end{array}
\rightB
 =
\leftB
\begin{array}{r}
-18 \\
3 \\
2
\end{array}
\rightB. 
\end{equation*}
Thus we can say that
\begin{equation*}
\vect{v}= \leftB
\begin{array}{r}
-18 \\
3 \\
2
\end{array}
\rightB
\end{equation*}
is a linear combination of the vectors 
\begin{equation*}
\vect{u}_1 = \leftB
\begin{array}{r}
-4 \\
1 \\
0
\end{array}
\rightB
\mbox{ and } 
\vect{u}_2 = 
\leftB
\begin{array}{r}
-3 \\
0\\
1
\end{array}
\rightB
\end{equation*}

For the specific case of $\mathbb{R}^3$, there are three special vectors which we often use. 
They are given by 
\begin{equation*}
\vect{i} = 
\leftB
\begin{array}{rrr}
1 & 0 & 0
\end{array}
\rightB^T
\end{equation*}
\begin{equation*}
\vect{j} = 
\leftB
\begin{array}{rrr}
0 & 1 & 0
\end{array}
\rightB^T
\end{equation*}
\begin{equation*}
\vect{k} = 
\leftB
\begin{array}{rrr}
0 & 0 & 1
\end{array}
\rightB^T
\end{equation*}
We can write any vector $\vect{u} = 
\leftB
\begin{array}{rrr}
a_1 & a_2 & a_3
\end{array}
\rightB^T$
as a linear combination of these vectors, written as $\vect{u} = a_1 \vect{i} + a_2 \vect{j} + a_3 \vect{k}$. This notation will be used throughout 
this chapter.

\begin{example}{Determining if Linear Combination}{determinelincomb}
Can $\vect{v}=\leftB \begin{array}{rrr}1 & 3 & 5 \end{array} \rightB^T$ be written as a linear combination of $\vect{u}_1=\leftB \begin{array}{rrr}2 & 2 & 6 \end{array} \rightB^T$, $\vect{u}_2=\leftB \begin{array}{rrr}1 & 6 & 8 \end{array} \rightB^T$ and$\vect{u}_3=\leftB \begin{array}{rrr}3 & 8 & 18 \end{array} \rightB^T$?
\end{example}

\begin{solution}
This question can be rephrased as: can we find $x,y,z$ such that
$$x\vect{u}_1+y\vect{u}_2+z\vect{u}_3=\vect{v}$$
where $x,y,z$ are scalars? Multiplying out produces the system of linear equations
\begin{align*}
2x+y+3z&=1\\
2x+6y+8z&=3\\
6x+8y+18z&=5
\end{align*}
now we row reduce the corresponding augmented matrix to solve
\begin{align*}
&\leftB \begin{array}{rrr|r} 2 & 1 & 3 & 1 \\ 2 & 6 & 8 & 3\\ 6 & 8 & 18 & 5 \end{array}\rightB \begin{matrix} R_1-R_2\to R_1\\ R_3-3R_1\to R_3 \end{matrix}
\leftB \begin{array}{rrr|r} 0 & -5 & -5 & -2 \\ 2 & 6 & 8 & 3 \\ 0 & 5 & 9 & 2 \end{array}\rightB R_1+R_3\to R_1
\leftB \begin{array}{rrr|r} 0 & 0 & 4 & 0 \\ 2 & 6 & 8 & 3 \\ 0 & 5 & 9 & 2 \end{array}\rightB R_2\leftrightarrow R_1 \\
&\leftB \begin{array}{rrr|r} 2 & 6 & 8 & 3 \\ 0 & 0 & 4 & 0 \\ 0 & 5 & 9 & 2 \end{array}\rightB \begin{matrix} \frac{1}{2}R_1 \\ R_2\leftrightarrow R_3 \end{matrix}
\leftB \begin{array}{rrr|r} 1 & 3 & 4 & \frac{3}{2} \\ 0 & 5 & 9 & 2\\  0 & 0 & 4 & 0\end{array}\rightB \frac{1}{4}R_3
\leftB \begin{array}{rrr|r} 1 & 3 & 4 & \frac{3}{2} \\ 0 & 5 & 9 & 2\\  0 & 0 & 1 & 0 \end{array}\rightB \begin{matrix} R_1-4R_3 \\ R_2-9R_3 \end{matrix} \\
& \leftB \begin{array}{rrr|r} 1 & 3 & 0 & \frac{3}{2} \\ 0 & 5 & 0 & 2\\  0 & 0 & 1 & 0 \end{array}\rightB  \frac{1}{5} R_2
\leftB \begin{array}{rrr|r} 1 & 3 & 0 & \frac{3}{2} \\ 0 & 1 & 0 & \frac{2}{5}\\  0 & 0 & 1 & 0 \end{array}\rightB R_1-3R_2
\leftB \begin{array}{rrr|r} 1 & 0 & 0 & \frac{3}{10} \\ 0 & 1 & 0 & \frac{2}{5}\\  0 & 0 & 1 & 0 \end{array}\rightB
\end{align*}
we are in the case where we have a unique solution:
\begin{align*}
x&=\frac{3}{10}\\
y&=\frac{2}{5}\\
z&=0
\end{align*}
this means that $\vect{v}$ is a linear combination of $\vect{u}_1$ and $\vect{u}_2$ only: $\vect{v}=\frac{3}{10}\vect{u}_1+\frac{2}{5}\vect{v}_2$.
\end{solution}

