\subsubsection{The Cholesky factorization}

Another important theorem is the existence of a specific factorization of positive definite matrices. It is called the Cholesky Factorization and factors the matrix into the product of an upper triangular matrix and its transpose.

\index{Cholesky factorization!positive definite}
\index{positive definite!Cholesky factorization} 
\begin{theorem}{Cholesky factorization}{cholesky-factorization}
Let $A$ be a positive definite matrix. Then
there exists an upper triangular matrix $U$ whose main diagonal entries are positive, such that $A$ can be written

\begin{equation*}
A=
U^TU
\end{equation*}
This factorization is unique.
\end{theorem}

The process for finding such a matrix $U$ relies on simple row operations.

\begin{procedure}{Finding the Cholesky factorization}{finding-cholesky}
Let $A$ be a positive definite matrix. The matrix $U$ that creates the Cholesky Factorization can be found through two steps.
\begin{enumerate}
\item Using only type $3$ elementary row operations (multiples of rows added to other rows) put $A$ in upper triangular form. Call this matrix $\hat{U}$. Then $\hat{U}$ has positive entries on the main diagonal. 
\item Divide each row of $\hat{U}$ by the square root of the diagonal entry in that row. The result is the matrix $U$. 
\end{enumerate}
\end{procedure}

Of course you can always verify that your factorization is correct by multiplying $U$ and $U^T$ to ensure the result is the original matrix $A$. 

Consider the following example.

\begin{example}{Cholesky factorization}{cholesky}
Show that
$A=\begin{mymatrix}{rrr}
9 & -6 & 3 \\ -6 & 5 & -3 \\ 3 & -3 & 6 
\end{mymatrix}$ 
is positive definite, and find the Cholesky factorization of $A$.
\end{example}

\begin{solution}
First we show that $A$ is positive definite. By Theorem~\ref{thm:positive-matrix-determinant-Ak} it suffices to show that the determinant of each submatrix is positive. 
\[ A_{1}=\begin{mymatrix}{c} 9 \end{mymatrix}
\mbox{ and }
A_{2}=\begin{mymatrix}{rr} 9 & -6 \\ -6 & 5 \end{mymatrix},\]
so $\det(A_{1})=9$ and $\det(A_{2})=9$.
Since $\det(A)=36$, it follows that $A$ is positive definite.

Now we use Procedure~\ref{proc:finding-cholesky} to find the Cholesky Factorization. Row reduce (using only type $3$ row operations) until an upper triangular matrix is obtained. 
\[ \begin{mymatrix}{rrr}
9 & -6 & 3 \\ -6 & 5 & -3 \\ 3 & -3 & 6
\end{mymatrix}
\rightarrow
\begin{mymatrix}{rrr}
9 & -6 & 3 \\ 0 & 1 & -1 \\ 0 & -1 & 5
\end{mymatrix}
\rightarrow
\begin{mymatrix}{rrr}
9 & -6 & 3 \\ 0 & 1 & -1 \\ 0 & 0 & 4
\end{mymatrix}
\]

Now divide the entries in each row by the square root of the diagonal
entry in that row, to give

\[ U=\begin{mymatrix}{rrr}
3 & -2 & 1 \\ 0 & 1 & -1 \\ 0 & 0 & 2
\end{mymatrix}
\]

You can verify that $U^TU = A$.
\end{solution}

\begin{example}{Cholesky factorization}{cholesky}
Let $A$ be a positive definite matrix given by 
\begin{equation*}
\begin{mymatrix}{ccc}
3 & 1 & 1 \\ 
1 & 4 & 2 \\ 
1 & 2 & 5
\end{mymatrix}
\end{equation*}
Determine its Cholesky factorization.
\end{example}

\begin{solution}
You can verify that $A$ is in fact positive definite. 

To find the Cholesky factorization we first row reduce to an upper triangular matrix. 
\[
\begin{mymatrix}{ccc}
3 & 1 & 1 \\ 
1 & 4 & 2 \\ 
1 & 2 & 5
\end{mymatrix}
\rightarrow
\begin{mymatrix}{ccc}
3 & 1 & 1 \\ 
0 & \vspace{0.05in}\frac{11}{3} & \vspace{0.05in}\frac{5}{3} \\ 
0 & \vspace{0.05in}\frac{5}{3}  & \vspace{0.05in}\frac{14}{5}
\end{mymatrix}
\rightarrow
\begin{mymatrix}{ccc}
3 & 1 & 1 \\ 
0 & \vspace{0.05in}\frac{11}{3} & \vspace{0.05in}\frac{5}{3} \\ 
0 & 0 & \vspace{0.05in}\frac{43}{11}
\end{mymatrix}
\]

Now divide the entries in each row by the square root of the diagonal entry in that row and simplify.
\[
U = \begin{mymatrix}{ccc}
\sqrt{3} & \vspace{0.05in}\frac{1}{3}\sqrt{3} & \vspace{0.05in}\frac{1}{3}\sqrt{3}  \\ 
0  & \vspace{0.05in}\frac{1}{3}\sqrt{3}\sqrt{11} &  \vspace{0.05in}\frac{5}{33}\sqrt{3}\sqrt{11} \\ 

0  & 0 & \vspace{0.05in}\frac{1}{11}\sqrt{11}\sqrt{43}
\end{mymatrix}
\]
\end{solution}