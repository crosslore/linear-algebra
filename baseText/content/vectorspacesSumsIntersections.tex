\section{Sums and intersections}

\begin{outcome}
  \begin{enumerate}
  \item Show that the sum of two subspaces is a subspace.
  \item Show that the intersection of two subspaces is a subspace.
  \end{enumerate}
\end{outcome}

We begin this section with a definition.

\begin{definition}{Sum and intersection}{sum-intersection}
Let $V$ be a vector space, and let $U$ and $W$ be subspaces of
$V$.  
Then
\begin{enumerate}
\item $U+W = \set{\vect{u}+\vect{w} ~|~ \vect{u}\in U\mbox{ and } \vect{w}\in W}$ and is 
called the sum of $U$ and $W$.

\item $U\cap W = \set{\vect{v} ~|~ \vect{v}\in U\mbox{ and } \vect{v}\in W}$ and is 
called the intersection of $U$ and $W$.
\end{enumerate}
\end{definition}

Therefore the intersection of two subspaces is all the vectors shared by both. If there are no vectors shared by both subspaces, meaning that $U \cap W = \set{\vect{0} }$, the sum $U+W$ takes on a special name.

\begin{definition}{Direct sum}{direct-sum}
Let $V$ be a vector space and suppose $U$ and $W$ are subspaces of $V$ such that  $U \cap W = \set{\vect{0} }$. Then the sum of $U$ and $W$ is called the direct sum and is denoted $U \oplus W$. 
\end{definition}

An interesting result is that both the sum $U + W$ and the intersection $U \cap W$ are subspaces of $V$. 

\begin{example}{Intersection is a subspace}{intersection-subspace}
Let $V$ be a vector space and suppose $U$ and $W$ are subspaces. Then the intersection $U \cap W$ is a subspace of $V$.
\end{example}

\begin{solution}
By the subspace test, we must show three things:
\begin{enumerate}
\item $\vect{0} \in U \cap W$
\item For vectors $\vect{v}_1, \vect{v}_2 \in U \cap W, \vect{v}_1+\vect{v}_2 \in U \cap W$
\item For scalar $a$ and vector $\vect{v} \in U \cap W, a\vect{v} \in U \cap W$
\end{enumerate}

We proceed to show each of these three conditions hold.
\begin{enumerate}
\item 
Since $U$ and $W$ are subspaces of $V$, they each contain $\vect{0}$. By definition of the intersection, $\vect{0} \in U \cap W$. 

\item
Let  $\vect{v}_1, \vect{v}_2 \in U \cap W$,. Then in particular,  $\vect{v}_1, \vect{v}_2 \in U$. Since $U$ is a subspace, it follows that $ \vect{v}_1+\vect{v}_2 \in U$. The same argument holds for $W$. Therefore $\vect{v}_1+\vect{v}_2$ is in both $U$ and $W$ and by definition is also in $U \cap W$. 

\item 
Let $a$ be a scalar and $\vect{v} \in U \cap W$. Then in particular, $\vect{v} \in U$. Since $U$ is a subspace, it follows that $a \vect{v} \in U$. The same argument holds for $W$ so $a\vect{v}$ is in both $U$ and $W$. By definition, it is in $U \cap W$. 
\end{enumerate}

Therefore $U \cap W$ is a subspace of $V$. 
\end{solution}

It can also be shown that $U + W$ is a subspace of $V$.

We conclude this section with an important theorem on dimension.

\begin{theorem}{Dimension of sum}{dimension-sum}
Let $V$ be a vector space with subspaces $U$ and $W$. Suppose $U$ and $W$ each have finite dimension. Then $U + W$ also has finite dimension which is given by\[
\func{dim} (U+W) = \func{dim}(U) + \func{dim}(W) - \func{dim} (U \cap W)
\]
\end{theorem}

Notice that when $U \cap W = \set{\vect{0} }$, the sum becomes the direct sum and the above equation becomes 
\[
\func{dim} (U \oplus W) = \func{dim}(U) + \func{dim}(W)
\]
