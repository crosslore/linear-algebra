\subsection{Complex eigenvalues}

In some applications, a matrix may have eigenvalues
\index{complex eigenvalues} which are complex numbers. For example, this often occurs in
differential equations. These questions are approached in the same way as above.

Consider the following example.

\begin{example}{A real matrix with complex eigenvalues}{realmatrixcomplexeigenvalues}
Let 
\begin{equation*}
A=\begin{mymatrix}{rrr}
1 & 0 &  0 \\
0 & 2 & -1 \\
0 & 1 &  2
\end{mymatrix} 
\end{equation*}
Find the eigenvalues and eigenvectors of $A$.
\end{example}

\begin{solution}
We will first find the eigenvalues as usual by solving the following equation. 

\begin{equation*}
\det \left( 
\eigenVar \begin{mymatrix}{rrr}
1 & 0 & 0 \\
0 & 1 & 0 \\
0 & 0 & 1
\end{mymatrix} 
- \begin{mymatrix}{rrr}
1 & 0 &  0 \\
0 & 2 & -1 \\
0 & 1 &  2
\end{mymatrix}  \right) =0
\end{equation*}
This reduces to $ \left( \eigenVar -1\right) \left(
\eigenVar^{2}-4 \eigenVar +5\right) =0.$ The solutions are $\lambda_1
=1,\lambda_2 = 2+i$ and $\lambda_3 =2-i.$

There is nothing new about finding the eigenvectors for $\lambda_1 =1$ so 
this is left as an exercise. 

Consider now the eigenvalue $\lambda_2 =2+i.$ As usual, we solve the equation $\left(\lambda I -A \right) X = 0$ as given by 
\begin{equation*}
\left(
\left( 2+i\right) \begin{mymatrix}{rrr}
1 & 0 & 0 \\
0 & 1 & 0 \\
0 & 0 & 1
\end{mymatrix} - 
\begin{mymatrix}{rrr}
1 & 0 & 0 \\
0 & 2 & -1 \\
0 & 1 & 2
\end{mymatrix}
 \right)
X
 =\begin{mymatrix}{r}
0 \\
0 \\
0
\end{mymatrix}
\end{equation*}
In other words, we need to solve the system represented by the augmented matrix
\begin{equation*}
\begin{mymatrix}{crr|r}
1+i &  0 & 0 & 0 \\
0   &  i & 1 & 0 \\
0   & -1 & i & 0
\end{mymatrix}
\end{equation*}

We now use our row operations to solve the system.
Divide the first row by $\left( 1+i\right) $ and then take 
$-i$ times the second row and add to the third row. This yields
\begin{equation*}
\begin{mymatrix}{rrr|r}
1 & 0 & 0 & 0 \\
0 & i & 1 & 0 \\
0 & 0 & 0 & 0
\end{mymatrix}
\end{equation*}
Now multiply the second row by $-i$ to obtain the {\rref}, given by 
\begin{equation*}
\begin{mymatrix}{rrr|r}
1 & 0 &  0 & 0 \\
0 & 1 & -i & 0 \\
0 & 0 &  0 & 0
\end{mymatrix}
\end{equation*}
Therefore, the eigenvectors are of the form
\begin{equation*}
t\begin{mymatrix}{r}
0 \\
i \\
1
\end{mymatrix} 
\end{equation*}
and the basic eigenvector is given by
\begin{equation*}
X_2 =
\begin{mymatrix}{r}
0 \\
i \\
1
\end{mymatrix}
\end{equation*}

As an exercise, verify that the eigenvectors for $\lambda_3 =2-i$ are of the form 
\begin{equation*}
t\begin{mymatrix}{r}
 0 \\
-i \\
 1
\end{mymatrix} 
\end{equation*}
Hence, the basic eigenvector is given by 
\begin{equation*}
X_3 = \begin{mymatrix}{r}
 0 \\
-i \\
 1
\end{mymatrix} 
\end{equation*}

As usual, be sure to check your answers! To verify, we check that 
$AX_3 = \left(2 - i \right) X_3$ as follows.
\begin{equation*}
\begin{mymatrix}{rrr}
1 & 0 &  0 \\
0 & 2 & -1 \\
0 & 1 &  2
\end{mymatrix} \begin{mymatrix}{r}
0 \\
-i \\
1
\end{mymatrix} = \begin{mymatrix}{c}
0 \\
-1-2i \\
2-i
\end{mymatrix} =\left( 2-i\right) \begin{mymatrix}{r}
0 \\
-i \\
1
\end{mymatrix}
\end{equation*}

Therefore, we know that this eigenvector and eigenvalue are correct. 
\end{solution}

Notice that in Example \ref{exa:realmatrixcomplexeigenvalues}, two of the eigenvalues were given by 
$\lambda_2 = 2 + i$ and $\lambda_3 = 2-i$. You may recall that these two complex numbers are \textbf{conjugates}. 
It turns out that whenever a matrix containing real entries has a complex eigenvalue $\lambda$, it also has an eigenvalue
equal to $\overline{\lambda}$, the conjugate of $\lambda$. 
