\section{Application: Systems of linear differential equations}

This type of behavior along with complex eigenvalues is typical of the
deviations from an equilibrium point in the Lotka Volterra system of
differential equations which is a famous model for predator-prey
interactions. These differential equations are given by
\begin{eqnarray*}
  x^{\prime } &=&x(a-by) \\
  y^{\prime } &=&-y(c-dx)
\end{eqnarray*}
where $a,b,c,d$ are positive constants. For example, you might have
$X$ be the population of moose and $Y$ the population of wolves on an
island.

Note that these equations make logical sense. The top says that the
rate at which the moose population increases would be $aX$ if there
were no predators $Y$.  However, this is modified by multiplying
instead by $(a-bY) $ because if there are predators, these will
militate against the population of moose.  The more predators there
are, the more pronounced is this effect. As to the predator equation,
you can see that the equations predict that if there are many prey
around, then the rate of growth of the predators would seem to be
high. However, this is modified by the term $-cY$ because if there are
many predators, there would be competition for the available food
supply and this would tend to decrease $Y^{\prime }$.

The behavior near an equilibrium point, which is a point where the
right side of the differential equations equals zero, is of great
interest. In this case, the equilibrium point is
\begin{equation*}
  x=\frac{c}{d}, y=\frac{a}{b}.
\end{equation*}
Then one defines new variables according to the formula
\begin{equation*}
  x+\frac{c}{d}=x,\ y=y+\frac{a}{b}.
\end{equation*}
In terms of these new variables, the differential equations become
\begin{eqnarray*}
  x^{\prime } &=&\paren{x+\frac{c}{d}} \paren{a-b\paren{y+\frac{a}{b}
                  }} \\
  y^{\prime } &=&-\paren{y+\frac{a}{b}} \paren{c-d\paren{x+\frac{c}{d}
                  }}.
\end{eqnarray*}
Multiplying out the right sides yields
\begin{eqnarray*}
  x^{\prime } &=&-bxy-b\frac{c}{d}y, \\
  y^{\prime } &=&dxy+\frac{a}{b}dx.
\end{eqnarray*}
The interest is for $x,y$ small and so these equations are essentially
equal to
\begin{equation*}
  x^{\prime }=-b\frac{c}{d}y,\ y^{\prime }=\frac{a}{b}dx.
\end{equation*}
Replace $x^{\prime }$ with the difference quotient
$\frac{x(t+h) -x(t) }{h}$ where $h$ is a small positive number and
$y^{\prime } $ with a similar difference quotient. For example one
could have $h$ correspond to one day or even one hour. Thus, for $h$
small enough, the following would seem to be a good approximation to
the differential equations.
\begin{eqnarray*}
  x(t+h) &=&x(t) -hb\frac{c}{d}y, \\
  y(t+h) &=&y(t) +h\frac{a}{b}dx.
\end{eqnarray*}
Let $1,2,3,\ldots$ denote the ends of discrete intervals of time
having length $h$ chosen above. Then the above equations take the form
\begin{equation*}
  \begin{mymatrix}{c}
    x(n+1) \\
    y(n+1)
  \end{mymatrix} =\begin{mymatrix}{cc}
    1 & -\frac{hbc}{d} \\
    \frac{had}{b} & 1
  \end{mymatrix} \begin{mymatrix}{c}
    x(n) \\
    y(n).
  \end{mymatrix}
\end{equation*}
Note that the eigenvalues of this matrix are always complex.

We are not interested in time intervals of length $h$ for $h$ very
small.  Instead, we are interested in much longer lengths of
time. Thus, replacing the time interval with $mh$,
\begin{equation*}
  \begin{mymatrix}{c}
    x(n+m) \\
    y(n+m)
  \end{mymatrix} =\begin{mymatrix}{cc}
    1 & -\frac{hbc}{d} \\
    \frac{had}{b} & 1
  \end{mymatrix} ^{m}\begin{mymatrix}{c}
    x(n) \\
    y(n)
  \end{mymatrix}.
\end{equation*}
For example, if $m=2$, you would have
\begin{equation*}
  \begin{mymatrix}{c}
    x(n+2) \\
    y(n+2)
  \end{mymatrix} =\begin{mymatrix}{cc}
    1-ach^{2} & -2b\frac{c}{d}h \\
    2\frac{a}{b}dh & 1-ach^{2}
  \end{mymatrix} \begin{mymatrix}{c}
    x(n) \\
    y(n)
  \end{mymatrix}.
\end{equation*}
Note that most of the time, the eigenvalues of the new matrix will be
complex.

You can also notice that the upper right corner will be negative by
considering higher powers of the matrix. Thus letting $1,2,3,\ldots$
denote the ends of discrete intervals of time, the desired discrete
dynamical system is of the form
\begin{equation*}
  \begin{mymatrix}{c}
    x(n+1) \\
    y(n+1)
  \end{mymatrix} =\begin{mymatrix}{rr}
    a & -b \\
    c & d
  \end{mymatrix} \begin{mymatrix}{c}
    x(n) \\
    y(n)
  \end{mymatrix},
\end{equation*}
where $a,b,c,d$ are positive constants and the matrix will likely have
complex eigenvalues because it is a power of a matrix which has
complex eigenvalues.

You can see from the above discussion that if the eigenvalues of the
matrix used to define the dynamical system are less than 1 in absolute
value, then the origin is stable in the sense that as
$n\rightarrow \infty$, the solution converges to the origin. If either
eigenvalue is larger than 1 in absolute value, then the solutions to
the dynamical system will usually be unbounded, unless the initial
condition is chosen very carefully. The next example exhibits the case
where one eigenvalue is larger than 1 and the other is smaller than 1.

The following example demonstrates a familiar concept as a dynamical system.
