\section{The dot product}

\begin{outcome}

\begin{enumerate}
\item[A.] Compute the dot product of vectors, and use this to compute vector projections.
\end{enumerate}
\end{outcome}

There are two ways of multiplying vectors which are of great importance in
applications. The first of these is called the \textbf{dot product}. When we take the dot product of vectors, the 
result is a scalar. For this reason, the dot product is also
called the \textbf{scalar product }and sometimes the \textbf{inner product}. The definition is as follows.
\index{dot product}
\index{scalar product}
\index{inner product}

\begin{definition}{Dot product}{dotproductdefn}
Let $\vect{u}=\begin{mymatrix}{r}
u_1 \\
u_2 \\
\vdots \\
u_n 
\end{mymatrix} ,\vect{v}= \begin{mymatrix}{r}
v_1 \\
v_2 \\
\vdots \\
v_n 
\end{mymatrix}$ be two vectors in $\R^{n}$. Then we
define the \textbf{dot product}  $\vect{u}\dotprod \vect{v}$ as
\begin{equation*}
\vect{u}\dotprod \vect{v} = u_1v_1+u_2v_2+\cdots+u_nv_n
\end{equation*}
\end{definition}

The dot product $\vect{u}\dotprod \vect{v}$ is sometimes denoted as $(\vect{u},\vect{v})$ where a comma replaces $\dotprod $.
%%% Move to later %%%% It can also be written as $\left\langle \vect{u},\vect{v}\right\rangle $. If we write the vectors as column or row matrices, it is equal to the matrix product $\vect{v}\vect{w}^{T}$.

Consider the following example.

\begin{example}{Compute a dot product}{dotproduct}
Find $\vect{u} \dotprod \vect{v}$ for
\begin{equation*}
\vect{u}
=
\begin{mymatrix}{r}
1 \\
2 \\
0 \\
-1 
\end{mymatrix},
\vect{v}
=
\begin{mymatrix}{r}
0 \\
1 \\
2 \\
3
\end{mymatrix} 
\end{equation*}
\end{example}

\begin{solution}
By Definition \ref{def:dotproductdefn}, this is given by 
\begin{eqnarray*}
\vect{u} \dotprod \vect{v}
&=&
(1)(0) + (2)(1) + (0)(2) + (-1)(3) \\
&=&
0 + 2 + 0 + -3 \\
&=&
-1
\end{eqnarray*}
\end{solution}

With this definition, there are several important properties satisfied
by the dot product.
\index{dot product!properties}

\begin{proposition}{Properties of the dot product}{propertiesdotproduct}
Let $k $ and $p $  denote scalars and $\vect{u},\vect{v},\vect{w}$ denote vectors.
Then the dot product $\vect{u} \dotprod \vect{v}$ satisfies the following properties.
\begin{itemize}
\item
$\vect{u}\dotprod \vect{v}= \vect{v}\dotprod \vect{u} $
\item
$\vect{u}\dotprod \vect{u}\geq 0 \text{ and equals zero if and only if }\vect{u}=\vect{0}$
\item
$\left( k\vect{u}+p\vect{v}\right) \dotprod \vect{w}=k\left( \vect{u}\dotprod \vect{w}\right) +p\left( \vect{v}\dotprod \vect{w}\right)$
\item
$\vect{u}\dotprod\left( k\vect{v}+p\vect{w}\right) =k\left(
\vect{u}\dotprod \vect{v}\right) +p\left( \vect{u}\dotprod \vect{w}\right)$
\item
$\vect{u}\dotprod \vect{u}=\vectlength \vect{u}\vectlength ^{2} $
\end{itemize}
\end{proposition}

The proof is left as an exercise. This proposition tells us that we can also use the dot product to find the length of a vector.

\begin{example}{Length of a vector}{dotproductlength}
Find the length of
\begin{equation*}
\vect{u}
=
\begin{mymatrix}{r}
2 \\
1 \\
4 \\
2
\end{mymatrix} 
\end{equation*}
 That is, find $\vectlength \vect{u} \vectlength .$
\end{example}

\begin{solution}
By Proposition \ref{prop:propertiesdotproduct},  $\vectlength \vect{u} \vectlength ^{2} = \vect{u} \dotprod \vect{u}$. 
Therefore, $\vectlength \vect{u} \vectlength = \sqrt {\vect{u} \dotprod \vect{u}}$.
First, compute $\vect{u} \dotprod \vect{u}$. 

This is given by 
\begin{eqnarray*}
\vect{u} \dotprod \vect{u}
&=&
(2)(2) + (1)(1) + (4)(4) + (2)(2) \\
&=&
4 + 1 + 16 + 4 \\
&=&
25
\end{eqnarray*}

Then, 
\begin{eqnarray*}
\vectlength \vect{u} \vectlength 
&=& \sqrt {\vect{u} \dotprod \vect{u}} \\
&=& \sqrt{25} \\
&=& 5
\end{eqnarray*}
\end{solution}

You may wish to compare this to our previous definition of length, 
given in Definition \ref{def:lengthofvector}. 

The \textbf{Cauchy Schwarz inequality} is a fundamental inequality satisfied by the dot product. 
\index{Cauchy Schwarz inequality} It is given in the following theorem.

\begin{theorem}{Cauchy Schwarz inequality}{cauchyschwarzinequality}
The dot product satisfies the inequality
\begin{equation}
\left\vert \vect{u}\dotprod \vect{v}\right\vert \leq \vectlength \vect{u}\vectlength \vectlength \vect{v}\vectlength   \label{cauchy}
\end{equation}
Furthermore equality is obtained if and only if one of $\vect{u}$ or $\vect{v}$ is a scalar multiple of the other.
\end{theorem}

\begin{proof}
First note that if $\vect{v}=\vect{0}$ both sides of \ref{cauchy}
equal zero and so the inequality holds in this case. Therefore, it will be
assumed in what follows that $\vect{v}\neq \vect{0}$.

Define a function of $t\in \R$ by 
\begin{equation*}
f\left( t\right) =\left( \vect{u}+t\vect{v}\right) \dotprod \left( \vect{u}+
t\vect{v}\right) 
\end{equation*}
Then by Proposition \ref{prop:propertiesdotproduct}, $f\left( t\right) \geq 0$ for all $t\in \R$.
Also from Proposition \ref{prop:propertiesdotproduct}
\begin{eqnarray*}
f\left( t\right) &=&\vect{u}\dotprod \left( \vect{u}+t\vect{v}\right) +
t\vect{v}\dotprod \left( \vect{u}+t\vect{v}\right) \\
&=&\vect{u}\dotprod \vect{u}+t\left( \vect{u}\dotprod \vect{v}\right) + t \vect{v}\dotprod \vect{u}+
t^{2}\vect{v}\dotprod \vect{v} \\
&=&\vectlength \vect{u}\vectlength ^{2}+2t\left( \vect{u}\dotprod \vect{v}\right) +\vectlength
\vect{v}\vectlength ^{2}t^{2}
\end{eqnarray*}

Now this means the graph of $y=f\left( t\right) $ is a parabola which opens
up and either its vertex touches the $t$ axis or else the entire graph is
above the $t$ axis. In the first case, there exists some $t$ where $f\left(
t\right) =0$ and this requires $\vect{u}+t\vect{v}=\vect{0}$ so one vector is a
multiple of the other. Then clearly equality holds in \ref{cauchy}. In the
case where $\vect{v}$ is not a multiple of $\vect{u}$, it follows 
$f\left( t\right) >0$ for all $t$ which says $f\left( t\right) $ has no real
zeros and so from the quadratic formula,
\begin{equation*}
\left( 2\left( \vect{u}\dotprod \vect{v}\right) \right) ^{2}-4\vectlength \vect{u}
\vectlength ^{2}\vectlength \vect{v}\vectlength ^{2}<0
\end{equation*}
which is equivalent to $\left\vert \vect{u}\dotprod \vect{v} \right\vert
<\vectlength \vect{u}\vectlength \vectlength \vect{v}\vectlength $.
\end{proof}

Notice that this proof was based only on the properties of
the dot product listed in Proposition \ref{prop:propertiesdotproduct}. This means that
whenever an operation satisfies these properties, the Cauchy Schwarz inequality
holds. There are many other instances of these properties besides vectors in
$\R^{n}$.

The Cauchy Schwarz inequality provides another proof of the \textbf{triangle
inequality} for
\index{triangle inequality}distances in $\R^{n}$.

\begin{theorem}{Triangle inequality}{triangleinequalitydotproduct}
  For $\vect{u},\vect{v}\in \R^{n}$
\begin{equation}
\vectlength \vect{u}+\vect{v}\vectlength \leq \vectlength \vect{u}\vectlength +\vectlength \vect{v}
\vectlength  \label{triangleineq1}
\end{equation}
and equality holds if and only if one of the vectors is a non-negative scalar
multiple of the other. 

Also
\begin{equation}
\big| \vectlength \vect{u}\vectlength -\vectlength \vect{v}\vectlength \big| \leq
\vectlength \vect{u}-\vect{v}\vectlength  \label{triangleineq2}
\end{equation}
\end{theorem}

\begin{proof} By properties of the dot product and the Cauchy Schwarz
inequality,
\begin{eqnarray*}
\vectlength \vect{u}+\vect{v}\vectlength ^{2} &=& \left( \vect{u}+\vect{v}\right) \dotprod \left( \vect{u}+\vect{v}\right) \\
& =&\left( \vect{u}\dotprod \vect{u}\right) +\left( \vect{u}\dotprod \vect{v}\right) +\left(\vect{v}\dotprod \vect{u}\right) +\left( \vect{v}\dotprod \vect{v}\right) \\
&=&\vectlength \vect{u}\vectlength ^{2}+2\left( \vect{u}\dotprod \vect{v}\right)+\vectlength \vect{v}\vectlength ^{2} \\
&\leq& \vectlength \vect{u}\vectlength ^{2}+2\left\vert \vect{u}\dotprod \vect{v}\right\vert +\vectlength \vect{v}\vectlength ^{2} \\
&\leq& \vectlength \vect{u}\vectlength ^{2}+2\vectlength \vect{u}\vectlength \vectlength \vect{v}\vectlength +\vectlength \vect{v}\vectlength ^{2} =\left( \vectlength \vect{u}\vectlength +\vectlength \vect{v}\vectlength\right) ^{2}
\end{eqnarray*}
Hence,
\begin{equation*}
\vectlength \vect{u}+\vect{v}\vectlength ^{2} \leq \left( \vectlength \vect{u}\vectlength +\vectlength \vect{v}\vectlength
\right) ^{2}
\end{equation*}
Taking square roots of both sides you obtain \ref{triangleineq1}.

It remains to consider when equality occurs. Suppose $\vect{u} = \vect{0}$.
Then, $\vect{u} = 0 \vect{v}$ and the claim about when
equality occurs is verified. The same argument holds if $\vect{v} = \vect{0}$. 
 Therefore, it can be assumed both vectors are
nonzero. To get equality in \ref{triangleineq1} above, Theorem \ref{thm:cauchyschwarzinequality} 
implies one of the vectors must be a multiple of
the other. Say $\vect{v}= k \vect{u}$. If $k <0$ then equality
cannot occur in \ref{triangleineq1} because in this case
\begin{equation*}
\vect{u}\dotprod \vect{v} =k \vectlength \vect{u}\vectlength
^{2}<0<\left| k \right| \vectlength \vect{u}\vectlength ^{2}=\left| \vect{u}\dotprod \vect{v}\right|
\end{equation*}
Therefore, $k \geq 0.$

To get the other form of the triangle inequality write 
\begin{equation*}
\vect{u}=\vect{u}-\vect{v}+\vect{v}
\end{equation*}
so
\begin{align*}
\vectlength \vect{u}\vectlength & =\vectlength \vect{u}-\vect{v}+\vect{v}\vectlength \\
& \leq \vectlength \vect{u}-\vect{v}\vectlength +\vectlength \vect{v}\vectlength 
\end{align*}
Therefore,
\begin{equation}
\vectlength \vect{u}\vectlength -\vectlength \vect{v}\vectlength \leq \vectlength \vect{u}-\vect{v}
\vectlength  \label{triangleineq3}
\end{equation}
Similarly,
\begin{equation}
\vectlength \vect{v}\vectlength -\vectlength \vect{u}\vectlength \leq \vectlength \vect{v}-\vect{u}
\vectlength =\vectlength \vect{u}-\vect{v}\vectlength  \label{triangleineq4}
\end{equation}
It follows from \ref{triangleineq3} and \ref{triangleineq4} that \ref{triangleineq2} holds. This
is because $\big| \vectlength \vect{u}\vectlength -\vectlength \vect{v}\vectlength
\big| $ equals the left side of either \ref{triangleineq3} or \ref{triangleineq4} and
either way, $\big| \vectlength \vect{u}\vectlength -\vectlength \vect{v}\vectlength
\big| \leq \vectlength \vect{u}-\vect{v}\vectlength $. 
\end{proof}