\subsection{Polynomial Interpolation}

In studying a set of data that relates variables $x$ and $y$, it may be the case that we can use a polynomial to ``fit'' to the data. If such a polynomial can be established, it can be used to estimate values of $x$ and $y$ which have not been provided. 

Consider the following example.

\begin{example}{Polynomial interpolation}{polynomialinterpolation}
Given data points $(1,4), (2,9), (3,12)$, find an interpolating polynomial $p(x)$ of degree at most $2$ and then estimate the value corresponding to $x = \frac{1}{2}$. 
\end{example}

\begin{solution}
We want to find a polynomial given by 
\[
p(x) = r_0 + r_1x_1 + r_2x_2^2
\]
such that $p(1)=4, p(2)=9$ and $p(3)=12$. 
To find this polynomial, substitute the known values in for $x$ and solve for $r_0, r_1$, and $r_2$. 
\begin{eqnarray*}
p(1) &=& r_0 + r_1 + r_2 = 4\\
p(2) &=& r_0 + 2r_1 + 4r_2 = 9\\
p(3) &=& r_0 + 3r_1 + 9r_2 = 12
\end{eqnarray*}

Writing the augmented matrix, we have
\[
\leftB
\begin{array}{rrr|r}
1 & 1 & 1 & 4 \\
1 & 2 & 4 & 9  \\
1 & 3 & 9 & 12 
\end{array}
\rightB
\]

After row operations, the resulting matrix is
\[
\leftB
\begin{array}{rrr|r}
1 & 0 & 0 & -3 \\
0 & 1 & 0 & 8 \\
0 & 0 & 1 & -1 
\end{array}
\rightB
\]

Therefore the solution to the system is $r_0 = -3, r_1 = 8, r_2 = -1$ and the required interpolating polynomial is 
\[
p(x) = -3 + 8x - x^2
\]

To estimate the value for $x = \frac{1}{2}$, we calculate $p(\frac{1}{2})$: 
\begin{eqnarray*}
p(\frac{1}{2}) &=& -3 + 8(\frac{1}{2}) - (\frac{1}{2})^2\\
&=& -3 + 4 - \frac{1}{4} \\
&=& \frac{3}{4}
\end{eqnarray*}
\end{solution}

This procedure can be used for any number of data points, and any degree of polynomial. The steps are outlined below.

\begin{procedure}{Finding an Interpolating Polynomial}{findinginterpolynomial}
Suppose that values of $x$ and corresponding values of $y$ are given, such that the actual relationship between $x$ and $y$ is unknown. Then, values of $y$ can be estimated using an \textbf{interpolating polynomial $p(x)$}. If given $x_1, ..., x_n$ and the corresponding $y_1, ..., y_n$, the procedure to find $p(x)$ is as follows:
\begin{enumerate}
\item The desired polynomial $p(x)$ is given by 
\[
p(x) = r_0 + r_1 x + r_2 x^2 + ... + r_{n-1}x^{n-1}
\]
\item $p(x_i) = y_i$ for all $i = 1, 2, ...,n$ so that
\[
\begin{array}{c}
r_0 + r_1x_1 + r_2 x_1^2 + ... + r_{n-1}x_1^{n-1} = y_1 \\
r_0 + r_1x_2 + r_2 x_2^2 + ... + r_{n-1}x_2^{n-1} = y_2 \\
\vdots \\
r_0 + r_1x_n + r_2 x_n^2 + ... + r_{n-1}x_n^{n-1} = y_n 
\end{array}
\]
\item Set up the augmented matrix of this system of equations
\[
\leftB
\begin{array}{rrrrr|r}
1 & x_1 & x_1^2 & \cdots & x_1^{n-1} & y_1 \\
1 & x_2 & x_2^2 & \cdots & x_2^{n-1} & y_2 \\
\vdots & \vdots & \vdots & &\vdots & \vdots \\
1 & x_n & x_n^2 & \cdots & x_n^{n-1} & y_n \\
\end{array}
\rightB
\]

\item Solving this system will result in a unique solution $r_0, r_1, \cdots, r_{n-1}$. Use these values to construct $p(x)$, and estimate the value of $p(a)$ for any $x=a$. 
\end{enumerate}

\end{procedure}

This procedure motivates the following theorem.

\begin{theorem}{Polynomial interpolation}{polynomialinterpolation}
Given $n$ data points $(x_1, y_1), (x_2, y_2), \cdots, (x_n, y_n)$ with the $x_i$ distinct, there is a unique polynomial $p(x) = r_0 + r_1x + r_2x^2 + \cdots + r_{n-1}x^{n-1}$ such that $p(x_i) = y_i$ for $i=1,2,\cdots, n$. The resulting polynomial $p(x)$ is called the \textbf{interpolating polynomial} for the data points. 
\end{theorem}

We conclude this section with another example. 

\begin{example}{Polynomial interpolation}{polynomialinterpolation2}
Consider the data points $(0,1), (1,2), (3,22), (5,66)$. Find an interpolating polynomial $p(x)$ of degree at most three, and estimate the value of $p(2)$. 
\end{example}

\begin{solution}
The desired polynomial $p(x)$ is given by:
\[
p(x) = r_0 + r_1 x + r_2x^2 + r_3x^3
\]

Using the given points, the system of equations is
\begin{eqnarray*}
p(0) &=& r_0 = 1 \\
p(1) &=& r_0 + r_1 + r_2 + r_3 = 2 \\
p(3) &=& r_0 + 3r_1 + 9r_2 + 27r_3 = 22 \\
p(5) &=& r_0 + 5r_1 + 25r_2 + 125r_3 = 66
\end{eqnarray*}

The augmented matrix is given by:
\[
\leftB
\begin{array}{rrrr|r}
1 & 0 & 0 & 0 & 1 \\
1 & 1 & 1 & 1 & 2 \\
1 & 3 & 9 & 27 & 22 \\
1 & 5 & 25 & 125 & 66
\end{array}
\rightB
\]

The resulting matrix is 
\[
\leftB
\begin{array}{rrrr|r}
1 & 0 & 0 & 0 & 1 \\
0 & 1 & 0 & 0 & -2 \\
0 & 0 & 1 & 0 & 3 \\
0 & 0 & 0 & 1 & 0
\end{array}
\rightB
\]

Therefore, $r_0 = 1, r_1 = -2, r_2 = 3, r_3 = 0$ and $p(x) = 1 -2x + 3x^2$. To estimate the value of $p(2)$, we compute $p(2) = 1 -2(2) + 3(2^2) = 1 - 4 + 12 = 9$.
\end{solution}