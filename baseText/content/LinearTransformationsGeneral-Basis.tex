\section{Linear transformations defined on a basis}

\begin{outcome}
  \begin{enumerate}
  \item Check whether two linear transformations are equal by
    considering their action on a spanning set.
  \item Specify a linear transformation by considering its action on a
    basis.
  \end{enumerate}
\end{outcome}

Recall that, by definition, two linear transformations $S,T:V\to W$
are equal if and only if for all $\vect{v}\in V$, we have
$S(\vect{v}) = T(\vect{v})$.  However, this is not a very practical
way of checking whether $S=T$, as it theoretically requires checking
$S(\vect{v}) = T(\vect{v})$ for each one of infinitely many vectors
$\vect{v}$. The following proposition states that it is sufficient to
check the actions of $S$ and $T$ on a spanning set of vectors.

\begin{proposition}{Equality of linear transformations}{transformation-spanning-set}
  Let $V$ and $W$ be vector spaces over a field $K$, and let
  $S,T:V\to W$ be linear transformations. Moreover, let $X\subseteq V$
  be a spanning set of $V$, i.e., such that $V=\sspan X$.
  If $S(\vect{v}) = T(\vect{v})$ for all $\vect{v}\in X$, then $S=T$.%
  \index{linear transformation!equality of}%
  \index{equality!of linear transformations}
\end{proposition}

\begin{proof}
  Assume that $S(\vect{v}) = T(\vect{v})$ holds for all $\vect{v}\in X$. To
  show that $S=T$, let $\vect{u}\in V$ be an arbitrary vector. Since
  $X$ is a spanning set, we can write
  $\vect{u}=a_1\vect{v}_1+\ldots+a_n\vect{v}_n$, for some
  $\vect{v}_1,\ldots,\vect{v}_n\in X$ and $a_1,\ldots,a_n\in K$.
  By assumption, $S(\vect{v}_i) = T(\vect{v}_i)$ for all $i$, because
  $\vect{v}_i\in X$. Then we have
  \begin{eqnarray*}
    S(\vect{u})
    &=& S(a_1\vect{v}_1+\ldots+a_n\vect{v}_n) \\
    &=& a_1S(\vect{v}_1)+\ldots+a_nS(\vect{v}_n) \\
    &=& a_1T(\vect{v}_1)+\ldots+a_nT(\vect{v}_n) \\
    &=& T(a_1\vect{v}_1+\ldots+a_n\vect{v}_n) \\
    &=& T(\vect{u}).
  \end{eqnarray*}
  Since $\vect{u}\in V$ was arbitrary, it follows that $S=T$.
\end{proof}

Therefore, if we know how a linear transformation acts on a spanning
set (and in particular, on a basis), then we know how it acts on the
entire space. There is also a kind of converse to this: given a basis
of $V$, we can map the basis vectors to any elements of $W$ we like,
and this will always determine a unique linear transformations. This
is the content of the following theorem.

\begin{theorem}{Linear transformation defined on a basis}{transformation-basis}
  Suppose $V$ and $W$ are vector spaces over a field $K$. Suppose
  $\vect{v}_1,\vect{v}_2,\ldots,\vect{v}_n$ is a basis of $V$, and
  $\vect{w}_1,\vect{w}_2,\ldots,\vect{w}_n$ are any vectors in $W$
  (which may or may not be distinct). Then there exists a unique
  linear transformation%
  \index{linear transformation!defined on a basis}
  $T:V\to W$ such that
  \begin{equation*}
    T(\vect{v}_1) = \vect{w}_1,
    \quad
    T(\vect{v}_2) = \vect{w}_2,
    \quad\ldots\quad
    T(\vect{v}_n) = \vect{w}_n.
  \end{equation*}
\end{theorem}

\begin{proof}
  To show that such a linear transformation $T$ exists, we first
  define a function $T:V\to W$ as follows. Given any $\vect{v}\in V$,
  there exists a unique set of coordinates $a_1,\ldots,a_n\in K$ such
  that
  \begin{equation*}
    \vect{v} = a_1\vect{v}_1 + \ldots + a_n\vect{v}_n.
  \end{equation*}
  Then define
  \begin{equation*}
    T(\vect{v}) = a_1\vect{w}_1 + \ldots + a_n\vect{w}_n.
  \end{equation*}
  This defines a function $T:V\to W$.  Next, we must check that $T$ is
  linear. To show that $T$ preserves addition, consider
  $\vect{v},\vect{v}'\in V$, with
  $\vect{v} = a_1\vect{v}_1 + \ldots + a_n\vect{v}_n$ and
  $\vect{v}' = b_1\vect{v}_1 + \ldots + b_n\vect{v}_n$.  Then
  \begin{eqnarray*}
    T(\vect{v}+\vect{v}')
    &=& T((a_1+b_1)\vect{v}_1 + \ldots + (a_n+b_n)\vect{v}_n) \\
    &=& (a_1+b_1)\vect{w}_1 + \ldots + (a_n+b_n)\vect{w}_n \\
    &=& (a_1\vect{w}_1 + \ldots + a_n\vect{w}_n)
        + (b_1\vect{w}_1 + \ldots + b_n\vect{w}_n) \\
    &=& T(\vect{v}) + T(\vect{v}').
  \end{eqnarray*}
  Therefore, $T$ preserves addition. To show that $T$ preserves scalar
  multiplication, consider $\vect{v} = a_1\vect{v}_1 + \ldots +
  a_n\vect{v}_n$ and $k\in K$. Then
  \begin{eqnarray*}
    T(k\vect{v})
    &=& T(ka_1\vect{v}_1 + \ldots + ka_n\vect{v}_n) \\
    &=& ka_1\vect{w}_1 + \ldots + ka_n\vect{w}_n \\
    &=& k(a_1\vect{w}_1 + \ldots + a_n\vect{w}_n) \\
    &=& kT(\vect{v}).
  \end{eqnarray*}
  Therefore, $T$ preserves scalar multiplication. It follows that $T$
  is linear. Next, we must show that $T$ satisfies the condition of
  the theorem, i.e., that $T(\vect{v}_i) = \vect{w}_i$ for each $i$.
  But this is clearly the case, because in this case, $a_i=1$ and
  $a_j=0$ for all $j\neq i$. We have shown that there exists a linear
  function $T$ satisfying all of the conditions required by the
  theorem.

  Finally, the only thing left to show is uniqueness. But this follows
  from Proposition~\ref{prop:transformation-spanning-set}. Namely, if
  $T'$ is another linear transformation such that
  $T'(\vect{v}_i) = \vect{w}_i$ for all $i$, then $T$ and $T'$ agree
  on $\vect{v}_1,\ldots,\vect{v}_n$, which is a basis and hence a
  spanning set. By Proposition~\ref{prop:transformation-spanning-set},
  $T=T'$.
\end{proof}

\begin{example}{Linear transformation defined on a basis}{transformation-basis}
  Recall that $\set{x^2, (x+1)^2, (x+2)^2}$ is a basis of $\Poly_2$.
  Consider the linear function $T:\Poly_2\to\Mat_{22}$ defined by
  \begin{equation*}
    T(x^2) = \begin{mymatrix}{rr} 1 & 1 \\ 0 & 0 \end{mymatrix},\quad
    T((x+1)^2) = \begin{mymatrix}{rr} 0 & 1 \\ 0 & 1 \end{mymatrix},\quad
    T((x+2)^2) = \begin{mymatrix}{rr} 0 & 0 \\ 1 & 1 \end{mymatrix}.
  \end{equation*}
  Find $T(4x)$.
\end{example}

\begin{solution}
  Let $\vect{v}_1=x^2$, $\vect{v}_2=(x+1)^2$, $\vect{v}_3=(x+2)^2$,
  $\vect{w}_1 = \begin{mysmallmatrix}{rr} 1 & 1 \\ 0 & 0 \end{mysmallmatrix}$,
  $\vect{w}_2 = \begin{mysmallmatrix}{rr} 0 & 1 \\ 0 & 1 \end{mysmallmatrix}$,
  and
  $\vect{w}_3 = \begin{mysmallmatrix}{rr} 0 & 0 \\ 1 & 1 \end{mysmallmatrix}$.
  We must first find $a,b,c$ such that
  $4x=a\vect{v}_1+b\vect{v}_2+c\vect{v}_3$. We do this by solving a
  system of equations, using the same method as in
  Example~\ref{exa:linear-combination-polynomials}. We find that
  $a=-3$, $b=4$, and $c=-1$.  Therefore
  \begin{equation*}
    T(4x)
    ~=~ T(-3\vect{v}_1 + 4\vect{v}_2 - \vect{v}_3)
    ~=~ -3\vect{w}_1 + 4\vect{w}_2 - \vect{w}_3
    ~=~ \begin{mymatrix}{rr}
      -3 & 1 \\
      -1 & 3 \\
    \end{mymatrix}.
  \end{equation*}
\end{solution}

