\subsubsection{Power methods}

While the $QR$ algorithm can be used to compute eigenvalues, there is a useful and fairly elementary technique for
finding the eigenvector and associated eigenvalue nearest to a given complex number which
is called the {\em shifted inverse power method}. It tends to work extremely well provided you
start with something which is fairly close to an eigenvalue.

Power methods are based the consideration of powers of a given matrix. Let $
\set{
\vect{x}_{1},\cdots ,\vect{x}_{n}} $ be a basis of eigenvectors for $
\C^{n}$ such that $A\vect{x}_{n}=\lambda _{n}\vect{x}_{n}$. Now let $
\vect{u}_{1}$ be some non-zero vector. Since $\set{\vect{x}_{1},\cdots ,\vect{x}_{n}} $ is a basis, there exists unique scalars, $c_{i}$ such that 
\begin{equation*}
\vect{u}_{1}=\sum_{k=1}^{n}c_{k}\vect{x}_{k}
\end{equation*}
Assume you have not been so unlucky as to pick $\vect{u}_{1}$ in such a way
that $c_{n}=0$. Then let $A\vect{u}_{k}=\vect{u}_{k+1}$ so that 
\begin{equation}
\vect{u}_{m}=A^{m}\vect{u}_{1}=\sum_{k=1}^{n-1}c_{k}\lambda _{k}^{m}\vect{x}
_{k}+\lambda _{n}^{m}c_{n}\vect{x}_{n}.  \label{20-may-e1}
\end{equation}
For large $m$ the last term, $\lambda _{n}^{m}c_{n}\vect{x}_{n}$, determines
quite well the direction of the vector on the right. This is because $
\abs{\lambda _{n}}$ is larger than $\abs{\lambda
_{k}}$ for $k<n$ and so for a large $m$, the sum, $\sum_{k=1}^{n-1}c_{k}\lambda _{k}^{m}\vect{x}_{k}$, on the right is fairly
insignificant. Therefore, for large $m$, $\vect{u}_{m}$ is essentially a
multiple of the eigenvector $\vect{x}_{n}$, the one which goes with $\lambda
_{n}$. The only problem is that there is no control of the size of the
vectors $\vect{u}_{m}$. You can fix this by scaling. Let $S_{2}$ denote the
entry of $A\vect{u}_{1}$ which is largest in absolute value. We call this a 
\index{scaling factor} \textbf{scaling factor}. Then $\vect{u}_{2}$ will not be just $A\vect{u}_{1}$ but $A\vect{u}_{1}/S_{2}$. Next
let $S_{3}$ denote the entry of $A\vect{u}_{2}$ which has largest absolute
value and define $\vect{u}_{3}\equiv A\vect{u}_{2}/S_{3}$. Continue this way.
The scaling just described does not destroy the relative insignificance of
the term involving a sum in {\eqref{20-may-e1}}. Indeed it amounts to nothing more
than changing the units of length. Also note that from this scaling
procedure, the absolute value of the largest element of $\vect{u}_{k}$ is
always equal to 1. Therefore, for large $m$,
\begin{equation*}
\vect{u}_{m}=\vspace{0.05in}\frac{\lambda _{n}^{m}c_{n}\vect{x}_{n}}{S_{2}S_{3}\cdots S_{m}}+\tup{\text{relatively insignificant term}} .
\end{equation*}
Therefore, the entry of $A\vect{u}_{m}$ which has the largest absolute value
is essentially equal to the entry having largest absolute value of 
\begin{equation*}
A\tup{\vspace{0.05in}\frac{\lambda _{n}^{m}c_{n}\vect{x}_{n}}{S_{2}S_{3}\cdots S_{m}}} =\vspace{0.05in}\frac{\lambda _{n}^{m+1}c_{n}
\vect{x}_{n}}{S_{2}S_{3}\cdots S_{m}}\approx \lambda _{n}\vect{u}_{m}
\end{equation*}%
and so for large $m$, it must be the case that $\lambda _{n}\approx S_{m+1}$.
This suggests the following procedure.

\begin{procedure}{Finding the largest eigenvalue with its eigenvector}{}
\begin{enumerate}
\item Start with a vector $\vect{u}_{1}$ which you hope has a component in
the direction of $\vect{x}_{n}$. The vector $\tup{1,\cdots ,1} ^{T}$
is usually a pretty good choice.

\item If $\vect{u}_{k}$ is known, 
\begin{equation*}
\vect{u}_{k+1}=\frac{A\vect{u}_{k}}{S_{k+1}}
\end{equation*}
where $S_{k+1}$ is the entry of $A\vect{u}_{k}$ which has largest absolute
value.

\item When the scaling factors, $S_{k}$ are not changing much, $S_{k+1}$
will be close to the eigenvalue and $\vect{u}_{k+1}$ will be close to an
eigenvector.

\item Check your answer to see if it worked well.
\end{enumerate}
\end{procedure}

The shifted inverse power method involves finding the eigenvalue closest to
a given complex number along with the associated eigenvalue. If $\mu $ is a
complex number and you want to find $\lambda $ which is closest to $\mu$,
you could consider the eigenvalues and eigenvectors of $\tup{A-\mu
I} ^{-1}$. Then $A\vect{x}=\lambda \vect{x}$ if and only if 
\begin{equation*}
\tup{A-\mu I} \vect{x}=\tup{\lambda -\mu } \vect{x}
\end{equation*}
If and only if 
\begin{equation*}
\frac{1}{\lambda -\mu }\vect{x}=\tup{A-\mu I} ^{-1}\vect{x}
\end{equation*}
Thus, if $\lambda $ is the closest eigenvalue of $A$ to $\mu $ then out of
all eigenvalues of $\tup{A-\mu I} ^{-1}$, you would have $\frac{1}{
\lambda -\mu }$ would be the largest. Thus all you have to do is apply the
power method to $\tup{A-\mu I} ^{-1}$ and the eigenvector you get
will be the eigenvector which corresponds to $\lambda $ where $\lambda $ is
the closest to $\mu $ of all eigenvalues of $A$. You could use the
eigenvector to determine this directly.

\begin{example}{Finding eigenvalue and eigenvector}{}
Find the eigenvalue and eigenvector for 
\begin{equation*}
\begin{mymatrix}{rrr}
3 & 2 & 1 \\ 
-2 & 0 & -1 \\ 
-2 & -2 & 0
\end{mymatrix}
\end{equation*}
which is closest to $0.9+0.9i$.
\end{example}

\begin{solution}
Form 
\begin{eqnarray*}
&&\tup{\begin{mymatrix}{rrr}
3 & 2 & 1 \\ 
-2 & 0 & -1 \\ 
-2 & -2 & 0
\end{mymatrix} -\tup{0.9+0.9i} \begin{mymatrix}{rrr}
1 & 0 & 0 \\ 
0 & 1 & 0 \\ 
0 & 0 & 1
\end{mymatrix} } ^{-1} \\
&=&\begin{mymatrix}{ccc}
-0.619\,19-10.\,\allowbreak 545i & -5.\,\allowbreak 524\,9-4.\,\allowbreak
972\,4i & -0.370\,57-5.\,\allowbreak 821\,3i \\ 
5.\,\allowbreak 524\,9+4.\,\allowbreak 972\,4i & 5.\,\allowbreak
276\,2+0.248\,62i & 2.\,\allowbreak 762\,4+2.\,\allowbreak 486\,2i \\ 
0.741\,14+11.\,\allowbreak 643i & 5.\,\allowbreak 524\,9+4.\,\allowbreak
972\,4i & 0.492\,52+6.\,\allowbreak 918\,9i%
\end{array}%
}
\end{eqnarray*}
Then pick an initial guess an multiply by this matrix raised to a large
power. 
\begin{equation*}
\begin{mymatrix}{ccc}
-0.619\,19-10.\,\allowbreak 545i & -5.\,\allowbreak 524\,9-4.\,\allowbreak
972\,4i & -0.370\,57-5.\,\allowbreak 821\,3i \\ 
5.\,\allowbreak 524\,9+4.\,\allowbreak 972\,4i & 5.\,\allowbreak
276\,2+0.248\,62i & 2.\,\allowbreak 762\,4+2.\,\allowbreak 486\,2i \\ 
0.741\,14+11.\,\allowbreak 643i & 5.\,\allowbreak 524\,9+4.\,\allowbreak
972\,4i & 0.492\,52+6.\,\allowbreak 918\,9i%
\end{mymatrix} ^{15}\begin{mymatrix}{c}
1 \\ 
1 \\ 
1
\end{mymatrix}
\end{equation*}
This equals 
\begin{equation*}
\begin{mymatrix}{c}
1.\,\allowbreak 562\,9\times 10^{13}-3.\,\allowbreak 899\,3\times 10^{12}i
\\ 
-5.\,\allowbreak 864\,5\times 10^{12}+9.\,\allowbreak 764\,2\times 10^{12}i
\\ 
-1.\,\allowbreak 562\,9\times 10^{13}+3.\,\allowbreak 899\,9\times 10^{12}i
\end{mymatrix}
\end{equation*}
Now divide by an entry to make the vector have reasonable size. This yields 
\begin{equation*}
\begin{mymatrix}{c}
-0.999\,99-3.\,\allowbreak 614\,0\times 10^{-5}i \\ 
0.499\,99-0.499\,99i \\ 
1.0
\end{mymatrix}
\end{equation*}
which is close to 
\begin{equation*}
\begin{mymatrix}{c}
-1 \\ 
0.5-0.5i \\ 
1.0
\end{mymatrix}
\end{equation*}
Then 
\begin{equation*}
\begin{mymatrix}{rrr}
3 & 2 & 1 \\ 
-2 & 0 & -1 \\ 
-2 & -2 & 0
\end{mymatrix} \begin{mymatrix}{c}
-1 \\ 
0.5-0.5i \\ 
1.0
\end{mymatrix} =\begin{mymatrix}{c}
-1.0-1.0i \\ 
1.0 \\ 
1.0+1.0i
\end{mymatrix} 
\end{equation*}
Now to determine the eigenvalue, you could just take the ratio of
corresponding entries. Pick the two corresponding entries which have the
largest absolute values. In this case, you would get the eigenvalue is $1+i$
which happens to be the exact eigenvalue. Thus an eigenvector and eigenvalue
are 
\begin{equation*}
\begin{mymatrix}{c}
-1 \\ 
0.5-0.5i \\ 
1.0
\end{mymatrix}, 1+i
\end{equation*}
\end{solution}

Usually it won't work out so well but you can still find what is desired.
Thus, once you have obtained approximate eigenvalues using the $QR$
algorithm, you can find the eigenvalue more exactly along with an
eigenvector associated with it by using the shifted inverse power method.
