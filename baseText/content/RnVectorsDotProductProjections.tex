\subsection{Projections}

In some applications, we wish to write a vector as a sum of two related vectors. Through the concept of projections, we can find these two vectors. First, we explore an important theorem. The result of 
this theorem will provide our definition of a vector projection.

\begin{theorem}{Vector Projections}{projection}
Let $\vect{v}$ and $\vect{u}$ be nonzero vectors. Then
there exist unique vectors $\vect{v}_{||}$ and $\vect{v}_{\bot }$ such
that
\begin{equation}
\vect{v}=\vect{v}_{||}+\vect{v}_{\bot }  \label{projection}
\end{equation}
where $\vect{v}_{||}$ is a scalar multiple of $\vect{u}$, 
and $\vect{v}_{\bot}$ is perpendicular to $\vect{u}$.

\end{theorem}

\begin{proof}
We start by proving that $\vect{v}_{||}$ and $\vect{v}_{\bot}$ are unique, suppose \ref{projection} holds and $\vect{v}_{||}= k \vect{u}$.
 Taking the dot product of both sides of \ref{projection} with $\vect{u}$ and using $\vect{v}_{\bot }\dotprod \vect{u}=0,$
 this yields
\begin{equation*}
\begin{array}{ll}
\vect{v}\dotprod \vect{u} & = ( \vect{v}_{||}+\vect{v}_{\bot }) \dotprod \vect{u} \\
& =  k\vect{u} \dotprod \vect{u} + \vect{v}_{\bot} \dotprod \vect{u} \\
& = k \vectlength \vect{u}\vectlength ^{2}
\end{array}
\end{equation*}
which requires $k =\vect{v}\dotprod \vect{u} / \vectlength \vect{u}\vectlength ^{2}.$
Thus there can be no more than one vector $\vect{v}_{||}$. It follows 
$\vect{v}_{\bot }$ must equal $\vect{v}-\vect{v}_{||}.$ This verifies there can
be no more than one choice for both $\vect{v}_{||}$ and $\vect{v}_{\bot
}$ and proves their uniqueness. 

Now let
\begin{equation*}
\vect{v}_{||} = 
\frac{\vect{v}\dotprod \vect{u}}{\vectlength \vect{u}\vectlength ^{2}}\vect{u}
\end{equation*}
and let
\begin{equation*}
\vect{v}_{\bot }=\vect{v}-\vect{v}_{||}=\vect{v}-\frac{\vect{v}\dotprod \vect{u}}
{\vectlength \vect{u}\vectlength ^{2}}\vect{u}
\end{equation*}
Then $\vect{v}_{||}= k\vect{u}$ where $k =\frac{\vect{v}\dotprod \vect{u}}{\vectlength \vect{u}\vectlength ^{2}}$.
 It only remains to
verify $\vect{v}_{\bot }\dotprod \vect{u}=0.$ But
\begin{eqnarray*}
\vect{v}_{\bot }\dotprod \vect{u} &=& \vect{v}\dotprod \vect{u}-\frac{\vect{v}\dotprod \vect{u}}{\vectlength \vect{u}\vectlength ^{2}}\vect{u}\dotprod \vect{u} \\
&=& \vect{v}\dotprod\vect{u}-\vect{v}\dotprod \vect{u}\\
&=& 0 
\end{eqnarray*}
\end{proof}

The vector $\vect{v}_{||}$ in Theorem \ref{thm:projection} is called the \textbf{projection}
\index{vector!projection}of $\vect{v}$ onto $\vect{u}$ and is denoted by
\begin{equation*}
\vect{v}_{||}
=
\func{proj}_{\vect{u}}\left( \vect{v}\right)
\end{equation*}

We now make a formal definition of the vector projection.

\begin{definition}{Vector Projection}{projection}
Let $\vect{u}$ and $\vect{v}$ be vectors. Then, the \textbf{projection of $\vect{v}$ onto
$\vect{u}$} is given by 
\begin{equation*}
\func{proj}_{\vect{u}}\left( \vect{v}\right) =\left( \vspace{0.05in}
\frac{\vect{v}\dotprod \vect{u}}{\vect{u}\dotprod \vect{u}}\right) \vect{u}
=
\frac{\vect{v}\dotprod \vect{u}}{\vectlength \vect{u}\vectlength ^{2}}\vect{u}
\end{equation*}
\index{vector!projection}
\end{definition}

Consider the following example of a projection.

\begin{example}{Find the Projection of One Vector Onto Another}{vectorprojection}
Find 
$\func{proj}_{\vect{u}}\left( \vect{v}\right) $ if 
\begin{equation*}
\vect{u}=
\leftB
\begin{array}{r}
2 \\
3 \\
-4
\end{array}
\rightB,
\vect{v}=
\leftB
\begin{array}{r}
1 \\
-2 \\
1
\end{array}
\rightB
\end{equation*}
\end{example}

\begin{solution}
We can use the formula provided in Definition \ref{def:projection} to find $\func{proj}_{\vect{u}}\left( \vect{v}\right)$.
First, compute $\vect{v} \dotprod \vect{u}$. 
This is given by 
\begin{eqnarray*}
\leftB
\begin{array}{r}
1 \\
-2 \\
1
\end{array}
\rightB
 \dotprod
\leftB
\begin{array}{r}
2 \\
3 \\
-4
\end{array}
\rightB
&=&
(2)(1) + (3)(-2) + (-4)(1) \\
&=& 
2 - 6 - 4 \\
&=& -8
\end{eqnarray*}
Similarly, $\vect{u} \dotprod \vect{u}$ is given by 
\begin{eqnarray*}
\leftB
\begin{array}{r}
2 \\
3 \\
-4 
\end{array}
\rightB
 \dotprod
\leftB
\begin{array}{r}
2 \\
3 \\
-4
\end{array}
\rightB
&=&
(2)(2) + (3)(3) + (-4)(-4) \\
&=& 
4 + 9 + 16 \\
&=& 29
\end{eqnarray*}

Therefore, the projection is equal to  
\begin{eqnarray*}
\func{proj}_{\vect{u}}\left( \vect{v}\right)
&=&-\frac{8}{29} 
\leftB
\begin{array}{r}
2 \\
3 \\
-4
\end{array}
\rightB \\
&=& 
\leftB
\begin{array}{r}
-\vspace{0.05in}\frac{16}{29} \\
-\vspace{0.05in}\frac{24}{29} \\
\vspace{0.05in}\frac{32}{29}
\end{array}
\rightB
\end{eqnarray*}
\end{solution}

We will conclude this section with an important application of projections. Suppose a line $L$ and a point $P$ are given such that $P$ is not contained in $L$. Through the use of projections, we can determine the shortest distance from $P$ to $L$. 

\begin{example}{Shortest Distance from a Point to a Line}{shortestpointline}
Let $P = (1,3,5)$ be a point in $\mathbb{R}^3$, and let $L$ be the line which goes through point $P_0 = (0,4,-2)$ with direction vector $\vect{d} = \leftB
\begin{array}{r}
2 \\
1 \\
2
\end{array}
\rightB
$.  Find the shortest distance from $P$ to the line $L$, and find the point $Q$ on $L$ that is closest to $P$. 
\end{example}

\begin{solution}
In order to determine the shortest distance from $P$ to $L$, we will first find the vector $\longvect{P_0P}$ and then find the projection of this vector onto $L$. 
The vector $\longvect{P_0P}$ is given by 
\[
\longvect{P_0P}=
\leftB
\begin{array}{r}
1 \\
3 \\
5
\end{array}
\rightB
-
\leftB
\begin{array}{r}
0 \\
4 \\
-2
\end{array}
\rightB
 = \leftB
\begin{array}{r}
1 \\
-1 \\
7
\end{array}
\rightB
\]

Then, if $Q$ is the point on $L$ closest to $P$, it follows that 
\begin{eqnarray*}
\longvect{P_0Q} &=& \func{proj}_{\vect{d}}\longvect{P_0P} \\
&=& \left( \frac{ \longvect{P_0P}\bullet \vect{d}}{\|\vect{d}\|^2}\right) \vect{d} \\
&=& 
\frac{15}{9} \leftB
\begin{array}{r}
2 \\
1 \\
2
\end{array}
\rightB \\
&=&
\frac{5}{3} \leftB
\begin{array}{r}
2 \\
1 \\
2
\end{array}
\rightB
\end{eqnarray*}

Now, the distance from $P$ to $L$ is given by 
\[
\| \longvect{QP} \| = \| \longvect{P_0P} - \longvect{P_0Q}\|
 = \sqrt{26} 
\]

The point $Q$ is found by adding the vector $\longvect{P_0Q}$ to the position vector $\longvect{0P_0}$ for $P_0$ as follows
\begin{eqnarray*}
\leftB
\begin{array}{r}
0 \\
4 \\
-2
\end{array}
\rightB
+
\frac{5}{3}
\leftB
\begin{array}{r}
2 \\
1 \\
2
\end{array}
\rightB 
&=& 
\leftB
\begin{array}{r}
\vspace{0.05in}\frac{10}{3} \\
\vspace{0.05in}\frac{17}{3} \\
\vspace{0.05in}\frac{4}{3}
\end{array}
\rightB
\end{eqnarray*}

Therefore, $Q = (\frac{10}{3}, \frac{17}{3}, \frac{4}{3})$. 

\end{solution}