\section{Application: Cramer's rule}

\begin{outcome}
  \begin{enumerate}
  \item Use Cramer's rule to solve a system of equations with
    invertible coefficient matrix.
  \end{enumerate}
\end{outcome}

Another application of determinants is \textbf{Cramer's rule} for
solving a system of equations. Recall that we can represent a system
of linear equations in the form $A\vect{x}=\vect{b}$, where $\vect{x}$
is a vector of variables. Cramer's rule gives a formula for the
solutions $\vect{x}$ in the special case that the coefficient matrix
$A$ is a square invertible matrix. Note that Cramer's rule does not
apply if you have a system of equations in which there is a different
number of equations than variables (in other words, when $A$ is not
square), or when $A$ is not invertible.

\begin{theorem}{Cramer's rule}{cramers-rule}
  Suppose $A$ is an invertible $n\times n$-matrix and we wish to solve
  the system $A\vect{x}=\vect{b}$, where
  $\vect{x}=\mat{x_1,\ldots,x_n}$.  Then $x_i$ can be computed by the
  rule%
  \index{Cramer's rule}%
  \index{system of linear equations!Cramer's rule}%
  \index{determinant!Cramer's rule}
  \begin{equation*}
    x_{i} = \frac{\det(A_{i})}{\det(A)},
  \end{equation*}
  where $A_{i}$ is the matrix obtained by replacing the $i\th$ column
  of $A$ with $\vect{b}$.
\end{theorem}

\begin{proof}
  Since $A$ is invertible, the solution to the system
  $A\vect{x}=\vect{b}$ is given by $\vect{x}=A^{-1}\vect{b}$.
  By Theorem~\ref{thm:inverse-and-determinant}, we have
  \begin{equation*}
    A^{-1} ~=~ \frac{1}{\det(A)} \adj(A),
  \end{equation*}
  and therefore
  \begin{equation*}
    \vect{x}=\frac{1}{\det(A)}\adj(A)\vect{b}.
  \end{equation*}
  Let $x_i$ be the $i\th$ component of $\vect{x}$ and $b_j$ be the $j\th$
  component of $\vect{b}$. Recall that the $\ijth$ entry of $\adj(A)$ is
  $\cofactor{A}{ji}$, the $ji\th$ cofactor of $A$. By definition of matrix
  multiplication, we have
  \begin{equation*}
    x_i = \frac{1}{\det(A)}(\cofactor{A}{1i}\,b_{1}+\ldots+\cofactor{A}{ni}\,b_{n}).
  \end{equation*}
  By the formula for the expansion of a determinant along a column,
  this is equal to
  \begin{equation*}
    x_{i}=\frac{1}{\det(A)}\begin{absmatrix}{ccccc}
      \ast & \cdots & b_{1} & \cdots & \ast \\
      \vdots &  & \vdots &  & \vdots \\
      \ast & \cdots & b_{n} & \cdots & \ast
    \end{absmatrix},
  \end{equation*}
  where the $i\th$ column of $A$ is replaced with the column vector
  $\vect{b}$. But this last formula is exactly Cramer's rule.
\end{proof}

\begin{example}{Using Cramer's rule}{cramers-rule}
  Use Cramer's rule to solve the system of equations
  \begin{equation*}
    \begin{mymatrix}{rrr}
      1 & 2 & 1 \\
      3 & 2 & 1 \\
      1 & 4 & 1 \\
    \end{mymatrix} \begin{mymatrix}{c}
      x \\
      y \\
      z \\
    \end{mymatrix} =\begin{mymatrix}{r}
      3 \\
      5 \\
      6 \\
    \end{mymatrix}.
  \end{equation*}
\end{example}

\begin{solution}
  The matrices $A_1$, $A_2$, and $A_3$ are obtained by respectively
  replacing the first, second, and third column of $A$ by $\vect{b}$.
  We compute
  \begin{equation*}
    \det(A) = \begin{absmatrix}{rrr}
      1 & 2 & 1 \\
      3 & 2 & 1 \\
      1 & 4 & 1 \\
    \end{absmatrix}
    = 4,
    \qquad
    \det(A_1) = \begin{absmatrix}{rrr}
      3 & 2 & 1 \\
      5 & 2 & 1 \\
      6 & 4 & 1 \\
    \end{absmatrix}
    = 4
  \end{equation*}
  \begin{equation*}
    \det(A_2) = \begin{absmatrix}{rrr}
      1 & 3 & 1 \\
      3 & 5 & 1 \\
      1 & 6 & 1 \\
    \end{absmatrix}
    = 6,
    \qquad
    \det(A_3) = \begin{absmatrix}{rrr}
      1 & 2 & 3 \\
      3 & 2 & 5 \\
      1 & 4 & 6 \\
    \end{absmatrix}
    = -4.
  \end{equation*}
  Then by Cramer's rule,
  \begin{equation*}
    x = \frac{\det(A_1)}{\det(A)} = \frac{4}{4} = 1,\quad
    y = \frac{\det(A_2)}{\det(A)} = \frac{6}{4} = \frac{3}{2},\quad\mbox{and}\quad
    z = \frac{\det(A_3)}{\det(A)} = \frac{-4}{4} = -1.\quad
  \end{equation*}
  Thus, the solution is $(x,y,z)=(1,\frac{3}{2},-1)$.
\end{solution}

Cramer's rule is sometimes useful in situations where row operations
would be difficult to do. One such situation is when a system of
equations involves functions rather than numbers, as in the following
example.

\begin{example}{Using Cramer's rule for non-constant matrix}{cramers-rule-non-constant-matrix}
  Solve the following system of equations for $z$.
  \begin{equation*}
    \begin{mymatrix}{ccc}
      1 & 0 & 0 \\
      0 & e^{t}\cos t & e^{t}\sin t \\
      0 & -e^{t}\sin t & e^{t}\cos t
    \end{mymatrix}
    \begin{mymatrix}{c}
      x \\
      y \\
      z
    \end{mymatrix}
    = \begin{mymatrix}{c}
      1 \\
      t \\
      t^{2}
    \end{mymatrix}.
  \end{equation*}
\end{example}

\begin{solution}
  We are asked to find the value of $z$ in the solution. By Cramer's
  rule, we have
  \begin{equation*}
    z ~=~ \frac{
      \begin{absmatrix}{ccc}
        1 & 0 & 1 \\
        0 & e^{t}\cos t & t \\
        0 & -e^{t}\sin t & t^{2}
      \end{absmatrix}
    }{
      \begin{absmatrix}{ccc}
        1 & 0 & 0 \\
        0 & e^{t}\cos t & e^{t}\sin t \\
        0 & -e^{t}\sin t & e^{t}\cos t
      \end{absmatrix}
    }
    ~=~ \frac{e^t(t^2\cos t - t\sin t)}{e^{2t}}
    ~=~ e^{-t}(t^2\cos t - t\sin t).
  \end{equation*}
\end{solution}
