\subsection{Elementary matrices and row operations}

Recall from Definition~\ref{def:row-operations} that there are three
kinds of elementary row operations%
\index{matrix!row operation}%
\index{matrix!elementary row operation}%
\index{row operation}%
\index{elementary row operation} on matrices:
\begin{enumerate}
\item Switch two rows.
\item Multiply a row by a non-zero number.
\item Add a multiple of one row to another row.
\end{enumerate}
The purpose of this section is to show that each of these row
operations corresponds to a special type of invertible matrix called
an \textbf{elementary matrix}%
\index{elementary matrix}%
\index{matrix!elementary matrix}.

\begin{example}{Elementary matrix for switching two rows}{elementary-matrix-1}
  Let
  \begin{equation*}
    E ~=~ \begin{mymatrix}{lll}
      1 & 0 & 0 \\
      0 & 0 & 1 \\
      0 & 1 & 0 \\
    \end{mymatrix}.
  \end{equation*}
  What is the effect of multiplying $E$ by an arbitrary $3\times
  n$-matrix $A$?
\end{example}

\begin{solution}
  Consider an arbitrary $3\times n$-matrix
  \begin{equation*}
    A ~=~ \begin{mymatrix}{cccc}
      a_{11} & a_{12} & \cdots & a_{1n} \\
      a_{21} & a_{22} & \cdots & a_{2n} \\
      a_{31} & a_{32} & \cdots & a_{3n} \\
    \end{mymatrix}.
  \end{equation*}
  We compute the product $EA$ by the row method:
  \begin{equation*}
    EA ~=~ \begin{mymatrix}{lll}
      1 & 0 & 0 \\
      0 & 0 & 1 \\
      0 & 1 & 0 \\
    \end{mymatrix}
    \begin{mymatrix}{cccc}
      a_{11} & a_{12} & \cdots & a_{1n} \\
      a_{21} & a_{22} & \cdots & a_{2n} \\
      a_{31} & a_{32} & \cdots & a_{3n} \\
    \end{mymatrix}
    ~=~
    \begin{mymatrix}{cccc}
      a_{11} & a_{12} & \cdots & a_{1n} \\
      a_{31} & a_{32} & \cdots & a_{3n} \\
      a_{21} & a_{22} & \cdots & a_{2n} \\
    \end{mymatrix}.
  \end{equation*}
  So the effect of multiplying $A$ by $E$ on the left is exactly the
  same as switching rows 2 and 3. We say that $E$ is the
  \textbf{elementary matrix for switching rows 2 and 3}.
\end{solution}

\begin{example}{Elementary matrix for multiplying a row by a non-zero number}{elementary-matrix-2}
  Let
  \begin{equation*}
    E ~=~ \begin{mymatrix}{lll}
      1 & 0 & 0 \\
      0 & k & 0 \\
      0 & 0 & 1 \\
    \end{mymatrix}.
  \end{equation*}
  What is the effect of multiplying $E$ by an arbitrary $3\times
  n$-matrix $A$?
\end{example}

\begin{solution}
  We compute the product $EA$ by the row method:
  \begin{equation*}
    EA ~=~ \begin{mymatrix}{lll}
      1 & 0 & 0 \\
      0 & k & 0 \\
      0 & 0 & 1 \\
    \end{mymatrix}
    \begin{mymatrix}{cccc}
      a_{11} & a_{12} & \cdots & a_{1n} \\
      a_{21} & a_{22} & \cdots & a_{2n} \\
      a_{31} & a_{32} & \cdots & a_{3n} \\
    \end{mymatrix}
    ~=~
    \begin{mymatrix}{rrrr}
      a_{11} & a_{12} & \cdots & a_{1n} \\
      ka_{21} & ka_{22} & \cdots & ka_{2n} \\
      a_{31} & a_{32} & \cdots & a_{3n} \\
    \end{mymatrix}.
  \end{equation*}
  So the effect of multiplying $A$ by $E$ on the left is exactly the
  same as multiplying row 2 by the scalar $k$. We say that $E$ is the
  \textbf{elementary matrix for multiplying row 2 by $k$}.
\end{solution}

\begin{example}{Elementary matrix for adding a multiple of one row to another row}{elementary-matrix-3}
  Let
  \begin{equation*}
    E ~=~ \begin{mymatrix}{lll}
      1 & 0 & 0 \\
      0 & 1 & 0 \\
      0 & k & 1 \\
    \end{mymatrix}.
  \end{equation*}
  What is the effect of multiplying $E$ by an arbitrary $3\times
  n$-matrix $A$?
\end{example}

\begin{solution}
  Once again we compute the product $EA$:
  \begin{equation*}
    EA ~=~ \begin{mymatrix}{lll}
      1 & 0 & 0 \\
      0 & 1 & 0 \\
      0 & k & 1 \\
    \end{mymatrix}
    \begin{mymatrix}{cccc}
      a_{11} & a_{12} & \cdots & a_{1n} \\
      a_{21} & a_{22} & \cdots & a_{2n} \\
      a_{31} & a_{32} & \cdots & a_{3n} \\
    \end{mymatrix}
    ~=~
    \begin{mymatrix}{cccc}
      a_{11} & a_{12} & \cdots & a_{1n} \\
      a_{21} & a_{22} & \cdots & a_{2n} \\
      a_{31}+ka_{21} & a_{32}+ka_{22} & \cdots & a_{3n}+ka_{2n} \\
    \end{mymatrix}.
  \end{equation*}
  So the effect of multiplying $A$ by $E$ on the left is exactly the
  same as adding $k$ times row 2 to row 3. We say that $E$ is the
  \textbf{elementary matrix for adding $k$ times row 2 to row 3}.
\end{solution}

As these examples show, performing each type of elementary row
operation is the same as multiplying (on the left) by a certain
invertible matrix. These matrices are called the \textbf{elementary
  matrices}%
\index{elementary matrix}%
\index{matrix!elementary matrix}. In the
above examples, we have only considered $3\times 3$-elementary
matrices, but they exist for other sizes too. The following definition
makes this precise. It also shows how to calculate the elementary
matrix corresponding to any elementary row operation.

\begin{definition}{Elementary matrices and row operations}{elementary-matrices-and-row-operations}
  Let $E$ be an $n\times n$-matrix. Then $E$ is an \textbf{elementary
    matrix}%
  \index{elementary matrix}%
  \index{matrix!elementary matrix}
  if it is the result of applying one elementary row operation to the
  $n\times n$ identity matrix.
\end{definition}

\begin{example}{Finding an elementary matrix}{finding-elementary-matrix}
  Consider the elementary row operation of adding $5$ times row 3 to
  row 1 of a $4\times n$-matrix. Find the elementary matrix $E$
  corresponding to this row operation.
\end{example}

\begin{solution}
  Following Definition~\ref{def:elementary-matrices-and-row-operations}, all
  we have to do is apply the desired row operation to the
  $4\times 4$-identity matrix:
  \begin{equation*}
    \begin{mymatrix}{rrrr}
      1 & 0 & 0 & 0 \\
      0 & 1 & 0 & 0 \\
      0 & 0 & 1 & 0 \\
      0 & 0 & 0 & 1 \\
    \end{mymatrix}
    \quad
    \stackrel{R_1\rowop R_1+5R_3}{\roweq}
    \quad
    \begin{mymatrix}{rrrr}
      1 & 0 & 5 & 0 \\
      0 & 1 & 0 & 0 \\
      0 & 0 & 1 & 0 \\
      0 & 0 & 0 & 1 \\
    \end{mymatrix}
    ~=~ E.
  \end{equation*}
\end{solution}

We can double-check that multiplying $E$ by any $4\times n$-matrix
does indeed have the desired effect:
\begin{equation*}
  \begin{mymatrix}{rrrr}
    1 & 0 & 5 & 0 \\
    0 & 1 & 0 & 0 \\
    0 & 0 & 1 & 0 \\
    0 & 0 & 0 & 1 \\
  \end{mymatrix}
  \begin{mymatrix}{cccc}
    a_{11} & a_{12} & \cdots & a_{1n} \\
    a_{21} & a_{22} & \cdots & a_{2n} \\
    a_{31} & a_{32} & \cdots & a_{3n} \\
    a_{41} & a_{42} & \cdots & a_{4n} \\
  \end{mymatrix}
  ~=~
  \begin{mymatrix}{cccc}
    a_{11}+5a_{31} & a_{12}+5a_{32} & \cdots & a_{1n}+5a_{3n} \\
    a_{21} & a_{22} & \cdots & a_{2n} \\
    a_{31} & a_{32} & \cdots & a_{3n} \\
    a_{41} & a_{42} & \cdots & a_{4n} \\
  \end{mymatrix}.
\end{equation*}
The fact that this always works is the content of the following
theorem.

\begin{theorem}{Multiplication by an elementary matrix and row operations}{multiplication-by-elementary-matrix}
  Performing any of the three elementary row operations on a matrix $A$ is the
  same as taking the product $EA$, where $E$ is the elementary matrix
  obtained by applying the desired row operation to the identity
  matrix.
\end{theorem}

