\section{Eigenvectors and eigenvalues}

When we multiply a square matrix $A$ by a non-zero vector $\vect{v}$,
we obtain another vector $A\vect{v}$. Most of the time, the vectors
$A\vect{v}$ and $\vect{v}$ are unrelated; they could point in
completely different directions. However, sometimes it can happen that
$A\vect{v}$ is a scalar multiple of $\vect{v}$. In that case,
$\vect{v}$ is called an \textbf{eigenvector} of $A$. We will see later
in this chapter that we can learn a lot about the matrix $A$ by
considering its eigenvectors.

\begin{definition}{Eigenvalues and eigenvectors}{eigenvalues-and-eigenvectors}
  Let $A$ be an $n\times n$-matrix. Suppose that $\vect{v}\in\R^n$ is
  a non-zero vector such that $A\vect{v}$ is a scalar multiple of
  $\vect{v}$. In other words, suppose that there exists a scalar
  $\lambda$ such that
  \begin{equation}\label{eigen1}
    A\vect{v}=\lambda\vect{v}.
  \end{equation}
  Then $\vect{v}$ is called an \textbf{eigenvector}%
  \index{eigenvector}%
  \index{vector!eigenvector}%
  \index{matrix!eigenvector} of $A$, and $\lambda$ is called the
  corresponding \textbf{eigenvalue}%
  \index{eigenvalue}%
  \index{matrix!eigenvalue}.
\end{definition}

\begin{example}{Eigenvalues and eigenvectors}{eigenvalues-and-eigenvectors}
  Consider the matrix
  \begin{equation*}
    A = \begin{mymatrix}{rrr}
      3 & -2 &  2 \\
      1 &  2 &  1 \\
      0 &  2 &  1 \\
    \end{mymatrix}.
  \end{equation*}
  Which of the following vectors are eigenvectors of $A$? Find the
  corresponding eigenvalues.
  \begin{equation*}
    \vect{v}_1 = \begin{mymatrix}{r} 2 \\ -1 \\ -2 \end{mymatrix},\quad
    \vect{v}_2 = \begin{mymatrix}{r} 0 \\  1 \\  1 \end{mymatrix},\quad
    \vect{v}_3 = \begin{mymatrix}{r} 1 \\  1 \\  1 \end{mymatrix},\quad
    \vect{v}_4 = \begin{mymatrix}{r} 0 \\  0 \\  0 \end{mymatrix}.
  \end{equation*}
\end{example}

\begin{solution}
  We compute
  \begin{equation*}
    A\vect{v}_1 = \begin{mymatrix}{r} 4 \\ -2 \\ -4 \end{mymatrix},\quad
    A\vect{v}_2 = \begin{mymatrix}{r} 0 \\  3 \\  3 \end{mymatrix},\quad
    A\vect{v}_3 = \begin{mymatrix}{r} 3 \\  4 \\  3 \end{mymatrix},\quad
    A\vect{v}_4 = \begin{mymatrix}{r} 0 \\  0 \\  0 \end{mymatrix}.
  \end{equation*}
  \begin{itemize}
  \item We see that $A\vect{v}_1$ is a scalar multiple of
    $\vect{v}_1$, namely $A\vect{v}_1=2\vect{v}_1$. Therefore,
    $\vect{v}_1$ is an eigenvector of $A$ with corresponding
    eigenvalue $\lambda=2$.
  \item Similarly, $A\vect{v}_2=3\vect{v}_2$, so $\vect{v}_2$ is an
    eigenvector of $A$ with corresponding eigenvalue $\lambda=3$.
  \item On the other hand, $A\vect{v}_3$ is not a scalar multiple of
    $\vect{v}_3$. Hence, $\vect{v}_3$ is not an eigenvector of $A$.
  \item Finally, although $A\vect{v}_4$ is a scalar multiple of
    $\vect{v}_4$, the zero vector is not considered an eigenvector.
  \end{itemize}
\end{solution}

\begin{example}{Find eigenvectors for the given eigenvalue}{find-eigenvectors-given-eigenvalue}
  Let
  \begin{equation*}
    A = \begin{mymatrix}{rrr}
      2  &  0 & 0 \\
      -1 &  3 & 1 \\
      2  & -2 & 0 \\
    \end{mymatrix}.
  \end{equation*}
  Find the eigenvectors corresponding to the eigenvalue $\lambda=2$.
\end{example}

\begin{solution}
  We have to solve the equation $A\vect{v}=2\vect{v}$. We can use
  algebra to rewrite this as
  \begin{eqnarray*}
    A\vect{v} = 2\vect{v}
    &\iff& A\vect{v} - 2\vect{v} = \vect{0} \\
    &\iff& (A-2I)\vect{v} = \vect{0} \\
    &\iff& \begin{mymatrix}{rrr}
      0  &  0 & 0 \\
      -1 &  1 & 1 \\
      2  & -2 & -2 \\
    \end{mymatrix}\vect{v}
    = \begin{mymatrix}{r} 0 \\ 0 \\ 0 \end{mymatrix}.
  \end{eqnarray*}
  This is a homogeneous system of equations with general
  solution
  \begin{equation*}
    \vect{v}
    ~=~ s \begin{mymatrix}{r} 1 \\ 1 \\ 0 \end{mymatrix}
    + t \begin{mymatrix}{r} 1 \\ 0 \\ 1 \end{mymatrix},
  \end{equation*}
  where $s$ and $t$ are parameters. These (except the zero vector) are
  exactly the eigenvectors corresponding to the eigenvalue $\lambda = 2$.
\end{solution}

As the last example shows, the eigenvectors for a given eigenvalue
$\lambda$, plus the zero vector, form a subspace of $\R^n$. This is
called the \textbf{eigenspace} of $\lambda$.

\begin{definition}{Eigenspace}{eigenspace}
  Let $A$ be an $n\times n$-matrix, and let $\lambda$ be an eigenvalue
  of $A$. The \textbf{eigenspace}%
  \index{eigenspace}%
  \index{matrix!eigenspace}%
  \index{subspace!eigenspace} of $\lambda$ is the set
  \begin{equation*}
    E_{\lambda} = \set{\vect{v} \mid A\vect{v}=\lambda\vect{v}}.
  \end{equation*}
  It is a subspace of $\R^n$.
\end{definition}

Instead of finding {\em all} eigenvectors for a given eigenvalue, it
is often sufficient to find a basis for the eigenspace. We also
sometimes call the basis vectors of the eigenspace \textbf{basic
  eigenvectors}%
\index{basic eigenvector}%
\index{eigenvector!basic}%
\index{matrix!eigenvector!basic}.

\begin{example}{Basis of eigenspace}{basis-eigenspace}
  Let
  \begin{equation*}
    A = \begin{mymatrix}{rrr}
      2  &  0 & 0 \\
      -1 &  3 & 1 \\
      2  & -2 & 0 \\
    \end{mymatrix}.
  \end{equation*}
  The matrix $A$ has eigenvalues $\lambda=1$ and $\lambda=2$. Find a
  basis for each eigenspace.
\end{example}

\begin{solution}
  We already found a basis for the eigenspace $E_2$ in
  Example~\ref{exa:find-eigenvectors-given-eigenvalue}.
  \begin{equation*}
    \mbox{Basis of $E_2$:}\quad\set{
      \begin{mymatrix}{r} 1 \\ 1 \\ 0 \end{mymatrix},\quad
      \begin{mymatrix}{r} 1 \\ 0 \\ 1 \end{mymatrix}
    }.
  \end{equation*}
  To find a basis for the eigenspace $E_1$, we proceed analogously.
  We must solve the equation $A\vect{v}=1\vect{v}$. We have:
  \begin{eqnarray*}
    A\vect{v} = 1\vect{v}
    &\iff& A\vect{v} - \vect{v} = \vect{0} \\
    &\iff& (A-I)\vect{v} = \vect{0} \\
    &\iff& \begin{mymatrix}{rrr}
      1  &  0 & 0 \\
      -1 &  2 & 1 \\
      2  & -2 & -1 \\
    \end{mymatrix}\vect{v}
    = \begin{mymatrix}{r} 0 \\ 0 \\ 0 \end{mymatrix}.
  \end{eqnarray*}
  This is a homogeneous system of rank 2, with general solution
  \begin{equation*}
    \vect{v}
    ~=~ t \begin{mymatrix}{r} 0 \\ -1 \\ 2 \end{mymatrix}.
  \end{equation*}
  Thus, the following is a basis for the eigenspace $E_1$:
  \begin{equation*}
    \mbox{Basis of $E_1$:}\quad\set{
      \begin{mymatrix}{r} 0 \\ -1 \\ 2 \end{mymatrix}
    }.
  \end{equation*}
\end{solution}
