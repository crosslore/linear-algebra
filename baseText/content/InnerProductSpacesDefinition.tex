\section{Real inner product spaces}

In this section, we fix $K$ to be the field of real numbers.

\begin{definition}{Real inner product space}{real-inner-product-space}
  A \textbf{(real) inner product space}%
  \index{inner product space!real}%
  \index{real inner product space} is a real vector space $V$ equipped
  with an operation that assigns to any pair of vectors
  $\vect{u},\vect{v}\in V$ a real number $\iprod{\vect{u},\vect{v}}$,
  called the \textbf{inner product}%
  \index{inner product}%
  \index{product!inner product} of $\vect{u}$ and $\vect{v}$. This
  operation must satisfy the following properties:
  \begin{enumerate}
  \item Symmetry: $\iprod{\vect{u},\vect{v}} = \iprod{\vect{v},\vect{u}}$.
  \item Linearity:
    $\iprod{\vect{u},k\vect{v}} = k\iprod{\vect{u},\vect{v}}$ and
    $\iprod{\vect{u},\vect{v}+\vect{v}'} = \iprod{\vect{u},\vect{v}} +
    \iprod{\vect{u},\vect{v}'}$.
  \item The positive definite property: $\iprod{\vect{u},\vect{u}}
    \geq 0$, and moreover, $\iprod{\vect{u},\vect{u}}=0$ if and only
    if $\vect{u} = \vect{0}$.
  \end{enumerate}
\end{definition}

\begin{example}{$\R^n$ with the usual dot product}{rn-with-dot-product}
  $V=\R^n$, with the usual dot product
  \begin{equation*}
    \iprod{\vect{u},\vect{v}} = \vect{u}\dotprod\vect{v} =
    u_1v_1 + \ldots + u_nv_n
  \end{equation*}
  is an inner product space.
\end{example}

\begin{proof}
  The relevant properties were shown in
  Theorem~\ref{thm:properties-dot-product}.
\end{proof}

There exist other inner product operations on $\R^n$ besides the dot
product. The following is an example.

\begin{example}{$\R^2$ with a non-standard inner product}{rn-with-inner-product}
  Let $V=\R^2$, and consider the matrix
  $A=\begin{mymatrix}{cc} 1 & 1 \\ 1 & 2 \end{mymatrix}$.
  Define an operation $\iprod{-,-}$ by
  \begin{equation*}
    \iprod{\vect{u},\vect{v}} = \vect{u}^T A \vect{v} =
    \begin{mymatrix}{cc} u_1 & u_2 \end{mymatrix}
    \begin{mymatrix}{cc} 1 & 1 \\ 1 & 2 \end{mymatrix}
    \begin{mymatrix}{c} v_1 \\ v_2 \end{mymatrix}
    = u_1v_1 + u_1v_2 + u_2v_1 + 2u_2v_2.
  \end{equation*}
  Then $\R^2$, with this operation, is an inner product space.
\end{example}

\begin{proof}
  We must show the three properties are satisfied. For symmetry, note
  that a scalar is equal to its own transpose. Therefore
  \begin{equation*}
    \iprod{\vect{u},\vect{v}}
    ~=~ \vect{u}^T A \vect{v}
    ~=~ (\vect{u}^T A \vect{v})^T
    ~=~ \vect{v}^T A^T \vect{u}.
  \end{equation*}
  But this is equal to $\iprod{\vect{v},\vect{u}}$ because $A$ is a
  symmetric matrix, i.e., $A^T=A$.
  For linearity, note that, by properties of matrix multiplication,
  \begin{equation*}
    \iprod{\vect{u},k\vect{v}}
    ~=~ \vect{u}^T A(k\vect{v})
    ~=~ k\vect{u}^T A\vect{v}
    ~=~ k\iprod{\vect{u},\vect{v}}
  \end{equation*}
  and
  \begin{equation*}
    \iprod{\vect{u},\vect{v}+\vect{v}'}
    ~=~ \vect{u}^T A(\vect{v}+\vect{v}')
    ~=~ \vect{u}^T A\vect{v}+\vect{u}^T A\vect{v}'
    ~=~ \iprod{\vect{u},\vect{v}} +
    \iprod{\vect{u},\vect{v}'}.
  \end{equation*}
  For the positive definite property, we have
  \begin{equation*}
    \iprod{\vect{u},\vect{u}}
    ~=~ u_1^2 + 2u_1u_2 + 2u_2^2
    ~=~ (u_1^2 + 2u_1u_2 + u_2^2) + u_2^2
    ~=~ (u_1+u_2)^2 + u_2^2
    ~\geq~ 0,
  \end{equation*}
  because the sum of two squares is always non-negative. Moreover, if
  equality holds in the last equation, then we must have $u_1+u_2=0$
  and $u_2=0$, and this is only possible if $\vect{u}=\vect{0}$.
\end{proof}

\begin{example}{Continuous functions on an interval}{continuous-interval}
  Let $a<b$ be real numbers, and let
  \begin{equation*}
    [a,b] = \set{x \mid a \leq x \leq b}
  \end{equation*}
  be the closed interval. Let $V=C[a,b]$%
  \index{Cab@$C[a,b]$} be the vector space of all continuous
  functions%
  \index{continuous function}%
  \index{function!continuous}%
  \index{inner product space!of continuous functions}
  $f:[a,b]\to\R$. Given two functions $f,g\in C[a,b]$, define
  \begin{equation*}
    \iprod{f,g} = \int_a^b f(t)g(t)\,dt.
  \end{equation*}
  This is an inner product.
\end{example}

\begin{proof}
  This follows from well-known properties of integrals. For symmetry,
  we have
  \begin{equation*}
    \iprod{f,g}
    ~=~ \int_a^b f(t)g(t)\,dt
    ~=~ \int_a^b g(t)f(t)\,dt
    ~=~ \iprod{g,f}.
  \end{equation*}
  For linearity, we have
  \begin{equation*}
    \iprod{f,kg}
    ~=~ \int_a^b f(t)kg(t)\,dt
    ~=~ k\int_a^b f(t)g(t)\,dt
    ~=~ k\iprod{f,g}.
  \end{equation*}
  and
  \begin{eqnarray*}
    \iprod{f,g_1+g_2}
    &=& \int_a^b f(t)(g_1(t)+g_2(t))\,dt \\
    &=& \int_a^b f(t)g_1(t) + f(t)g_2(t)\,dt \\
    &=& \int_a^b f(t)g_1(t)\,dt + \int_a^b f(t)g_2(t)\,dt \\
    &=& \iprod{f,g_1} + \iprod{f,g_2}.
  \end{eqnarray*}
  For the positive definite property, we have
  \begin{equation*}
    \iprod{f,f}
    = \int_a^b f(t)^2\,dt
    \geq 0.
  \end{equation*}
  This is $\geq 0$ because $a<b$ and the integral over a non-negative
  function is non-negative. Moreover, since $f$ is continuous, we know
  from calculus that the last integral can only be equal to $0$ if $f$
  is the constant zero function.
\end{proof}

\begin{example}{Polynomials}{inner-product-polynomials}
  Let $\Poly$ be the vector space of polynomials (of any degree) with
  real coefficients. Let $a<b$ be real numbers, and consider the inner
  product defined by
  \begin{equation*}
    \iprod{p,q} = \int_a^b p(t)q(t)\,dt.
  \end{equation*}
  This is an inner product space.
\end{example}

\begin{proof}
  The proof is the same as for $C[a,b]$. 
\end{proof}

\begin{example}{Real Hilbert space}{hilbert-space}
  Let $V$ be the vector space of all infinite sequences of real
  numbers $a=(a_0,a_1,a_2,\ldots)$ satifying
  \begin{equation*}
    a_0^2 + a_1^2 + a_2^2 + \ldots < \infty.
  \end{equation*}
  These are called the \textbf{square summable} sequences%
  \index{sequence!square summable}%
  \index{square summable sequence}. (One needs
  to do a little bit of work to show that it is indeed a vector space;
  in particular, to show that the sum of two square summable sequences
  is square summable.) On this space, we can
  define an inner product as follows:
  \begin{equation*}
    \iprod{a,b} = a_0b_0 + a_1b_1 + a_2b_2 + \ldots.
  \end{equation*}
  The details will be worked out in Exercise~\ref{ex:hilbert-space}.
  This inner product space is called \textbf{(real) Hilbert space}%
  \index{Hilbert space}.
\end{example}

We now look at some properties of inner products. The first thing to
note is that while the axioms require linearity in the right
component, symmetry ensures that linearity in the left component also
holds.

\begin{proposition}{Left linearity}{inner-product-left-linearity}
  Let $V$ be an inner product space. Then
  $\iprod{k\vect{u},\vect{v}} = k\iprod{\vect{u},\vect{v}}$ and
  $\iprod{\vect{u}+\vect{u}',\vect{v}} = \iprod{\vect{u},\vect{v}} +
  \iprod{\vect{u}',\vect{v}}$.
\end{proposition}

\begin{proof}
  This follows directly from symmetry. We have
  \begin{equation*}
    \iprod{k\vect{u},\vect{v}}
    = \iprod{\vect{v},k\vect{u}}
    = k\iprod{\vect{v},\vect{u}}
    = k\iprod{\vect{u},\vect{v}}
  \end{equation*}
  and
  \begin{equation*}
    \iprod{\vect{u}+\vect{u}',\vect{v}}
    = \iprod{\vect{v},\vect{u}+\vect{u}'}
    = \iprod{\vect{v},\vect{u}} + \iprod{\vect{v},\vect{u}'}
    = \iprod{\vect{u},\vect{v}} + \iprod{\vect{u}',\vect{v}}
  \end{equation*}
\end{proof}

Another important property of inner products is the Cauchy-Schwarz
inequality. We have already encountered this in the context of the dot
product in Section~\ref{sec:dot-product}.

\begin{theorem}{Cauchy-Schwarz inequality}{inner-product-cauchy-schwarz}
  Let $\vect{u},\vect{v}$ be vectors in an inner product space. Then%
  \index{Cauchy-Schwarz inequality!in inner product space}
  \begin{equation}\label{cauchy-inner}
    \iprod{\vect{u},\vect{v}}^2
    \leq \iprod{\vect{u},\vect{u}} \cdot \iprod{\vect{v},\vect{v}}.
  \end{equation}
\end{theorem}

\begin{proof}
  The proof is almost identical to that of
  Theorem~\ref{thm:cauchy-schwarz-inequality}.  First note that if
  $\vect{u}=\vect{0}$, then both sides of {\eqref{cauchy-inner}} are
  equal to zero, and so there is nothing to show. Therefore, we will
  assume in what follows that $\vect{u}\neq \vect{0}$.  Define a
  function of $t\in \R$ by
  \begin{equation*}
    f(t) = \iprod{t\vect{u}+\vect{v}, t\vect{u}+ \vect{v}}.
  \end{equation*}
  Then by the positive definite property, we know that $f(t)\geq 0$
  for all $t\in \R$.  Also from linearity and symmetry, we have
  \begin{eqnarray*}
    f(t) &=& \iprod{t\vect{u}, t\vect{u}+\vect{v}} +
             \iprod{\vect{v}, t\vect{u}+\vect{v}} \\
         &=& t^2\iprod{\vect{u}, \vect{u}}+t\iprod{\vect{u}, \vect{v}} + t\iprod{\vect{v}, \vect{u}}+
             \iprod{\vect{v}, \vect{v}} \\
         &=&t^2\iprod{\vect{u},\vect{u}}+2t\iprod{\vect{u}, \vect{v}}
             + \iprod{\vect{v}, \vect{v}}.
  \end{eqnarray*}
  This means the graph of $y=f(t)$ is a parabola which opens upwards
  and is never negative. It follows that this function has at most one
  root. From the quadratic formula, we know that a quadratic function
  $at^2+bt+c$ has one or zero roots if and only if $b^2-4ac\leq
  0$. Applying this reasoning to the function $f(t)$, we obtain
  \begin{equation*}
    (2\iprod{\vect{u}, \vect{v}})^2
    - 4\iprod{\vect{u},\vect{u}}\iprod{\vect{v},\vect{v}} < 0,
  \end{equation*}
  which is equivalent to
  $\iprod{\vect{u},\vect{v}}^2 \leq \iprod{\vect{u},\vect{u}} \cdot
  \iprod{\vect{v},\vect{v}}$.
\end{proof}

Finally, we can use the inner product to define the norm of a vector.

\begin{definition}{Norm}{inner-product-norm}
  Let $\vect{u}$ be a vector in an inner product space. Then the
  \textbf{norm}%
  \index{norm!in inner product space}%
  \index{vector!norm} of $\vect{u}$ is defined as
  \begin{equation*}
    \norm{\vect{u}} = \sqrt{\iprod{\vect{u},\vect{u}}}.
  \end{equation*}
\end{definition}

\begin{example}{Norm in $C[-1,1]$}{norm-in-continuous-functions}
  Calculate the norm of $f(t) = t^2$ in $C[-1,1]$.
\end{example}

\begin{solution}
  In the vector space $C[-1,1]$, the inner product is defined as
  \begin{equation*}
    \iprod{f,g} = \int_{-1}^{1} f(t)g(t)\,dt.
  \end{equation*}
  We therefore have
  \begin{equation*}
    \iprod{f,f}
    = \int_{-1}^{1} f(t)^2\, dt
    = \int_{-1}^{1} t^4\, dt
    = \bigbracket{\frac{1}{5}t^5}_{-1}^{1}
    = \frac{2}{5}.
  \end{equation*}
  Therefore, $\norm{f} = \sqrt{\iprod{f,f}} = \sqrt{\frac{2}{5}}$.
\end{solution}
