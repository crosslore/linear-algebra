\section{Addition}

\begin{outcome}
  \begin{enumerate}
  \item Perform the operations of matrix addition and subtraction.
  \item Identify when these operations are not defined.
  \item Apply the algebraic properties of matrix addition to
    manipulate an algebraic expression involving matrices.
  \end{enumerate}
\end{outcome}

To add two matrices, the matrices have to be of the same size. The
addition works by simply adding corresponding entries of the matrices.

\begin{definition}{Addition of matrices}{addition-of-matrices}
  Let $A=\mat{a_{ij}}$ and $B=\mat{b_{ij}}$ be two
  $m\times n$-matrices. Then $A+B=C$%
  \index{matrix!addition}%
  \index{sum|see{addition}}%
  \index{addition!of matrices} where $C$ is the $m\times n$-matrix
  $C=\mat{c_{ij}}$ defined by
  \begin{equation*}
    c_{ij}=a_{ij}+b_{ij}
  \end{equation*}
\end{definition}

\begin{example}{Addition of matrices}{same-size-matrix-addition}
  Add the following matrices.
  \begin{equation*}
    A = \begin{mymatrix}{rrr}
      1 & 2 & 3 \\
      1 & 0 & 4
    \end{mymatrix},\quad
    B = \begin{mymatrix}{rrr}
      5 & 2 & 3 \\
      -6 & 2 & 1
    \end{mymatrix}
  \end{equation*}
\end{example}

\begin{solution}
  Notice that both $A$ and $B$ are of size $2\times 3$.
  Since $A$ and $B$ are of the same size, the addition is possible. Using Definition~\ref{def:addition-of-matrices},
  the addition is done as follows.
  \begin{equation*}
    A + B = \begin{mymatrix}{rrr}
      1 & 2 & 3 \\
      1 & 0 & 4
    \end{mymatrix}
    +
    \begin{mymatrix}{rrr}
      5 & 2 & 3 \\
      -6 & 2 & 1
    \end{mymatrix}
    =\allowbreak
    \begin{mymatrix}{ccc}
      1+5 & 2+2 & 3+3 \\
      1+(-6) & 0+2 & 4+1
    \end{mymatrix}
    =
    \begin{mymatrix}{rrr}
      6 & 4 & 6 \\
      -5 & 2 & 5
    \end{mymatrix}.
  \end{equation*}
\end{solution}

On the other hand, the matrices
\begin{equation*}
  \begin{mymatrix}{rr}
    1 & 2 \\
    3 & 4 \\
    5 & 2
  \end{mymatrix}
  \quad\mbox{and}\quad
  \begin{mymatrix}{rrr}
    -1 & 4 & 8 \\
    2 & 8 & 5
  \end{mymatrix}
\end{equation*}
cannot be added, because one has size $3\times 2$ while the other has size $2\times 3$.

\begin{definition}{The zero matrix}{zero-matrix}
  The \textbf{$m\times n$ zero matrix}%
  \index{zero matrix}%
  \index{matrix!zero matrix} is the $m\times n$-matrix in which all
  entries are equal to zero. It is denoted by $0$.
\end{definition}

Note there is a zero matrix for every size. For example, there is a
$2\times 3$ zero matrix, a $3\times 4$ zero matrix, and so on.

\begin{example}{The zero matrix}{zero-matrix}
  The $2\times 3$ zero matrix is $0= \begin{mymatrix}{rrr}
    0 & 0 & 0 \\
    0 & 0 & 0
  \end{mymatrix}$.
\end{example}

\begin{definition}{Negative of a matrix and subtraction}{matrix-negative}
  The \textbf{negative}%
  \index{matrix!negative}%
  \index{negative!of a matrix} of a matrix $A=\mat{a_{ij}}$ is defined
  to be $-A = [-a_{ij}]$. In other words, it is obtained by negating
  every entry of $A$. To \textbf{subtract}%
  \index{matrix!subtraction}%
  \index{subtraction!of matrices} two matrices, we simply add the
  negative of the second matrix to the first one, i.e.,
  $A-B = A+(-B)$. This is just the same as componentwise subtraction.
\end{definition}

\begin{example}{Subtraction}{matrix-subtraction}
  Subtract the following matrices.
  \begin{equation*}
    A = \begin{mymatrix}{rrr}
      1 & 2 & 3 \\
      1 & 0 & 4
    \end{mymatrix},\quad
    B = \begin{mymatrix}{rrr}
      5 & 2 & 3 \\
      -6 & 2 & 1
    \end{mymatrix}
  \end{equation*}
\end{example}

\begin{solution}
  \begin{equation*}
    A-B =
    \begin{mymatrix}{rrr}
      1 & 2 & 3 \\
      1 & 0 & 4
    \end{mymatrix}
    - \begin{mymatrix}{rrr}
      5 & 2 & 3 \\
      -6 & 2 & 1
    \end{mymatrix}
    =
    \begin{mymatrix}{ccc}
      1-5 & 2-2 & 3-3 \\
      1-(-6) & 0-2 & 4-1
    \end{mymatrix}
    =
    \begin{mymatrix}{rrr}
      -4 & 0 & 0 \\
      7 & -2 & 3
    \end{mymatrix}.
  \end{equation*}
\end{solution}

Addition of matrices obeys the same properties as addition of vectors.

\begin{theorem}{Properties of matrix addition}{properties-of-addition}
  Let $A,B$ and $C$ be matrices of the same size. Then, the following
  properties%
  \index{matrix!properties of addition}%
  \index{matrix!addition!properties}%
  \index{properties of addition!matrices} hold.

  \begin{itemize}
  \item The commutative law of addition
    \begin{equation*}
      A+B=B+A.
    \end{equation*}
  \item The associative law of addition
    \begin{equation*}
      (A+B)+C=A+(B+C).
    \end{equation*}
  \item The existence of an additive unit
    \begin{equation*}
      \begin{array}{c}
        A+0=A.
      \end{array}
    \end{equation*}
  \item The existence of an additive inverse
    \begin{equation*}
      \begin{array}{c}
        A+(-A) = 0.
      \end{array}
    \end{equation*}
  \end{itemize}
\end{theorem}

\begin{proof}
  To prove the commutative law of addition, let $A$ and $B$ be
  matrices of the same size. We want to show that $A+B=B+A$. To do so,
  we use the definition of matrix addition given in Definition~\ref{def:addition-of-matrices}.  We have
  \begin{equation*}
    A+B = \mat{a_{ij}+b_{ij}} = \mat{b_{ij}+a_{ij}} = B+A.
  \end{equation*}
  The proof of the other properties are similar, and are left as an
  exercise.
\end{proof}
