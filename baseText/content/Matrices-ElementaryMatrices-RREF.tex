\subsection{Elementary matrices and {\rref}s}

Suppose an $m\times n$-matrix $A$ is row reduced to its {\rref}. By
tracking each row operation completed, this row reduction can be
performed through multiplication by elementary matrices. The following
theorem uses this fact.

\begin{theorem}{The form $R=UA$}{form-rua}
  Let $A$ be any $m\times n$-matrix and let $R$ be its {\rref}. Then
  there exists an invertible $m\times m$-matrix $U$ such that
  \begin{equation*}
    R=UA.
  \end{equation*}
  Specifically, $U$ can be computed as the product (from right to
  left) of the elementary matrices of all row operations used to
  convert $A$ to {\rref}.
\end{theorem}

\begin{example}{The form $R=UA$}{form-rua}
  Let
  \begin{equation*}
    A = \begin{mymatrix}{rr}
      0 & 1 \\
      1 & 0 \\
      2 & 0
    \end{mymatrix}.
  \end{equation*}
  Find the {\rref} of $A$ and write it in the form $R=UA$, where $U$
  is invertible.
\end{example}

\begin{solution}
  To find the {\rref} $R$, we row reduce $A$. For each step, we will
  record the appropriate elementary matrix. First, switch rows $1$
  and $2$.
  \begin{equation*}
    \begin{mymatrix}{rr}
      0 & 1 \\
      1 & 0 \\
      2 & 0
    \end{mymatrix}
    \quad\stackrel{R_1\rowswap R_2}{\roweq}\quad
    \begin{mymatrix}{rr}
      1 & 0 \\
      0 & 1 \\
      2 & 0
    \end{mymatrix}.
  \end{equation*}
  The corresponding elementary matrix is
  $E_1 = \begin{mymatrix}{rrr}
    0 & 1 & 0 \\
    1 & 0 & 0 \\
    0 & 0 & 1
  \end{mymatrix}$, i.e.,
  \begin{equation*}
    \begin{mymatrix}{rrr}
      0 & 1 & 0 \\
      1 & 0 & 0 \\
      0 & 0 & 1
    \end{mymatrix}
    \begin{mymatrix}{rr}
      0 & 1 \\
      1 & 0 \\
      2 & 0
    \end{mymatrix}
    ~=~\begin{mymatrix}{rr}
      1 & 0 \\
      0 & 1 \\
      2 & 0
    \end{mymatrix}.
  \end{equation*}
  Next, subtract $2$ times the first row from the third row.
  \begin{equation*}
    \begin{mymatrix}{rr}
      1 & 0 \\
      0 & 1 \\
      2 & 0
    \end{mymatrix}
    \quad\stackrel{R_3\rowop R_3-2R_1}{\roweq}\quad
    \begin{mymatrix}{rr}
      1 & 0 \\
      0 & 1 \\
      0 & 0
    \end{mymatrix}.
  \end{equation*}
  The corresponding elementary matrix is
  $E_2 = \begin{mymatrix}{rrr}
    1  & 0 & 0 \\
    0  & 1 & 0 \\
    -2 & 0 & 1
  \end{mymatrix}$, i.e.,
  \begin{equation*}
    \begin{mymatrix}{rrr}
    1  & 0 & 0 \\
    0  & 1 & 0 \\
    -2 & 0 & 1
    \end{mymatrix}
    \begin{mymatrix}{rr}
      1 & 0 \\
      0 & 1 \\
      2 & 0
    \end{mymatrix}
    ~=~\begin{mymatrix}{rr}
      1 & 0 \\
      0 & 1 \\
      0 & 0
    \end{mymatrix}.
  \end{equation*}
  Notice that the resulting matrix is $R$, the required {\rref} of
  $A$. We can then write
  \begin{eqnarray*}
    R &=& E_2E_1A \\
      &=& U A.
  \end{eqnarray*}
  It remains to compute $U$:
  \begin{equation*}
    U ~=~ E_2E_1 ~=~
    \begin{mymatrix}{rrr}
      1 & 0 & 0 \\
      0 & 1 & 0 \\
      -2 & 0 & 1
    \end{mymatrix}
    \begin{mymatrix}{rrr}
      0 & 1 & 0 \\
      1 & 0 & 0 \\
      0 & 0 & 1
    \end{mymatrix} \\
    ~=~ \begin{mymatrix}{rrr}
      0 & 1 & 0\\
      1 & 0 & 0 \\
      0 & -2  & 1
    \end{mymatrix}.
  \end{equation*}
  We can verify that $R = UA$ holds for this matrix $U$:
  \begin{equation*}
    UA ~=~ \begin{mymatrix}{rrr}
      0 & 1 & 0\\
      1 & 0 & 0 \\
      0 & -2  & 1
    \end{mymatrix}
    \begin{mymatrix}{rr}
      0 & 1 \\
      1 & 0 \\
      2 & 0
    \end{mymatrix} \\
    ~=~ \begin{mymatrix}{rr}
      1 & 0 \\
      0 & 1 \\
      0 & 0
    \end{mymatrix} \\
    ~=~ R.
  \end{equation*}
\end{solution}

While the process used in the above example is reliable and simple
when only a few row operations are used, it becomes cumbersome in a
case where many row operations are needed to carry $A$ to $R$. The
following theorem provides an alternate way to find the matrix $U$.

\begin{theorem}{Finding the matrix $U$}{finding-u}
  Let $A$ be an $m\times n$-matrix and let $R$ be its {\rref}. Then
  $R = UA$, where $U$ is an invertible $m \times m$-matrix found by
  forming the augmented matrix $\mat{A\mid I}$ and row reducing to
  $\mat{R\mid U}$.
\end{theorem}

Let's revisit the above example using the process outlined in
Theorem~\ref{thm:finding-u}.

\begin{example}{The form $R=UA$, revisited}{form-rua-revisited}
  Let $A = \begin{mymatrix}{rr}
    0 & 1 \\
    1 & 0 \\
    2 & 0
  \end{mymatrix}$. Use the process of Theorem~\ref{thm:finding-u} to
  find $U$ such that $R=UA$.
\end{example}

\begin{solution}
  First, we set up the augmented matrix $\mat{A\mid I}$:
  \begin{equation*}
    \begin{mymatrix}{rr|rrr}
      0 & 1 & 1 & 0 & 0 \\
      1 & 0 & 0 & 1 & 0 \\
      2 & 0 & 0 & 0 & 1
    \end{mymatrix}.
  \end{equation*}
  Now, we row reduce until the left-hand side is in {\rref}:
  \begin{eqnarray*}
    \begin{mymatrix}{rr|rrr}
      0 & 1 & 1 & 0 & 0 \\
      1 & 0 & 0 & 1 & 0 \\
      2 & 0 & 0 & 0 & 1
    \end{mymatrix}
        &\stackrel{R_1\rowswap R_2}{\roweq}
            &
              \begin{mymatrix}{rr|rrr}
                1 & 0 & 0 & 1 & 0 \\
                0 & 1 & 1 & 0 & 0 \\
                2 & 0 & 0 & 0 & 1
              \end{mymatrix} \\
        &\stackrel{R_3\rowop R_3-2R_1}{\roweq}
            &
              \begin{mymatrix}{rr|rrr}
                1 & 0 & 0 & 1 & 0 \\
                0 & 1 & 1 & 0 & 0 \\
                0 & 0 & 0 & -2 & 1
              \end{mymatrix}.
  \end{eqnarray*}
  The left-hand side of this augmented matrix is $R$, and the
  right-hand side is $U$. Comparing this to the matrices $R$ and $U$
  we found in Example~\ref{exa:form-rua}, we see that the same
  matrices are obtained regardless of which process is used.
\end{solution}

