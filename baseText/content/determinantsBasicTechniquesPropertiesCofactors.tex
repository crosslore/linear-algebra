\section{Cofactors and \texorpdfstring{$2\times 2$}{2x2} determinants}

Let $A$ be an $n\times n$-matrix. That is, let $A$ be a square
matrix. The \textbf{determinant}\index{determinant} of $A$, denoted by
$\det(A)$ is a very important number which we will explore
throughout this section.

If $A$ is a $2\times 2$-matrix, the determinant is given by the
following formula.

\begin{definition}{Determinant of a two by two matrix}{two-by-two-determinant}
  Let $A=\begin{mymatrix}{rr}
    a & b \\
    c & d
  \end{mymatrix}$. Then
  \begin{equation*}
    \det(A) = ad-cb
  \end{equation*}
\end{definition}

The determinant is also often denoted by enclosing the matrix with two
vertical lines. Thus
\begin{equation*}
  \det \begin{mymatrix}{rr}
    a & b \\
    c & d
  \end{mymatrix} =\begin{absmatrix}{rr}
    a & b \\
    c & d
  \end{absmatrix} 
  =ad - bc
\end{equation*}

The following is an example of finding the determinant of a
$2 \times 2$-matrix.

\begin{example}{A two by two determinant}{two-by-two-determinant}
  Find $\det(A)$ for the matrix
  $A =  \begin{mymatrix}{rr}
    2 & 4 \\
    -1 & 6
  \end{mymatrix}$.
\end{example}

\begin{solution}
  From Definition~\ref{def:two-by-two-determinant},
  \begin{equation*}
    \det(A) = (2) (6) - (-1) (4) = 12 + 4 = 16
  \end{equation*}
\end{solution} 

The $2\times 2$ determinant can be used to find the determinant of
larger matrices.  We will now explore how to find the determinant of a
$3\times 3$-matrix, using several tools including the $2\times 2$
determinant.

We begin with the following definition. 

\begin{definition}{The $ij\th$ minor of a matrix}{ijth-minor}
  Let $A$ be a $3\times 3$-matrix. The $ij\th$
  \textbf{minor}\index{determinant!minor} of $A$, denoted as
  $minor(A) _{ij}$, is the determinant of the $2\times 2$-matrix which
  results from deleting the $i\th$ row and the $j\th$ column of
  $A$.

  In general, if $A$ is an $n\times n$-matrix, then the $ij\th$
  minor of $A$ is the determinant of the $n-1\times n-1$-matrix which
  results from deleting the $i\th$ row and the $j\th$ column of
  $A$.
\end{definition}

Hence, there is a minor associated with each entry of $A$. Consider
the following example which demonstrates this definition.

\begin{example}{Finding minors of a matrix}{finding-minors}
  Let 
  \begin{equation*}
    A = \begin{mymatrix}{rrr}
      1 & 2 & 3 \\
      4 & 3 & 2 \\
      3 & 2 & 1
    \end{mymatrix} 
  \end{equation*}
  Find $minor(A) _{12}$ and $minor(A) _{23}$.
\end{example}

\begin{solution}
  First we will find $minor(A) _{12}$. By
  Definition~\ref{def:ijth-minor}, this is the determinant of the
  $2\times 2$-matrix which results when you delete the first row and
  the second column. This minor is given by
  \begin{equation*}
    minor (A)_{12}
    =
    \det \begin{mymatrix}{rr}
      4 & 2 \\
      3 & 1
    \end{mymatrix}
  \end{equation*}
  Using Definition~\ref{def:two-by-two-determinant}, we see that
  \begin{equation*}
    \det \begin{mymatrix}{rr}
      4 & 2 \\
      3 & 1
    \end{mymatrix} = (4)(1) - (3)(2) = 4 - 6 = -2
  \end{equation*}

  Therefore $minor (A)_{12} = -2$. 

  Similarly, $minor(A)_{23}$ is the determinant of the
  $2\times 2$-matrix which results when you delete the second row and
  the third column. This minor is therefore
  \begin{equation*}
    minor (A)_{23} 
    =
    \det \begin{mymatrix}{rr}
      1 & 2 \\
      3 & 2
    \end{mymatrix} = -4
  \end{equation*}
  Finding the other minors of $A$ is left as an exercise.
\end{solution}

The $ij\th$ minor of a matrix $A$ is used in another important
definition, given next.

\begin{definition}{The $ij\th$ cofactor of a matrix}{ijth-cofactor}
  Suppose $A$ is an $n\times n$-matrix. The $ij\th$
  \textbf{cofactor}\index{determinant!cofactor}, denoted by
  $\func{cof}(A) _{ij}$ is defined to be
  \begin{equation*}
    \func{cof}(A) _{ij} = (-1) ^{i+j} minor(A)_{ij} 
  \end{equation*}
\end{definition}

It is also convenient to refer to the cofactor of an entry of a matrix
as follows. If $a_{ij}$ is the $ij\th$ entry of the matrix, then its
cofactor is just $\func{cof}(A) _{ij}$.

\begin{example}{Finding cofactors of a matrix}{finding-cofactors}
  Consider the matrix
  \begin{equation*}
    A=\begin{mymatrix}{rrr}
      1 & 2 & 3 \\
      4 & 3 & 2 \\
      3 & 2 & 1
    \end{mymatrix} 
  \end{equation*}
  Find $\func{cof}(A) _{12}$ and $\func{cof}(A) _{23}$.
\end{example}

\begin{solution}
  We will use Definition~\ref{def:ijth-cofactor} to compute these
  cofactors.

  First, we will compute $\func{cof}(A) _{12}$.  Therefore, we need to
  find $minor(A)_{12}$. This is the determinant of the
  $2\times 2$-matrix which results when you delete the first row and
  the second column. Thus $minor(A)_{12}$ is given by
  \begin{equation*}
    \det \begin{mymatrix}{rr}
      4 & 2 \\
      3 & 1
    \end{mymatrix} = -2
  \end{equation*}
  Then,
  \begin{equation*}
    \func{cof}(A) _{12}=(-1) ^{1+2} minor(A)_{12} =(-1) ^{1+2}(-2) =2
  \end{equation*}
  Hence, $\func{cof}(A) _{12}=2$.

  Similarly, we can find $\func{cof}(A) _{23}$. First, find
  $minor(A)_{23}$, which is the determinant of the $2\times 2$-matrix
  which results when you delete the second row and the third
  column. This minor is therefore
  \begin{equation*}
    \det \begin{mymatrix}{rr}
      1 & 2 \\
      3 & 2
    \end{mymatrix} = -4
  \end{equation*}
  Hence,
  \begin{equation*}
    \func{cof}(A) _{23}=(-1) ^{2+3} minor(A)_{23} =(-1) ^{2+3}(-4) =4
  \end{equation*}
\end{solution}

You may wish to find the remaining cofactors for the above
matrix. Remember that there is a cofactor for every entry in the
matrix.

We have now established the tools we need to find the determinant of a
$3\times3$-matrix.

\begin{definition}{The determinant of a three by three matrix}{three-by-three-determinant}
  Let $A$ be a $3\times 3$-matrix. Then, $\det(A)$ is calculated by
  picking a row (or column) and taking the product of each entry in
  that row (column) with its cofactor and adding these products
  together.

  This process when applied to the $i\th$ row (column) is known as
  \textbf{expanding along the $i\th$ row (column)}%
  \index{determinant!expanding along row or column} as is given by
  \[
    \det(A) =
    a_{i1}\func{cof}(A)_{i1} + a_{i2}\func{cof}(A)_{i2} + a_{i3}\func{cof}(A)_{i3}
  \]
\end{definition}

When calculating the determinant, you can choose to expand any row or
any column. Regardless of your choice, you will always get the same
number which is the determinant of the matrix $A$.  This method of
evaluating a determinant by expanding along a row or a column is
called \textbf{Laplace Expansion}\index{Laplace expansion} or
\textbf{Cofactor Expansion}\index{Cofactor Expansion}.

Consider the following example.

\begin{example}{Finding the determinant of a three by three matrix}{determinant-three-by-three}
  Let
  \begin{equation*}
    A=\begin{mymatrix}{rrr}
      1 & 2 & 3 \\
      4 & 3 & 2 \\
      3 & 2 & 1
    \end{mymatrix} 
  \end{equation*}
  Find $\det(A)$ using the method of Laplace Expansion.
\end{example}

\begin{solution}
  First, we will calculate $\det(A)$ by expanding along the first
  column.  Using Definition~\ref{def:three-by-three-determinant}, we
  take the $1$ in the first column and multiply it by its cofactor,
  \begin{equation*}
    1  (-1) ^{1+1}\begin{absmatrix}{rr}
      3 & 2 \\
      2 & 1
    \end{absmatrix} 
    =
    (1)(1)(-1) = -1 
  \end{equation*}
  Similarly, we take the $4$ in the first column and multiply it by
  its cofactor, as well as with the $3$ in the first column. Finally,
  we add these numbers together, as given in the following equation.
  \begin{equation*}
    \det(A) = 
    1 \overset{
      \func{cof}(A) _{11}}{\overbrace{(-1) ^{1+1}\begin{absmatrix}{rr}
          3 & 2 \\
          2 & 1
        \end{absmatrix} }}+4 \overset{\func{cof}(A) _{21}}{\overbrace{(
        -1) ^{2+1}\begin{absmatrix}{rr}
          2 & 3 \\
          2 & 1
        \end{absmatrix} }}+3 \overset{\func{cof}(A) _{31}}{\overbrace{(
        -1) ^{3+1}\begin{absmatrix}{rr}
          2 & 3 \\
          3 & 2
        \end{absmatrix} }}
  \end{equation*}
  Calculating each of these, we obtain
  \begin{equation*}
    \det(A)
    = 1 (1)(-1)
    + 4 (-1)(-4)
    + 3 (1)(-5)
    = -1 + 16 + -15
    = 0
  \end{equation*}
  Hence, $\det(A) = 0$. 

  As mentioned in Definition~\ref{def:three-by-three-determinant}, we
  can choose to expand along any row or column. Let's try now by
  expanding along the second row.  Here, we take the $4$ in the second
  row and multiply it to its cofactor, then add this to the $3$ in the
  second row multiplied by its cofactor, and the $2$ in the second row
  multiplied by its cofactor. The calculation is as follows.
  \begin{equation*}
    \det(A)
    = 4 \overset{\func{cof}(A)_{21}}{\overbrace{(-1)^{2+1}
        \begin{absmatrix}{rr}
          2 & 3 \\
          2 & 1
        \end{absmatrix}
      }}
    +3 \overset{\func{cof}(A) _{22}}{\overbrace{(-1)^{2+2}
        \begin{absmatrix}{rr}
          1 & 3 \\
          3 & 1
        \end{absmatrix}
      }}
    +2 \overset{\func{cof}(A)_{23}}{\overbrace{(-1)^{2+3}
        \begin{absmatrix}{rr}
          1 & 2 \\
          3 & 2
        \end{absmatrix}
      }}
  \end{equation*}
  Calculating each of these products, we obtain
  \begin{equation*}
    \det(A)
    =
    4(-1)(-2) 
    +
    3(1)(-8) 
    +
    2 (-1)(-4)
    =
    0
  \end{equation*}
  You can see that for both methods, we obtained $\det(A) = 0$. 
\end{solution}

As mentioned above, we will always come up with the same value for
$\det(A)$ regardless of the row or column we choose to expand
along. You should try to compute the above determinant by expanding
along other rows and columns. This is a good way to check your work,
because you should come up with the same number each time!

We present this idea formally in the following theorem.

\begin{theorem}{The determinant is well defined}{well-defined-determinant}
  Expanding the $n\times n$-matrix along any row or column always
  gives the same answer, which is the determinant.
\end{theorem}

We have now looked at the determinant of $2\times 2$ and
$3\times 3$-matrices. It turns out that the method used to calculate
the determinant of a $3\times 3$-matrix can be used to calculate the
determinant of any sized matrix. Notice that
Definitions~\ref{def:ijth-minor}, {\ref{def:ijth-cofactor}}, and
{\ref{def:three-by-three-determinant}} can all be applied to a matrix
of any size.

For example, the $ij\th$ minor of a $4\times 4$-matrix is the
determinant of the $3\times 3$-matrix you obtain when you delete the
$i\th$ row and the $j\th$ column.  Just as with the $3\times 3$
determinant, we can compute the determinant of a $4\times 4$-matrix
by Laplace Expansion, along any row or column

Consider the following example. 

\begin{example}{Determinant of a four by four matrix}{four-by-four-determinant}
  Find $\det(A)$ where
  \begin{equation*}
    A=\begin{mymatrix}{rrrr}
      1 & 2 & 3 & 4 \\
      5 & 4 & 2 & 3 \\
      1 & 3 & 4 & 5 \\
      3 & 4 & 3 & 2
    \end{mymatrix}
  \end{equation*}
\end{example}

\begin{solution}
  As in the case of a $3\times 3$-matrix, you can expand this along
  any row or column. Lets pick the third column. Then, using Laplace
  Expansion,
  \begin{equation*}
    \det(A) 
    = 3(-1)^{1+3}
    \begin{absmatrix}{rrr}
      5 & 4 & 3 \\
      1 & 3 & 5 \\
      3 & 4 & 2
    \end{absmatrix}
    `+2(-1)^{2+3}
    \begin{absmatrix}{rrr}
      1 & 2 & 4 \\
      1 & 3 & 5 \\
      3 & 4 & 2
    \end{absmatrix}+
  \end{equation*}
  \begin{equation*}
    4(-1)^{3+3}
    \begin{absmatrix}{rrr}
      1 & 2 & 4 \\
      5 & 4 & 3 \\
      3 & 4 & 2
    \end{absmatrix}
    +3(-1)^{4+3}
    \begin{absmatrix}{rrr}
      1 & 2 & 4 \\
      5 & 4 & 3 \\
      1 & 3 & 5
    \end{absmatrix}
  \end{equation*}
  Now, you can calculate each $3\times 3$ determinant using Laplace
  Expansion, as we did above.  You should complete these as an
  exercise and verify that $\det(A)= -12$.
\end{solution}

The following provides a formal definition for the determinant of an
$n\times n$-matrix. You may wish to take a moment and consider the
above definitions for $2\times 2$ and $3\times 3$ determinants in
context of this definition.

\begin{definition}{The determinant of an $n\times n$-matrix}{n-by-n-determinant}
  Let $A$ be an $n\times n$-matrix where $n\geq 2$ and suppose the
  determinant of an $(n-1)\times (n-1) $ has been defined. Then
  \begin{equation*}
    \det(A) =\sum_{j=1}^{n}a_{ij}\func{cof}(A)
    _{ij}=\sum_{i=1}^{n}a_{ij}\func{cof}(A) _{ij}
  \end{equation*}
  The first formula consists of expanding the determinant along the
  $i\th$ row and the second expands the determinant along the
  $j\th$ column.
\end{definition}

In the following sections, we will explore some important properties
and characteristics of the determinant.
