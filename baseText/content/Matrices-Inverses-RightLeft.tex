\subsection{Right and left inverses}

So far, we have only talked about the inverses of square matrices. But
what about matrices that are not square? Can they be invertible? It
turns out that non-square matrices can never be invertible. However,
they can have left inverses or right inverses.

\begin{definition}{Left and right inverses}{left-and-right-inverse}
  Let $A$ be an $m\times n$-matrix and $B$ an $n\times m$-matrix.  We
  say that $B$ is a \textbf{left inverse}%
  \index{inverse!left inverse}%
  \index{left inverse}%
  \index{matrix!left inverse}%
  \index{matrix!inverse!left inverse} of $A$ if
  \begin{equation*}
    BA=I.
  \end{equation*}
  We say that $B$ is a \textbf{right inverse}%
  \index{inverse!right inverse}%
  \index{right inverse}%
  \index{matrix!right inverse}%
  \index{matrix!inverse!right inverse} of $A$ if
  \begin{equation*}
    AB=I.
  \end{equation*}
  If $A$ has a left inverse, we also say that $A$ is
  \textbf{left invertible}. Similarly, if $A$ has a right inverse, we
  say that $A$ is \textbf{right invertible}.
\end{definition}

\begin{example}{Right inverse}{right-inverse}
  Let
  \begin{equation*}
    A = \begin{mymatrix}{rrr}
      1 & 0 & 0 \\
      0 & 1 & 0 \\
    \end{mymatrix}
    \quad\mbox{and}\quad
    B = \begin{mymatrix}{rr}
      1 & 0 \\
      0 & 1 \\
      0 & 0 \\
    \end{mymatrix}.
  \end{equation*}
  Show that $B$ is a right inverse, but not a left inverse, of $A$.
\end{example}

\begin{solution}
  We compute
  \begin{equation*}
    AB
    ~=~ \begin{mymatrix}{rrr}
      1 & 0 & 0 \\
      0 & 1 & 0 \\
    \end{mymatrix}
    \begin{mymatrix}{rr}
      1 & 0 \\
      0 & 1 \\
      0 & 0 \\
    \end{mymatrix}
    ~=~ \begin{mymatrix}{rrr}
      1 & 0 \\
      0 & 1 \\
    \end{mymatrix}
    ~=~ I,
  \end{equation*}
  \begin{equation*}
    BA
    ~=~ \begin{mymatrix}{rr}
      1 & 0 \\
      0 & 1 \\
      0 & 0 \\
    \end{mymatrix}
    \begin{mymatrix}{rrr}
      1 & 0 & 0 \\
      0 & 1 & 0 \\
    \end{mymatrix}
    ~=~ \begin{mymatrix}{rrr}
      1 & 0 & 0 \\
      0 & 1 & 0 \\
      0 & 0 & 0 \\
    \end{mymatrix}
    ~\neq~ I.
  \end{equation*}
  Therefore, $B$ is a right inverse, but not a left inverse, of $A$.
\end{solution}

Recall from Definition~\ref{def:invertible-matrix} that $B$ is called
an \textbf{inverse}%
\index{inverse!of a matrix}%
\index{matrix!inverse} of $A$ if it is both a left inverse and a right
inverse. A crucial fact is that invertible matrices are always square.

\begin{theorem}{Invertible matrices are square}{invertible-square}
  Let $A$ be an $m\times n$-matrix.
  \begin{itemize}
  \item If $A$ is left invertible, then $m\geq n$.
  \item If $A$ is right invertible, then $m\leq n$.
  \item If $A$ is invertible, then $m=n$.
  \end{itemize}
  In particular, only square matrices can be invertible.
\end{theorem}

\begin{proof}
  To prove the first claim, assume that $A$ is left invertible, i.e.,
  assume that $BA=I$ for some $n\times m$-matrix $B$. We must show
  that $m\geq n$. Assume, for the sake of obtaining a contradiction,
  that this is not the case, i.e., that $m<n$. Then the matrix $A$ has
  more columns than rows. It follows that the homogeneous system of
  equations $A\vect{x}=\vect{0}$ has a non-trivial solution; let
  $\vect{x}$ be such a solution. We obtain a contradiction by a
  similar method as in
  Example~\ref{exa:non-invertible-matrix}. Namely, we have
  \begin{equation*}
    \vect{x} ~=~ I\,\vect{x} ~=~ (BA)\vect{x} ~=~ B(A\vect{x}) ~=~ B\vect{0} ~=~
    \vect{0},
  \end{equation*}
  contradicting the fact that $\vect{x}$ was non-trivial.  Since we
  got a contradiction from the assumption that $m<n$, it follows that
  $m\geq n$.

  The second claim is proved similarly, but exchanging the roles of
  $A$ and $B$.  The third claim follows directly from the first two
  claims, because every invertible matrix is both left and right
  invertible.
\end{proof}

Of course, not all square matrices are invertible. In particular, zero
matrices are not invertible, along with many other square matrices.
