\section{Application: Matrix powers}

\begin{outcome}
  \begin{enumerate}
  \item Use diagonalization to raise a matrix to a high power.
  \item Use diagonalization to compute a square root of a matrix.
  \end{enumerate}
\end{outcome}

Suppose we have a matrix $A$ and we want to find $A^{50}$. One could
try to multiply $A$ with itself 50 times, but this is a lot of work
(try it!). However, diagonalization allows us to compute high powers
of a matrix relatively easily.  Suppose $A$ is diagonalizable, so that
$P^{-1}AP=D$. We can rearrange this equation to write $A=PDP^{-1}$.
Now, consider $A^{2}$. Since $A=PDP^{-1}$, it follows that
\begin{equation*}
  A^{2} = (PDP^{-1}) ^{2}=PDP^{-1}PDP^{-1}=PD^{2}P^{-1}.
\end{equation*}
Similarly, 
\begin{equation*}
  A^3 = (PDP^{-1}) ^{3}=PDP^{-1}PDP^{-1}PDP^{-1}=PD^{3}P^{-1}.
\end{equation*}
In general,
\begin{equation*}
  A^n = (PDP^{-1}) ^{n}=PD^{n}P^{-1}.
\end{equation*}
Therefore, we have reduced the problem to finding $D^{n}$. But as we
saw in Example~\ref{exa:sums-products-diagonal}, computing a power of
a diagonal matrix is easy. To compute $D^{n}$, we only need to raise
every entry on the diagonal to the power of $n$. Through this method,
we can compute large powers of matrices.

\begin{example}{Raising a matrix to a high power}{matrix-high-power}
  Let\index{matrix!raising to a power}%
  \index{power!of a matrix}
  $A=\begin{mymatrix}{rrr}
    2 & 1 & 0 \\
    0 & 1 & 0 \\
    -1 & -1 & 1
  \end{mymatrix}$. Find $A^{50}$.
\end{example}

\begin{solution}
  First, we will diagonalize $A$. Following the usual steps, we find
  that the eigenvalues are $\eigenvar=1$ and $\eigenvar=2$. The basic
  eigenvectors corresponding to $\eigenvar=1$ are
  \begin{equation*}
    \vect{v}_1 = \begin{mymatrix}{r} 0 \\ 0 \\ 1 \end{mymatrix}
    \quad\mbox{and}\quad
    \vect{v}_2 = \begin{mymatrix}{r} -1 \\ 1 \\ 0 \end{mymatrix},
  \end{equation*}
  and the basic eigenvector corresponding to $\eigenvar = 2$ is
  \begin{equation*}
    \vect{v}_3 = \begin{mymatrix}{r} -1 \\ 0 \\ 1 \end{mymatrix}.
  \end{equation*}
  Now we construct $P$ by using the basic eigenvectors of $A$ as the
  columns of $P$.  Thus
  \begin{equation*}
    P =
    \begin{mymatrix}{rrr}
      0 & -1 & -1 \\
      0 & 1 & 0 \\
      1 & 0 & 1
    \end{mymatrix}.
  \end{equation*}
  The inverse of $P$ is
  \begin{equation*}
    P^{-1}=\begin{mymatrix}{rrr}
      1 & 1 & 1 \\
      0 & 1 & 0 \\
      -1 & -1 & 0
    \end{mymatrix}.
  \end{equation*}
  Then
  \begin{equation*}
    P^{-1}AP
    ~=~
    \begin{mymatrix}{rrr}
      1 & 1 & 1 \\
      0 & 1 & 0 \\
      -1 & -1 & 0
    \end{mymatrix} \begin{mymatrix}{rrr}
      2 & 1 & 0 \\
      0 & 1 & 0 \\
      -1 & -1 & 1
    \end{mymatrix} \begin{mymatrix}{rrr}
      0 & -1 & -1 \\
      0 & 1 & 0 \\
      1 & 0 & 1
    \end{mymatrix} \\
    ~=~
    \begin{mymatrix}{rrr}
      1 & 0 & 0 \\
      0 & 1 & 0 \\
      0 & 0 & 2
    \end{mymatrix} \\
    ~=~
    D.
  \end{equation*}
  Now it follows by rearranging the equation that $A=PDP^{-1}$, and
  therefore, as noted above,
  \begin{eqnarray*}
    A^{50}
    \quad=\quad
    PD^{50}P^{-1} 
    &=&\begin{mymatrix}{rrr}
      0 & -1 & -1 \\
      0 & 1 & 0 \\
      1 & 0 & 1
    \end{mymatrix} \begin{mymatrix}{rrr}
      1^{50} & 0      & 0 \\
      0      & 1^{50} & 0 \\
      0      & 0      & 2^{50}
    \end{mymatrix} \begin{mymatrix}{rrr}
      1 & 1 & 1 \\
      0 & 1 & 0 \\
      -1 & -1 & 0
    \end{mymatrix} \\
    &=&\begin{mymatrix}{ccc}
      2^{50} & -1+2^{50} & 0 \\
      0 & 1 & 0 \\
      1-2^{50} & 1-2^{50} & 1
    \end{mymatrix}.
  \end{eqnarray*}
  \vspace{-6ex}\par  
\end{solution}

Thus, through diagonalization, we have efficiently computed a high
power of $A$. The following example shows that we can also use the
same technique for finding a square root of a matrix.

\begin{example}{Square root of a matrix}{matrix-root}
  Let\index{matrix!square root of}%
  \index{square root!of a matrix}
  $A=\begin{mymatrix}{rrr}
    1  &  3 & 3 \\
    -1 &  5 & 3 \\
    1  & -1 & 1 \\
  \end{mymatrix}$. Find a square root of $A$, i.e., find a matrix $B$
  such that $A=B^2$.
\end{example}

\begin{solution}
  We first diagonalize $A$. The characteristic polynomial is
  \begin{eqnarray*}
    \begin{absmatrix}{ccc}
      1-\eigenvar  &  3 & 3 \\
      -1 &  5-\eigenvar & 3 \\
      1  & -1 & 1-\eigenvar \\
    \end{absmatrix}
    &=&
    (1-\eigenvar)(5-\eigenvar)(1-\eigenvar) + 9 + 3
    - 3(5-\eigenvar) +3(1-\eigenvar) +3(1-\eigenvar)
    \\
    &=& -\eigenvar^3+7\eigenvar^2 -14\eigenvar + 8,
  \end{eqnarray*}
  with roots $\eigenvar=1$, $\eigenvar=2$, and $\eigenvar=4$. The
  corresponding eigenvectors are
  \begin{equation*}
    \vect{v}_1 = \begin{mymatrix}{r} 1 \\ 1 \\ -1 \end{mymatrix},
    \quad
    \vect{v}_2 = \begin{mymatrix}{r} 0 \\ 1 \\ -1 \end{mymatrix},
    \quad\mbox{and}\quad
    \vect{v}_3 = \begin{mymatrix}{r} 1 \\ 1 \\  0 \end{mymatrix},
  \end{equation*}
  respectively. Therefore we have $P^{-1}AP = D$, where
  \begin{equation*}
    P = \begin{mymatrix}{rrr}
      1  &  0 & 1 \\
      1  &  1 & 1 \\
      -1 & -1 & 0 \\
    \end{mymatrix},
    \quad
    P^{-1} = \begin{mymatrix}{rrr}
      1  & -1 & -1 \\
      -1 &  1 &  0 \\
      0  &  1 &  1 \\
    \end{mymatrix},
    \quad\mbox{and}\quad
    D = \begin{mymatrix}{rrr}
      1 & 0 & 0 \\
      0 & 2 & 0 \\
      0 & 0 & 4 \\
    \end{mymatrix}.
  \end{equation*}
  We can equivalently write $A = PDP^{-1}$. Finding a square root of a
  diagonal matrix is easy:
  \begin{equation*}
    D^{\frac{1}{2}} =
    \begin{mymatrix}{ccc}
      1 & 0        & 0 \\
      0 & \sqrt{2} & 0 \\
      0 & 0        & 2 \\
    \end{mymatrix}.    
  \end{equation*}
  If we now define $B=PD^{\frac{1}{2}}P^{-1}$, we clearly have $B^2 =
  PD^{\frac{1}{2}}P^{-1}PD^{\frac{1}{2}}P^{-1} = PDP^{-1} = A$. So the
  desired square root of $A$ is
  \begin{equation*}
    B
    ~=~
    PD^{\frac{1}{2}}P^{-1}
    ~=~
    \begin{mymatrix}{rrr}
      1  &  0 & 1 \\
      1  &  1 & 1 \\
      -1 & -1 & 0 \\
    \end{mymatrix}
    \begin{mymatrix}{ccc}
      1 & 0        & 0 \\
      0 & \sqrt{2} & 0 \\
      0 & 0        & 2 \\
    \end{mymatrix}
    \begin{mymatrix}{rrr}
      1  & -1 & -1 \\
      -1 &  1 &  0 \\
      0  &  1 &  1 \\
    \end{mymatrix}
    ~=~
    \begin{mymatrix}{ccc}
      1           &  1          &  1 \\
      1-\sqrt{2}  &  1+\sqrt{2} &  1 \\
      -1+\sqrt{2} &  1-\sqrt{2} &  1 \\
    \end{mymatrix}.
  \end{equation*}
  Finally, we verify that we have computed $B$ correctly by squaring
  it and double-checking that we really get $A$.
  \begin{equation*}
    B^2 ~=~
    \begin{mymatrix}{ccc}
      1           &  1          &  1 \\
      1-\sqrt{2}  &  1+\sqrt{2} &  1 \\
      -1+\sqrt{2} &  1-\sqrt{2} &  1 \\
    \end{mymatrix}
    \begin{mymatrix}{ccc}
      1           &  1          &  1 \\
      1-\sqrt{2}  &  1+\sqrt{2} &  1 \\
      -1+\sqrt{2} &  1-\sqrt{2} &  1 \\
    \end{mymatrix}
    ~=~
    \begin{mymatrix}{rrr}
      1  &  3 & 3 \\
      -1 &  5 & 3 \\
      1  & -1 & 1 \\
    \end{mymatrix}
    ~=~
    A.
  \end{equation*}
  We note that the square root of a matrix is not unique. In fact,
  $D$ has 8 different square roots, all of the form
  \begin{equation*}
    \begin{mymatrix}{ccc}
      \pm 1 & 0            & 0 \\
      0     & \pm \sqrt{2} & 0 \\
      0     & 0            & \pm 2 \\
    \end{mymatrix}.
  \end{equation*}
  It follows that $A$ has 8 different square roots as well. We leave
  it as an exercise to compute them all.
\end{solution}

The same method can also be used to compute other powers of a matrix,
for example a cube root.
