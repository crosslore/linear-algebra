\subsection{Computing inverses}

In Example \ref{exa:verifying-inverse}, we verified that a matrix $A$
had an inverse. But we did not actually compute the inverse: the
inverse $B$ was already given, and we merely checked that $AB=I$ and
$BA=I$. We now explore a method for finding the inverse when it is not
already known what it is.

\begin{example}{Finding the inverse of a matrix}{finding-inverse}
  Find the inverse of the matrix
  \begin{equation*}
    A=\begin{mymatrix}{rr}
      1 & -2 \\
      2 & -3
    \end{mymatrix}.
  \end{equation*}
\end{example}

\begin{solution}
  To find $A^{-1}$, we need to find a matrix $\begin{mymatrix}{rr}
    x & z \\
    y & w
  \end{mymatrix} $ such that
  \begin{equation*}
    \begin{mymatrix}{rr}
      1 & -2 \\
      2 & -3
    \end{mymatrix} \begin{mymatrix}{rr}
      x & z \\
      y & w
    \end{mymatrix} =\begin{mymatrix}{rr}
      1 & 0 \\
      0 & 1
    \end{mymatrix}.
  \end{equation*}
  We can multiply these two matrices, and see that in order for this
  equation to be true, we must solve the systems of equations
  \begin{equation*}
    \begin{array}{c}
      x  - 2y = 1, \\
      2x - 3y = 0,
    \end{array}
  \end{equation*}
  and
  \begin{equation*}
    \begin{array}{c}
      z  - 2w = 0, \\
      2z - 3w = 1.
    \end{array}
  \end{equation*}
  Writing the augmented matrix for these two systems gives
  \begin{equation*}
    \begin{mymatrix}{rr|r}
      1 & -2 & 1 \\
      2 & -3 & 0
    \end{mymatrix}  
    % \label{inverse-1a}
  \end{equation*}
  for the first system and
  \begin{equation*}
    \begin{mymatrix}{rr|r}
      1 & -2 & 0 \\
      2 & -3 & 1
    \end{mymatrix}
  \end{equation*}
  for the second one. Note that both systems have $A$ as their coefficient
  matrix. Since both systems have the same coefficient matrix, they both
  require exactly the same row operations, and we can use the method of
  Example~\ref{exa:multiple-systems} to solve both systems at the same
  time.  To do so, we create a single augmented matrix containing both of the
  right-hand sides:
  \begin{equation*}
    \begin{mymatrix}{rr|rr}
      1 & -2 & 1 & 0 \\
      2 & -3 & 0 & 1
    \end{mymatrix}.
  \end{equation*}
  Then we perform row operations until the coefficient matrix is in
  {\rref}:
  \begin{equation}\label{eqn:finding-inverse}
    \begin{mymatrix}{rr|rr}
      1 & -2 & 1 & 0 \\
      2 & -3 & 0 & 1
    \end{mymatrix}
    \stackrel{R_2\leftarrow R_2-2R_1}{\sim}
    \begin{mymatrix}{rr|rr}
      1 & -2 &  1 & 0 \\
      0 &  1 & -2 & 1
    \end{mymatrix}
    \stackrel{R_1\leftarrow R_1+2R_2}{\sim}
    \begin{mymatrix}{rr|rr}
      1 & 0 & -3 & 2 \\
      0 & 1 & -2 & 1
    \end{mymatrix}.
  \end{equation}
  This corresponds to the following {\rref}s for the two original
  systems of equations:
  \begin{equation*}
    \begin{mymatrix}{rr|r}
      1 & 0 & -3 \\
      0 & 1 & -2
    \end{mymatrix}
    \quad\mbox{and}\quad
    \begin{mymatrix}{rr|rr}
      1 & 0 & 2 \\
      0 & 1 & 1
    \end{mymatrix}.
  \end{equation*}
  The solution of the first system is $x=-3$ and $y=-2$. The solution
  for the second system is $z=2$ and $w=1$. If we take the values
  found for $x$, $y$, $z$, and $w$ and put them into our inverse
  matrix, we see that the inverse is
  \begin{equation*}
    A^{-1} = 
    \begin{mymatrix}{rr}
      x & z \\
      y & w
    \end{mymatrix}
    =
    \begin{mymatrix}{rr}
      -3 & 2 \\
      -2 & 1
    \end{mymatrix}.
  \end{equation*}
  Notice that this is exactly the right-hand side in the last
  augmented matrix of {\eqref{eqn:finding-inverse}}. In other words,
  all we really had to do to find the inverses were the row operations
  in {\eqref{eqn:finding-inverse}}. The inverse can be read off
  directly from the result.
\end{solution}

The example suggests a general method for finding the inverse of a
matrix, which we summarize in the following algorithm.

\begin{algorithm}{Finding the inverse of a matrix}{matrix-inversion-algorithm}
  Suppose $A$ is an $n\times n$ matrix. To find $A^{-1}$ if it
  exists\index{matrix!finding the inverse}, form the augmented
  $n\times 2n$ matrix
  \begin{equation*}
    \mat{A\mid I}.
  \end{equation*}
  If possible, do row operations until you obtain an $n\times 2n$
  matrix of the form
  \begin{equation*}
    \mat{I\mid B}.
  \end{equation*}
  If this can be done, then $A$ is invertible and $A^{-1}=B$. If it is
  not possible (i.e., if the reduced echelon form of $A$ has less than
  $n$ pivot entries), then $A$ is not invertible.
\end{algorithm}

This algorithm shows how to find the inverse if it exists. It also
tells us if $A$ does not have an inverse.

\begin{example}{Finding the inverse of a matrix}{finding-inverse}
  Let $A=\begin{mymatrix}{rrr}
    1 & 2 & 2 \\
    1 & 0 & 2 \\
    3 & 1 & -1
  \end{mymatrix}$. Find $A^{-1}$ if it exists.
\end{example}

\begin{solution}
  We set up the augmented matrix and reduce it to {\rref}.
  \begin{eqnarray*}
    \mat{A\mid I} &=&  
    \begin{mymatrix}{rrr|rrr}
      1 & 2 &  2 & 1 & 0 & 0 \\
      1 & 0 &  2 & 0 & 1 & 0 \\
      3 & 1 & -1 & 0 & 0 & 1
    \end{mymatrix}\\[1ex]
    &\stackrel{R_2\leftarrow R_2-R_1}{\stackrel{R_3\leftarrow R_3-3R_1}{\sim}}&
    \begin{mymatrix}{rrr|rrr}
      1 &  2 &  2 &  1 & 0 & 0 \\
      0 & -2 &  0 & -1 & 1 & 0 \\
      0 & -5 & -7 & -3 & 0 & 1
    \end{mymatrix}\\[1ex]
    &\stackrel{R_1\leftarrow 7R_1}{\stackrel{R_3\leftarrow -2R_3}{\sim}}&
    \begin{mymatrix}{rrr|rrr}
      7 & 14 & 14 &  7 & 0 &  0 \\
      0 & -2 &  0 & -1 & 1 &  0 \\
      0 & 10 & 14 &  6 & 0 & -2
    \end{mymatrix}\\[1ex]
    &\stackrel{R_1\leftarrow R_1+7R_2}{\stackrel{R_3\leftarrow R_3+5R_2}{\sim}}&
    \begin{mymatrix}{rrr|rrr}
      7 &  0 & 14 &  0 & 7 &  0 \\
      0 & -2 &  0 & -1 & 1 &  0 \\
      0 &  0 & 14 &  1 & 5 & -2
    \end{mymatrix}\\[1ex]
    &\stackrel{R_1\leftarrow R_1-R_3}{\sim}&
    \begin{mymatrix}{rrr|rrr}
      7 &  0 &  0 & -1 & 2 &  2 \\
      0 & -2 &  0 & -1 & 1 &  0 \\
      0 &  0 & 14 &  1 & 5 & -2
    \end{mymatrix}\\[-1ex]
    &\stackrel{R_1\leftarrow \frac{1}{7}R_1}{\stackrel{R_2\leftarrow -\frac{1}{2}R_2}{\stackrel{R_3\leftarrow \frac{1}{14}R_3}{\sim}}}&
    \def\arraystretch{1.5}
    \begin{mymatrix}{rrr|rrr}
      1 & 0 & 0 & -\frac{1}{7} & \frac{2}{7} & \frac{2}{7} \\
      0 & 1 & 0 & \frac{1}{2} & -\frac{1}{2} & 0           \\
      0 & 0 & 1 & \frac{1}{14} & \frac{5}{14} & -\frac{1}{7}
    \end{mymatrix}.
  \end{eqnarray*}
  Notice that the last augmented matrix is of the form
  $\mat{I\mid B}$, where the left-hand side is the $3 \times 3$
  identity matrix.  Therefore, the inverse is the $3 \times 3$ matrix
  on the right-hand side, given by
  \begin{equation*}
    A^{-1} ~=~
    \def\arraystretch{1.5}
    \begin{mymatrix}{rrr}
      -\frac{1}{7} & \frac{2}{7} & \frac{2}{7} \\
      \frac{1}{2} & -\frac{1}{2} & 0 \\
      \frac{1}{14} & \frac{5}{14} & -\frac{1}{7}
    \end{mymatrix}.
  \end{equation*}
\end{solution}

When looking for the inverse of a matrix, it can happen that the
left-hand side cannot be row reduced to the identity matrix. The
following is an example of this situation.

\begin{example}{A non-invertible matrix}{matrix-no-inverse}
  Let $A=\begin{mymatrix}{rrr}
    1 & -2 & 2 \\
    1 &  0 & 2 \\
    2 & -2 & 4
  \end{mymatrix}$. Find $A^{-1}$ if it exists.%
  \index{matrix!inverse!does not exist}%
  \index{inverse!of a matrix!does not exist}
\end{example}

\begin{solution} 
  We write the augmented matrix 
  \begin{equation*}
    \mat{A\mid I} 
    ~=~ 
    \begin{mymatrix}{rrr|rrr}
      1 & -2 & 2 & 1 & 0 & 0 \\
      1 &  0 & 2 & 0 & 1 & 0 \\
      2 & -2 & 4 & 0 & 0 & 1
    \end{mymatrix}
  \end{equation*}
  and proceed to do row operations attempting to obtain
  $\mat{I\mid A^{-1}}$. After a few row operations, we have
  \begin{equation*}
    \begin{mymatrix}{rrr|rrr}
      \circled{1} & -2 & 2 & 1 & 0 & 0 \\
      0 & \circled{2} & 0 & -1 & 1 & 0 \\
      0 & 0 & 0 & -1 & -1 & 1
    \end{mymatrix}.
  \end{equation*}
  At this point, we see that the coefficient matrix has rank $2$,
  i.e., there are only two pivot entries. This means there is no way
  to obtain $I$ on the left-hand side of this augmented matrix.
  Hence, there is no way to complete the algorithm, and the inverse of
  $A$ does not exist.
\end{solution}

If the algorithm provides an inverse, it is always possible to
double-check that your answer is correct.  To do so, use the method
demonstrated in Example \ref{exa:verifying-inverse}. Check that the
products $AA^{-1}$ and $A^{-1}A$ both equal the identity
matrix. Through this method, you can always ensure that you have
calculated $A^{-1}$ properly.

\subsection{CONTINUE HERE}

One way in which the inverse of a matrix is useful is to find the
solution of a system of linear equations.  Recall from Definition
\ref{def:matrix-form} that we can write a system of equations in
matrix form, which is of the form $AX=B$. Suppose you find the inverse
of the matrix $A^{-1}$. Then you could multiply both sides of this
equation on the left by $A^{-1}$ and simplify to obtain
\begin{equation*}
  \begin{array}{c}
    \tup{A^{-1} } AX =A^{-1}B \\
    \tup{A^{-1}A} X = A^{-1}B \\
    IX = A^{-1}B \\
    X = A^{-1}B
  \end{array}
\end{equation*}
Therefore we can find $X$, the solution to the system, by computing
$X=A^{-1}B$.  Note that once you have found $A^{-1}$, you can easily
get the solution for different right hand sides (different $B$). It is
always just $A^{-1}B$.

We will explore this method of finding the solution to a system in the
following example.

\begin{example}{Using the inverse to solve a system of equations}{inverse-to-solve-system}
  Consider the following system of equations. Use the inverse of a
  suitable matrix to give the solutions to this system.
  \begin{equation*}
    \begin{array}{c}
      x+z=1 \\
      x-y+z=3 \\
      x+y-z=2
    \end{array}
  \end{equation*}
\end{example}

\begin{solution} First, we can write the system of equations in matrix form
  \begin{equation}
    AX = 
    \begin{mymatrix}{rrr}
      1 & 0 & 1 \\
      1 & -1 & 1 \\
      1 & 1 & -1
    \end{mymatrix} \begin{mymatrix}{r}
      x \\
      y \\
      z
    \end{mymatrix} =\begin{mymatrix}{r}
      1 \\
      3 \\
      2
    \end{mymatrix}  = B  \label{inverse-system-1}
  \end{equation}

  The inverse of the matrix 
  \begin{equation*}
    A = \begin{mymatrix}{rrr}
      1 & 0 & 1 \\
      1 & -1 & 1 \\
      1 & 1 & -1
    \end{mymatrix}
  \end{equation*}
  is
  \begin{equation*}
    A^{-1} = \begin{mymatrix}{rrr}
      0 & \vspace{0.05in}\frac{1}{2} & \vspace{0.05in}\frac{1}{2} \\
      1 & -1 & 0 \\
      1 & -\vspace{0.05in}\frac{1}{2} & -\vspace{0.05in}\frac{1}{2}
    \end{mymatrix}
  \end{equation*}

  Verifying this inverse is left as an exercise.

  From here, the solution to the given system \ref{inverse-system-1} is
  found by
  \begin{equation*}
    \begin{mymatrix}{r}
      x \\
      y \\
      z
    \end{mymatrix} = A^{-1}B = \begin{mymatrix}{rrr}
      0 & \vspace{0.05in}\frac{1}{2} & \vspace{0.05in}\frac{1}{2} \\
      1 & -1 & 0 \\
      1 & -\vspace{0.05in}\frac{1}{2} & -\vspace{0.05in}\frac{1}{2}
    \end{mymatrix} \begin{mymatrix}{r}
      1 \\
      3 \\
      2
    \end{mymatrix} =\begin{mymatrix}{r}
      \vspace{0.05in}\frac{5}{2} \\
      -2 \\
      -\vspace{0.05in}\frac{3}{2}
    \end{mymatrix} 
  \end{equation*}
\end{solution}

What if the right side, $B$, of \ref{inverse-system-1} had been
$\begin{mymatrix}{r}
  0 \\
  1 \\
  3
\end{mymatrix}$? In other words, what would be the solution to
\begin{equation*}
  \begin{mymatrix}{rrr}
    1 & 0 & 1 \\
    1 & -1 & 1 \\
    1 & 1 & -1
  \end{mymatrix} \begin{mymatrix}{r}
    x \\
    y \\
    z
  \end{mymatrix} =\begin{mymatrix}{r}
    0 \\
    1 \\
    3
  \end{mymatrix}?
\end{equation*}
By the above discussion, the solution is given by 
\begin{equation*}
  \begin{mymatrix}{r}
    x \\
    y \\
    z
  \end{mymatrix} = A^{-1}B = \begin{mymatrix}{rrr}
    0 & \vspace{0.05in}\frac{1}{2} & \vspace{0.05in}\frac{1}{2} \\
    1 & -1 & 0 \\
    1 & -\vspace{0.05in}\frac{1}{2} & -\vspace{0.05in}\frac{1}{2}
  \end{mymatrix} \begin{mymatrix}{r}
    0 \\
    1 \\
    3
  \end{mymatrix} =\begin{mymatrix}{r}
    2 \\
    -1 \\
    -2
  \end{mymatrix} 
\end{equation*}
This illustrates that for a system $AX=B$ where $A^{-1}$ exists, 
it is easy to find the solution when the vector $B$ is changed.

We conclude this section with some important properties of the inverse. 

\begin{theorem}{Inverses of transposes and products}{inverse-transpose-prod}
  Let $A, B$, and $A_i$ for $i=1,...,k$ be $n \times n$ matrices. 
  \begin{enumerate}
  \item
    If $A$ is an invertible matrix, then $(A^{T})^{-1} = (A^{-1})^{T}$
  \item
    If $A$ and $B$ are invertible matrices, then $AB$ is invertible and $(AB)^{-1} = B^{-1}A^{-1}$
  \item
    If $A_1, A_2, ..., A_k$ are invertible, then the product $A_1A_2 \cdots A_k$ is invertible, and $(A_1A_2 \cdots A_k)^{-1} = A_k^{-1}A_{k-1}^{-1} \cdots A_2^{-1}A_1^{-1}$
  \end{enumerate}
\end{theorem}

Consider the following theorem.

\begin{theorem}{Properties of the inverse}{inverse-properties}
  Let $A$ be an $n \times n$ matrix and $I$ the usual identity matrix. 
  \begin{enumerate}
  \item
    $I$ is invertible and $I^{-1} = I$
  \item
    If $A$ is invertible then so is $A^{-1}$, and $(A^{-1})^{-1} = A$
  \item
    If $A$ is invertible then so is $A^k$, and $(A^k)^{-1} = (A^{-1})^k$
  \item
    If $A$ is invertible and $p$ is a non-zero real number, then $pA$ is invertible and $(pA)^{-1} = \frac{1}{p}A^{-1}$
  \end{enumerate}
\end{theorem}
