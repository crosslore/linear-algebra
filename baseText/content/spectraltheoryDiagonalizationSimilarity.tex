\subsection{Similarity and Diagonalization}

We begin this section by recalling the definition  of similar matrices. 
Recall that if $A,B$ are two $n\times n$ matrices, then they are \textbf{similar} \index{similar matrices}
if and only if there exists an invertible matrix $P$
such that
\begin{equation*}
A=P^{-1}BP
\end{equation*}

In this case we write $A \sim B$. The concept of similarity is an example of an \textbf{equivalence relation}. \index{equivalence relation} 

\begin{lemma}{Similarity is an Equivalence Relation}{similarityequivalence}
Similarity is an equivalence relation, i.e. for $n \times n$ matrices $A,B,$ and $C$, 
\begin{enumerate}
\item $A \sim A$ (reflexive)
\item If $A \sim B$, then $B \sim A$ (symmetric)
\item If $A \sim B$ and $B \sim C$, then $A \sim C$ (transitive)
\end{enumerate}
\end{lemma}

\begin{proof}
It is clear that $A\sim A$, taking $P=I$. 

Now, if $A\sim B,$ then for some $P$ invertible,
\begin{equation*}
A=P^{-1}BP
\end{equation*}
and so
\begin{equation*}
PAP^{-1}=B
\end{equation*}
But then
\begin{equation*}
\left( P^{-1}\right) ^{-1}AP^{-1}=B
\end{equation*}
which shows that $B\sim A$.

Now suppose $A\sim B$ and $B\sim C$. Then there exist invertible matrices 
$P,Q$ such that
\begin{equation*}
A=P^{-1}BP,\ B=Q^{-1}CQ
\end{equation*}
Then,
\begin{equation*}
A=P^{-1} \left( Q^{-1}CQ \right)P=\left( QP\right) ^{-1}C\left( QP\right)
\end{equation*}
showing that $A$ is similar to $C$.
\end{proof}

Another important concept necessary to this section is the trace of a matrix. Consider the definition.

\begin{definition}{Trace of a Matrix}{matrixtrace}
\index{trace of a matrix}
If $A=[a_{ij}]$ is an $n\times n$ matrix, then the
trace of $A$ is
\[ \func{trace}(A) = \sum_{i=1}^n a_{ii}.\]
\end{definition}

In words, the trace of a matrix is the sum of the entries on the main diagonal. 

\begin{lemma}{Properties of Trace}{traceproperties}
For $n\times n$ matrices $A$ and $B$, and any $k\in\mathbb{R}$,
\begin{enumerate}
\item $\func{trace}(A+B)=\func{trace}(A) + \func{trace}(B)$
\item $\func{trace}(kA)=k\cdot\func{trace}(A)$
\item $\func{trace}(AB)=\func{trace}(BA)$
\end{enumerate}
\end{lemma}

The following theorem includes a reference to the characteristic polynomial of a matrix. Recall that for any $n \times n$ matrix $A$, the characteristic polynomial of $A$ is $c_A(x)=\det(xI-A)$.

\begin{theorem}{Properties of Similar Matrices}{propertiessimilar}
If $A$ and $B$ are $n\times n$ matrices and $A\sim B$, then
\begin{enumerate}
\item $\det(A) = \det(B)$
\item $\func{rank}(A) = \func{rank}(B)$
\item $\func{trace}(A)= \func{trace}(B)$
\item $c_A(x)=c_B(x)$
\item $A$ and $B$ have the same eigenvalues
\end{enumerate}
\end{theorem}

We now proceed to the main concept of this section. When a matrix is similar to a diagonal matrix, the matrix is said to
be
\index{diagonalizable} diagonalizable. 
We define a diagonal matrix $D$ as a matrix containing a zero in every entry 
except those on the main diagonal.\index{matrix!diagonal matrix} \index{matrix!main diagonal} More precisely, if $d_{ij}$ is the $ij^{th}$ entry of a diagonal matrix $D$, then
$d_{ij}=0$ unless $i=j$. Such
matrices look like the following.
\begin{equation*}
D = \leftB
\begin{array}{ccc}
\ast &  & 0 \\
& \ddots &  \\
0 &  & \ast
\end{array}
\rightB
\end{equation*}
where $\ast $ is a number which might not be zero.

The following is the formal definition of a diagonalizable matrix. 

\begin{definition}{Diagonalizable}{diagonalizable}
Let $A$ be an $n\times n$ matrix. Then $A$ is said to be \textbf{diagonalizable}
\index{diagonalizable} if there exists an invertible matrix $P$ such that
\begin{equation*}
P^{-1}AP=D
\end{equation*}
where $D$ is a diagonal matrix.
\end{definition}

Notice that the above equation can be rearranged as $A=PDP^{-1}$. Suppose we wanted to compute $A^{100}$. By diagonalizing $A$ first it suffices to then compute $\left(PDP^{-1}\right)^{100}$, which reduces to $PD^{100}P^{-1}$. This last computation is much simpler than $A^{100}$. While this process is described in detail later, it provides motivation for diagonalization. 