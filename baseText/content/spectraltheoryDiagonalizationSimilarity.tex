\section{Similarity}

We begin this section by recalling the definition  of similar matrices. 
Recall that if $A,B$ are two $n\times n$-matrices, then they are \textbf{similar}\index{similar matrices}
if and only if there exists an invertible matrix $P$
such that
\begin{equation*}
A=P^{-1}BP
\end{equation*}

In this case we write $A\similar B$. The concept of similarity is an example of an \textbf{equivalence relation}\index{equivalence relation}.

\begin{lemma}{Similarity is an equivalence relation}{similarity-equivalence}
Similarity is an equivalence relation, i.e. for $n \times n$-matrices $A,B$, and $C$, 
\begin{enumerate}
\item $A \similar A$ (reflexive)
\item If $A \similar B$, then $B \similar A$ (symmetric)
\item If $A \similar B$ and $B \similar C$, then $A \similar C$ (transitive)
\end{enumerate}
\end{lemma}

\begin{proof}
It is clear that $A\similar A$, taking $P=I$. 

Now, if $A\similar B$, then for some $P$ invertible,
\begin{equation*}
A=P^{-1}BP
\end{equation*}
and so
\begin{equation*}
PAP^{-1}=B
\end{equation*}
But then
\begin{equation*}
(P^{-1}) ^{-1}AP^{-1}=B
\end{equation*}
which shows that $B\similar A$.

Now suppose $A\similar B$ and $B\similar C$. Then there exist invertible matrices 
$P,Q$ such that
\begin{equation*}
A=P^{-1}BP,\ B=Q^{-1}CQ
\end{equation*}
Then,
\begin{equation*}
A=P^{-1} (Q^{-1}CQ)P=(QP) ^{-1}C(QP)
\end{equation*}
showing that $A$ is similar to $C$.
\end{proof}

Another important concept necessary to this section is the trace of a matrix. Consider the definition.

\begin{definition}{Trace of a matrix}{matrix-trace}
\index{trace of a matrix}%
If $A=[a_{ij}]$ is an $n\times n$-matrix, then the
trace of $A$ is
\[ \tr(A) = \sum_{i=1}^n a_{ii}.\]
\end{definition}

In words, the trace of a matrix is the sum of the entries on the main diagonal. 

\begin{lemma}{Properties of trace}{trace-properties}
For $n\times n$-matrices $A$ and $B$, and any $k\in\R$,
\begin{enumerate}
\item $\tr(A+B)=\tr(A) + \tr(B)$
\item $\tr(kA)=k\cdot\tr(A)$
\item $\tr(AB)=\tr(BA)$
\end{enumerate}
\end{lemma}

The following theorem includes a reference to the characteristic polynomial of a matrix. Recall that for any $n \times n$-matrix $A$, the characteristic polynomial of $A$ is $c_A(x)=\det(xI-A)$.

\begin{theorem}{Properties of similar matrices}{properties-similar}
If $A$ and $B$ are $n\times n$-matrices and $A\similar B$, then
\begin{enumerate}
\item $\det(A) = \det(B)$
\item $\rank(A) = \rank(B)$
\item $\tr(A)= \tr(B)$
\item $c_A(x)=c_B(x)$
\item $A$ and $B$ have the same eigenvalues
\end{enumerate}
\end{theorem}

