\subsection{Row space, column space, and null space of a matrix}

We begin this section with a new definition.

\begin{definition}{Row and column space}{rowcolumnspace}
Let $A$ be an $m\times n$ matrix. The \textbf{column space}\index{column space}\index{matrix!column space} of $A$, written $\func{col}(A)$, is the
span of the columns. The \textbf{row space}\index{row space}\index{matrix!row space} of $A$, written $\func{row}(A)$, is the span of the rows.
\end{definition}

Using the {\rref}, we can obtain an efficient description of the row and column space of
a matrix. Consider the following lemma.

\begin{lemma}{Effect of row operations on row space}{rowoperationsrowspace}
Let $A$ and $B$ be $m\times n$ matrices such that $A$ can be carried to $B$ by elementary row $\mat{\mbox{column}}$ operations. Then $\func{row}(A)=\func{row}(B)$ $\mat{\func{col}(A)=\func{col}(B)}$.
\end{lemma}

\begin{proof}
We will prove that the above is true for row operations, which can be easily applied to column operations. 

Let $\vect{r}_1, \vect{r}_2, \ldots, \vect{r}_m$ denote the rows of $A$.

\begin{itemize}
\item If $B$ is obtained from $A$ by a interchanging two rows of $A$, then
$A$ and $B$ have exactly the same rows, so $\func{row}(B)=\func{row}(A)$.

\item
Suppose $p\neq 0$, and suppose that for some $j$, $1\leq j\leq m$,
$B$ is obtained from $A$ by multiplying row $j$ by $p$.
Then 

\[ \func{row}(B)=\func{span}\{\vect{r}_1, \ldots, 
p\vect{r}_{j}, \ldots, \vect{r}_m\}.  \]

Since
\[ \{\vect{r}_1, \ldots, 
p\vect{r}_{j}, \ldots, \vect{r}_m\} \subseteq\func{row}(A),\]

it follows that $\func{row}(B)\subseteq\func{row}(A)$.
Conversely, since

\[ \{\vect{r}_1, \ldots, \vect{r}_m\}\subseteq\func{row}(B),\]

it follows that $\func{row}(A)\subseteq\func{row}(B)$.
Therefore, $\func{row}(B)=\func{row}(A)$.

\item
Suppose $p\neq 0$, and suppose that for some $i$ and $j$,
$1\leq i,j\leq m$,
$B$ is obtained from $A$ by adding $p$ time row $j$ to row $i$.
Without loss of generality, we may assume $i<j$.

Then 

\[ \func{row}(B)=\func{span}\{\vect{r}_1, \ldots, \vect{r}_{i-1},
\vect{r}_i+p\vect{r}_j, \ldots,\vect{r}_j,\ldots, \vect{r}_m\}.  \]

Since

\[ \{\vect{r}_1, \ldots, \vect{r}_{i-1}, 
\vect{r}_i+p\vect{r}_{j}, \ldots, \vect{r}_m\} \subseteq\func{row}(A),\]

it follows that $\func{row}(B)\subseteq\func{row}(A)$.

Conversely, since

\[ \{\vect{r}_1, \ldots, \vect{r}_m\}\subseteq\func{row}(B),\]

it follows that $\func{row}(A)\subseteq\func{row}(B)$.
Therefore, $\func{row}(B)=\func{row}(A)$.
\end{itemize}
\end{proof}

Consider the following lemma. 

\begin{lemma}{Row space of a {\ef} matrix}{echelonformrowspace}
Let $A$ be an $m \times n$ matrix and let $R$ be its {\ef}. Then the nonzero rows of $R$ form a basis of $\func{row}(R)$, and consequently of $\func{row}(A)$. 
\end{lemma}

This lemma suggests that we can examine the {\ef} of a matrix in order to obtain the row space. Consider now the column space. The column space can be obtained by simply saying that
it equals the span of all the columns. However, you can often get the column
space as the span of fewer columns than this. A variation of the previous lemma provides a solution. Suppose $A$ is row reduced to its {\ef} $R$. Identify the pivot columns of $R$ (columns which have leading ones), and take the corresponding columns of $A$. It turns out that this forms a basis of $\func{col}(A)$. 

Before proceeding to an example of this concept, we revisit the definition of rank. 

\begin{definition}{Rank of a matrix}{rankrowspace}
Previously, we defined $\func{rank}(A)$ to be the number of leading entries in the {\ef} of $A$. Using an understanding of dimension and row space, we can now define rank as follows:
\[
\mbox{rank}(A) = \dim(\func{row}(A))
\]
\end{definition}

Consider the following example. 

\begin{example}{Rank, column and row space}{rankcolumnrowspace}
Find the rank of the following matrix and describe the column and row spaces.
\begin{equation*}
A = 
\begin{mymatrix}{rrrrr}
1 & 2 & 1 & 3 & 2 \\
1 & 3 & 6 & 0 & 2 \\
3 & 7 & 8 & 6 & 6
\end{mymatrix}  
\end{equation*}
\end{example}

\begin{solution}
The {\rref} of $A$ is 
\begin{equation*}
\begin{mymatrix}{rrrrr}
1 & 0 & -9 & 9 & 2 \\ 
0 & 1 & 5 & -3 & 0 \\ 
0 & 0 & 0 & 0 & 0
\end{mymatrix} 
\end{equation*}
Therefore, the rank is $2$.

Notice that the first two columns of $R$ are pivot columns. By the discussion following Lemma \ref{lem:echelonformrowspace}, we find the corresponding columns of $A$, in this case the first two columns. Therefore a basis for $\func{col}(A)$ is given by
\[
\set{\begin{mymatrix}{r}
1 \\ 
1 \\ 
3
\end{mymatrix} , \begin{mymatrix}{r}
2 \\ 
3 \\ 
7
\end{mymatrix}
}
\] 

 For
example, consider the third column of the original matrix. It can be written as a linear combination of the first two columns of the original matrix as follows.
\begin{equation*}
\begin{mymatrix}{r}
1 \\ 
6 \\ 
8
\end{mymatrix} =-9\begin{mymatrix}{r}
1 \\ 
1 \\ 
3
\end{mymatrix} +5\begin{mymatrix}{r}
2 \\ 
3 \\ 
7
\end{mymatrix} 
\end{equation*}

What about an efficient description of the row space? By Lemma \ref{lem:echelonformrowspace} we know that the nonzero rows of $R$ create a basis of $\func{row}(A)$.  For the above matrix, the row space equals 
\[
\func{row}(A) = 
\func{span} \set{
\begin{mymatrix}{rrrrr}
1 & 0 & -9 & 9 & 2
\end{mymatrix}, \begin{mymatrix}{rrrrr}
0 & 1 & 5 & -3 & 0
\end{mymatrix}
}
\]
\end{solution}

Notice that the column space of $A$ is given as the span of columns of the original matrix, while the row space of $A$ is the span of rows of the {\rref} of $A$. 

Consider another example.

\begin{example}{Rank, column and row space}{rankcolumnrowspace2}
Find the rank of the following matrix and describe the
column and row spaces.
\begin{equation*}
\begin{mymatrix}{rrrrrr}
1 & 2 & 1 & 3 & 2 \\
1 & 3 & 6 & 0 & 2 \\
1 & 2 & 1 & 3 & 2 \\
1 & 3 & 2 & 4 & 0
\end{mymatrix}
\end{equation*}
\end{example}

\begin{solution}
The {\rref} is 
\begin{equation*}
\begin{mymatrix}{rrrrrr}
1 & 0 & 0 & 0 & \vspace{0.05in}\frac{13}{2} \\ 
0 & 1 & 0 & 2 & -\vspace{0.05in}\frac{5}{2} \\ 
0 & 0 & 1 & -1 & \vspace{0.05in}\frac{1}{2} \\ 
0 & 0 & 0 & 0 & 0
\end{mymatrix} 
\end{equation*}
and so the rank is $3$. The row space is given by 
\begin{equation*}
\func{row}(A) = \func{span} \set{
\begin{mymatrix}{ccccc}
1 & 0 & 0 & 0 & \frac{13}{2}
\end{mymatrix},
\begin{mymatrix}{rrrrr}
0 & 1 & 0 & 2 & -\frac{5}{2}
\end{mymatrix} , 
\begin{mymatrix}{rrrrr}
0 & 0 & 1 & -1 & \frac{1}{2}
\end{mymatrix}
 }
\end{equation*}
Notice that the first three columns of the {\rref} are pivot columns. The column space is the span of the first three columns in the \textbf{original matrix}, 
\begin{equation*}
\func{col}(A) = \func{span} \set{\begin{mymatrix}{r}
1 \\ 
1 \\ 
1 \\ 
1
\end{mymatrix}, \; \begin{mymatrix}{r}
2 \\ 
3 \\ 
2 \\ 
3
\end{mymatrix} , \; \begin{mymatrix}{r}
1 \\ 
6 \\ 
1 \\ 
2
\end{mymatrix} }
\end{equation*}

\end{solution}

Consider the solution given above for Example
\ref{exa:rankcolumnrowspace2}, where the rank of $A$ equals
$3$. Notice that the row space and the column space each had dimension
equal to $3$. It turns out that this is not a coincidence, and this
essential result is referred to as the Rank Theorem and is given
now. Recall that we defined $\func{rank}(A) = \func{dim}(\func{row}(A))$. 

\begin{theorem}{Rank theorem}{ranktheorem}
Let $A$ be an $m \times n$ matrix. Then $\func{dim}(\func{col} (A))$, the dimension of the column space, is equal to the dimension of the row space, $\func{dim}(\func{row}(A))$.
\end{theorem}

The following statements all follow from the Rank Theorem.

\begin{corollary}{Results of the rank theorem}{ranktheorem}
Let $A$ be a matrix. Then the following are true:
\begin{enumerate}
\item
 $\func{rank}(A) = \func{rank}(A^T)$.
\item
For $A$ of size $m \times n$, $\func{rank}(A) \leq m$ and $\func{rank}(A) \leq n$.
\item
For $A$ of size $n \times n$,  $A$ is invertible if and only if $\func{rank}(A) = n$.
\item
For invertible matrices $B$ and $C$ of appropriate size, 
$\func{rank}(A) = \func{rank}(BA) = \func{rank}(AC)$. 
\end{enumerate}
\end{corollary}

Consider the following example.

\begin{example}{Rank of the transpose}{ranktranspose}
Let \[
A = 
\begin{mymatrix}{rr}
1 & 2 \\
-1 & 1 
\end{mymatrix}
\]
Find $\func{rank}(A)$ and $\func{rank}(A^T)$.
\end{example}

\begin{solution}
To find $\func{rank}(A)$ we first row reduce to find the {\rref}.
\[
A = 
\begin{mymatrix}{rr}
1 & 2 \\
-1 & 1 
\end{mymatrix}
\rightarrow \cdots \rightarrow
\begin{mymatrix}{rr}
1 & 0 \\
0 & 1 
\end{mymatrix}
\]

Therefore the rank of $A$ is $2$. Now consider $A^T$ given by 
\[
A^T = \begin{mymatrix}{rr}
1 & -1 \\
2 & 1 
\end{mymatrix}
\]
Again we row reduce to find the {\rref}.
\[
\begin{mymatrix}{rr}
1 & -1 \\
2 & 1 
\end{mymatrix}
\rightarrow \cdots \rightarrow
\begin{mymatrix}{rr}
1 & 0 \\
0 & 1 
\end{mymatrix}
\]

You can see that $\func{rank}(A^T) = 2$, the same as $\func{rank}(A)$. 
\end{solution}

We now define what is meant by the null space of a general $m\times n$ matrix.

\begin{definition}{Null space, or kernel, of $A$}{nullspace}
The null space of a matrix $A$, also referred to as the kernel of $A$, 
is defined as follows.\index{kernel}\index{matrix!kernel}%
\index{null space}\index{matrix!null space}
\begin{equation*}
\func{null} \tup{A} =\set{\vect{x} :A \vect{x} =\vect{0}}
\end{equation*}
\end{definition}

It can also be referred to using the notation $\ker \tup{
A} $. 
Similarly, we can discuss the image of $A$, denoted by
$\func{im}\tup{A}$. The image of $A$ consists of the vectors
of $\R^{m}$ which ``get hit'' by $A$.  The formal definition
is as follows.

\begin{definition}{Image of $A$}{image}
The image of $A$, written $\func{im}\tup{A}$ is given by
\[
\func{im}\tup{A } = \set{A\vect{x} : \vect{x} \in \R^n }
\]
\end{definition}

Consider
$A$ as a mapping from $\R^{n}$ to $\R^{m}$ whose action is given by multiplication. The following diagram displays this scenario. 
\begin{equation*}
\overset{\func{null} \tup{A} }{\R^{n}}\ \overset{A}{\rightarrow }\ 
\overset{
\func{im}\tup{A} }{\R^{m}}
\end{equation*}
As indicated, $\func{im}\tup{A} $ is a subset of $\R^{m}$
while $\func{null} \tup{A} $ is a subset of $\R^{n}$.

It turns out that the null space and image of $A$ are both subspaces. Consider the following example.

\begin{example}{Null space}{nullsubspace}
Let $A$ be an $m\times n$ matrix. Then the null space of $A$, $\func{null}(A)$ is 
a subspace of $\R^n$.
\end{example}

\begin{solution}
\begin{itemize}
\item Since $A\vect{0}_n=\vect{0}_m$, 
$\vect{0}_n\in\func{null}(A)$.

\item Let $\vect{x},\vect{y}\in\func{null}(A)$. 
Then $A\vect{x}=\vect{0}_m$ and $A\vect{y}=\vect{0}_m$, so
\[ A(\vect{x}+\vect{y})=A\vect{x}+A\vect{y} = \vect{0}_m+\vect{0}_m=\vect{0}_m,\]
and thus $\vect{x}+\vect{y}\in\func{null}(A)$.
\item Let $\vect{x}\in\func{null}(A)$ and $k\in\R$.
Then $A\vect{x}=\vect{0}_m$, so

\[ A(k\vect{x}) = k(A\vect{x})=k\vect{0}_m=\vect{0}_m,\]

and thus $k\vect{x}\in\func{null}(A)$.
\end{itemize}
Therefore by the subspace test, $\func{null}(A)$ is a subspace of $\R^n$.

\end{solution}

The proof that $\func{im}(A)$ is a subspace of $\R^m$ is similar and is left as an exercise to the reader. 

We now wish to find a way to describe $\func{null}(A)$ for a matrix $A$. However, finding $\func{null} \tup{A}$ is not new! There is just some new
terminology being used, as $\func{null} \tup{A} $ is simply the solution
to the system $A\vect{x}=\vect{0}$.

\begin{theorem}{Basis of $\func{null}(A)$}{nullspacebasis}
Let $A$ be an $m \times n$ matrix such that $\func{rank}(A) = r$. Then the system $A\vect{x}=\vect{0}_m$ has $n-r$ basic solutions, providing a basis of $\func{null}(A)$ with $\dim(\func{null}(A))=n-r$.
\end{theorem}

Consider the following example. 

\begin{example}{Null space of $A$}{nullspace}
Let
\begin{equation*}
A=\begin{mymatrix}{rrr}
1 & 2 & 1 \\
0 & -1 & 1 \\
2 & 3 & 3
\end{mymatrix} 
\end{equation*}
Find $\func{null} \tup{A} $ and $\func{im}\tup{A}$.
\end{example}

\begin{solution}
In order to find $\func{null} \tup{A} $, we simply need to solve the
equation $A\vect{x}=\vect{0}$. This is the usual procedure of writing
the augmented matrix, finding the {\rref} and then the solution. The
augmented matrix and corresponding {\rref} are
\begin{equation*}
\begin{mymatrix}{rrr|r}
1 & 2 & 1 & 0 \\ 
0 & -1 & 1 & 0 \\ 
2 & 3 & 3 & 0
\end{mymatrix}
\rightarrow \cdots \rightarrow
\begin{mymatrix}{rrr|r}
1 & 0 & 3 & 0 \\ 
0 & 1 & -1 & 0 \\ 
0 & 0 & 0 & 0
\end{mymatrix}
\end{equation*}
The third column is not a pivot column, and therefore the solution
will contain a parameter. The solution to the system
$A\vect{x}=\vect{0}$ is given by
\begin{equation*}
\begin{mymatrix}{r}
-3t \\ 
t \\ 
t
\end{mymatrix} :t\in \R
\end{equation*}
which can be written as 
\begin{equation*}
t
\begin{mymatrix}{r}
-3 \\ 
1 \\ 
1
\end{mymatrix} :t\in \R
\end{equation*}

Therefore, the null space of $A$ is all multiples of this vector, which we can write as
\begin{equation*}
\func{null} (A) = \func{span} \set{\begin{mymatrix}{r}
-3 \\ 
1 \\ 
1
\end{mymatrix}
}
\end{equation*}

Finally $\func{im}\tup{A} $ is just $\set{A\vect{x} :
\vect{x} \in \R^n }$ and hence consists of the span of
all columns of $A$, that is $\func{im}\tup{A} = \func{col} (A)$. 

Notice from the above calculation that that the first two columns of the {\rref} are pivot
columns. Thus the  column space is the span of the first two  columns in
the \textbf{original matrix}, and we get 
\begin{equation*}
\func{im}\tup{A} = \func{col}(A) =
\func{span} \set{\begin{mymatrix}{r}
1 \\ 
0 \\ 
2 
\end{mymatrix}, \; \begin{mymatrix}{r}
2 \\ 
-1 \\ 
3 
\end{mymatrix}  }
\end{equation*}

\end{solution}

Here is a larger example, but the method is entirely similar.

\begin{example}{Null space of $A$}{nullspace2}
Let
\begin{equation*}
A=\begin{mymatrix}{rrrrr}
1 & 2 & 1 & 0 & 1 \\
2 & -1 & 1 & 3 & 0 \\
3 & 1 & 2 & 3 & 1 \\
4 & -2 & 2 & 6 & 0
\end{mymatrix}
\end{equation*}
Find the null space of $A$.
\end{example}

\begin{solution}
To find the null space, we need to solve the equation $AX=0$. The augmented matrix and corresponding {\rref} are given by 
\begin{equation*}
\begin{mymatrix}{rrrrr|r}
1 & 2 & 1 & 0 & 1 & 0 \\ 
2 & -1 & 1 & 3 & 0 & 0 \\ 
3 & 1 & 2 & 3 & 1 & 0 \\ 
4 & -2 & 2 & 6 & 0 & 0
\end{mymatrix}
\rightarrow \cdots \rightarrow
\begin{mymatrix}{rrrrr|r}
1 & 0 & \vspace{0.05in}\frac{3}{5} & \vspace{0.05in}\frac{6}{5} & \vspace{0.05in}\frac{1}{5} & 0 \\ 
0 & 1 & \vspace{0.05in}\frac{1}{5} & -\vspace{0.05in}\frac{3}{5} & \vspace{0.05in}\frac{2}{5} & 0 \\ 
0 & 0 & 0 & 0 & 0 & 0 \\ 
0 & 0 & 0 & 0 & 0 & 0
\end{mymatrix}
\end{equation*}
It follows that the first two columns are pivot columns, and the next three correspond to parameters. Therefore, $\func{null} \tup{A} $ is given by 
\begin{equation*}
\begin{mymatrix}{c}
\tup{-\vspace{0.05in}\frac{3}{5}} s +\tup{-\vspace{0.05in}\frac{6}{5}} t+\tup{
\vspace{0.05in}\frac{1}{5}} r \\ 
\tup{-\vspace{0.05in}\frac{1}{5}} s +\tup{\vspace{0.05in}\frac{3}{5}} t +\tup{-
\vspace{0.05in}\frac{2}{5}} r \\ 
s \\ 
t \\ 
r
\end{mymatrix} :s ,t ,r\in \R\text{.}
\end{equation*}
We write this in the form 
\begin{equation*}
s \begin{mymatrix}{r}
-\vspace{0.05in}\frac{3}{5} \\ 
-\vspace{0.05in}\frac{1}{5} \\ 
1 \\ 
0 \\ 
0
\end{mymatrix} + t \begin{mymatrix}{r}
-\vspace{0.05in}\frac{6}{5} \\ 
\vspace{0.05in}\frac{3}{5} \\ 
0 \\ 
1 \\ 
0
\end{mymatrix} + r \begin{mymatrix}{r}
\vspace{0.05in}\frac{1}{5} \\ 
-\vspace{0.05in}\frac{2}{5} \\ 
0 \\ 
0 \\ 
1
\end{mymatrix} :s , t , r\in \R\text{.}
\end{equation*}
In other words, the null space of this matrix equals the span of the three
vectors above. Thus 
\begin{equation*}
\func{null} \tup{A} =\func{span}\set{\begin{mymatrix}{r}
-\vspace{0.05in}\frac{3}{5} \\ 
-\vspace{0.05in}\frac{1}{5} \\ 
1 \\ 
0 \\ 
0
\end{mymatrix} ,\begin{mymatrix}{r}
-\vspace{0.05in}\frac{6}{5} \\ 
\vspace{0.05in}\frac{3}{5} \\ 
0 \\ 
1 \\ 
0
\end{mymatrix} ,\begin{mymatrix}{r}
\vspace{0.05in}\frac{1}{5} \\ 
-\vspace{0.05in}\frac{2}{5} \\ 
0 \\ 
0 \\ 
1
\end{mymatrix} } 
\end{equation*}
\end{solution}

Notice also that the three vectors above are linearly independent and
so the dimension of $\func{null} \tup{A} $ is 3. The following is
true in general, the number of parameters in the solution of $AX=0$
equals the dimension of the null space. Recall also that the number of
leading ones in the {\rref} equals the number of pivot columns, which is
the rank of the matrix, which is the same as the dimension of either
the column or row space.

Before we proceed to an important theorem, we first define what is meant by the nullity of a matrix. 

\begin{definition}{Nullity}{nullity}
The dimension of the null space of a matrix is called the \textbf{nullity}\index{nullity}, denoted $\dim( \func{null}\tup{A})$.
\end{definition}

From our observation above we can now state an important theorem.

\begin{theorem}{Rank and nullity}{ranknullity}
Let $A$ be an $m\times n$ matrix. Then $\func{rank}\tup{A} + \dim( \func{null}\tup{A}) =n$. 
\end{theorem}

Consider the following example, which we first explored above in Example \ref{exa:nullspace}

\begin{example}{Rank and nullity}{ranknullity}
Let
\begin{equation*}
A=\begin{mymatrix}{rrr}
1 & 2 & 1 \\
0 & -1 & 1 \\
2 & 3 & 3
\end{mymatrix} 
\end{equation*}

Find $\func{rank}\tup{A}$ and $\dim( \func{null}\tup{A})$. 
\end{example}

\medskip
\begin{solution}
In the above Example \ref{exa:nullspace} we determined that the {\rref} of $A$ is given by 
\[
\begin{mymatrix}{rrr}
1 & 0 & 3 \\ 
0 & 1 & -1  \\
0 & 0 & 0 
\end{mymatrix}
\]

Therefore the rank of $A$ is $2$. We also determined that the null space of $A$ is given by 
\[
\func{null} (A) = \func{span} \set{\begin{mymatrix}{r}
-3 \\ 
1 \\ 
1
\end{mymatrix}
}
\]

Therefore the nullity of $A$ is $1$. It follows from Theorem \ref{thm:ranknullity} that $\func{rank}\tup{A} + \dim( \func{null}\tup{A}) = 2 + 1 = 3$, which is the number of columns of $A$.
\end{solution} 

We conclude this section with two similar, and important, theorems.

\begin{theorem}{}{}
Let $A$ be an $m\times n$ matrix.
The following are equivalent.
\begin{enumerate}
\item $\func{rank}(A)=n$.
\item $\func{row}(A)=\R^n$, i.e., the rows of $A$ span $\R^n$.
\item The columns of $A$ are independent in $\R^m$.
\item The $n\times n$ matrix $A^TA$ is invertible.
\item There exists an $n\times m$ matrix $C$ so that $CA=I_n$.
\item If $A\vect{x}=\vect{0}_m$ for some $\vect{x}\in\R^n$,
then $\vect{x}=\vect{0}_n$.
\end{enumerate}
\end{theorem}

\begin{theorem}{}{}
Let $A$ be an $m\times n$ matrix.
The following are equivalent.
\begin{enumerate}
\item $\func{rank}(A)=m$.
\item $\func{col}(A)=\R^m$, i.e., the columns of $A$ span $\R^m$.
\item The rows of $A$ are independent in $\R^n$.
\item The $m\times m$ matrix $AA^T$ is invertible.
\item There exists an $n\times m$ matrix $C$ so that $AC=I_m$.
\item The system $A\vect{x}=\vect{b}$ is consistent for
every $\vect{b}\in\R^m$.
\end{enumerate}
\end{theorem}
