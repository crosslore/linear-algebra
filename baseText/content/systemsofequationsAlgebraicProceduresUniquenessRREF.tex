\section{Uniqueness of the {\rref}}

\begin{outcome}
  \begin{enumerate}
  \item Determine whether two systems of equations are row equivalent,
    by comparing their {\rref}.
  \item For two homogeneous systems of equations that are not row
    equivalent, find a solution to one system that is not a solution
    to the other.
  \end{enumerate}
\end{outcome}

We have seen in earlier sections that every matrix can be brought into
{\rref} by a sequence of elementary row operations. Here we will prove
that the resulting matrix is unique; in other words, the resulting
matrix in {\rref} does not depend upon the particular sequence of
elementary row operations or the order in which they were performed.

Let $A$ be the augmented matrix of a homogeneous system of linear
equations in the variables $x_1, x_2, \cdots, x_n$ which is also in
{\rref}. Recall that the matrix $A$ divides the set of variables in
two different types: $x_i$ is a
{\em pivot variable}\index{variable!pivot} when column $i$ is a pivot
column, and a {\em free variable}\index{variable!free} otherwise.

\begin{example}{Pivot and free variables}{pivot-free}
Find the pivot and free variables in the following system, and find
the general solution.
\[
\begin{array}{c}
x+2y-z+w=0 \\
x+y-z+w=0 \\
x+3y-z+w=0
\end{array}
\]
\end{example}

\begin{solution}
The {\rref} of the augmented matrix is
\[
\begin{mymatrix}{rrrr|r}
\circled{1} & 0 & -1 & 1 & 0 \\
0 & \circled{1} & 0 & 0 & 0 \\
0 & 0 & 0 & 0 & 0
\end{mymatrix}.
\]
From this, we see that columns $1$ and $2$ are pivot
columns. Therefore, $x$ and $y$ are pivot variables and $z$ and $w$
are free variables. We can write the solution to this system as
\[
\begin{array}{r@{~}c@{~}l}
x &=& s-t \\
y &=& 0 \\
z &=& s \\
w &=& t.
\end{array}
\]
\end{solution}

In general, all solutions can be written in terms of the free
variables. In such a description, the free variables are written as
parameters, while the pivot variables are written as functions of
these parameters. Indeed, a pivot variable $x_i$ is a function of {\em
  only} those free variables $x_j$ with $j>i$. This leads to the
following observation.

\begin{proposition}{Pivot and free variables}{pivot-free}
  If $x_i$ is a pivot variable of a homogeneous system of linear
  equations, then any solution of the system with $x_j=0$ for all
  those free variables $x_j$ with $j>i$ must also have $x_i=0$.
\end{proposition}

Using this proposition, we prove a lemma which will be used in the
proof of the main result of this section.

\begin{lemma}{Solutions and the {\rref} of a matrix}{rref-solutions}
  Let $A$ and $B$ be two augmented matrices for two homogeneous
  systems of $m$ equations in $n$ variables, such that $A$ and $B$ are
  each in {\rref}. If $A$ and $B$ are different, then the two systems
  do not have exactly the same solutions.
\end{lemma}

\begin{proof}
With respect to the linear systems associated with the matrices $A$ and $B$, there are two cases to consider:
\begin{itemize}
\item Case $1$: the two systems have the same pivot variables
\item Case $2$: the two systems do not have the same pivot variables
\end{itemize}
In case $1$, the two matrices will have exactly the same pivot positions. However, since $A$ and $B$ are not identical, there is some row of $A$ which is different from the corresponding row of $B$ and yet the rows each have a pivot in the same column position. Let $i$ be the index of this column position. Since the matrices are in {\rref}, the two rows must differ at some entry in a column $j>i$. Let these entries be $a$ in $A$ and $b$ in $B$, where $a \neq b$. Since $A$ is in {\rref}, if $x_j$ were a pivot variable for its linear system, we would have $a=0$. Similarly, if $x_j$ were a pivot variable for the linear system of the matrix $B$, we would have $b=0$. Since $a$ and $b$ are unequal, they cannot both be equal to $0$, and hence $x_j$ cannot be a pivot variable for both linear systems. However, since the systems have the same pivot variables, $x_j$ must then be a free variable for each system. We now look at the solutions of the systems in which $x_j$ is set equal to $1$ and all other free variables are set equal to $0$. For this choice of parameters, the solution of the system for matrix $A$ has $x_j=-a$, while the solution of the system for matrix $B$ has $x_j=-b$, so that the two systems have different solutions.

In case $2$, there is a variable $x_i$ which is a pivot variable for one matrix, let's say $A$, and a free variable for the other matrix $B$. The system for matrix $B$ has a solution in which $x_i=1$ and $x_j=0$ for all other free variables $x_j$. However, by Proposition \ref{prop:pivot-free} this cannot be a solution of the system for the matrix $A$. This completes the proof of case $2$.
\end{proof}

Now, we say that the matrix $B$ is \textbf{row equivalent}\index{matrix!row equivalence}\index{row equivalence} to the matrix $A$ if $B$ can be obtained from $A$ by performing a sequence of elementary row operations. By Theorem~\ref{thm:elementary-operations-and-solns}, we know that row equivalent systems have exactly the same solutions. Now, we can use Lemma \ref{lem:rref-solutions} to prove the main result of this section, which is that each matrix $A$ has a unique {\rref}.

\begin{theorem}{Uniqueness of the {\rref}}{unique-rref}
  Every matrix $A$ is row equivalent to a unique matrix in {\rref}.
\end{theorem}

\begin{proof}
  By Gauss-Jordan elimination, we already know that every matrix is
  row equivalent to some {\rref}. What we must show is that the
  resulting {\rref} is unique, i.e., does not depend on the order in
  which row operations are performed.
  
  Therefore, let $A$ be an $m \times n$-matrix and let $B$ and $C$ be
  matrices in {\rref}, each row equivalent to $A$. We have to show
  that $B=C$.

  Let $A^{+}$ be the matrix $A$ augmented with a new rightmost column
  consisting entirely of zeros. Similarly, augment matrices $B$ and
  $C$ each with a rightmost column of zeros to obtain $B^{+}$ and
  $C^{+}$. Note that $B^{+}$ and $C^{+}$ are augmented matrices in
  {\rref}, and that both $B^{+}$ and $C^{+}$ are row equivalent to
  $A^{+}$, because the addition of a column of zeros does not change
  the effect of any row operations.

  Now, $A^{+}$, $B^{+}$, and $C^{+}$ can all be considered as
  augmented matrices of homogeneous linear systems in the variables
  $x_1, x_2, \cdots, x_n$. Because all three systems are row
  equivalent, they have exactly the same solutions. By
  Lemma~\ref{lem:rref-solutions}, we conclude that $B^{+}=C^{+}$.
  Omitting the final column of zeros, we must also have $B=C$.
\end{proof}

\begin{example}{Row equivalent systems}{det-row-eq}
  Determine whether the following two systems of equations row equivalent.
  \begin{equation*}
    \begin{array}{r@{~}c@{~}l}
      2x + 3y + z &=& 12 \\
      x - 2y + 4z &=& -1 \\
      x + 2z &=& 3,
    \end{array}
    \quad\quad
    \begin{array}{r@{~}c@{~}l}
      x + 2y &=& 7 \\
      3x - y + 7z &=& 7 \\
      y - z &=& 2.
    \end{array}
  \end{equation*}
\end{example}

\begin{solution}
  The augmented matrices for the two systems are:
  \begin{equation*}
    \begin{mymatrix}{rrr|r}
      2 & 3 & 1 & 12 \\
      1 & -2 & 4 & -1\\
      1 & 0 & 2 & 3
    \end{mymatrix},
    \quad\quad
    \begin{mymatrix}{rrr|r}
      1 & 2 & 0 & 7 \\
      3 & -1 & 7 & 7\\
      0 & 1 & -1 & 2
    \end{mymatrix}.
  \end{equation*}
  The {\rref}s of the two augmented matrices are:
  \begin{equation*}
    \begin{mymatrix}{rrr|r}
      1 & 0 & 2 & 3 \\
      0 & 1 & -1 & 2 \\
      0 & 0 & 0 & 0
    \end{mymatrix},
    \quad\quad
    \begin{mymatrix}{rrr|r}
      1 & 0 & 2 & 3 \\
      0 & 1 & -1 & 2 \\
      0 & 0 & 0 & 0
    \end{mymatrix}.
  \end{equation*}
  Since both systems have the same {\rref}, they are row equivalent.
\end{solution}

\begin{example}{Non-row equivalent systems}{non-row-eq}
  Determine whether the following two systems of equations row
  equivalent. If they are not row equivalent, find a solution to one
  system that is not a solution to the other.
  \begin{equation*}
    \begin{array}{r@{~}c@{~}l}
      x - 2y - 5z &=& 0 \\
      x + z &=& 0 \\
      x + y + 4z &=& 0,
    \end{array}
    \quad\quad
    \begin{array}{r@{~}c@{~}l}
      2x + 2y + z &=& 0 \\
      x + y + 3z &=& 0 \\
      -x - y + 2z &=& 0.  
    \end{array}
  \end{equation*}
\end{example}

\begin{solution}
  The augmented matrices for the two systems are:
  \begin{equation*}
    \begin{mymatrix}{rrr|r}
      1 & -2 & -5 & 0\\
      1 & 0 & 1 & 0 \\
      1 & 1 & 4 & 0
    \end{mymatrix},
    \quad\quad
    \begin{mymatrix}{rrr|r}
      2 & 2 & 1 & 0 \\
      1 & 1 & 3 & 0\\
      -1 & -1 & 2 & 0
    \end{mymatrix}.
  \end{equation*}
  The {\rref}s of the two augmented matrices are:
  \begin{equation*}
    \begin{mymatrix}{rrr|r}
      1 & 0 & 1 & 0 \\
      0 & 1 & 3 & 0 \\
      0 & 0 & 0 & 0
    \end{mymatrix},
    \quad\quad
    \begin{mymatrix}{rrr|r}
      1 & 1 & 0 & 0 \\
      0 & 0 & 1 & 0 \\
      0 & 0 & 0 & 0
    \end{mymatrix}.
  \end{equation*}
  Since the two systems have different {\rref}s, they are not row
  equivalent.  Following the proof of Lemma~\ref{lem:rref-solutions},
  we see that $z$ is a free variable for the first system, but a pivot
  variable for the second system. Therefore, there exists a solution
  of the first system with $z=1$, namely $(x,y,z) = (-1,-3,1)$. But
  there exists no solution for the second system with $z=1$, and in
  particular, $(x,y,z) = (-1,-3,1)$ is not a solution of the second
  system. 
\end{solution}

We finish this section by pointing out an important consequence of
Theorem~\ref{thm:unique-rref}, namely that the rank\index{rank} of a
matrix is well-defined. Recall that in Definition~\ref{def:rank}, we
defined the rank of a matrix $A$ to be the number of pivot entries of
``any'' {\ef} of $A$. It was not clear, however, why different {\ef}s
of $A$ could not have different numbers of pivot entries. Now we can
answer this question. By the Gauss-Jordan algorithm, we know that
every {\ef} can be converted to a {\rref} without changing the number
or position of the pivots. Since the {\rref} is unique, it follows
that all {\ef}s of $A$ have the same number of pivot entries (and in
fact the same pivot columns). Therefore, the rank of $A$ is a
well-defined quantity.
