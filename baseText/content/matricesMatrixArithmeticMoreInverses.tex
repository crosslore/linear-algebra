\section{More on matrix inverses}

In this section, we will prove three theorems which will clarify the concept of matrix inverses. In order to do this, first recall some important properties of elementary matrices. 

Recall that an elementary matrix is a square matrix obtained by performing an elementary operation on an identity matrix. Each elementary matrix is invertible, and its inverse is also an elementary matrix. If $E$ is an $m \times m$ elementary matrix and $A$ is an $m \times n$-matrix, then the product $EA$ is the result of applying to $A$ the same elementary row operation that was applied to the $m \times m$ identity matrix in order to obtain $E$.

Let $R$ be the {\rref} of an $m \times n$-matrix $A$. $R$ is obtained by iteratively applying a sequence of elementary row operations to $A$. Denote by $E_1, E_2, \cdots, E_k$ the elementary matrices associated with the elementary row operations which were applied, in order, to the matrix $A$ to obtain the resulting $R$. We then have that $R = \tup{E_k \cdots \tup{E_2 \tup{E_1A } }} = E_k \cdots E_2E_1A$. Let $E$ denote the product matrix $E_k \cdots E_2E_1$ so that we can write $R=EA$ where $E$ is an invertible matrix whose inverse is the product $(E_1)^{-1}(E_2)^{-1} \cdots (E_k)^{-1}$.

Now, we will consider some preliminary lemmas. 

\begin{lemma}{Invertible matrix and zeros}{no-zero-row}
Suppose that $A$ and $B$ are matrices such that the product $AB$ is an identity matrix. Then the {\rref} of $A$ does not have a row of zeros.
\end{lemma}

\begin{proof}
Let $R$ be the {\rref} of $A$. Then $R=EA$ for some invertible square matrix $E$ as described above. By hypothesis $AB=I$ where $I$ is an identity matrix, so we have a chain of equalities
\begin{equation*}
R(BE^{-1})
=
(EA)(BE^{-1})
=
E(AB)E^{-1}
=
EIE^{-1}
=
EE^{-1}
=
I
\end{equation*}
If $R$ would have a row of zeros, then so would the product $R(BE^{-1})$. But since the identity matrix $I$ does not have a row of zeros, neither can $R$ have one.
\end{proof}

We now consider a second important lemma.

\begin{lemma}{Size of invertible matrix}{size-invertible}
Suppose that $A$ and $B$ are matrices such that the product $AB$ is an identity matrix. Then $A$ has at least as many columns as it has rows.
\end{lemma}

\begin{proof}
Let $R$ be the {\rref} of $A$. By Lemma \ref{lem:no-zero-row}, we know that $R$ does not have a row of zeros, and therefore each row of $R$ has a leading $1$. Since each column of $R$ contains at most one of these leading $1$s, $R$ must have at least as many columns as it has rows.
\end{proof}

An important theorem follows from this lemma. 

\begin{theorem}{Invertible matrices are square}{square-invertible}
Only square matrices can be invertible.
\end{theorem}

\begin{proof}
Suppose that $A$ and $B$ are matrices such that both products $AB$ and $BA$ are identity matrices. We will show that $A$ and $B$ must be square matrices of the same size. Let the matrix $A$ have $m$ rows and $n$ columns, so that $A$ is an $m \times n$-matrix. Since the product $AB$ exists, $B$ must have $n$ rows, and since the product $BA$ exists, $B$ must have $m$ columns so that $B$ is an $n \times m$-matrix. To finish the proof, we need only verify that $m=n$. 

We first apply Lemma \ref{lem:size-invertible} with $A$ and $B$, to obtain the inequality $m \leq n$. We then apply Lemma \ref{lem:size-invertible} again (switching the order of the matrices), to obtain the inequality $n \leq m$. It follows that $m=n$, as we wanted.
\end{proof}

Of course, not all square matrices are invertible. In particular, zero matrices are not invertible, along with many other square matrices. 

The following proposition will be useful in proving the next theorem.

\begin{proposition}{{\Rref} of a square matrix}{rref-square}
If $R$ is the {\rref} of a square matrix, then either $R$ has a row of zeros or $R$ is an identity matrix. 
\end{proposition}

The proof of this proposition is left as an exercise to the reader. 
We now consider the second important theorem of this section. 

\begin{theorem}{Unique inverse of a matrix}{unique-inverse-of-matrix}
Suppose $A$ and $B$ are square matrices such that $AB=I$ where $I$ is an identity matrix. Then it follows that $BA=I$. Further, both $A$ and $B$ are invertible and $B=A^{-1}$ and $A=B^{-1}$.
\end{theorem}

\begin{proof}
Let $R$ be the {\rref} of a square matrix $A$. Then, $R=EA$ where $E$ is an invertible matrix. Since $AB=I$, Lemma \ref{lem:no-zero-row} gives us that $R$ does not have a row of zeros. By noting that $R$ is a square matrix and applying Proposition \ref{prop:rref-square}, we see that $R=I$. Hence, $EA=I$. 

Using both that $EA=I$ and $AB=I$, we can finish the proof with a chain of equalities as given by
\begin{eqnarray*}
BA = IBIA &=& (EA)B(E^{-1}E)A \\
&=& E(AB)E^{-1}(EA) \\
&=& EIE^{-1}I \\
&=& EE^{-1} = I
\end{eqnarray*}

It follows from the definition of the inverse of a matrix that $B=A^{-1}$ and $A=B^{-1}$.
\end{proof}

This theorem is very useful, since with it we need only test one of the products $AB$ or $BA$ in order to check that $B$ is the inverse of $A$. The hypothesis that $A$ and $B$ are square matrices is very important, and without this the theorem does not hold.

We will now consider an example.

\begin{example}{Non square matrices}{non-square}
Let 
\begin{equation*}
A =
\begin{mymatrix}{rr}
1 & 0 \\
0 & 1 \\
0 & 0 
\end{mymatrix},
\end{equation*}
Show that $A^{T}A = I$ but $AA^{T} \neq 0$. 
\end{example}

\begin{solution}
Consider the product $A^{T}A$ given by 
\begin{equation*}
\begin{mymatrix}{rrr}
1 & 0 & 0\\
0 & 1 & 0
\end{mymatrix}
\begin{mymatrix}{rr}
1 & 0 \\
0 & 1 \\
0 & 0 
\end{mymatrix}
=
\begin{mymatrix}{rr}
1 & 0 \\
0 & 1
\end{mymatrix}
\end{equation*}
Therefore, $A^{T}A = I_2$, where $I_2$ is the $2 \times 2$ identity matrix.
However, the product $AA^{T}$ is 
\begin{equation*}
\begin{mymatrix}{rr}
1 & 0 \\
0 & 1 \\
0 & 0 
\end{mymatrix}
\begin{mymatrix}{rrr}
1 & 0 & 0\\
0 & 1 & 0
\end{mymatrix}
=
\begin{mymatrix}{rrr}
1 & 0 & 0 \\
0 & 1 & 0 \\
0 & 0 & 0
\end{mymatrix}
\end{equation*}
Hence $AA^{T}$ is not the $3 \times 3$ identity matrix. This shows that for Theorem \ref{thm:unique-inverse-of-matrix}, it is essential that both matrices be square and of the same size.
\end{solution}

Is it possible to have matrices $A$ and $B$ such that $AB=I$, while $BA=0$? This question is left to the reader to answer, and you should take a moment to consider the answer.

We conclude this section with an important theorem.

\begin{theorem}{The {\rref} of an invertible matrix}{invertible-rref}
For any matrix $A$ the following conditions are equivalent:
\begin{itemize}
\item $A$ is invertible
\item The {\rref} of $A$ is an identity matrix
\end{itemize}
\end{theorem}

\begin{proof}
In order to prove this, we show that for any given matrix $A$, each condition implies the other. We first show that if $A$ is invertible, then its {\rref} is an identity matrix, then we show that if the {\rref} of $A$ is an identity matrix, then $A$ is invertible. 

If $A$ is invertible, there is some matrix $B$ such that $AB = I$. By Lemma \ref{lem:no-zero-row}, we get that the {\rref} of $A$ does not have a row of zeros. Then by Theorem \ref{thm:square-invertible}, it follows that $A$ and the {\rref} of $A$ are square matrices. Finally, by Proposition \ref{prop:rref-square}, this {\rref} of $A$ must be an identity matrix. This proves the first implication.

Now suppose the {\rref} of $A$ is an identity matrix $I$. Then $I=EA$ for some product $E$ of elementary matrices. By Theorem \ref{thm:unique-inverse-of-matrix}, we can conclude that $A$ is invertible.
\end{proof}

Theorem \ref{thm:invertible-rref} corresponds to Algorithm \ref{algo:matrix-inversion-algorithm}, which claims that $A^{-1}$ is found by row reducing the augmented matrix $\mat{A\mid I}$ to the form $\mat{I\mid A^{-1}}$. This will be a matrix product $E\mat{A\mid I}$ where $E$ is a product of elementary matrices. By the rules of matrix multiplication, we have that $E\mat{A\mid I} = \mat{EA\mid EI} = \mat{EA\mid E}$.

It follows that the {\rref} of $\mat{A\mid I}$ is $\mat{EA\mid E}$, where $EA$ gives the {\rref} of $A$. By Theorem \ref{thm:invertible-rref}, if $EA \neq I$, then $A$ is not invertible, and if $EA=I$, $A$ is invertible. If $EA=I$, then by Theorem \ref{thm:unique-inverse-of-matrix}, $E=A^{-1}$. This proves that Algorithm \ref{algo:matrix-inversion-algorithm} does in fact find $A^{-1}$.
