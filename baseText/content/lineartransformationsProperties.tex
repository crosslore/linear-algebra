\section{Properties of linear transformations}

\begin{outcome}
  \begin{enumerate}
  \item Use properties of linear transformations to solve problems.
  \item Find the composite of transformations and the inverse of a
    transformation.
  \end{enumerate}
\end{outcome}

We begin by noting that linear transformations preserve the zero
vector, negation, and linear combinations.

\begin{proposition}{Properties of linear transformations}{properties-linear-transformation}
  Let $T: \R^n \to \R^m$ be a linear transformation. Then
  \begin{itemize}
  \item $T$ preserves the zero vector: $T(\vect{0}) = \vect{0}$.
  \item $T$ preserves negation: $T(-\vect{v}) = -T(\vect{v})$.
  \item $T$ preserves linear combinations:
    \begin{equation*}
      T(a_1\vect{v}_1 + a_2\vect{v}_2 + \ldots + a_k \vect{v}_k)
      \quad=\quad
      a_1T(\vect{v}_1) + a_2T(\vect{v}_2) + \ldots + a_k T(\vect{v}_k).
    \end{equation*}
  \end{itemize}
\end{proposition}

\begin{example}{Linear combination}{lin-transf-combination}
  Let $T:\R^3 \to \R^4$ be a linear transformation such that
  \begin{equation*}
    T \paren{\begin{mymatrix}{r} 1 \\ 3 \\ 1 \end{mymatrix}}
    = \begin{mymatrix}{r} 4 \\ 4 \\ 0 \\ -2 \end{mymatrix}
    \quad\mbox{and}\quad
    T \paren{\begin{mymatrix}{r} 4 \\ 0 \\ 5 \end{mymatrix}}
    = \begin{mymatrix}{r} 4 \\ 5 \\ -1 \\ 5 \end{mymatrix}.
  \end{equation*}
  Find $T\paren{\begin{mymatrix}{r} -7 \\ 3 \\ -9 \end{mymatrix}}$.
\end{example}

\begin{solution}
  Using the third property in
  Proposition~\ref{prop:properties-linear-transformation}, we can find
  $T\paren{\begin{mymatrix}{r} -7 \\ 3 \\ -9 \end{mymatrix}}$ by writing
  $\begin{mymatrix}{r} -7 \\ 3 \\ -9 \end{mymatrix}$ as a linear
  combination of $\begin{mymatrix}{r} 1 \\ 3 \\ 1 \end{mymatrix}$ and
  $\begin{mymatrix}{r} 4 \\ 0 \\ 5 \end{mymatrix}$.  By solving the
  appropriate system of equations, we find that
  \begin{equation*}
    \begin{mymatrix}{r} -7 \\ 3 \\ -9 \end{mymatrix}
    = \begin{mymatrix}{r} 1 \\ 3 \\ 1 \end{mymatrix}
    - 2 \begin{mymatrix}{r} 4 \\ 0 \\ 5 \end{mymatrix}.
  \end{equation*}
  Therefore,
  \begin{eqnarray*}
    T \paren{\begin{mymatrix}{r} -7 \\ 3 \\ -9 \end{mymatrix}}
    &=&
        T \paren{
        \begin{mymatrix}{r} 1 \\ 3 \\ 1 \end{mymatrix}
    -2 \begin{mymatrix}{r} 4 \\ 0 \\ 5 \end{mymatrix}
}
    \\ &=&
           T \begin{mymatrix}{r} 1 \\ 3 \\ 1 \end{mymatrix}
    -2T \begin{mymatrix}{r} 4 \\ 0 \\ 5 \end{mymatrix}
    ~=~
        \begin{mymatrix}{r} 4 \\ 4 \\ 0 \\ -2 \end{mymatrix}
    -2 \begin{mymatrix}{r} 4 \\ 5 \\ -1 \\ 5 \end{mymatrix}
    ~=~
        \begin{mymatrix}{r} -4 \\ -6 \\ 2 \\ -12 \end{mymatrix}.
  \end{eqnarray*}
\end{solution}

Suppose that we first apply a linear transformation $T$ to a vector,
and then the linear transformation $S$ to the result. The resulting
two-step transformation is also a linear transformation, called the
\textbf{composition} of $T$ and $S$.

\begin{definition}{Composition of linear transformations}{composite-transformations}
  Let $S: \R^k \to \R^n$ and $T: \R^n \to \R^m$ be linear
  transformations. Then the \textbf{composition}%
  \index{linear transformation!composition}%
  \index{composition of linear transformations} of $S$ and $T$ (also
  called the \textbf{composite transformation}%
  \index{composite transformation} of $S$ and $T$) is the linear
  transformation
  \begin{equation*}
    T\circ S: \R^k \to \R^m
  \end{equation*}
  that is defined by
  \begin{equation*}
    (T\circ S) (\vect{v}) = T(S(\vect{v})),
  \end{equation*}
  for all $\vect{v}\in\R^k$.
\end{definition}

Notice that the resulting vector will be in $\R^m$. Be careful to
observe the order of transformations. The composite transformation
$T\circ S$ means that we are {\em first} applying $S$, and {\em then}
$T$. Composition of linear transformations is written from right to
left. The composition $T\circ S$ is sometimes pronounced ``{\em $T$
  after $S$}''.

\begin{theorem}{Matrix of a composite transformation}{composite-transformation}
  Let $S: \R^k \to \R^n$ and $T: \R^n \to \R^m$ be linear
  transformations. Let $A$ be the matrix corresponding to $S$, and let
  $B$ be the matrix corresponding to $T$. Then the matrix
  corresponding to the composite linear transformation $T\circ S$ is
  $BA$.
\end{theorem}

\begin{proof}
  For all $\vect{v}\in\R^k$, we have
  \begin{equation*}
    (T\circ S)(\vect{v}) = T(S(\vect{v})) = B(A \vect{v}) = (BA) \vect{v}.
  \end{equation*}
  Therefore, $BA$ is the matrix corresponding to $T\circ S$.
\end{proof}

\begin{example}{Two rotations}{two-rotations}
  Find the matrix for a counterclockwise rotation%
  \index{matrix!of a rotation}%
  \index{rotation!matrix of}%
  \index{linear transformation!rotation} by angle
  $\theta+\phi$ in two different ways, and compare.
\end{example}

\begin{solution}
  Let $A_{\theta}$ be the matrix of a rotation by $\theta$, and let
  $A_{\phi}$ be the matrix of a rotation by angle $\phi$.  We
  calculated these matrices in Example~\ref{exa:rotation-theta-R2}.
  Then a rotation by the angle $\theta+\phi$ is given by the product
  of these two matrices:
  \begin{eqnarray*}
    A_{\theta}A_{\phi}
    &=& \begin{mymatrix}{cc}
      \cos\theta & -\sin\theta \\
      \sin\theta & \cos\theta \\
    \end{mymatrix}
    \begin{mymatrix}{cc}
      \cos\phi & -\sin\phi \\
      \sin\phi & \cos\phi \\
    \end{mymatrix}
    \\
    &=& \begin{mymatrix}{cc}
      \cos\theta \cos\phi - \sin\theta \sin\phi &
      -\cos\theta \sin\phi - \sin\theta \cos\phi \\
      \sin\theta \cos\phi + \cos\theta \sin\phi &
      \cos\theta \cos\phi - \sin\theta \sin\phi
    \end{mymatrix}.
  \end{eqnarray*}
  On the other hand, we can compute the matrix for a rotation by angle
  $\theta+\phi$ directly:
  \begin{eqnarray*}
    A_{\theta+\phi}
    &=& \begin{mymatrix}{cc}
      \cos(\theta+\phi) & -\sin(\theta+\phi) \\
      \sin(\theta+\phi) & \cos(\theta+\phi) \\
    \end{mymatrix}.
  \end{eqnarray*}
  The fact that these matrices are equal amounts to the well-known
  trigonometric identities for the sum of two angles%
  \index{trigonometry!sum of two angles}%
  \index{sum!of two angles}, which we have here derived using linear
  algebra concepts:
  \begin{eqnarray*}
    \sin(\theta+\phi) &=& \sin\theta \cos\phi + \cos\theta \sin\phi, \\
    \cos(\theta+\phi) &=& \cos\theta \cos\phi - \sin\theta \sin\phi.
  \end{eqnarray*}
\end{solution}

\begin{example}{Multiple rotations in $\R^3$}{multiple-rotations}
  Find the matrix of the linear transformation $T:\R^3\to\R^3$ that is
  given as follows: a rotation by $30$ degrees about the $z$-axis,
  followed by a rotation by $45$ degrees about the $x$-axis.
\end{example}

\begin{solution}
  It would be quite difficult to picture the transformation $T$ in one
  step. Fortunately, we don't have to do this. All we have to do is
  find the matrix for each rotation separately, then multiply the two
  matrices. We have the be careful to multiply the matrices in the
  correct order.

  Let $B$ be the matrix for a $30$-degree rotation about the
  $z$-axis. It is given exactly as in
  Example~\ref{exa:rotation-R3}:
  \begin{equation*}
    \def\arraystretch{1.4}
    B = \begin{mymatrix}{ccc}
      \cos 30^{\circ} & -\sin 30^{\circ} & 0 \\
      \sin 30^{\circ} & \cos 30^{\circ} & 0 \\
      0 & 0 & 1 \\
    \end{mymatrix}
    = \begin{mymatrix}{ccc}
      \frac{\sqrt3}{2} & -\frac{1}{2} & 0 \\
      \frac{1}{2} & \frac{\sqrt3}{2} & 0 \\
      0 & 0 & 1 \\
    \end{mymatrix}.
  \end{equation*}
  Let $C$ be the matrix for a $45$-degree rotation about the $x$-axis.
  It is analogous to Example~\ref{exa:rotation-R3}, except that the
  rotation takes place in the $yz$-plane instead of the $xy$-plane.
  \begin{equation*}
    \def\arraystretch{1.4}
    C = \begin{mymatrix}{ccc}
      1 & 0 & 0 \\
      0 & \cos 45^{\circ} & -\sin 45^{\circ} \\
      0 & \sin 45^{\circ} & \cos 45^{\circ} \\
    \end{mymatrix}
    = \begin{mymatrix}{ccc}
      1 & 0 & 0 \\
      0 & \frac{1}{\sqrt2} & -\frac{1}{\sqrt2} \\
      0 & \frac{1}{\sqrt2} & \frac{1}{\sqrt2} \\
    \end{mymatrix}.
  \end{equation*}
  Finally, to apply the linear transformation $T$ to a vector
  $\vect{v}$, we must first apply $B$ and then $C$. This means that
  $T(\vect{v}) = C(B\vect{v})$. Therefore, the matrix corresponding to
  $T$ is $CB$. Note that it is important that we multiply the matrices
  corresponding to each subsequent rotation {\em from right to left}.
  \begin{equation*}
    \def\arraystretch{1.4}
    A ~=~ CB
    ~=~ \begin{mymatrix}{ccc}
      1 & 0 & 0 \\
      0 & \frac{1}{\sqrt2} & -\frac{1}{\sqrt2} \\
      0 & \frac{1}{\sqrt2} & \frac{1}{\sqrt2} \\
    \end{mymatrix}
    \begin{mymatrix}{ccc}
      \frac{\sqrt3}{2} & -\frac{1}{2} & 0 \\
      \frac{1}{2} & \frac{\sqrt3}{2} & 0 \\
      0 & 0 & 1 \\
    \end{mymatrix}
    ~=~ \begin{mymatrix}{ccc}
      \frac{\sqrt3}{2} & -\frac{1}{2} & 0 \\
      \frac{1}{2\sqrt2} & \frac{\sqrt3}{2\sqrt2} & -\frac{1}{\sqrt2} \\
      \frac{1}{2\sqrt2} & \frac{\sqrt3}{2\sqrt2} & \frac{1}{\sqrt2} \\
    \end{mymatrix}.
  \end{equation*}
\end{solution}

We can also consider the inverse of a linear transformation. The
inverse of $T$, if it exists, is a linear transformation that undoes
the effect of $T$.

\begin{definition}{Inverse of a transformation}{inverse-transformation}
  Let $T,S: \R^n \to \R^n$ be linear transformations. Suppose that for
  each $\vect{v} \in \R^n$,
  \begin{equation*}
    (S\circ T)(\vect{v}) = \vect{v}
  \end{equation*}
  and
  \begin{equation*}
    (T\circ S)(\vect{v}) = \vect{v}.
  \end{equation*}
  Then $S$ is called the \textbf{inverse}%
  \index{linear transformation!inverse}%
  \index{inverse!of a linear transformation} of $T$, and we write
  $S=T^{-1}$.
\end{definition}

\begin{example}{Inverse of a transformation}{inverse-transformation}
  What is the inverse of a counterclockwise rotation by the angle
  $\theta$ in $\R^2$?
\end{example}

\begin{solution}
  The inverse is a clockwise rotation by the same angle.
\end{solution}

It is perhaps not entirely unexpected that the matrix of $T^{-1}$ is
exactly the inverse of the matrix of $T$, if it exists.

\begin{theorem}{Matrix of the inverse transformation}{inverse-transformation}
  Let $T:\R^n \to \R^n$ be a linear transformation and let $A$ be the
  corresponding $n\times n$-matrix. Then $T$ has an inverse if and
  only if the matrix $A$ is invertible. In this case, the matrix of
  $T^{-1}$ is $A^{-1}$.
\end{theorem}

\begin{example}{Matrix of the inverse transformation}{matrix-inverse-transformation}
  Find the inverse of the linear transformation $T:\R^2\to\R^2$ given
  by
  \begin{equation*}
    T\paren{\begin{mymatrix}{c} x \\ y \end{mymatrix}}
    = \begin{mymatrix}{c} 2x+y \\ 7x+4y \end{mymatrix}.
  \end{equation*}
\end{example}

\begin{solution}
  The easiest way to do this is to find the matrix of $T$. We have
  \begin{equation*}
    T\paren{\begin{mymatrix}{c} 1 \\ 0 \end{mymatrix}}
    = \begin{mymatrix}{c} 2 \\ 7 \end{mymatrix}
    \quad\mbox{and}\quad
    T\paren{\begin{mymatrix}{c} 0 \\ 1 \end{mymatrix}}
    = \begin{mymatrix}{c} 1 \\ 4 \end{mymatrix}.
  \end{equation*}
  Therefore, the matrix of $T$ is
  \begin{equation*}
    A = \begin{mymatrix}{rr}
      2 & 1 \\
      7 & 4 \\
    \end{mymatrix}.
  \end{equation*}
  The inverse of $A$ is
  \begin{equation*}
    A^{-1} = \begin{mymatrix}{rr}
      4 & -1 \\
      -7 & 2 \\
    \end{mymatrix}.
  \end{equation*}
  Therefore, $T^{-1}$ is the linear transformation defined by
  \begin{equation*}
    T^{-1}\paren{\begin{mymatrix}{c} x \\ y \end{mymatrix}}
    = \begin{mymatrix}{rr}
      4 & -1 \\
      -7 & 2 \\
    \end{mymatrix}
    \begin{mymatrix}{c} x \\ y \end{mymatrix}
    = \begin{mymatrix}{c} 4x-y \\ -7x+2y \end{mymatrix}.
  \end{equation*}
\end{solution}
