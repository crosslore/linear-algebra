\section{Properties of Linear Transformations}

\begin{outcome}
\begin{enumerate}
\item[A.] Use properties of linear transformations to solve problems. 

\item[B.] Find the composite of transformations and the inverse of a transformation.
\end{enumerate}
\end{outcome}

Let $T: \mathbb{R}^n \mapsto \mathbb{R}^m$ be a linear transformation. Then there are some important properties of $T$ which will be examined in this section. 
Consider the following theorem.

\begin{theorem}{Properties of Linear Transformations}{properties}
Let $T: \mathbb{R}^n \mapsto \mathbb{R}^m$ be a linear transformation and let $\vect{x} \in \mathbb{R}^n$. 

\begin{itemize}
\item $T$ preserves the zero vector. 
\[
T(0\vect{x}) = 0 T(\vect{x}). \mbox{ Hence }T(\vect{0}) = \vect{0}
\]
\item $T$ preserves the negative of a vector:
\[
T( (-1)\vect{x})=(-1)T(\vect{x}). \mbox{ Hence }T(-\vect{x}) = -T(\vect{x}).
\]
\item $T$ preserves linear combinations:
\[
\mbox{Let }\vect{x}_1, ..., \vect{x}_k \in \mathbb{R}^n \mbox{ and }a_1, ..., a_k \in \mathbb{R}.
\]
\[
\mbox{Then if }\vect{y} = a_1\vect{x}_1 + a_2\vect{x}_2 + ...+a_k \vect{x}_k, \mbox{it follows that }
\]
\[
T(\vect{y}) = T(a_1\vect{x}_1 + a_2\vect{x}_2 + ...+a_k \vect{x}_k) = a_1T(\vect{x}_1) + a_2T(\vect{x}_2) + ...+a_k T(\vect{x}_k).
\] 
\end{itemize}
\end{theorem}

These properties are useful in determining the action of a transformation on a given vector. Consider the following example.

\begin{example}{Linear combination}{lintransfcombination}
Let $T:\mathbb{R}^3 \mapsto \mathbb{R}^4$ be a linear transformation such that 
\[
T \leftB \begin{array}{r}
1 \\
3 \\
1
\end{array} \rightB
=
\leftB \begin{array}{r}
4 \\
4 \\
0 \\
-2
\end{array} \rightB,
T \leftB \begin{array}{r}
4 \\
0 \\
5
\end{array} \rightB
=
\leftB \begin{array}{r}
4 \\
5 \\
-1 \\
5
\end{array} \rightB
\]
Find $T \leftB \begin{array}{r}
-7 \\
3 \\
-9
\end{array} \rightB$.
\end{example}

\begin{solution}
Using the third property in Theorem \ref{thm:properties}, we can find $T \leftB \begin{array}{r}
-7 \\
3 \\
-9
\end{array} \rightB$ by writing $\leftB \begin{array}{r}
-7 \\
3 \\
-9
\end{array} \rightB$ as a linear combination of $\leftB \begin{array}{r}
1 \\
3 \\
1
\end{array} \rightB$ and $\leftB \begin{array}{r}
4 \\
0 \\
5
\end{array} \rightB$. 

Therefore we want to find $a,b \in \mathbb{R}$ such that 
\[
\leftB \begin{array}{r}
-7 \\
3 \\
-9
\end{array} \rightB
=
a
\leftB \begin{array}{r}
1 \\
3 \\
1
\end{array} \rightB
+
b
\leftB \begin{array}{r}
4 \\
0 \\
5
\end{array} \rightB
\]

The necessary augmented matrix and resulting {\rref} are given by:
\[
\leftB
\begin{array}{rr|r}
1 & 4 & -7 \\
3 & 0 & 3 \\
1 & 5 & -9 
\end{array}
\rightB
\rightarrow \cdots \rightarrow
\leftB
\begin{array}{rr|r}
1 & 0 & 1 \\
0 & 1 & -2 \\
0 & 0 & 0 
\end{array}
\rightB
\]

Hence $a = 1, b = -2$ and \[
\leftB \begin{array}{r}
-7 \\
3 \\
-9
\end{array} \rightB
=
1
\leftB \begin{array}{r}
1 \\
3 \\
1
\end{array} \rightB
+
(-2)
\leftB \begin{array}{r}
4 \\
0 \\
5
\end{array} \rightB
\]

Now, using the third property above, we have 
\begin{eqnarray*}
T \leftB \begin{array}{r}
-7 \\
3 \\
-9
\end{array} \rightB
&=&
T \left(
1
\leftB \begin{array}{r}
1 \\
3 \\
1
\end{array} \rightB
+
(-2)
\leftB \begin{array}{r}
4 \\
0 \\
5
\end{array} \rightB
\right) \\
&=& 
1T 
\leftB \begin{array}{r}
1 \\
3 \\
1
\end{array} \rightB
-2T
\leftB \begin{array}{r}
4 \\
0 \\
5
\end{array} \rightB
\\
&=&
\leftB \begin{array}{r}
4 \\
4 \\
0 \\
-2
\end{array} \rightB
-2
\leftB \begin{array}{r}
4 \\
5 \\
-1 \\
5
\end{array} \rightB
\\
&=& 
\leftB
\begin{array}{r}
-4 \\
-6 \\
2 \\
-12
\end{array}
\rightB
\end{eqnarray*}

Therefore, $T \leftB \begin{array}{r}
-7 \\
3 \\
-9
\end{array}
\rightB
=
\leftB
\begin{array}{r}
-4 \\
-6 \\
2 \\
-12
\end{array}
\rightB
$.
\end{solution}

Suppose two linear transformations act in the same way on $\vect{x}$ for all vectors. Then we say that these transformations are equal.

\begin{definition}{Equal Transformations}{equaltransformations}
Let $S$ and $T$ be linear transformations from $\mathbb{R}^n$ to $\mathbb{R}^m$. Then $S = T$ if and only if for every $\vect{x} \in \mathbb{R}^n$, 
\[
S \left( \vect{x} \right) = T \left( \vect{x} \right)
\]
\end{definition}

Suppose two linear transformations act on the same vector $\vect{x}$, first the transformation $T$ and then a second transformation given by $S$. We can find the \textbf{composite} transformation that results from applying both transformations.

\begin{definition}{Composition of Linear Transformations}{compositetransformations}
Let $T: \mathbb{R}^k \mapsto \mathbb{R}^n$ and $S: \mathbb{R}^n \mapsto \mathbb{R}^m$ be linear transformations. Then the \textbf{composite}\index{linear transformation!composite}\index{composite} of $S$ and $T$ is 
\[
S \circ T: \mathbb{R}^k \mapsto \mathbb{R}^m
\]
The action of $S \circ T$ is given by 
\[
(S \circ T) (\vect{x}) = S(T(\vect{x})) \; \mbox{for all} \; \vect{x} \in \mathbb{R}^k
\]
\end{definition}

Notice that the resulting vector will be in $\mathbb{R}^m$. Be careful to observe the order of transformations. We write $S \circ T$ but apply the transformation $T$ first, followed by $S$. 

\begin{theorem}{Composition of Transformations}{compositetransformation}
Let $T: \mathbb{R}^k \mapsto \mathbb{R}^n$ and $S: \mathbb{R}^n \mapsto \mathbb{R}^m$ be linear transformations such that $T$ is induced by the matrix $A$ and $S$ is induced by the matrix $B$. Then $S \circ T$ is a linear transformation which is induced by the matrix $BA$.
\end{theorem}

Consider the following example. 

\begin{example}{Composition of transformations}{compositetransformation}
Let $T$ be a linear transformation induced by the matrix 
\[
A = 
\leftB
\begin{array}{rr}
1 & 2 \\
2 & 0 
\end{array}
\rightB
\]
and $S$ a linear transformation induced by the matrix
\[
B = 
\leftB
\begin{array}{rr}
2 & 3 \\
0 & 1
\end{array}
\rightB
\]
Find the matrix of the composite transformation $S \circ T$. Then, find $(S \circ T)(\vect{x})$ for $\vect{x} = \leftB
\begin{array}{r}
1 \\
4 
\end{array}
\rightB$.
\end{example}

\begin{solution}
By Theorem \ref{thm:compositetransformation}, the matrix of $S \circ T$ is given by $BA$. 
\[
BA 
=
\leftB
\begin{array}{rr}
2 & 3 \\
0 & 1
\end{array}
\rightB
\leftB
\begin{array}{rr}
1 & 2 \\
2 & 0 
\end{array}
\rightB
 =
\leftB
\begin{array}{rr}
8 & 4 \\
2 & 0
\end{array}
\rightB
\]

To find $(S \circ T)(\vect{x})$, multiply $\vect{x}$ by $BA$ as follows
\[
\leftB
\begin{array}{rr}
8 & 4 \\
2 & 0
\end{array}
\rightB
\leftB
\begin{array}{rr}
1 \\
4
\end{array}
\rightB
=
\leftB
\begin{array}{r}
24 \\
2
\end{array}
\rightB
\]

To check, first determine $T(\vect{x})$:
\[
\leftB
\begin{array}{rr}
1 & 2 \\
2 & 0 
\end{array}
\rightB
\leftB
\begin{array}{r}
1 \\
4
\end{array}
\rightB
=
\leftB
\begin{array}{r}
9 \\
2
\end{array}
\rightB
\]

Then, compute $S(T(\vect{x}))$ as follows:
\[
\leftB
\begin{array}{rr}
2 & 3 \\
0 & 1 
\end{array}
\rightB
\leftB
\begin{array}{r}
9 \\
2
\end{array}
\rightB
=
\leftB
\begin{array}{r}
24 \\
2
\end{array}
\rightB
\]
\end{solution}

Consider a composite transformation $S \circ T$, and suppose that this transformation acted such that $(S \circ T) (\vect{x}) = \vect{x}$. That is, the transformation $S$ took the vector $T(\vect{x})$ and returned it to $\vect{x}$. In this case, $S$ and $T$ are inverses of each other. Consider the following definition.  

\begin{definition}{Inverse of a Transformation}{inversetransformation}
Let $T: \mathbb{R}^n \mapsto \mathbb{R}^n$ and $S:\mathbb{R}^n \mapsto \mathbb{R}^n$ be linear transformations. Suppose that for each $\vect{x} \in \mathbb{R}^n$, 
\[
(S \circ T)(\vect{x}) = \vect{x}
\]
and 
\[
(T \circ S)(\vect{x}) = \vect{x}
\]
Then, $S$ is called an inverse of $T$  and $T$ is called an inverse of $S$. Geometrically, they reverse the action of each other. 
\end{definition}

The following theorem is crucial, as it claims that the above inverse transformations are unique. 

\begin{theorem}{Inverse of a Transformation}{inversetransformation}
Let $T:\mathbb{R}^n \mapsto \mathbb{R}^n$ be a linear transformation induced by the matrix $A$. Then $T$ has an inverse transformation if and only if the matrix $A$ is invertible. In this case, the inverse transformation is unique and denoted $T^{-1}: \mathbb{R}^n \mapsto \mathbb{R}^n$. $T^{-1}$ is induced by the matrix $A^{-1}$. 
\end{theorem}

Consider the following example. 

\begin{example}{Inverse of a transformation}{inversetransformation}
Let $T: \mathbb{R}^2 \mapsto \mathbb{R}^2$ be a linear transformation induced by the matrix 
\[
A = 
\leftB
\begin{array}{rr}
2 & 3 \\
3 & 4
\end{array}
\rightB
\]
Show that $T^{-1}$ exists and find the matrix $B$ which it is induced by. 
\end{example}

\begin{solution}
Since the matrix $A$ is invertible, it follows that the transformation $T$ is invertible. Therefore, $T^{-1}$ exists. 

You can verify that $A^{-1}$ is given by:
\[
A^{-1}
=
\leftB
\begin{array}{rr}
-4 & 3 \\
3 & -2
\end{array}
\rightB
\]
Therefore the linear transformation $T^{-1}$ is induced by the matrix $A^{-1}$. 
\end{solution}
