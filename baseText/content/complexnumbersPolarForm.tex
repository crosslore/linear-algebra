\section{Polar Form}

\begin{outcome}
\begin{enumerate}
\item[A.] Convert a complex number from standard form to polar form, and from polar form to standard form. 
\end{enumerate}
\end{outcome}

In the previous section, we identified a complex number $z=a+bi$ with
a point $\left( a, b\right)$ in the coordinate plane. There is 
another form in which we can express the same number, called the {\em
polar form\em}. The polar form is the focus of this section. It will turn out to be
very useful if not crucial for certain calculations as we shall soon
see.

Suppose $z=a+bi$ is a complex number,  and let 
$r=\sqrt{a^{2}+b^{2}} = |z|$. Recall that $r$ is the \textbf{modulus}\index{complex numbers!modulus}\index{modulus} of $z$. Note first that
\begin{equation*}
\left( \frac{a}{r} \right) ^{2}+\left( \frac{b}{r}\right) ^{2}=   \frac{a^2+b^2}{r^2}=1
\end{equation*}
and so $\left( \frac{a}{r},\frac{b}{r}\right)$
is a point on the unit circle. Therefore, there exists an angle  $
\theta$ (in radians) such that
\begin{equation*}
\cos \theta =\frac{a}{r},\ \sin \theta =\frac{b}{r}
\end{equation*}
In other words $\theta$ is an angle such
that $ a = r\cos \theta$ and $b=r \sin \theta$, that is $\theta = \cos^{-1}(a/r)$ and $\theta = \sin^{-1}(b/r)$. We call
this angle $\theta$ the \textbf{argument}\index{complex numbers!argument}\index{argument} of $z$. 

We often speak of the \textbf{principal argument}\index{principal argument} of $z$. This is the unique angle $\theta \in (-\pi, \pi]$ such that 
\begin{equation*}
\cos \theta =\frac{a}{r},\ \sin \theta =\frac{b}{r}
\end{equation*}

The polar form of the complex number $z=a+bi = r \left( \cos \theta +i\sin \theta \right)$ is for convenience written as:
\begin{equation*}
z = r e^{i \theta}
\end{equation*}
where $\theta $ is the argument of
$z$. 

\begin{definition}{Polar Form of a Complex Number}{polarform}
Let $z = a + bi$ be a complex number. Then the \textbf{polar form}\index{complex numbers!polar form}\index{polar form} of $z$ is written as 
\[
z = re^{i\theta}
\]
where $r = \sqrt{a^2 + b^2}$ and $\theta$ is the argument of $z$. 
\end{definition}

When given $z = re^{i\theta}$, the identity $e^{i\theta} = \cos\theta + i \sin\theta$ will convert $z$ back to standard form. Here we think of $ e^{i \theta}$ as a short cut for $ \cos \theta
+i\sin \theta$. This is all we will need in this course, but in
reality $e^{i \theta}$ can be considered as the complex equivalent of
the exponential function where this turns out to be a true equality.

\begin{center}
\begin{tikzpicture}
\draw(-2,0)--(2,0);
\draw(0,-2)--(0,2);
\draw[ultra thick, blue, ->](0,0)--(1.5,1.5);
\node[right] at (1.5,1.5){$z = a+bi = re^{i\theta}$};
\node[above right] at (0.5,0){$\theta$};
\node[left] at (1,1){$r$};
\node at (-1.5,1){$r = \sqrt{a^2 + b^2}$};
\end{tikzpicture}
\end{center}

Thus we can convert any complex number in the standard (Cartesian) form $z = a+bi$
into its polar form. Consider the following example.

\begin{example}{Standard to polar form}{polarform}
Let $z = 2 + 2i$ be a complex number. 
Write $z$ in the polar form 
\begin{equation*}
z = re^{i \theta}
\end{equation*}
\end{example}

\begin{solution}
First, find $r$.
By the above discussion, $r=\sqrt{
a^{2}+b^{2}} = |z|$. Therefore,

\begin{equation*}
r = \sqrt{2^{2} + 2^{2}} = \sqrt{8} =2\sqrt{2}
\end{equation*}

Now, to find $\theta$, we plot the point $\left( 2, 2 \right)$ and
find the angle from the positive $x$ axis to the line between this
point and the origin. In this case, $\theta = 45^{\circ} =
\frac{\pi}{4}$.  That is we found the unique angle $\theta$ such that 
$\theta = \cos^{-1}(1/\sqrt{2})$ and $\theta = \sin^{-1}(1/\sqrt{2})$. 

Note that in polar form, we always express angles in radians, not degrees.

Hence, we can write $z$ as

\begin{equation*}
z = 2\sqrt{2} e^{i\frac{\pi}{4}}
\end{equation*}

\end{solution}

Notice that the standard and polar forms are completely equivalent. That is not only can we transform a complex number from standard form
to its polar form, we can also take a complex number in polar form and
convert it back to standard form.

\begin{example}{Polar to standard form}{polarstandardform}
Let $z = 2 e^{ 2\pi i/3}$. Write $z$ in the standard form 
\begin{equation*}
z = a+bi
\end{equation*}
\end{example}

\begin{solution}
Let $z = 2 e^{2\pi i/3}$ be the polar form of a complex number. Recall that 
$e^{i\theta} = \cos \theta + i \sin \theta$. Therefore using standard values of $\sin$ and $\cos$ we get:
\begin{eqnarray*}
z = 2 e^{i 2\pi/3} &=& 2 (\cos (2\pi/3)+i\sin (2\pi/3))\\
&=& 2 \left ( -\frac{1}{2} + i \frac{\sqrt{3}}{2} \right) \\
&=&-1 + \sqrt{3}i 
\end{eqnarray*}
which is the standard form of this complex number.
\end{solution}

You can always verify your answer by converting it back to polar form and ensuring you reach the original answer. 
