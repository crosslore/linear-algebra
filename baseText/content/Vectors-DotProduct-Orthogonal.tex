\subsection{Orthogonal vectors}

Two non-zero vectors are said to be \textbf{orthogonal}%
\index{vector!orthogonal}%
\index{orthogonal vectors}, sometimes also called
\textbf{perpendicular}%
\index{vector!perpendicular}%
\index{perpendicular vectors}, if the included angle is $\pi /2$
radians ($90^{\circ })$. By convention, we also say that the zero
vector is orthogonal to all vectors.

\begin{proposition}{Orthogonal vectors}{orthogonal-vectors}
  Let $\vect{u}$ and $\vect{v}$ be vectors in $\R^n$. Then $\vect{u}$
  and $\vect{v}$ are orthogonal\index{vector!orthogonal} if and only
  if
  \begin{equation*}
    \vect{u} \dotprod \vect{v} = 0.
  \end{equation*}
  We also write $\vect{u}\orth\vect{v}$ to indicate that $\vect{u}$
  and $\vect{v}$ are orthogonal.
\end{proposition}

\begin{proof}
  If $\vect{u}$ or $\vect{v}$ is zero, the vectors are orthogonal by
  definition, and the dot product is $0$ in that case, so the
  proposition holds. Now assume $\vect{u}$ and $\vect{v}$ are both
  non-zero.  Then by Proposition~\ref{prop:dot-product-angle}, we have
  $\vect{u} \dotprod \vect{v} = 0$ if and only if
  $\norm{\vect{u}} \norm{\vect{v}} \cos \theta$ if and only if
  $\cos\theta=0$. Recall that the included angle is between $0$ and
  $\pi$. Therefore, $\cos\theta=0$ if and only if $\theta=\pi/2$.
\end{proof}

\begin{example}{Determine whether two vectors are orthogonal}{orthogonal-vectors}
  Determine whether the vectors
  \begin{equation*}
    \vect{u}=
    \begin{mymatrix}{r}
      2 \\
      1 \\
      -1
    \end{mymatrix}
    \quad\mbox{and}\quad
    \vect{v}
    =
    \begin{mymatrix}{r}
      1 \\
      3 \\
      5
    \end{mymatrix}
  \end{equation*}
  are orthogonal.
\end{example}

\begin{solution}
  In order to determine if these two vectors are orthogonal, we
  compute the dot product. We have
  \begin{equation*}
    \vect{u} \dotprod \vect{v}
    =
    (2)(1) + (1)(3) + (-1)(5)
    =
    0,
  \end{equation*}
  and therefore, by Proposition~\ref{prop:orthogonal-vectors}, the two vectors are orthogonal.
\end{solution}
