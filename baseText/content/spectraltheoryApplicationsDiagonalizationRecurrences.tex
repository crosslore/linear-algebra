\section{Application: Solving recurrences}

\begin{outcome}
  \begin{enumerate}
  \item Solve a linear recurrence relation using diagonalization.
  \end{enumerate}
\end{outcome}

Consider the following sequence of integers, called the
\textbf{Fibonacci sequence}%
\index{Fibonacci sequence}%
\index{sequence!Fibonacci}%
\index{number!Fibonacci}:
\begin{equation*}
  0,\ 1,\ 1,\ 2,\ 3,\ 5,\ 8,\ 13,\ 21,\ \ldots
\end{equation*}
The first two Fibonacci numbers are 0 and 1. Every subsequent
Fibonacci number is the sum of the previous two numbers. For example,
$0+1=1$, $1+1=2$, $1+2=3$, $2+3=5$, and so on. Thus, if we write $F_n$ for the
$n\th$ Fibonacci number, then the Fibonacci sequence is given by the
following conditions:
\begin{equation*}
  \begin{array}{l@{~}c@{~}l}
    F_0 &=& 0, \\
    F_1 &=& 1, \\
    F_{n+2} &=& F_{n} + F_{n+1}, \quad\mbox{for all $n\geq 0$.}\\
  \end{array}
\end{equation*}
The condition $F_{n+2} = F_{n} + F_{n+1}$ is known as a
\textbf{recurrence relation}, or simply as a \textbf{recurrence}%
\index{recurrence}%
\index{sequence!defined by a recurrence}, because we compute each
member of the sequence from previous members (the word ``recurrence''
comes from Latin ``recurrere'', which means ``to go back''). The
conditions $F_0=0$ and $F_1=1$ are known as the \textbf{base cases}%
\index{recurrence!base case}%
\index{base case!of a recurrence} of the recurrence. Note that we
start counting from zero, i.e., we call $F_0=0$ the ``zeroth Fibonacci
number'', $F_1=1$ the ``first Fibonacci number'', and so on. Counting
from zero will help simplify our calculations later.

\begin{example}{Computing a Fibonacci number}{fibonacci-10}
  Compute the $10\th$ Fibonacci number $F_{10}$.
\end{example}

\begin{solution}
  To compute the $10\th$ Fibonacci number using the recurrence, we
  have to compute all the previous Fibonacci numbers as well.
  \begin{eqnarray*}
    F_0 &=& 0, \\
    F_1 &=& 1, \\
    F_2 &=& F_0 + F_1 ~=~ 0 + 1 ~=~ 1, \\
    F_3 &=& F_1 + F_2 ~=~ 1 + 1 ~=~ 2, \\
    F_4 &=& F_2 + F_3 ~=~ 1 + 2 ~=~ 3, \\
    F_5 &=& F_3 + F_4 ~=~ 2 + 3 ~=~ 5, \\
    F_6 &=& F_4 + F_5 ~=~ 3 + 5 ~=~ 8, \\
    F_7 &=& F_5 + F_6 ~=~ 5 + 8 ~=~ 13, \\
    F_8 &=& F_6 + F_7 ~=~ 8 + 13 ~=~ 21, \\
    F_9 &=& F_7 + F_8 ~=~ 13 + 21 ~=~ 34, \\
    F_{10} &=& F_8 + F_9 ~=~ 21 + 34 ~=~ 55.
  \end{eqnarray*}
  Thus, the $10\th$ Fibonacci number is $55$.
\end{solution}

Suppose we want to compute the $100\th$ Fibonacci number. As the
previous example shows, computing this by using the recurrence
relation would be lot of work. We will therefore explore how to use
linear algebra, and in particular diagonalization, to find a closed
formula for the $n\th$ Fibonacci number. By a \textbf{closed formula}%
\index{closed formula}%
\index{formula!closed}, we mean a formula to calculate $F_n$ directly
in one step, i.e., without using a recurrence. The process of finding
a closed formula is called \textbf{solving the recurrence}%
\index{recurrence!solving}.

The first step in solving the recurrence is to replace the recurrence
relation $F_{n+2} = F_{n} + F_{n+1}$, which requires {\em two}
previous terms of the sequence, by another recurrence relation
requiring only {\em one} previous term. To that end, we define
$\vect{v}_n$ to be the vector consisting of the $n\th$ and $n+1\st$
Fibonacci numbers, for all $n\geq 0$:
\begin{equation*}
  \vect{v}_n = \begin{mymatrix}{c} F_n \\ F_{n+1} \end{mymatrix}.
\end{equation*}
Using the recurrence relation for $F_{n+2}$, we have
\begin{equation*}
  \vect{v}_{n+1}
  = \begin{mymatrix}{c} F_{n+1} \\ F_{n+2} \end{mymatrix}
  = \begin{mymatrix}{c} F_{n+1} \\ F_n + F_{n+1} \end{mymatrix}
  = \begin{mymatrix}{rr}
    0 & 1 \\
    1 & 1 \\
  \end{mymatrix}
  \begin{mymatrix}{c} F_n \\ F_{n+1} \end{mymatrix}
  = \begin{mymatrix}{rr}
    0 & 1 \\
    1 & 1 \\
  \end{mymatrix}
  \vect{v}_n.
\end{equation*}
Therefore, to compute $\vect{v}_{n+1}$, we only need to know
$\vect{v}_n$ (and not $\vect{v}_{n-1}$). Let
\begin{equation*}
  A = \begin{mymatrix}{rr}
    0 & 1 \\
    1 & 1 \\
  \end{mymatrix}.
\end{equation*}
Since $\vect{v}_n$ is obtained from $\vect{v}_0$ by applying the
matrix $A$ $n$ times, we have $\vect{v}_n = A^n\,\vect{v}_0$ for all
$n\geq 0$. We can therefore get a formula for $\vect{v}_n$, and thus
for $F_n$, by diagonalizing the matrix $A$.

\begin{problem}{Diagonalizing $A$}{fibonacci-diagonalize}
  Diagonalize the matrix
  $A=\begin{mymatrix}{rr} 0 & 1 \\ 1 & 1 \end{mymatrix}$.
\end{problem}

\begin{solution}
  The characteristic polynomial is
  \begin{equation*}
    p(\eigenvar) =
    \begin{absmatrix}{cc}
      -\eigenvar & 1 \\
      1 & 1-\eigenvar \\
    \end{absmatrix}
    = (-\eigenvar)(1-\eigenvar) - 1
    = \eigenvar^2 - \eigenvar - 1.
  \end{equation*}
  The eigenvalues are the roots of the characteristic polynomial. We
  find them by using the quadratic formula. The eigenvalues are
  \begin{equation*}
    \eigenvar_1 = \frac{1+\sqrt{5}}{2}
    \quad\mbox{and}\quad
    \eigenvar_2 = \frac{1-\sqrt{5}}{2}.
  \end{equation*}
  To simplify later calculations, we note that
  $\eigenvar_1+\eigenvar_2 = 1$, or equivalently,
  \begin{equation}\label{eqn:fibonacci-1}
    1-\eigenvar_1 = \eigenvar_2.
  \end{equation}
  We also note that
  $\eigenvar_1\eigenvar_2 = -1$, or equivalently,
  \begin{equation}\label{eqn:fibonacci-2}
    \eigenvar_2 = -\frac{1}{\eigenvar_1}.
  \end{equation}
  To find the eigenvector corresponding to the eigenvalue $\eigenvar_1$,
  we solve the equation
  \begin{equation*}
    \begin{mymatrix}{cc}
      -\eigenvar_1 & 1 \\
      1 & 1-\eigenvar_1 \\
    \end{mymatrix}
    \vect{v} = \vect{0}.
  \end{equation*}
  By {\eqref{eqn:fibonacci-1}} and {\eqref{eqn:fibonacci-2}}, this is
  equivalent to
  \begin{equation*}
    \begin{mymatrix}{cc}
      -\eigenvar_1 & 1 \\
      0 & -\frac{1}{\eigenvar_1} \\
    \end{mymatrix}
    \vect{v} = \vect{0},
  \end{equation*}
  and we find the basic solution
  \begin{equation*}
    \vect{u} = \begin{mymatrix}{c} 1 \\ \eigenvar_1 \end{mymatrix}.
  \end{equation*}
  By a similar method, we find that the basic eigenvector
  corresponding to the eigenvalue $\eigenvar_2$ is
  \begin{equation*}
    \vect{w} = \begin{mymatrix}{c} 1 \\ \eigenvar_2 \end{mymatrix}.
  \end{equation*}
  Therefore, by Theorem~\ref{thm:eigenvectors-and-diagonalizable}, $A$ is
  diagonalizable, and we have $P^{-1}AP=D$, where
  \begin{equation*}
    P = \begin{mymatrix}{cc}
      1 & 1 \\
      \eigenvar_1 & \eigenvar_2 \\
    \end{mymatrix}
    \quad\mbox{and}\quad
    D = \begin{mymatrix}{cc}
      \eigenvar_1 & 0 \\
      0 & \eigenvar_2 \\
    \end{mymatrix}.
  \end{equation*}
  For later reference, we note that the inverse of $P$ is given by
  \begin{equation*}
    P^{-1} =
    \frac{1}{\sqrt{5}} \begin{mymatrix}{cc}
      -\eigenvar_2 & 1 \\
      \eigenvar_1 & -1 \\
    \end{mymatrix}.
  \end{equation*}
\end{solution}

We are now ready to find our formula for the $n\th$ Fibonacci number.

\begin{problem}{Solving the recurrence}{fibonacci-formula}
  Find a formula for the $n\th$ Fibonacci number.
\end{problem}

\begin{solution}
  Since
  \begin{equation*}
    \vect{v}_n = \begin{mymatrix}{c} F_n \\ F_{n+1} \end{mymatrix},
  \end{equation*}
  we know that the $n\th$ Fibonacci number is the first component of
  $\vect{v}_n$, i.e.,
  \begin{equation*}
    F_n = \begin{mymatrix}{cc}1 & 0\end{mymatrix}\vect{v}_n.
  \end{equation*}
  Putting together all of the above calculations, we then have:
  \begin{eqnarray*}
    F_n
    &=& \begin{mymatrix}{cc}1 & 0\end{mymatrix}\vect{v}_n \\
    &=& \begin{mymatrix}{cc}1 & 0\end{mymatrix}A^n\,\vect{v}_0 \\
    &=& \begin{mymatrix}{cc}1 & 0\end{mymatrix}PD^nP^{-1}\,\vect{v}_0 \\
    &=& \begin{mymatrix}{cc}1 & 0\end{mymatrix}
        \begin{mymatrix}{cc} 1 & 1 \\ \eigenvar_1 & \eigenvar_2 \end{mymatrix}
        \begin{mymatrix}{cc} \eigenvar_1^n & 0 \\ 0 & \eigenvar_2^n \end{mymatrix}
        \frac{1}{\sqrt{5}}
        \begin{mymatrix}{cc} -\eigenvar_2 & 1 \\ \eigenvar_1 & -1 \end{mymatrix}
        \begin{mymatrix}{c} 0 \\ 1 \end{mymatrix} \\
    &=& \frac{1}{\sqrt{5}}
        \begin{mymatrix}{cc}1 & 1\end{mymatrix}
        \begin{mymatrix}{cc} \eigenvar_1^n & 0 \\ 0 & \eigenvar_2^n \end{mymatrix}
        \begin{mymatrix}{c} 1 \\ -1 \end{mymatrix} \\
    &=& \frac{1}{\sqrt{5}} (\eigenvar_1^n - \eigenvar_2^n).
  \end{eqnarray*}
  So the $n\th$ Fibonacci number is
  \begin{equation*}
    F_n
    ~=~ \frac{1}{\sqrt{5}} (\eigenvar_1^n - \eigenvar_2^n)
    ~=~ \frac{1}{\sqrt{5}}
    \paren{\paren{\frac{1+\sqrt{5}}{2}}^n - \paren{\frac{1-\sqrt{5}}{2}}^n}.
  \end{equation*}
\end{solution}

\begin{example}{Computing the $100\th$ Fibonacci number}{fibonacci-100}
  Calculate the $100\th$ Fibonacci number without using the recurrence.
\end{example}

\begin{solution}
  Note that this calculation requires about 25 digits of precision to
  give the correct result. We
  have
  \begin{eqnarray*}
    F_{100}
    &=& \frac{1}{\sqrt{5}} (\eigenvar_1^{100} - \eigenvar_2^{100}) \\
    &=& \frac{1}{\sqrt{5}} (1.6180339887498948482045868^{100} - (-0.6180339887498948482045868)^{100}) \\
    &=& 354224848179261915075.
  \end{eqnarray*}
\end{solution}

We can use the same method to solve other linear recurrences. Here is
another example:

\begin{example}{Solving a recurrence}{recurrence}
  Consider the sequence of numbers defined by the recurrence
  \begin{equation*}
    \begin{array}{l@{~}c@{~}l}
      b_0 &=& 1, \\
      b_1 &=& 2, \\
      b_{n+2} &=& 6b_{n} + b_{n+1}, \quad\mbox{for all $n\geq 0$.}\\
    \end{array}
  \end{equation*}
  Find the first 5 members of this sequence. Then solve the recurrence
  and find $b_{20}$.
\end{example}

\begin{solution}
  The first 5 members of the sequence are:
  \begin{equation*}
    \begin{array}{l@{~}c@{~}l}
      b_0 &=& 1, \\
      b_1 &=& 2, \\
      b_2 &=& 6b_0 + b_1 ~=~ 6 + 2 ~=~ 8, \\
      b_3 &=& 6b_1 + b_2 ~=~ 12 + 8 ~=~ 20, \\
      b_4 &=& 6b_2 + b_3 ~=~ 48 + 20 ~=~ 68. \\
    \end{array}
  \end{equation*}
  To solve the recurrence, let
  \begin{equation*}
    \vect{w}_n = \begin{mymatrix}{c} b_n \\ b_{n+1} \end{mymatrix}, 
  \end{equation*}
  so that for all $n\geq 0$,
  \begin{equation*}
    \vect{w}_{n+1}
    = \begin{mymatrix}{c} b_{n+1} \\ b_{n+2} \end{mymatrix}
    = \begin{mymatrix}{c} b_{n+1} \\ 6b_n + b_{n+1} \end{mymatrix}
    = \begin{mymatrix}{rr}
      0 & 1 \\
      6 & 1 \\
    \end{mymatrix}
    \begin{mymatrix}{c} b_n \\ b_{n+1} \end{mymatrix}
    = \begin{mymatrix}{rr}
      0 & 1 \\
      6 & 1 \\
    \end{mymatrix}
    \vect{w}_n.
  \end{equation*}
  We then diagonalize the matrix
  \begin{equation*}
    B = \begin{mymatrix}{rr}
      0 & 1 \\
      6 & 1 \\
    \end{mymatrix}.
  \end{equation*}
  The characteristic polynomial is
  \begin{equation*}
    p(\eigenvar) = \eigenvar^2 - \eigenvar - 6,
  \end{equation*}
  with roots $\eigenvar=-2$ and $\eigenvar=3$. The respective eigenvectors
  are 
  \begin{equation*}
    \begin{array}{ll}
      \vect{v} = \begin{mymatrix}{c} 1 \\ -2 \end{mymatrix}
        &\quad\mbox{for the eigenvalue $\eigenvar=-2$,}\\\\[-2ex]
      \vect{u} = \begin{mymatrix}{c} 1 \\ 3 \end{mymatrix}
        &\quad\mbox{for the eigenvalue $\eigenvar=3$.}
    \end{array}
  \end{equation*}
  Therefore, $B = PDP^{-1}$, where
  \begin{equation*}
    P = \begin{mymatrix}{cc}
      1 & 1 \\
      -2 & 3 \\
    \end{mymatrix}
    \quad\mbox{and}\quad
    D = \begin{mymatrix}{cc}
      -2 & 0 \\
      0 & 3 \\
    \end{mymatrix}.
  \end{equation*}
  The inverse of $P$ is
  \begin{equation*}
    P^{-1} =
    \frac{1}{5}
    \begin{mymatrix}{cc}
      3 & -1 \\
      2 &  1 \\
    \end{mymatrix}.
  \end{equation*}
  Finally, we use this information to solve the recurrence:
  \begin{eqnarray*}
    b_n
    &=& \begin{mymatrix}{cc} 1 & 0 \end{mymatrix}\vect{w}_n \\
    &=& \begin{mymatrix}{cc} 1 & 0 \end{mymatrix}B^n\,\vect{w}_0 \\
    &=& \begin{mymatrix}{cc} 1 & 0 \end{mymatrix}PD^nP^{-1}\,\vect{w}_0 \\
    &=& \begin{mymatrix}{cc} 1 & 0 \end{mymatrix}
        \begin{mymatrix}{cc} 1 & 1 \\ -2 & 3 \end{mymatrix}
        \begin{mymatrix}{cc} (-2)^n & 0 \\ 0 & 3^n \end{mymatrix}
        \frac{1}{5}
        \begin{mymatrix}{cc} 3 & -1 \\ 2 &  1 \end{mymatrix}
        \begin{mymatrix}{c} 1 \\ 2 \end{mymatrix} \\
    &=& \frac{1}{5}
        \begin{mymatrix}{cc}1 & 1\end{mymatrix}
        \begin{mymatrix}{cc} (-2)^n & 0 \\ 0 & 3^n \end{mymatrix}
        \begin{mymatrix}{c} 1 \\ 4 \end{mymatrix} \\
    &=& \frac{(-2)^n + 4\cdot 3^n}{5}
  \end{eqnarray*}
  Finally, we calculate
  \begin{equation*}
    b_{20}
    = \frac{(-2)^{20} + 4\cdot 3^{20}}{5}
    = \frac{1048576 + 4\cdot 3486784401}{5}
    = 2789637236.
  \end{equation*}
  
\end{solution}
