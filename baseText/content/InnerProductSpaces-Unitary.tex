\section{Unitary and hermitian matrices}

\begin{outcome}
  \begin{enumerate}
  \item Determine whether a complex linear transformation is an
    isometry and/or unitary.
  \item Determine whether a complex matrix is unitary.
  \item Determine whether a complex matrix is hermitian.
  \item Calculate the eigenvalues and eigenvectors of a hermitian
    matrix.
  \item Compute an orthogonal basis of eigenvectors for a hermitian
    matrix.
  \item Unitarily diagonalize a hermitian matrix. 
  \end{enumerate}
\end{outcome}

In the context of real inner product spaces, we studied orthogonal
functions, orthogonal matrices, and symmetric matrices. The
corresponding concepts in the context of complex inner product spaces
are unitary functions, unitary matrices, and hermitian matrices. We
will introduce these concepts in this section. Since the proofs are
similar to those in Section~\ref{sec:orthogonal-matrices}, we omit
most of them.

\begin{definition}{Isometries and unitary maps of complex inner product spaces}{complex-isometries-unitary}
  Let $V,W$ be complex inner product spaces. A linear transformation
  $T:V\to W$ is called an \textbf{isometry}%
  \index{isometry}%
  \index{linear transformation!isometry} if for all
  $\vect{u},\vect{v}\in V$,
  \begin{equation*}
    \iprod{T(\vect{u}),T(\vect{v})} = \iprod{\vect{u},\vect{v}}.
  \end{equation*}
  An isometry that is also invertible is called a \textbf{unitary
    transformation}%
  \index{unitary transformation}%
  \index{linear transformation!unitary}, or simply \textbf{unitary}.
\end{definition}

In the case of real inner product spaces, we found in
Proposition~\ref{prop:matrix-orthogonal} that a square matrix $P$ is
the matrix of an orthogonal transformation (with respect to
orthonormal bases) if and only if $P^TP=I$. In the complex case, we
have an analogous property, except that we must use the adjoint
instead of the transpose.

\begin{proposition}{The matrix of a unitary transformation}{matrix-unitary}
  Let $T:V\to W$ be a linear transformation between finite-dimensional
  complex inner product spaces $V$ and $W$. Let $B$ and $C$ be
  orthonormal bases of $V$ and $W$, respectively, and let
  $P=\coord{T}_{C,B}$ be the matrix of $T$ with respect to the bases
  $B$ and $C$. Then
  \begin{enumialphparenastyle}
    \begin{enumerate}
    \item $T$ is an isometry if and only if $P^{\adjoint}P = I$.
    \item $T$ is unitary if and only if $P^{\adjoint}P = I$ and
      $\dim V=\dim W$.
    \end{enumerate}
  \end{enumialphparenastyle}
\end{proposition}

This motivates the following definition.  We therefore define a
\textbf{unitary}%
\index{unitary matrix} matrix to be an $n\times n$-matrix satisfying
$P^{\adjoint}P=I$.

\begin{definition}{Unitary matrix}{unitary-matrix}
  An $n\times n$-matrix $P$ is called \textbf{unitary}%
  \index{unitary matrix}%
  \index{matrix!unitary} if $P^{\adjoint}P=I$.
\end{definition}

We have the following analogue of
Proposition~\ref{prop:conditions-orthogonal-matrix}:

\begin{proposition}{Conditions for unitary matrices}{conditions-unitary-matrix}
  The following are equivalent for a complex $n\times n$-matrix $P$:
  \begin{enumialphparenastyle}
    \begin{enumerate}
    \item $P$ is unitary.
    \item $P^{\adjoint}P=I$.
    \item $P$ is invertible and $P^{-1}=P^{\adjoint}$.
    \item $PP^{\adjoint}=I$.
    \item $P^{\adjoint}$ is unitary.
    \item The columns of $P$ form an orthonormal set of vectors.
    \item The rows of $P$ form an orthonormal set of vectors.
    \end{enumerate}
  \end{enumialphparenastyle}
\end{proposition}

\begin{example}{Unitary matrices}{unitary-matrices}
  Determine which of the following matrices are unitary.
  \begin{equation*}
    A = \begin{mymatrix}{cc}
      1 & 0 \\
      0 & i \\
    \end{mymatrix},
    \quad
    B = \frac{1}{\sqrt{2}}
    \begin{mymatrix}{rr}
      1 & i \\
      i & 1 \\
    \end{mymatrix},
    \quad
    C = \frac{1}{\sqrt{2}}
    \begin{mymatrix}{rr}
      1 & i \\
      -i & 1 \\
    \end{mymatrix}.
  \end{equation*}
\end{example}

\begin{solution}
  We have
  \begin{equation*}
    A^{\adjoint}A =
    \begin{mymatrix}{cc}
      1 & 0 \\
      0 & -i \\
    \end{mymatrix}\begin{mymatrix}{cc}
      1 & 0 \\
      0 & i \\
    \end{mymatrix}
    = \begin{mymatrix}{cc}
      1 & 0 \\
      0 & 1 \\
    \end{mymatrix}
    = I,
  \end{equation*}
  so $A$ is unitary. Similarly, we have
  \begin{equation*}
    B^{\adjoint}B =
    \frac{1}{2}
    \begin{mymatrix}{rr}
      1 & i \\
      i & 1 \\
    \end{mymatrix}\begin{mymatrix}{rr}
      1 & -i \\
      -i & 1 \\
    \end{mymatrix}
    = \frac{1}{2}
    \begin{mymatrix}{rr}
      2 & 0 \\
      0 & 2 \\
    \end{mymatrix}
    = I,
  \end{equation*}
  so $B$ is unitary as well. On the other hand,
  \begin{equation*}
    C^{\adjoint}C =
    \frac{1}{2}
    \begin{mymatrix}{rr}
      1 & i \\
      -i & 1 \\
    \end{mymatrix}\begin{mymatrix}{rr}
      1 & i \\
      -i & 1 \\
    \end{mymatrix}
    = \frac{1}{2}
    \begin{mymatrix}{rr}
      2 & 2i \\
      -2i & 2 \\
    \end{mymatrix}
    \neq I,
  \end{equation*}
  so $C$ is not unitary. Equivalently, we could have checked whether
  the columns of $A$, $B$, and $C$ form an orthonormal set of vectors
  (they do, in the case of $A$ and $B$, but don't, in the case of
  $C$. See also Example~\ref{exa:orthogonal-vectors-complex}).
\end{solution}

Of course, if $P$ happens to be a matrix with real entries, then $P$
is unitary if and only if it is orthogonal, because in that case
$P^{\adjoint}=\conjugate{P}^T = P^T$.

Recall that a matrix $A$ is called \textbf{symmetric} if $A=A^T$. In
the complex world, we are more often interested in the property
$A=A^{\adjoint}$. A matrix with this property is called
\textbf{hermitian} (after the French mathematician Charles Hermite,
1822--1901).

\begin{definition}{Hermitian matrix}{hermitian-matrix}
  A complex $n\times n$-matrix $A$ is called \textbf{hermitian}%
  \index{hermitian matrix}%
  \index{matrix!hermitian}
  if $A=A^{\adjoint}$.
\end{definition}

\begin{example}{Hermitian vs. symmetric matrices}{hermitian-matrix}
  Which of the following matrices are hermitian? Which ones are
  symmetric?
  \begin{equation*}
    A = \begin{mymatrix}{cc} 1 & 2 \\ 2 & 1 \end{mymatrix}, \quad
    B = \begin{mymatrix}{cc} 1 & 2i \\ 2i & 1 \end{mymatrix}, \quad
    C = \begin{mymatrix}{cc} 1 & 2i \\ -2i & 1 \end{mymatrix}, \quad
    D = \begin{mymatrix}{cc} i & 2 \\ 2 & i \end{mymatrix}, \quad
  \end{equation*}
\end{example}

\begin{solution}
  The matrix $A$ is symmetric and also hermitian. The matrix $B$ is
  symmetric but not hermitian. In fact, we have
  \begin{equation*}
    B^{\adjoint} = \begin{mymatrix}{cc} 1 & -2i \\ -2i & 1 \end{mymatrix} \neq B.
  \end{equation*}
  The matrix $C$ is hermitian but not symmetric. The matrix $D$ is
  symmetric but not hermitian.
\end{solution}

A matrix $A=\mat{a_{ij}}$ is hermitian if and only if
$a_{ij} = \conjugate{a_{ji}}$, for all $i,j$. In particular, the
diagonal entries of a hermitian matrix are always real, and the
off-diagonal entries come in complex conjugate pairs. If all of the
entries in a matrix $A$ are real, then $A$ is hermitian if and only if
it is symmetric.

Hermitian matrices are of interest, among other things, because their
eigenvalues are always real. Moreover, eigenvectors for distinct
eigenvalues are orthogonal. The following proposition is analogous to
Proposition~\ref{prop:eigenvalues-symmetric}.

\begin{proposition}{Eigenvalues and eigenvectors of hermitian matrices}{eigenvalues-hermitian}
  Let $A$ be a hermitian matrix. Then
  \begin{enumialphparenastyle}
    \begin{enumerate}
    \item All eigenvalues of $A$ are real.
    \item Eigenvectors for distinct eigenvalues of $A$ are orthogonal.
    \end{enumerate}
  \end{enumialphparenastyle}
\end{proposition}

\begin{proof}
  (a) Suppose $\eigenvar$ is an eigenvalue of $A$, with eigenvector
  $\vect{v}$. We will evaluate $\vect{v}^{\adjoint}A\vect{v}$ in two
  different ways:
  \begin{eqnarray*}
    \vect{v}^{\adjoint}A\vect{v}
    &=& \vect{v}^{\adjoint}(A\vect{v})
        = \vect{v}^{\adjoint}(\eigenvar\vect{v})
        = \eigenvar\vect{v}^{\adjoint}\vect{v}, \\
    \vect{v}^{\adjoint}A\vect{v}
    &=& (\vect{v}^{\adjoint}A)\vect{v}
        = (\vect{v}^{\adjoint}A^{\adjoint})\vect{v}
        = (A\vect{v})^{\adjoint}\vect{v}
        = (\eigenvar\vect{v})^{\adjoint}\vect{v}
        = \conjugate{\eigenvar}\vect{v}^{\adjoint}\vect{v}.
  \end{eqnarray*}
  Therefore,
  $\eigenvar\vect{v}^{\adjoint}\vect{v} =
  \conjugate{\eigenvar}\vect{v}^{\adjoint}\vect{v}$. Since
  $\vect{v}^{\adjoint}\vect{v}$ is non-zero, this implies
  $\eigenvar=\conjugate{\eigenvar}$, i.e., $\eigenvar$ is real.

  \noindent
  (b) Suppose $\vect{v}$ and $\vect{w}$ are eigenvectors for
  eigenvalues $\eigenvarA$ and $\eigenvarB$, respectively, and
  $\eigenvarA\neq\eigenvarB$. From part (a), we know that both
  $\eigenvarA$ and $\eigenvarB$ are real.  By evaluating
  $\vect{v}^{\adjoint}A\vect{w}$ in two different ways, we find that
  \begin{eqnarray*}
    \vect{v}^{\adjoint}A\vect{w}
    &=& \vect{v}^{\adjoint}(A\vect{w})
    = \vect{v}^{\adjoint}(\eigenvarB\vect{w})
    = \eigenvarB \vect{v}^{\adjoint}\vect{w}, \\
    \vect{v}^{\adjoint}A\vect{w}
    &=& (\vect{v}^{\adjoint}A)\vect{w}
    = (A^{\adjoint}\vect{v})^{\adjoint}\vect{w}
    = (A\vect{v})^{\adjoint}\vect{w}
    = (\eigenvarA\vect{v})^{\adjoint}\vect{w}
    = \eigenvarA\vect{v}^{\adjoint}\vect{w}.
  \end{eqnarray*}
  Therefore,
  $\eigenvarB\vect{v}^{\adjoint}\vect{w} = \eigenvarA\vect{v}^{\adjoint}\vect{w}$, or
  equivalently $(\eigenvarA - \eigenvarB)\vect{v}^{\adjoint}\vect{w} =
  0$. Since by assumption, $\eigenvarA - \eigenvarB\neq 0$, we must
  have $\vect{v}^{\adjoint}\vect{w} = 0$, i.e., $\vect{v}\orth\vect{w}$.
\end{proof}

We say that a matrix $A$ is \textbf{unitarily diagonalizable}%
\index{unitarily diagonalizable matrix}%
\index{diagonalizable matrix!unitarily diagonalizable}%
\index{diagonalization!unitary diagonalization}%
\index{matrix!unitarily diagonalizable}%
\index{matrix!diagonalizable!unitarily} if there exists a unitary
matrix $P$ and a diagonal matrix $D$ such that $D = P^{-1}AP$.  The
following is the main theorem about the diagonalization of hermitian
matrices. It is analogous to
Theorem~\ref{thm:diagonalization-symmetric} in the real case.

\begin{theorem}{Diagonalization of hermitian matrices}{diagonalization-hermitian}
  Every hermitian matrix $A$ is unitarily diagonalizable as
  $D = P^{-1}AP$. Moreover, the entries of $D$ are real.
\end{theorem}

\begin{example}{Diagonalization of hermitian matrices}{diagonalization-hermitian}
  The matrix
  \begin{equation*}
    A = \begin{mymatrix}{cc}  3 & -2i \\ 2i & 6 \end{mymatrix}
  \end{equation*}
  is hermitian. Unitarily diagonalize $A$, i.e., find a unitary matrix
  $P$ and a real diagonal matrix $D$ such that $D = P^{-1}AP$.
\end{example}

\begin{solution}
  The characteristic polynomial is
  \begin{equation*}
    \det(A-\eigenvar I)
    = \begin{absmatrix}{cc} 3-\eigenvar & -2i \\ 2i & 6-\eigenvar \end{absmatrix}
    = (3-\eigenvar)(6-\eigenvar) + 4i\,^2
    = \eigenvar^2 - 9\eigenvar + 14.
  \end{equation*}
  Its roots are $\eigenvar_1=2$ and $\eigenvar_2=7$. For the eigenvalue
  $\eigenvar_1=2$, we find the normalized eigenvector
  \begin{equation*}
    \vect{v}_1 = \frac{1}{\sqrt{5}}\begin{mymatrix}{c} 2i \\ 1 \end{mymatrix}.
  \end{equation*}
  For the eigenvalue
  $\eigenvar_2=7$, we find the normalized eigenvector
  \begin{equation*}
    \vect{v}_2 = \frac{1}{\sqrt{5}}\begin{mymatrix}{c} 1 \\ 2i \end{mymatrix}.
  \end{equation*}
  Note that these eigenvectors are orthogonal to each other,
  confirming Proposition~\ref{prop:eigenvalues-hermitian}. We
  therefore have $D=P^{-1}AP$, where
  \begin{equation*}
    D = \begin{mymatrix}{cc} 2 & 0 \\ 0 & 7 \end{mymatrix}
    \quad\mbox{and}\quad
    P = \frac{1}{\sqrt{5}} \begin{mymatrix}{cc} 2i & 1 \\ 1 & 2i \end{mymatrix}.
  \end{equation*}
  Note that $P$ is unitary and $D$ is real diagonal.
\end{solution}
