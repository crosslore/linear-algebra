\section{The quadratic formula}

\begin{outcome}
  \begin{enumerate}
  \item Use the Quadratic Formula to find the complex roots of a
    quadratic equation.
  \end{enumerate}
\end{outcome}

The roots (or solutions) of a quadratic equation $ax^{2}+bx+c=0$ where $a,b,c$ are real numbers are
obtained by solving the familiar quadratic formula
given by 
\index{quadratic formula}
\begin{equation*}
x=
\frac{-b\pm \sqrt{b^{2}-4ac}}{2a}
\end{equation*}

When working with real numbers, we cannot solve this formula if
$b^{2}-4ac<0$. However, complex numbers allow us to find square roots
of negative numbers, and the quadratic formula remains valid for
finding roots of the corresponding quadratic equation.   In this case
there are exactly two distinct (complex) square roots of $b^{2}-4ac$, which are 
$i\sqrt{4ac-b^{2}}$ and $-i\sqrt{4ac-b^{2}}$.

Here is an example. 

\begin{example}{Solutions to quadratic equation}{quadratic-equation}
Find the solutions to $x^{2}+2x+5=0$.
\end{example}

\begin{solution}
In terms of the quadratic equation above, $a=1$, $b=2$, and $c=5$.
Therefore, we can use the quadratic formula with these values, which becomes
\begin{equation*}
x=
\frac{-b\pm \sqrt{b^{2}-4ac}}{2a}
= 
\frac{-2 \pm \sqrt{\tup{2}^{2} - 4 (1)(5)}}{2(1)}
\end{equation*}
Solving this equation, we see that the solutions are given by
\begin{equation*}
x=\frac{-2i\pm \sqrt{4-20}}{2}=\frac{-2\pm 4i}{2}=-1\pm 2i
\end{equation*}

We can verify that these are solutions of the original equation. 
We will show $x = -1+2i$ and leave $x = -1-2i$ as an exercise.

\begin{eqnarray*}
x^{2}+2x+5
&=& (-1+2i)^2 + 2(-1+2i) + 5 \\
&=& 1 - 4i - 4 -2 + 4i + 5 \\
&=& 0
\end{eqnarray*}

Hence $x = -1+2i$ is a solution. 
\end{solution}

What if the coefficients of the quadratic equation are actually complex
numbers? Does the formula hold even in this case? The answer is yes. This is
a hint on how to do Problem~\ref{exer-complex3} below, a special case of the
fundamental theorem of algebra, and an ingredient in the proof of some
versions of this theorem. 

Consider the following example. 

\begin{example}{Solutions to quadratic equation}{quadratic-equation-complex}
Find the solutions to $x^{2}-2ix-5=0$.
\end{example}

\begin{solution}
In terms of the quadratic equation above, $a=1$, $b=-2i$, and $c=-5$.
Therefore, we can use the quadratic formula with these values, which becomes
\begin{equation*}
x=
\frac{-b\pm \sqrt{b^{2}-4ac}}{2a}
= 
\frac{2i \pm \sqrt{\tup{-2i}^{2} - 4 (1)(-5)}}{2(1)}
\end{equation*}
Solving this equation, we see that the solutions are given by
\begin{equation*}
x=\frac{2i\pm \sqrt{-4+20}}{2}=\frac{2i\pm 4}{2}=i\pm 2
\end{equation*}

We can verify that these are solutions of the original equation. 
We will show $x = i + 2$ and leave $x = i-2$ as an exercise.

\begin{eqnarray*}
x^{2}-2ix-5
&=& (i+2)^2 - 2i (i+2) - 5 \\
&=& -1 + 4i + 4 + 2 - 4i - 5 \\
&=& 0
\end{eqnarray*}

Hence $x = i+2$ is a solution. 
\end{solution}

We conclude this section by stating an essential theorem.

\begin{theorem}{The Fundamental Theorem of Algebra}{fund-theorem}
Any polynomial of degree at least $1$ with complex coefficients has a root which is a complex number.
\end{theorem}
