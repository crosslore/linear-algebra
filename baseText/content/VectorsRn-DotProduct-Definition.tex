\subsection{Definition and properties}

When we take the dot product of two vectors, the result is a
scalar. For this reason, the dot product is also called the
\textbf{scalar product}. Sometimes it is also called the \textbf{inner
  product}. The definition is as follows\index{dot product}%
\index{vector!dot product}%
\index{scalar product|see{dot product}}%
\index{inner product|seealso{dot product}}.

\begin{definition}{Dot product}{dot-product}
  Let $\vect{u}=\begin{mymatrix}{c}
    u_1 \\
    u_2 \\
    \vdots \\
    u_n
  \end{mymatrix}$, $\vect{v}= \begin{mymatrix}{c}
    v_1 \\
    v_2 \\
    \vdots \\
    v_n
  \end{mymatrix}$ be two vectors in $\R^{n}$. We
  define their \textbf{dot product} as
  \begin{equation*}
    \vect{u}\dotprod \vect{v} = u_1v_1+u_2v_2+\ldots+u_nv_n.
  \end{equation*}
\end{definition}

\begin{example}{Compute a dot product}{dot-product}
  Find $\vect{u} \dotprod \vect{v}$ for $\vect{u}=\mat{1,2,0,-1}^T$
  and $\vect{v}=\mat{0,1,2,3}^T$.
\end{example}

\begin{solution}
  We have
  \begin{eqnarray*}
    \vect{u} \dotprod \vect{v}
    &=&
        (1)(0) + (2)(1) + (0)(2) + (-1)(3) \\
    &=&
        0 + 2 + 0 + -3 \\
    &=&
        -1.
  \end{eqnarray*}
\end{solution}

The dot product satisfies a number of important properties.

\begin{proposition}{Properties of the dot product}{properties-dot-product}
  \index{dot product!properties}%
  \index{properties of dot product}%
  \index{vector!dot product!properties}%
  \index{vector!properties of dot product}%
  The dot product satisfies the following properties, where
  $\vect{u},\vect{v},\vect{w}$ are vectors and $k,\ell$ are
  scalars.
  \begin{itemize}
  \item $\vect{u}\dotprod\vect{v}=\vect{v}\dotprod\vect{u}$.
  \item $\vect{u}\dotprod\vect{u}\geq 0$, and $\vect{u}\dotprod\vect{u}=0$ if and only if $\vect{u}=\vect{0}$.
  \item $(k\vect{u}+\ell\vect{v})\dotprod\vect{w}=k(\vect{u}\dotprod\vect{w})+\ell(\vect{v}\dotprod\vect{w})$.
  \item $\vect{u}\dotprod(k\vect{v}+\ell\vect{w})
    =k(\vect{u}\dotprod \vect{v})+\ell(\vect{u}\dotprod\vect{w})$.
  \item $\vect{u}\dotprod\vect{u}=\norm{\vect{u}}^{2}$.
  \end{itemize}
\end{proposition}

The proof is left as an exercise. Note that, by the last part of the
proposition, we can also use the dot product to find the length of a
vector.

\begin{example}{Length of a vector}{dot-product-length}
  Use a dot product to find $\norm{\vect{u}}$, where
  \begin{equation*}
    \vect{u}
    =
    \begin{mymatrix}{r}
      2 \\
      1 \\
      4 \\
      2
    \end{mymatrix}.
  \end{equation*}
\end{example}

\begin{solution}
  By the last part of Proposition~\ref{prop:properties-dot-product}, we have
  $\norm{\vect{u}} = \sqrt {\vect{u} \dotprod \vect{u}}$. We have
  $\vect{u} \dotprod \vect{u} = 2^2+1^2+4^2+2^2 = 25$, and therefore
  $\norm{\vect{u}} = \sqrt{\vect{u} \dotprod \vect{u}} = \sqrt{25} = 5$.
\end{solution}
