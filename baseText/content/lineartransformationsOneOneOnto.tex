\section{One to one and onto transformations}

\begin{outcome}
\begin{enumerate}
\item[A.]  Determine if a linear transformation is onto or one to one.
\end{enumerate}
\end{outcome}

Let $T: \mathbb{R}^n \mapsto \mathbb{R}^m$ be a linear transformation. We define the \textbf{range}\index{linear transformation!range} or \textbf{image}\index{linear transformation!image} of $T$ as the set of vectors of $\mathbb{R}^{m}$ which are of the form
$T \left(\vect{x}\right)$ (equivalently, $A\vect{x}$) for some $\vect{x}\in \mathbb{R}^{n}$. It is common
to write $T\mathbb{R}^{n}$, $T\left( \mathbb{R}^{n}\right)$, or
$\func{Im}\left( T\right) $ to denote these vectors.  

\begin{lemma}{Range of a matrix transformation}{Ax}

Let $A$ be an $m\times n$ matrix where $A_{1},\cdots , A_{n}$ denote the columns of
$A$\index{range of matrix transformation}. Then, for a vector $\vect{x}=\leftB 
\begin{array}{c}
x_{1} \\
\vdots \\
 x_{n}
\end{array}
\rightB$ in $\mathbb{R}^n$,

\begin{equation*}
A\vect{x}=\sum_{k=1}^{n}x_{k}A_{k}
\end{equation*}

Therefore, $A \left( \mathbb{R}^n \right)$ is the collection of all
linear combinations of these products.
\end{lemma}

\begin{proof}
This follows from the definition of matrix multiplication.
\end{proof}

This section is devoted to studying two important characterizations of linear transformations, called one to one and onto. We define them now. 

\begin{definition}{One to one}{onetoone}
Suppose $\vect{x}_1$ and $\vect{x}_2$ are vectors in $\mathbb{R}^n$. A linear transformation $T: \mathbb{R}^n \mapsto \mathbb{R}^m$ is called \textbf{one to one}\index{one to one} (often written as $1-1)$ if whenever
 $\vect{x}_1 \neq \vect{x}_2$ it follows that :
\begin{equation*}
T\left( \vect{x}_1 \right) \neq T \left(\vect{x}_2\right)
\end{equation*}

Equivalently, if $T\left( \vect{x}_1 \right) =T\left( \vect{x}_2\right) ,$
then $\vect{x}_1 = \vect{x}_2$. Thus,  $T$ is one to one if it never takes two different
vectors to the same vector.
\end{definition}

The second important characterization is called onto.

\begin{definition}{Onto}{onto}
Let $T: \mathbb{R}^n \mapsto \mathbb{R}^m$ be a linear transformation. Then $T$ is called \textbf{onto}\index{onto} if whenever $\vect{x}_2 \in \mathbb{R}^{m}$ there exists 
$\vect{x}_1 \in \mathbb{R}^{n}$ such that $T\left( \vect{x}_1\right) = \vect{x}_2. $
\end{definition}

We often call a linear transformation which is one-to-one an \textbf{injection}\index{injection}. Similarly, a linear transformation which is onto is often called a \textbf{surjection}\index{surjection}.

The following proposition is an important result. 

\begin{proposition}{One to One}{onetoonematrices}
Let $T:\mathbb{R}^n \mapsto \mathbb{R}^m$ be a linear transformation. Then $T$ is one to one if
and only if $T(\vect{x}) = \vect{0}$ implies $\vect{x}=\vect{0}$.
\end{proposition}

\begin{proof}
We need to prove two things here. First, we will prove that if $T$ is one to one, then 
$T(\vect{x}) = \vect{0}$ implies that $\vect{x}=\vect{0}$. Second, we will show that if $T(\vect{x})=\vect{0}$ implies that $\vect{x}=\vect{0}$, then 
it follows that $T$ is one to one. Recall that a linear transformation has the property that $T(\vect{0}) = \vect{0}$. 

%%Note that since $T$ is linear, it is induced by an $m \times n$ matrix $A$. Therefore we can rewrite the statement ``$T_A(\vect{x}) = \vect{0}$ implies $\vect{x}=\vect{0}$'' in terms of the matrix $A$ as ``$A\vect{x}=\vect{0}$ implies $\vect{x}=\vect{0}$''. Therefore we can prove this theorem using $A$. 
%%
%%Observe that $A\vect{0}=A\left( \vect{0}+\vect{0} \right) =A\vect{0} +A\vect{0}$ and so $A\vect{0}=\vect{0}$.
%% 
%%Now suppose $A$ is one to one and $A\vect{x}=\vect{0}$. We need to show that this implies $\vect{x}=\vect{0}$. Since $A$ is one to one, by Definition  \ref{def:onetoone} $A$ can only map one vector to the zero vector $\vect{0}$. Now $A\vect{x}=\vect{0}$ and $A\vect{0}=\vect{0}$, so it follows that $\vect{x}=\vect{0}$. Thus if $A$ is one to one and $A\vect{x}=\vect{0}$, then $\vect{x}=\vect{0}$. 
%%
%%Next assume that $A\vect{x}=\vect{0}$ implies $\vect{x}=\vect{0}$.  We need to show that $A$ is one to one. Suppose $A\vect{x}=A\vect{y}$. Then $A\vect{x} - A\vect{y} = \vect{0}$. 
%%Hence $A\vect{x}-A\vect{y} =  A\left(\vect{x}-\vect{y}\right) = \vect{0}$. However, we have assumed that $A\vect{x}=\vect{0}$ implies $\vect{x}=\vect{0}$. This means that
%%whenever $A$ times a vector equals $\vect{0}$, that vector is also equal to $\vect{0}$. Therefore,  $\vect{x}-\vect{y} = \vect{0}$ and so $\vect{x}=\vect{y}$.
%%Thus $A$ is one to one by Definition \ref{def:onetoone}.
%%\end{proof}

Suppose first that $T$ is one to one and consider $T(\vect{0})$. 
\begin{equation*}
T(\vect{0})=T\left( \vect{0}+\vect{0}\right) =T(\vect{0})+T(\vect{0})
\end{equation*}
and so, adding the additive inverse of $T(\vect{0})$ to both sides, one sees
that $T(\vect{0})=\vect{0}$. If $T(\vect{x})=\vect{0}$ it must be the
case that $\vect{x}=\vect{0}$ because it was just shown that $T(\vect{0})=\vect{0}$ and $T$ is assumed to be one to one. 

Now assume that if $T(\vect{x})=\vect{0},$ then it follows that $\vect{x}=\vect{0}.$ If $T(\vect{v})=T(\vect{u}),$ then 
\[
T(\vect{v})-T(\vect{u})=T\left( \vect{v}-\vect{u}\right) =\vect{0}
\]
which shows that $\vect{v}-\vect{u}=0$. In other words, $\vect{v}=\vect{u}$, and $T$ is one to one. 
\end{proof}

Note that this proposition says that if $A=\leftB
\begin{array}{ccc}
A_{1} & \cdots & A_{n}
\end{array}
\rightB $ then $A$ is one to one if and only if whenever
\begin{equation*}
0 = \sum_{k=1}^{n}c_{k}A_{k}
\end{equation*}
it follows that each scalar $c_{k}=0$. 

We will now take a look at an example of a one to one and onto linear transformation. 

\begin{example}{A one to one and onto linear transformation}{onetooneontolineartransformation}
Suppose
\begin{equation*}
T\leftB
\begin{array}{c}
x \\
y
\end{array}
\rightB =\leftB
\begin{array}{rr}
1 & 1 \\
1 & 2
\end{array}
\rightB \leftB
\begin{array}{r}
x \\
y
\end{array}
\rightB
\end{equation*}
Then, $T:\mathbb{R}^{2}\rightarrow \mathbb{R}^{2}$ is a linear
transformation. Is $T$ onto? Is it one to one?
\end{example}

\begin{solution} Recall that because $T$ can be expressed as matrix
multiplication, we know that $T$ is a linear transformation.  We will
start by looking at onto.  So suppose $\leftB
\begin{array}{c}
a \\
b
\end{array}
\rightB \in \mathbb{R}^{2}.$ Does there exist $\leftB
\begin{array}{c}
x \\
y
\end{array}
\rightB  \in \mathbb{R}^2 $ such that $T\leftB
\begin{array}{c}
x \\
y
\end{array}
\rightB =\leftB
\begin{array}{c}
a \\
b
\end{array}
\rightB ?$ If so, then since $\leftB
\begin{array}{c}
a \\
b
\end{array}
\rightB $ is an arbitrary vector in $\mathbb{R}^{2},$ it will follow that $T$
is onto. 

This question is familiar to you. It is asking whether
there is a solution to the equation
\begin{equation*}
\leftB
\begin{array}{cc}
1 & 1 \\
1 & 2
\end{array}
\rightB \leftB
\begin{array}{c}
x \\
y
\end{array}
\rightB =\leftB
\begin{array}{c}
a \\
b
\end{array}
\rightB
\end{equation*}
This is the same thing as asking for a solution to the following system of
equations.
\begin{equation*}
\begin{array}{c}
x+y=a \\
x+2y=b
\end{array}
\end{equation*}
Set up the augmented matrix and row reduce.
\begin{equation}
\leftB
\begin{array}{rr|r}
1 & 1 & a \\
1 & 2 & b
\end{array}
\rightB \rightarrow \leftB
\begin{array}{rr|r}
1 & 0 & 2a-b \\
0 & 1 & b-a
\end{array}
\rightB
\label{ontomatrix}
\end{equation}
You can see from this point that the system has a solution. Therefore,
we have shown that for any $a, b$, there is a $
\leftB
\begin{array}{c}
x \\
y
\end{array}
\rightB$ such that $T\leftB
\begin{array}{c}
x \\
y
\end{array}
\rightB =\leftB
\begin{array}{c}
a \\
b
\end{array}
\rightB$.
Thus $T$ is onto.

Now we want to know if $T$ is one to one. 
By Proposition \ref{prop:onetoonematrices} it is enough to show that $A\vect{x}=0$ implies $\vect{x}=0$. 
Consider the system $A\vect{x}=0$ given by:
\begin{equation*}
\leftB
\begin{array}{cc}
1 & 1 \\
1 & 2\\
\end{array}
\rightB
\leftB
\begin{array}{c}
x\\
y
\end{array}
\rightB
=
\leftB
\begin{array}{c}
0 \\
0
\end{array}
\rightB
\end{equation*}

This is the same as the system given by

\begin{equation*}
\begin{array}{c}
x + y = 0 \\
x + 2y = 0
\end{array}
\end{equation*}

We need to show that the solution to this system is $x = 0$ and $y = 0$. By setting up the augmented matrix and row reducing, we end up with
\begin{equation*} \leftB
\begin{array}{rr|r}
1 & 0 & 0 \\
0 & 1 & 0
\end{array}
\rightB
\end{equation*}

This tells us that $x = 0$ and $y = 0$. Returning to the original system, this says that if 

\begin{equation*}
\leftB
\begin{array}{cc}
1 & 1 \\
1 & 2\\
\end{array}
\rightB
\leftB
\begin{array}{c}
x\\
y
\end{array}
\rightB
=
\leftB
\begin{array}{c}
0 \\
0
\end{array}
\rightB
\end{equation*}

then 
\begin{equation*}
\leftB
\begin{array}{c}
x \\
y
\end{array}
\rightB
=
\leftB
\begin{array}{c}
0 \\
0
\end{array}
\rightB
\end{equation*}

In other words, $A\vect{x}=0$ implies that $\vect{x}=0$. By 
Proposition \ref{prop:onetoonematrices}, $A$ is one to one, and so $T$ is also one to one.

We also could have seen that $T$ is one to one from our above solution for onto. By looking at the matrix given 
by \ref{ontomatrix}, you can see that there is a \textbf{unique} solution given
by $x=2a-b$ and $y=b-a$. Therefore, there
is only one vector, specifically 
$\leftB
\begin{array}{c}
x \\
y
\end{array}
\rightB
=
\leftB
\begin{array}{c}
2a-b\\
b-a
\end{array}
\rightB $ such that $T\leftB
\begin{array}{c}
x \\
y
\end{array}
\rightB =\leftB
\begin{array}{c}
a \\
b
\end{array}
\rightB $. Hence by Definition \ref{def:onetoone}, $T$ is one to one.
\end{solution}

\begin{example}{An onto transformation}{ontotransformation}
Let $T: \mathbb{R}^4 \mapsto \mathbb{R}^2$ be a linear transformation defined by
\[
T \leftB \begin{array}{c}
a \\
b \\
c \\
d
\end{array}
\rightB = 
\leftB \begin{array}{c}
a + d \\
b + c 
\end{array}
\rightB
\mbox{ for all } \leftB \begin{array}{c}
a \\
b \\
c \\
d
\end{array}
\rightB \in \mathbb{R}^4
\]
Prove that $T$ is onto but not one to one.
\end{example}

\begin{solution} 
You can prove that $T$ is in fact linear. 

To show that $T$ is onto, let $\leftB \begin{array}{c} 
x \\
y
\end{array} \rightB$ be an arbitrary vector in $\mathbb{R}^2$. Taking the vector $\leftB \begin{array}{c}
x \\
y \\
0 \\
0 
\end{array} \rightB \in \mathbb{R}^4$ we have 
\[
T \leftB \begin{array}{c}
x \\
y \\
0 \\
0
\end{array}
\rightB = 
\leftB \begin{array}{c}
x + 0 \\
y + 0 
\end{array}
\rightB
= \leftB \begin{array}{c}
x \\
y 
\end{array}
\rightB
\]
This shows that $T$ is onto. 

By Proposition \ref{prop:onetoonematrices} $T$ is one to one if and only if $T(\vect{x}) = \vect{0}$ implies that $\vect{x} = \vect{0}$. Observe that 
\[
T \leftB \begin{array}{r}
1 \\
0 \\
0 \\
-1
\end{array}
\rightB = 
\leftB \begin{array}{c}
1 + -1 \\
0 + 0 
\end{array}
\rightB
= \leftB \begin{array}{c}
0 \\
0 
\end{array}
\rightB
\]
There exists a nonzero vector $\vect{x}$ in $\mathbb{R}^4$ such that $T(\vect{x}) = \vect{0}$. It follows that $T$ is not one to one.
\end{solution}

The above examples demonstrate a method to determine if a linear transformation $T$ is one to one or onto. It turns out that the matrix $A$ of $T$ can provide this information.

\begin{theorem}{Matrix of a one to one or onto transformation}{matrixonetooneonto}
Let $T: \mathbb{R}^n \mapsto \mathbb{R}^m$ be a linear transformation induced by the $m \times n$ matrix $A$. Then $T$ is one to one if and only if the rank of $A$ is $n$. $T$ is onto if and only if the rank of $A$ is $m$. 
\end{theorem}

Consider Example \ref{exa:ontotransformation}. Above we showed that $T$ was onto but not one to one. We can now use this theorem to determine this fact about $T$. 

\begin{example}{An onto transformation}{ontotransformationmatrix}
Let $T: \mathbb{R}^4 \mapsto \mathbb{R}^2$ be a linear transformation defined by
\[
T \leftB \begin{array}{c}
a \\
b \\
c \\
d
\end{array}
\rightB = 
\leftB \begin{array}{c}
a + d \\
b + c 
\end{array}
\rightB
\mbox{ for all } \leftB \begin{array}{c}
a \\
b \\
c \\
d
\end{array}
\rightB \in \mathbb{R}^4
\]
Prove that $T$ is onto but not one to one.
\end{example}

\begin{solution}
Using Theorem \ref{thm:matrixonetooneonto} we can show that $T$ is onto but not one to one from the matrix of $T$. Recall that to find the matrix $A$ of $T$, we apply $T$ to each of the standard basis vectors $\vect{e}_i$ of $\mathbb{R}^4$. The result is the $2 \times 4$ matrix A given by 
\[
A = \leftB \begin{array}{rrrr}
1 & 0 & 0 & 1 \\
0 & 1 & 1 & 0 
\end{array} \rightB
\]
Fortunately, this matrix is already in {\rref}. The rank of $A$ is $2$. Therefore by the above theorem $T$ is onto but not one to one. 
\end{solution}

Recall that if $S$ and $T$ are linear transformations, we can discuss their composite denoted $S \circ T$. The following examines what happens if both $S$ and $T$ are onto. 

\begin{example}{Composite of onto transformations}{compositeonto}
Let $T: \mathbb{R}^k \mapsto \mathbb{R}^n$ and $S: \mathbb{R}^n \mapsto \mathbb{R}^m$ be linear transformations. 
If $T$ and $S$ are onto, then $S \circ T$ is onto.
\end{example}

\begin{solution}
Let $\vect{z}\in \mathbb{R}^m$.  
Since $S$ is onto, there exists a vector $\vect{y}\in \mathbb{R}^n$
such that $S(\vect{y})=\vect{z}$.
Furthermore, since $T$ is onto, there exists a vector $\vect{x}\in \mathbb{R}^k$
such that $T(\vect{x})=\vect{y}$.
Thus
\[ \vect{z} = S(\vect{y}) = S(T(\vect{x})) = (ST)(\vect{x}),\]
showing that for each $\vect{z}\in \mathbb{R}^m$ there exists and $\vect{x}\in \mathbb{R}^k$
such that $(ST)(\vect{x})=\vect{z}$.
Therefore, $S \circ T$ is onto.
\end{solution}

The next example shows the same concept with regards to one-to-one transformations. 

\begin{example}{Composite of one to one transformations}{compositeonetoone}
Let $T: \mathbb{R}^k \mapsto \mathbb{R}^n$ and $S: \mathbb{R}^n \mapsto \mathbb{R}^m$ be linear transformations. 
Prove that if $T$ and $S$ are one to one, then $S \circ T$ 
is one-to-one.
\end{example}

\begin{solution}
To prove that $S \circ T$ is one to one, we need to show that if $S(T (\vect{v})) = \vect{0}$ it follows that $\vect{v} = \vect{0}$. 
Suppose that  $S(T (\vect{v})) = \vect{0}$. Since $S$ is one to one, it follows that  $T (\vect{v}) = \vect{0}$. Similarly, since $T$ is one to one, it follows that $\vect{v} = \vect{0}$. Hence $S \circ T$ is one to one. 
\end{solution}
