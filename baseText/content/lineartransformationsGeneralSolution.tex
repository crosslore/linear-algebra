\section{The general solution of a linear system}

\begin{outcome}
  \begin{enumerate}
  \item Use linear transformations to determine the particular
    solution and general solution to a system of equations.
  \item Find the kernel of a linear transformation.
  \end{enumerate}
\end{outcome}

Recall the definition of a linear transformation discussed above. 
$T$ is a \textbf{linear transformation} if whenever $\vect{x}, \vect{y}$ are
vectors and $k,p$ are scalars,
\begin{equation*}
T\tup{k\vect{x}+p\vect{y}} =k T \tup{\vect{x} } +p T\tup{\vect{y} }
\end{equation*}
Thus linear transformations distribute across addition and pass scalars to
the outside.

It turns out that we can use linear transformations to solve
systems of linear equations. Indeed given a system of linear equations of the
form $A\vect{x}=\vect{b}$, one may rephrase this as $T(\vect{x})=\vect{b}$ where $T$ is the linear
transformation $T_A$ induced by the coefficient matrix $A$. With this in mind consider the following definition. 

\begin{definition}{Particular solution of a system of equations}{particular-solutions}
Suppose a system of linear equations can be written in the form
\begin{equation*}
T\tup{\vect{x}}=\vect{b}
\end{equation*}
If $T\tup{\vect{x}_{p}}=\vect{b}$, 
then $\vect{x}_{p}$ is called a \textbf{particular solution}\index{particular solution} of
the linear system.
\end{definition}

Recall that a system is called homogeneous if every equation in the system is equal to $0$. 
Suppose we represent a homogeneous system of equations by $T\tup{\vect{x}}=0$. It turns out
that the $\vect{x}$ for which $T \tup{\vect{x}} = 0$ are part of a special set called the \textbf{null space}
of $T$. We may also refer to the null space as the \textbf{kernel} of $T$, and we write $ker\tup{T}$. 

Consider the following definition.

\begin{definition}{Null space or kernel of a linear transformation}{null-space-of-linear-transformation}
Let $T$ be a linear transformation. Define
\begin{equation*}
\ker \tup{T} = \set{\vect{x}:T \tup{\vect{x} }= \vect{0} } 
\end{equation*}
The kernel\index{kernel}, $\ker \tup{T} $ consists of the set of all vectors $\vect{x}$ for which
$T (\vect{x}) = \vect{0}$. This is also called the
\textbf{null space}\index{null space} of $T$. 
\end{definition}

We may also refer to the kernel of $T$ as the
\textbf{solution space}\index{solution space} of the equation $T \tup{\vect{x}} = \vect{0}$.


Consider the following example.

\begin{example}{The kernel of the derivative}{kernel-derivative}
Let $\frac{d}{dx}$ denote the linear transformation defined on $f$, the functions
which are defined on $\R$ and have a continuous derivative. Find 
$\ker \tup{\frac{d}{dx}}$.
\end{example}

\begin{solution} The example asks for functions $f$ which the property that $\frac{df}{dx}
=0. $ As you may know from calculus, these functions are the constant functions.
Thus $\ker \tup{\frac{d}{dx}}$ is the set of constant functions.
\end{solution} 

Definition~\ref{def:null-space-of-linear-transformation} states that $\ker \tup{T} $ is the set of
solutions to the equation,
\begin{equation*}
T\tup{\vect{x} } = \vect{0}
\end{equation*}
Since we can write $T\tup{\vect{x} }$ as $A\vect{x}$, you have been solving such
equations for quite some time.

We have spent a lot of time finding solutions to systems of equations in general, as well as
homogeneous systems. Suppose we look at a system given by $A\vect{x}=\vect{b}$, and consider the 
related homogeneous system. By this, we mean that we replace $\vect{b}$ by $\vect{0}$ and look at $A\vect{x}=\vect{0}$. 
It turns out that there is a very important relationship between the solutions of the original
system and the solutions of the associated homogeneous system. In the following 
theorem, we use linear transformations to denote a system of equations. Remember that
$T\tup{\vect{x}} = A\vect{x}$.

\begin{theorem}{Particular solution and general solution}{particular-and-general-solution}
Suppose $\vect{x}_{p}$ is a solution to the linear system given by
\begin{equation*}
T\tup{\vect{x} } = \vect{b}.
\end{equation*}
Then if $\vect{y}$ is any other solution to $T\tup{\vect{x}}=\vect{b}$, 
\index{general solution!solution space} there exists $\vect{x}_0 \in \ker
\tup{T} $ such that
\begin{equation*}
\vect{y} = \vect{x}_{p}+ \vect{x}_0.
\end{equation*}
Hence, every solution to the linear system can be written as a sum of a particular solution, $\vect{x}_p$,
 and a solution $\vect{x}_0$ to the associated 
homogeneous system given by $T\tup{\vect{x}}=\vect{0}$.
\end{theorem}

\begin{proof}
Consider $\vect{y} - \vect{x}_{p}= \vect{y} + \tup{
-1} \vect{x}_{p}$. Then $T\tup{\vect{y} - \vect{x}_{p}} =T\tup{\vect{y}}
-T\tup{\vect{x}_{p} }$. Since $\vect{y}$ and $\vect{x}_{p}$ are both solutions to the system, it follows that $T\tup{\vect{y}}= \vect{b} $
and $T\tup{\vect{x}_p} = \vect{b}$. 

Hence, $T\tup{\vect{y}}-T\tup{\vect{x}_{p} }
=\vect{b} - \vect{b} = \vect{0}$.  Let $\vect{x}_0 = \vect{y} - \vect{x}_{p}$.
Then, $T\tup{\vect{x}_0}= \vect{0} $ so $\vect{x}_0$ is a solution to the associated homogeneous system and so is in $\ker \tup{T}$.
\end{proof}

Sometimes people remember the above theorem in the following form. The
solutions to the system $T\tup{\vect{x}}=\vect{b}$ are given by 
$\vect{x}_{p}+\ker \tup{T} $ where $\vect{x}_{p}$ is a particular
solution to $T\tup{\vect{x}}=\vect{b}$.

For now, we have been speaking about the kernel or null space of a linear transformation $T$. However, 
we know that every linear transformation $T$ is determined by some matrix $A$. Therefore,
we can also speak about the null space of a matrix. Consider the following example.  

\begin{example}{The null space of a matrix}{matrix-null-space}
Let
\begin{equation*}
A=\begin{mymatrix}{rrrr}
1 & 2 & 3 & 0 \\
2 & 1 & 1 & 2 \\
4 & 5 & 7 & 2
\end{mymatrix}
\end{equation*}
Find $\nullspace(A)$. Equivalently, find the solutions to the
system of equations $A\vect{x}=\vect{0}$.
\end{example}

\begin{solution}  We are asked to find $\set{\vect{x} : A\vect{x} = \vect{0}}$. In other
words we want to solve the system, $A\vect{x}=\vect{0}$. Let $\vect{x} =
\begin{mymatrix}{r}
x \\
y \\
z \\
w
\end{mymatrix}$. Then this amounts to solving
\begin{equation*}
\begin{mymatrix}{rrrr}
1 & 2 & 3 & 0 \\
2 & 1 & 1 & 2 \\
4 & 5 & 7 & 2
\end{mymatrix} \begin{mymatrix}{c}
x \\
y \\
z \\
w
\end{mymatrix} =\begin{mymatrix}{c}
0 \\
0 \\
0
\end{mymatrix}
\end{equation*}
This is the linear system
\begin{equation*}
\begin{array}{c}
x+2y+3z=0 \\
2x+y+z+2w=0 \\
4x+5y+7z+2w=0
\end{array}
\end{equation*}
To solve, set up the augmented matrix and row reduce to find the {\rref}.
\begin{equation*}
\begin{mymatrix}{rrrr|r}
1 & 2 & 3 & 0 & 0 \\
2 & 1 & 1 & 2 & 0 \\
4 & 5 & 7 & 2 & 0
\end{mymatrix}
\sim\ldots\sim
\begin{mymatrix}{rrrr|r}
1 & 0 & -
\vspace{0.05in}\frac{1}{3} & \vspace{0.05in}\frac{4}{3} &  0 \\
0 & 1 & \vspace{0.05in}\frac{5}{3} & -\vspace{0.05in}\frac{2}{3} & 0 \\
0 & 0 & 0 & 0 & 0
\end{mymatrix} 
\end{equation*}
This yields $x=\vspace{0.05in}\frac{1}{3}z-\vspace{0.05in}\frac{4}{3}w$ and 
$y=\vspace{0.05in}\frac{2}{3}w-\vspace{0.05in}\frac{5}{3}z$.
Since $\nullspace(A) $ consists of the solutions to this system, it consists vectors of the form,
\begin{equation*}
\begin{mymatrix}{c}
\vspace{0.05in}\frac{1}{3}z-\vspace{0.05in}\frac{4}{3}w \\
\vspace{0.05in}\frac{2}{3}w-\vspace{0.05in}\frac{5}{3}z \\
z \\
w
\end{mymatrix} =z \begin{mymatrix}{r}
\vspace{0.05in}\frac{1}{3} \\
-\vspace{0.05in}\frac{5}{3} \\
1 \\
0
\end{mymatrix} +w \begin{mymatrix}{r}
-\vspace{0.05in}\frac{4}{3} \\
\vspace{0.05in}\frac{2}{3} \\
0 \\
1
\end{mymatrix} 
\end{equation*}
\end{solution}

Consider the following example.

\begin{example}{A general solution}{general-solution}
The \textbf{general solution}\index{general solution} of a system of linear equations 
is the set of
all possible solutions. Find the general solution to the linear system,
\begin{equation*}
\begin{mymatrix}{rrrr}
1 & 2 & 3 & 0 \\
2 & 1 & 1 & 2 \\
4 & 5 & 7 & 2
\end{mymatrix} \begin{mymatrix}{r}
x \\
y \\
z \\
w
\end{mymatrix} =\begin{mymatrix}{r}
9 \\
7 \\
25
\end{mymatrix}
\end{equation*}
given that $\begin{mymatrix}{r}
x \\
y \\
z \\
w
\end{mymatrix}=\begin{mymatrix}{r}
1 \\
1 \\
2 \\
1
\end{mymatrix}$ is one solution.
\end{example}

\begin{solution} Note the matrix of this system is the same as the matrix in Example~\ref{exa:matrix-null-space}. Therefore, from Theorem~\ref{thm:particular-and-general-solution}, you will obtain all
solutions to the above linear system by adding a particular solution $\vect{x}_p$ to the solutions of the associated homogeneous 
system, $\vect{x}$. One particular solution is given above by
\begin{equation}
\vect{x}_p
=
\begin{mymatrix}{r}
x \\
y \\
z \\
w
\end{mymatrix}=\begin{mymatrix}{r}
1 \\
1 \\
2 \\
1
\end{mymatrix}
\end{equation}

Using this particular solution along with the solutions found in Example~\ref{exa:matrix-null-space}, we
obtain the following solutions, 
\begin{equation*}
z\begin{mymatrix}{r}
\vspace{0.05in}\frac{1}{3} \\
-\vspace{0.05in}\frac{5}{3} \\
1 \\
0
\end{mymatrix} +w\begin{mymatrix}{r}
-\vspace{0.05in}\frac{4}{3} \\
\vspace{0.05in}\frac{2}{3} \\
0 \\
1
\end{mymatrix} +\begin{mymatrix}{r}
1 \\
1 \\
2 \\
1
\end{mymatrix} 
\end{equation*}

Hence, any solution to the above linear system is of this form.
\end{solution} 
