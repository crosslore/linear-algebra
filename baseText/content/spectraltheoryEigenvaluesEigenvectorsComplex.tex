\section{Complex eigenvalues and eigenvectors}

An $n\times n$-matrix is diagonalizable if and only if it has $n$
linearly independent eigenvectors. But as we saw in
Example~\ref{exa:no-real-eigenvalue}, if we work over the real
numbers, it can sometimes happen that a matrix has no eigenvalues, and
therefore no eigenvectors, at all. For example, the matrix
\begin{equation*}
  A=\begin{mymatrix}{rr}
      0 & -1 \\
      1 &  0 \\
    \end{mymatrix}
\end{equation*}
has characteristic polynomial $\eigenvar^2+1$. Since the equation
$\eigenvar^2+1$ does not have any roots in the real numbers, there are
no real eigenvalues.

On the other hand, the fundamental theorem of algebra tell us that
over the {\em complex} numbers, every non-constant polynomial has a
root. In fact, every polynomial of degree $n$ factors into $n$ linear
factors. Therefore, every matrix has at least one eigenvalue over the
complex numbers. Some matrices are diagonalizable over the complex
numbers but not over the real numbers. An introduction to complex
numbers and the fundamental theorem of algebra can be found in
Appendix~\ref{app:complex}.

\begin{example}{Complex eigenvalues and eigenvectors}{complex-eigenvalue1}
  Find the eigenvectors and eigenvalues of
  \begin{equation*}
    A=\begin{mymatrix}{rr}
      0 & -1 \\
      1 &  0 \\
    \end{mymatrix}
  \end{equation*}
  over the complex numbers. Diagonalize $A$ if possible.
\end{example}

\begin{solution}
  The characteristic polynomial is $\eigenvar^2+1$. This has no roots
  in the real numbers, but it has two roots $\eigenvar=i$ and
  $\eigenvar=-i$\, in the complex numbers. To find the eigenvectors for
  $\eigenvar=i$, we solve $(A-iI)\vect{v}=\vect{0}$:
  \begin{equation*}
    \begin{mymatrix}{rr|r}
      -i & -1 & 0 \\
      1 & -i & 0 \\
    \end{mymatrix}
    \quad\stackrel{R_1\rowswap R_2}{\roweq}\quad
    \begin{mymatrix}{rr|r}
      1 & -i & 0 \\
      -i & -1 & 0 \\
    \end{mymatrix}
    \quad\stackrel{R_2\rowop R_2+iR_1}{\roweq}\quad
    \begin{mymatrix}{rr|r}
      1 & -i & 0 \\
      0 & 0 & 0 \\
    \end{mymatrix}.
  \end{equation*}
  Thus, the basic eigenvector for $\eigenvar=i$\, is
  \begin{equation*}
    \vect{v}_1 = \begin{mymatrix}{c} i \\ 1 \end{mymatrix}.
  \end{equation*}
  Similarly, to find the eigenvectors for $\eigenvar=-i$, we solve
  $(A+iI)\vect{v}=\vect{0}$:
  \begin{equation*}
    \begin{mymatrix}{cc|c}
      i & -1 & 0 \\
      1 &  i & 0 \\
    \end{mymatrix}
    \quad\stackrel{R_1\rowswap R_2}{\roweq}\quad
    \begin{mymatrix}{cc|c}
      1 &  i & 0 \\
      i & -1 & 0 \\
    \end{mymatrix}
    \quad\stackrel{R_2\rowop R_2-iR_1}{\roweq}\quad
    \begin{mymatrix}{cc|c}
      1 & i & 0 \\
      0 & 0 & 0 \\
    \end{mymatrix}.
  \end{equation*}
  Thus, the basic eigenvector for $\eigenvar=-i$\, is
  \begin{equation*}
    \vect{v}_2 = \begin{mymatrix}{r} -i \\ 1 \end{mymatrix}.
  \end{equation*}
  Since we have found two linearly independent eigenvectors, the
  matrix $A$ is diagonalizable. We have $A=PDP^{-1}$, where
  \begin{equation*}
    P =
    \begin{mymatrix}{cr}
      i & -i \\
      1 &  1 \\
    \end{mymatrix}
    \quad\mbox{and}\quad
    D =
    \begin{mymatrix}{cr}
      i &  0 \\
      0 & -i \\
    \end{mymatrix}.
  \end{equation*}
\end{solution}

\begin{example}{Complex eigenvalues and eigenvectors}{complex-eigenvalue2}
  Find the eigenvectors and eigenvalues of
  \begin{equation*}
    A=\begin{mymatrix}{rr}
      1 & -1 \\
      1 &  1 \\
    \end{mymatrix}
  \end{equation*}
  over the complex numbers. Diagonalize $A$ if possible.
\end{example}

\begin{solution}
  The characteristic polynomial is
  \begin{equation*}
    \det(A-\eigenvar I)
    = \begin{absmatrix}{cc}
      1-\eigenvar & -1 \\
      1 & 1-\eigenvar \\
    \end{absmatrix}
    = (1-\eigenvar)^2 + 1
    = \eigenvar^2 - 2\eigenvar + 2.
  \end{equation*}
  To find the roots, we use the quadratic formula. The roots are given
  by:
  \begin{equation*}
    \eigenvar
    = \frac{-b\pm\sqrt{b^2-4ac}}{2a}
    = \frac{2\pm\sqrt{-4}}{2} = 1\pm i.
  \end{equation*}
  Note that since the discriminant $b^2-4ac$ is negative, there are no
  real solutions. However, we find two complex solutions $\eigenvar=1+i$
  and $\eigenvar=1-i$. To find the eigenvectors for $\eigenvar=1+i$, we
  solve the equation $(A-(1+i)I)\vect{v}=\vect{0}$:
  \begin{equation*}
    \begin{mymatrix}{rr|r}
      -i & -1 & 0 \\
      1 & -i & 0 \\
    \end{mymatrix}
    \quad\roweq\ldots\roweq\quad
    \begin{mymatrix}{rr|r}
      1 & -i & 0 \\
      0 & 0 & 0 \\
    \end{mymatrix}.
  \end{equation*}
  The basic eigenvector for $\eigenvar=1+i$\, is
  \begin{equation*}
    \vect{v}_1 = \begin{mymatrix}{c} i \\ 1 \end{mymatrix}.
  \end{equation*}
  Similarly, the basic eigenvector for $\eigenvar=1-i$\, is
  \begin{equation*}
    \vect{v}_2 = \begin{mymatrix}{r} -i \\ 1 \end{mymatrix}.
  \end{equation*}
  Since we have found two linearly independent eigenvectors, the
  matrix $A$ is diagonalizable. We have $A=PDP^{-1}$, where
  \begin{equation*}
    P =
    \begin{mymatrix}{cr}
      i & -i \\
      1 &  1 \\
    \end{mymatrix}
    \quad\mbox{and}\quad
    D =
    \begin{mymatrix}{cc}
      1+i &  0 \\
      0 & 1-i \\
    \end{mymatrix}.
  \end{equation*}
\end{solution}

\begin{example}{Diagonalize a matrix over the complex numbers}{complex-diagonalize}
  Diagonalize the matrix
  \begin{equation*}
    A = \begin{mymatrix}{rrr}
      -3 & -2 & 4 \\
      2  &  1 & 0 \\
      -2 & -2 & 3 \\
    \end{mymatrix}.
  \end{equation*}
\end{example}

\begin{solution}
  The characteristic polynomial is
  \begin{equation*}
    p(\eigenvar)
    = (-3-\eigenvar)(1-\eigenvar)(3-\eigenvar) - 16 +8(1-\eigenvar) + 4(3-\eigenvar)
    = -\eigenvar^3 + \eigenvar^2 - 3\eigenvar - 5.
  \end{equation*}
  By trial and error, we find that $\eigenvar=-1$ is one of the
  roots. We factor out $(\eigenvar + 1)$:
  \begin{equation*}
    p(\eigenvar) = (\eigenvar+1)(-\eigenvar^2 + 2\eigenvar - 5).
  \end{equation*}
  We then use the quadratic formula to find the other two eigenvalues,
  i.e., the roots of $-\eigenvar^2 + 2\eigenvar - 5$. They are:
  \begin{equation*}
    \eigenvar = \frac{-2\pm\sqrt{-16}}{-2} = 1\pm 2i.
  \end{equation*}
  Eigenvectors:
  \begin{itemize}
  \item For $\lambda=-1$, we solve $(A-(-1)I)\vect{v}=\vect{0}$:
    \begin{equation*}
      \begin{mymatrix}{rrr|r}
        -2 & -2 & 4 & 0 \\
        2  &  2 & 0 & 0 \\
        -2 & -2 & 4 & 0 \\
      \end{mymatrix}
      \quad\roweq\ldots\roweq\quad
      \begin{mymatrix}{rrr|r}
        1 & 1 &  0 & 0 \\
        0 & 0 &  1 & 0 \\
        0 & 0 &  0 & 0 \\
      \end{mymatrix}.
    \end{equation*}
    The basic eigenvector is
    \begin{equation*}
      \vect{v}_1 = \begin{mymatrix}{r} -1 \\ 1 \\ 0 \end{mymatrix}.
    \end{equation*}
  \item For $\lambda=1+2i$, we solve $(A-(1+2i)I)\vect{v}=\vect{0}$:
    \begin{eqnarray*}
      \begin{mymatrix}{ccc|c}
        -4-2i & -2  & 4    & 0 \\
        2     & -2i & 0    & 0 \\
        -2    & -2  & 2-2i & 0 \\
      \end{mymatrix}
      & \stackrel{R_1\rowop-R_1/2}{
        \stackrel{R_2\rowop R_2/2}{
        \stackrel{R_3\rowop -R_3/2}{\roweq}}} &
      \begin{mymatrix}{ccc|c}
        2+i &  1 &  -2 & 0 \\
        1   & -i &  0  & 0 \\
        1   &  1 & i-1 & 0 \\
      \end{mymatrix}
      \\
      &\stackrel{R_1\rowswap R_2}{\roweq} &
      \begin{mymatrix}{ccc|c}
        1   & -i &  0  & 0 \\
        2+i &  1 &  -2 & 0 \\
        1   &  1 & i-1 & 0 \\
      \end{mymatrix}
      \\
      & \stackrel{R_2\rowop R_2-(2+i)R_1}{
        \stackrel{R_3\rowop R_3-R_1}{\roweq}} &
      \begin{mymatrix}{ccc|c}
        1   & -i  &  0  & 0 \\
        0   & 2i  & -2  & 0 \\
        0   & 1+i & i-1 & 0 \\
      \end{mymatrix}
      \\
      & \stackrel{R_2 \rowop R_2/2i}{
        \stackrel{R_3 \rowop R_3/(1+i)}{\roweq}} &
      \begin{mymatrix}{ccc|c}
        1   & -i  &  0  & 0 \\
        0   & 1  &   i  & 0 \\
        0   & 1   &  i  & 0 \\ 
      \end{mymatrix}
      \\
      & \stackrel{R_1 \rowop R_1+iR_2}{
        \stackrel{R_3 \rowop R_3-R_2}{\roweq}} &
      \begin{mymatrix}{ccc|c}
        1     & 0   & -1  & 0 \\
        0     & 1   & i   & 0 \\
        0     & 0   & 0   & 0 \\
      \end{mymatrix}.
    \end{eqnarray*}
    The basic eigenvector is
    \begin{equation*}
      \vect{v}_2 = \begin{mymatrix}{r} 1 \\ -i \\ 1 \end{mymatrix}.
    \end{equation*}
  \item For $\lambda=1-2i$, we solve $(A-(1-2i)I)\vect{v}=\vect{0}$:
    \begin{equation*}
      \begin{mymatrix}{ccc|c}
        -4+2i & -2  & 4    & 0 \\
        2     &  2i & 0    & 0 \\
        -2    & -2  & 2+2i & 0 \\
      \end{mymatrix}
      \quad\roweq\ldots\roweq\quad
      \begin{mymatrix}{ccc|c}
        1     & 0   & -1  & 0 \\
        0     & 1   & -i  & 0 \\
        0     & 0   & 0   & 0 \\
      \end{mymatrix}.
    \end{equation*}
    The basic eigenvector is
    \begin{equation*}
      \vect{v}_3 = \begin{mymatrix}{c} 1 \\ i \\ 1 \end{mymatrix}.
    \end{equation*}
    Therefore, $A=PDP^{-1}$, where
    \begin{equation*}
      P =
      \begin{mymatrix}{ccc}
        -1 &  1 & 1 \\
        1  & -i & i \\
        0  &  1 & 1 \\
      \end{mymatrix}
      \quad\mbox{and}\quad
      D =
      \begin{mymatrix}{ccc}
        -1 &  0   & 0    \\
        0  & 1+2i & 0    \\
        0  &  0   & 1-2i \\
      \end{mymatrix}.
    \end{equation*}
  \end{itemize}
\end{solution}

It is important to note that even over the complex numbers, not all
matrices are diagonalizable. On the one hand, the characteristic
polynomial of an $n\times n$-matrix always factors into $n$ linear
factors over the complex numbers. Therefore, the sum of the algebraic
multiplicities of the eigenvalues is always $n$. However, it can still
happen that the geometric multiplicity of some eigenvalue is less than
its algebraic multiplicity. In that case, the matrix is not
diagonalizable, even over the complex numbers. We have:

\begin{proposition}{Diagonalizability criterion}{complex-diagonalizability}
  A square matrix $A$ is diagonalizable over the complex numbers if
  and only if the geometric multiplicity of each eigenvalue is equal
  to its algebraic multiplicity.
\end{proposition}

\begin{proof}
  Let $A$ be an $n\times n$-matrix. By
  Proposition~\ref{prop:complex-factoring}, the characteristic
  polynomial factors into $n$ linear factors:
  \begin{equation*}
    p(\eigenvar) = (b_1-\eigenvar)(b_2-\eigenvar)\cdots(b_n-\eigenvar).
  \end{equation*}
  Therefore, the sum of the algebraic multiplicities of all the
  eigenvalues is $n$. If the geometric multiplicity of each eigenvalue
  is equal to its algebraic multiplicity, then the sum of the
  geometric multiplicities is also $n$, and therefore $A$ is
  diagonalizable. On the other hand, if the geometric multiplicity of
  some eigenvalue is less than its algebraic multiplicity, then the
  sum of the geometric multiplicities is less than $n$, and $A$ is not
  diagonalizable.
\end{proof}

\begin{example}{Non-diagonalizable matrix}{complex-non-diagonalizable}
  Show that the matrix $A =
    \begin{mymatrix}{rr}
      1 & 1 \\
      0 & 1
    \end{mymatrix}$ cannot be diagonalized, even over the complex numbers.
\end{example}

\begin{solution}
  The characteristic polynomial is $(1-\eigenvar)^2$, and therefore
  the only eigenvalue is $\lambda=1$, with algebraic multiplicity
  $2$. On the other hand, the eigenspace for $\lambda=1$ is
  1-dimensional:
  \begin{equation*}
    \begin{mymatrix}{rr}
      0 & 1 \\
      0 & 0
    \end{mymatrix}
    \vect{v}
    = \begin{mymatrix}{r} 0 \\ 0 \end{mymatrix}
  \end{equation*}
  has a 1-dimensional solution space. Therefore, we can find only one
  basic eigenvector, and the matrix is not diagonalizable.
\end{solution}
