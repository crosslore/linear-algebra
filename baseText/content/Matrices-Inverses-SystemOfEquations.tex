\subsection{Using the inverse to solve a system of equations}

One way in which the inverse of a matrix is useful is to find the
solution of a system of linear equations.  Recall from Definition~\ref{def:matrix-form} that we can write a system of equations in
matrix form, which is in the form
\begin{equation*}
  A\vect{x}=\vect{b}.
\end{equation*}
Suppose we find the inverse $A^{-1}$ of the matrix $A$. Then we can
multiply both sides of this equation by $A^{-1}$ on the left and
simplify to obtain
\begin{equation*}
  \vect{x} = A^{-1}\vect{b}.
\end{equation*}
Therefore we can find $\vect{x}$, the solution to the system, by
computing $\vect{x} = A^{-1}\vect{b}$. Note that once we have found
$A^{-1}$, we can easily get the solution for different right-hand
sides (different $\vect{b}$). It is always just $A^{-1}\vect{b}$.

\begin{example}{Using the inverse to solve a system of equations}{inverse-to-solve-system}
  Consider the following system of equations. Use the inverse of a
  suitable matrix to solve this system.
  \begin{equation*}
    \begin{array}{c}
      x+z=1 \\
      x-y+z=3 \\
      x+y-z=2
    \end{array}
  \end{equation*}
\end{example}

\begin{solution}
  First, we can write the system in matrix form
  \begin{equation*}
    A\vect{x} =
    \begin{mymatrix}{rrr}
      1 & 0 & 1 \\
      1 & -1 & 1 \\
      1 & 1 & -1
    \end{mymatrix} \begin{mymatrix}{r}
      x \\
      y \\
      z
    \end{mymatrix} =\begin{mymatrix}{r}
      1 \\
      3 \\
      2
    \end{mymatrix} = \vect{b}.
  \end{equation*}
  The inverse of $A$ is
  \begin{equation*}
    A^{-1} =
    \def\arraystretch{1.2}
    \begin{mymatrix}{rrr}
      0 & \frac{1}{2} & \frac{1}{2} \\
      1 & -1 & 0 \\
      1 & -\frac{1}{2} & -\frac{1}{2}
    \end{mymatrix}.
  \end{equation*}
  From here, the solution to the system $A\vect{x}=\vect{b}$ is found
  by $\vect{x}=A^{-1}\vect{b}$, i.e.,
  \begin{equation*}
    \def\arraystretch{1.2}
    \begin{mymatrix}{r}
      x \\
      y \\
      z
    \end{mymatrix}
    =
    \begin{mymatrix}{rrr}
      0 & \frac{1}{2} & \frac{1}{2} \\
      1 & -1 & 0 \\
      1 & -\frac{1}{2} & -\frac{1}{2}
    \end{mymatrix} \begin{mymatrix}{r}
      1 \\
      3 \\
      2
    \end{mymatrix} =\begin{mymatrix}{r}
      \frac{5}{2} \\
      -2 \\
      -\frac{3}{2}
    \end{mymatrix}.
  \end{equation*}
\end{solution}

What if the right-hand side had been $\vect{b}=\begin{mymatrix}{r}
  0 \\
  1 \\
  3
\end{mymatrix}$? In this case, the solution would be given by
\begin{equation*}
  \def\arraystretch{1.2}
  \begin{mymatrix}{r}
    x \\
    y \\
    z
  \end{mymatrix} = \begin{mymatrix}{rrr}
    0 & \frac{1}{2} & \frac{1}{2} \\
    1 & -1 & 0 \\
    1 & -\frac{1}{2} & -\frac{1}{2}
  \end{mymatrix} \begin{mymatrix}{r}
    0 \\
    1 \\
    3
  \end{mymatrix} =\begin{mymatrix}{r}
    2 \\
    -1 \\
    -2
  \end{mymatrix}.
\end{equation*}
This illustrates that for a system $A\vect{x}=\vect{b}$ where $A^{-1}$ exists,
it is easy to find the solution when the vector $\vect{b}$ is changed.
