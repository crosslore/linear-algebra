\section{Linear transformations}

\begin{outcome}
  \begin{enumerate}
  \item Determine whether a vector function $T:\R^n\to\R^m$ is a
    linear transformation.
  \end{enumerate}
\end{outcome}

In calculus, a \textbf{function}%
\index{function} $f:\R\to\R$ is a rule that maps a real number
$x\in\R$ to a real number $f(x)\in\R$. In linear algebra, we can
generalize this concept to vectors. A \textbf{vector function}%
\index{vector function}%
\index{function!vector function} $T:\R^n\to\R^m$ is a rule that inputs
an $n$-dimensional vector $\vect{v}\in\R^n$ and outputs an
$m$-dimensional vector $T(\vect{v})\in\R^m$. The following are some
examples of vector functions:
\begin{equation}\label{eq:vector-functions}
  T_1\paren{\begin{mymatrix}{c} x \\ y \end{mymatrix}}
  = \begin{mymatrix}{c} x^2 \\ x+y \\ y^2 \end{mymatrix},\quad
  T_2\paren{\begin{mymatrix}{c} x \\ y \\ z \end{mymatrix}}
  = \begin{mymatrix}{c} x+y \\ x+y+z \\ 0 \end{mymatrix},\quad
  T_3\paren{\begin{mymatrix}{c} x \\ y \\ z \end{mymatrix}}
  = \begin{mymatrix}{c} e^{x+z} \\ \sqrt{y} \end{mymatrix}.
\end{equation}
Of these, the first is a function $T_1:\R^2\to\R^3$, the second is a
function $T_2:\R^3\to\R^3$, and the third is a function
$T_3:\R^3\to\R^2$.  We can evaluate a vector function by applying it
to a vector, for example,
\begin{equation*}
  T_1\paren{\begin{mymatrix}{c} 1 \\ 2 \end{mymatrix}}
  = \begin{mymatrix}{c} 1^2 \\ 1+2 \\ 2^2 \end{mymatrix}
  = \begin{mymatrix}{c} 1 \\ 3 \\ 4 \end{mymatrix},\quad
  T_1\paren{\begin{mymatrix}{c} 0 \\ 1 \end{mymatrix}}
  = \begin{mymatrix}{c} 0^2 \\ 1+1 \\ 1^2 \end{mymatrix}
  = \begin{mymatrix}{c} 0 \\ 1 \\ 1 \end{mymatrix},
\end{equation*}
and so on. The study of arbitrary vector functions and their
derivatives and integrals is the subject of {\em multivariable
  calculus}%
\index{calculus!multivariable}%
\index{multivariable calculus}. In linear algebra, we will only be
concerned with \textbf{linear vector functions}%
\index{vector function!linear|see{linear transformation}}, which are
also called \textbf{linear transformations}%
\index{linear transformation}. They are defined as follows.

\begin{definition}{Linear transformation}{linear-transformation}
  A vector function $T:\R^{n}\to \R^{m}$ is called a \textbf{linear
    transformations}%
  \index{linear transformation}, or simply \textbf{linear}, if it
  satisfies the following two conditions:
  \begin{enumerate}
  \item $T$ preserves addition, i.e., for all\/
    $\vect{v},\vect{w}\in\R^n$, we have
    $T(\vect{v}+\vect{w}) = T(\vect{v}) + T(\vect{w})$;
  \item $T$ preserves scalar multiplication, i.e, for all\/
    $\vect{v}\in\R^n$ and scalars $k$, we have
    $T(k\vect{v}) = kT(\vect{v})$.
  \end{enumerate}
\end{definition}

\begin{example}{Linear and non-linear transformations}{linear-transformation}
  Which of the vector functions in {\eqref{eq:vector-functions}} are
  linear transformations?
\end{example}

\begin{solution}
  \begin{enumerate}
  \item[(a)] The function $T_1$ is not a linear transformation. For
    example, let $\vect{v}=\begin{mymatrix}{r} 1 \\
      0 \end{mymatrix}$. Then
    \begin{equation*}
      T_1(\vect{v})
      ~=~ T_1\paren{\begin{mymatrix}{c} 1 \\ 0 \end{mymatrix}}
      ~=~ \begin{mymatrix}{c} 1 \\ 1 \\ 0 \end{mymatrix}
      \quad\mbox{and}\quad
      T_1(2\vect{v})
      ~=~ \paren{\begin{mymatrix}{c} 2 \\ 0 \end{mymatrix}}
      ~=~ \begin{mymatrix}{c} 4 \\ 2 \\ 0 \end{mymatrix}
      ~\neq~ 2\begin{mymatrix}{c} 1 \\ 1 \\ 0 \end{mymatrix}.
    \end{equation*}
    Since $T_1(2\vect{v}) \neq 2T_1(\vect{v})$, the vector function
    $T_1$ does not preserve scalar multiplication, and therefore it is
    not a linear transformation.
  \item[(b)] The function $T_2$ is a linear transformation. For
    example, to prove that $T_2$ preserves addition, consider two
    arbitrary vectors
    \begin{equation*}
      \vect{v} =
      \begin{mymatrix}{c}
        x_1 \\
        y_1 \\
        z_1 \\
      \end{mymatrix}
      \quad\mbox{and}\quad
      \vect{w} =
      \begin{mymatrix}{c}
        x_2 \\
        y_2 \\
        z_2 \\
      \end{mymatrix}.
    \end{equation*}
    We have
    \begin{equation*}
      T_2(\vect{v}+\vect{w})
      ~=~ T_2\paren{
        \begin{mymatrix}{c}
          x_1+x_2 \\
          y_1+y_2 \\
          z_1+z_2 \\
        \end{mymatrix}}
      ~=~ \begin{mymatrix}{c}
        (x_1+x_2)+(y_1+y_2) \\
        (x_1+x_2)+(y_1+y_2)+(z_1+z_2) \\
        0
      \end{mymatrix}
    \end{equation*}
    and
    \begin{equation*}
      T_2(\vect{v})+T_2(\vect{w})
      ~=~
      \begin{mymatrix}{c}
        x_1+y_1 \\
        x_1+y_1+z_1 \\
        0 \\
      \end{mymatrix}
      + \begin{mymatrix}{c}
        x_2+y_2 \\
        x_2+y_2+z_2 \\
        0 \\
      \end{mymatrix}
      ~=~ \begin{mymatrix}{c}
        (x_1+y_1)+(x_2+y_2) \\
        (x_1+y_1+z_1)+(x_2+y_2+z_2) \\
        0 \\
      \end{mymatrix}.
    \end{equation*}
    Since the two sides are evidently equal, $T_2$ preserves
    addition. The fact that it preserves scalar multiplication can be
    shown by a similar calculation.
  \item[(c)] The function $T_3$ is not a linear transformation. For
    example, consider
    $\vect{v}=\begin{mymatrix}{c}0\\1\\0\end{mymatrix}$ and
    $\vect{w}=\begin{mymatrix}{c}1\\1\\0\end{mymatrix}$.
    Then
    \begin{equation*}
      T_3(\vect{v}+\vect{w})
      ~=~ T_3\paren{\begin{mymatrix}{c} 1 \\ 2 \\ 0\end{mymatrix}}
      ~=~ \begin{mymatrix}{c} e \\ \sqrt{2} \end{mymatrix},
    \end{equation*}
    and
    \begin{equation*}
      T_3(\vect{v})+T_3(\vect{w})
      ~=~ \begin{mymatrix}{c} 1 \\ 1 \end{mymatrix}
      + \begin{mymatrix}{c} e \\ 1 \end{mymatrix}
      ~=~ \begin{mymatrix}{c} e+1 \\ 2 \end{mymatrix}.
    \end{equation*}
    Since $T_3(\vect{v}+\vect{w})\neq T_3(\vect{v})+T_3(\vect{w})$,
    the vector function $T_3$ does not preserve addition, and therefore
    it is not linear.
  \end{enumerate}
\end{solution}

An easy fact about linear transformation is that they preserve the
origin, i.e., they satisfy $T(\vect{0}) = \vect{0}$. This can be seen,
for example, by considering
$T(\vect{0}) = T(\vect{0}+\vect{0}) = T(\vect{0}) + T(\vect{0})$ and
then subtracting $T(\vect{0})$ from both sides of the equation.  This
gives an easier way to see that $T_3$ in the above example is not a
linear transformation, since $T_3(\vect{0}) \neq \vect{0}$. On the
other hand, of course not every function that preserves the origin is
linear. For example, $T_1$ is not linear although it satisfies
$T_1(\vect{0})=\vect{0}$.

The following characterization of linearity is often useful, as it
permits us to check just one property instead of two.

\begin{proposition}{Alternative characterization of linear transformations}{linear-transformation-alternative}
  A vector function $T:\R^{n}\to \R^{m}$ is linear if and only if it
  satisfies the following condition, for all
  $\vect{v},\vect{w}\in\R^n$ and scalars $a,b$:
  \begin{equation*}
    T(a\vect{v}+b\vect{w}) = aT(\vect{v}) + bT(\vect{w}).
  \end{equation*}
\end{proposition}

\begin{proof}
  First, assume that $T$ is linear. Then from preservation of addition
  and scalar multiplication, we have
  $T(a\vect{v}+b\vect{w}) = T(a\vect{v})+T(b\vect{w}) = aT(\vect{v}) +
  bT(\vect{w})$. Conversely, assume that $T$ satisfies
  $T(a\vect{v}+b\vect{w}) = aT(\vect{v}) + bT(\vect{w})$ for all
  vectors $\vect{v},\vect{w}$ and scalars $a,b$. Then we get
  preservation of addition by setting $a=b=1$, and preservation of
  scalar multiplication by setting $b=0$.
\end{proof}
