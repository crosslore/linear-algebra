\section{Linear transformations}

\begin{outcome}
\begin{enumerate}
\item[A.]  Understand the definition of a linear transformation, and that all linear transformations are determined by matrix multiplication.
\end{enumerate}
\end{outcome}

Recall that when we multiply an $m\times n$ matrix by an $n\times 1 $
column vector, the result is an $m\times 1$ column vector. In this
section we will discuss how, through matrix multiplication, an $m
\times n$ matrix \textbf{transforms} an $n\times 1$ column vector into
an $m \times 1$ column vector.

Recall that the $n \times 1$ vector given by
\begin{equation*}
\vect{x} = 
\begin{mymatrix}{r}
x_1 \\
x_2\\ 
\vdots \\
x_n
\end{mymatrix}
\end{equation*}
is said to belong to $\R^n$, which is the set of all $n \times 1$ vectors. In this section, we will discuss transformations of vectors in $\R^n.$ 

Consider the following example. 

\begin{example}{A function which transforms vectors}{functiontransformation}
Consider the matrix $A = \begin{mymatrix}{ccc}
1 & 2 & 0 \\
2 & 1 & 0
\end{mymatrix} .$ 
Show that by matrix multiplication $A$ transforms vectors in $\R^3$ into vectors in $\R^2$.
\end{example}

\begin{solution}
First, recall that vectors in $\R^3$ are vectors of size $ 3 \times 1$, while vectors in $
\R^{2}$ are of size $2 \times 1$. If we multiply $A$, which is a $2 \times 3$ matrix, by a $3 \times 1$ vector,
the result will be a $2 \times 1$ vector. This what we mean when we say that $A$ {\em transforms \em} vectors.

Now, for $\begin{mymatrix}{c}
x \\
y \\
z
\end{mymatrix} $ in $\R^3$ , multiply on the left by the given matrix to obtain the new
vector. This product looks like 
\begin{equation*}
\begin{mymatrix}{rrr}
1 & 2 & 0 \\
2 & 1 & 0
\end{mymatrix} 
\begin{mymatrix}{r}
x \\
y \\
z
\end{mymatrix} = 
\begin{mymatrix}{c}
x+2y \\
2x+y
\end{mymatrix}
\end{equation*}
The resulting product is a $2 \times 1$ vector which is determined by the choice of $x$ and $y$. 
 Here are some numerical examples.
\begin{equation*}
\begin{mymatrix}{ccc}
1 & 2 & 0 \\
2 & 1 & 0
\end{mymatrix} \begin{mymatrix}{c}
1 \\
2 \\
3
\end{mymatrix} =\allowbreak \begin{mymatrix}{c}
5 \\
4
\end{mymatrix}
\end{equation*}
Here, the vector
$\begin{mymatrix}{c}
1 \\
2 \\
3
\end{mymatrix}$
in $\R^3$ was transformed by the matrix into the vector
$\begin{mymatrix}{c} 
5 \\
4
\end{mymatrix}$
in $\R^2$. 
 
Here is another example:
\begin{equation*}
\begin{mymatrix}{rrr}
1 & 2 & 0 \\
2 & 1 & 0
\end{mymatrix} \begin{mymatrix}{r}
10 \\
5 \\
-3
\end{mymatrix} =\allowbreak \begin{mymatrix}{r}
20 \\
25
\end{mymatrix}
\end{equation*}
\end{solution}

The idea is to define a function which takes vectors in
$\R^{3}$ and delivers new vectors in $\R^{2}.$ In this
case, that function is multiplication by the matrix $A$.

Let $T$ denote such a function. The notation $T:\R^{n}\mapsto \R^{m}$ means that the function $T$
transforms vectors in $\R^{n}$ into vectors in $\R^{m}$. The notation $T(\vect{x})$ means the transformation $T$ applied to the vector $\vect{x}$. The above example demonstrated a transformation achieved by matrix multiplication. In this case,  we often write
\begin{equation*}
T_{A}\left( \vect{x}\right) =A \vect{x}
\end{equation*}
Therefore, $T_{A}$ is the transformation determined by the matrix $A$. In this case we say that $T$ is a matrix transformation. 

Recall the property of matrix multiplication that states that for 
$k $ and $p$ scalars,
\begin{equation*}
A\left( kB+pC\right) =kAB+pAC
\end{equation*}
In particular, for $A$ an $m\times n$ matrix and $B$ and $C,$ $n\times 1$
vectors in $\R^{n}$,  this formula holds.

In other words, this means that matrix multiplication gives an
example of a linear transformation, which we will now define. 

\begin{definition}{Linear transformation}{lineartransformation}
 Let $T:\R^{n}\mapsto \R^{m}$ be a function, where for each
$\vect{x} \in \R^{n},T\left(\vect{x}\right)\in \R^{m}.$ Then $T$ is a
\textbf{linear transformation}\index{linear transformation} if whenever $k ,p $ are scalars and 
$\vect{x}_1$ and $\vect{x}_2$ are vectors in $\R^{n}$ $(
n\times 1$ vectors$),$
\begin{equation*}
T\left( k \vect{x}_1 + p \vect{x}_2 \right) = kT\left(\vect{x}_1\right)+ pT\left(\vect{x}_{2} \right)
\end{equation*}
\end{definition}

Consider the following example.

\begin{example}{Linear transformation}{lineartransformation}
Let $T$ be a transformation defined by
$T:\R^3\to\R^2$ is defined by
\[
T\begin{mymatrix}{c} x \\ y \\ z \end{mymatrix}
= 
\begin{mymatrix}{c} x+y \\ x-z \end{mymatrix}
\mbox{ for all }
\begin{mymatrix}{c} x \\ y \\ z \end{mymatrix} \in\R^3
\]
Show that $T$ is a linear transformation.
\end{example}

\begin{solution}
By Definition \ref{def:lineartransformation} we need to show that $T\left( k \vect{x}_1 + p \vect{x}_2 \right) = kT\left(\vect{x}_1\right)+ pT\left(\vect{x}_{2} \right)$ for all scalars $k,p$ and vectors $\vect{x}_1, \vect{x}_2$.
Let
\[
\vect{x}_1 = \begin{mymatrix}{c} x_1 \\ y_1 \\ z_1 \end{mymatrix}, 
\vect{x}_2 = \begin{mymatrix}{c} x_2 \\ y_2 \\ z_2 \end{mymatrix}
\]
Then
\begin{eqnarray*}
T\left( k \vect{x}_1 + p \vect{x}_2 \right) &=& T \left( k \begin{mymatrix}{c} x_1 \\ y_1 \\ z_1 \end{mymatrix} + p \begin{mymatrix}{c} x_2 \\ y_2 \\ z_2 \end{mymatrix} \right) \\
&=& T \left(  \begin{mymatrix}{c} kx_1 \\ ky_1 \\ kz_1 \end{mymatrix} +  \begin{mymatrix}{c} px_2 \\ py_2 \\ pz_2 \end{mymatrix} \right) \\
&=& T \left(  \begin{mymatrix}{c} kx_1 + px_2 \\ ky_1 + py_2 \\ kz_1 + pz_2 \end{mymatrix}  \right) \\
&=& \begin{mymatrix}{c} (kx_1 + px_2) + (ky_1 + py_2) \\ (kx_1 + px_2)- (kz_1 + pz_2) \end{mymatrix} \\
&=& \begin{mymatrix}{c} (kx_1 + ky_1) + (px_2 + py_2) \\ (kx_1 - kz_1) + (px_2 - pz_2) \end{mymatrix} \\
&=& \begin{mymatrix}{c} kx_1 + ky_1 \\ kx_1 - kz_1 \end{mymatrix} + \begin{mymatrix}{c} px_2 + py_2 \\  px_2 - pz_2 \end{mymatrix} \\
&=& k \begin{mymatrix}{c} x_1 + y_1 \\ x_1 - z_1 \end{mymatrix} + p \begin{mymatrix}{c} x_2 + y_2 \\  x_2 - z_2 \end{mymatrix} \\
&=& k T(\vect{x}_1) + p T(\vect{x}_2) 
\end{eqnarray*}
Therefore $T$ is a linear transformation. 
\end{solution}

Two important examples of linear transformations are the zero transformation\index{zero transformation} and identity transformation\index{identity transformation}. The zero transformation defined by $T\left( \vect{x} \right) = \vect(0)$ for all $\vect{x}$ is an example of a linear transformation. Similarly the identity transformation defined by $T\left( \vect{x} \right) = \vect(x)$ is also linear. Take the time to prove these using the method demonstrated in Example \ref{exa:lineartransformation}.

We began this section by discussing matrix transformations, where multiplication by a matrix transforms vectors. These matrix transformations are in fact linear transformations. 

\begin{theorem}{Matrix transformations are linear transformations}{matrixarelinear}
Let $T:\R^{n}\mapsto \R^{m}$ be a transformation defined by $T(\vect{x}) = A\vect{x}$. Then $T$ is a linear transformation. 
\end{theorem}

It turns out that every linear transformation can be expressed as a matrix transformation, and thus linear transformations are exactly the same as matrix transformations. 
