\section{Application: Error correcting codes}

\begin{outcome}
  \begin{enumerate}
  \item Determine the block length, message length, Hamming distance,
    and rate of a code.
  \item Determine whether a code is $m$-error detecting and $m$-error
    correcting.
  \item Find generator and check matrices for a linear code.
  \item Use the syndrome method to correct errors in code blocks.
  \item Construct and use Hamming codes.
  \end{enumerate}
\end{outcome}

% ----------------------------------------------------------------------
\subsection*{Binary codes}

When transmitting or storing information on digital media, the
information is usually encoded as a sequence of bits, i.e., 0s and 1s.
For example, in the ASCII code, each symbol is encoded as a sequence
of 8 bits. The letter ``A'' is encoded as $01000001$, the letter ``B''
is encoded as $01000010$, and so on. We can think of a sequence of $n$
bits as an $n$-dimensional column vector over the field $\Z_2$, i.e.,
as an element of $\Z_2^n$. For our purposes, it is convenient to
continue writing bit sequences horizontally, but we will consider this
to be merely an alternate notation for a column vector.

One issue with digital data is that the data can sometimes be
corrupted. DVDs may be scratched, magnetic storage may depolarize, and
data sent by radio transmission may be subject to interference. This
can result in errors in the data, such as some 0s being changed to 1s
or vice versa. One of the ways to deal with such errors is to
introduce \textbf{redundancy}%
\index{redundancy} in the way the data is encoded.

A very simple example of redundant encoding is the so-called
\textbf{3-repetition code}%
\index{error correcting code!3-repetition code}%
\index{repetition code}%
\index{3-repetition code}. This simply means to repeat each bit 3
times. Thus, the bit $0$ is encoded as $000$ and the bit $1$ as
$111$. For example, the bit string $100101$ is encoded as
$111000000111000111$. With this encoding, single bit errors are easy
to detect and correct. Assuming that at most one error occurs within
each 3-bit block, the blocks can be decoded by ``majority
decision''. Namely, if the bits in a block are not the same, we assume
that the error occurred in the bit that is in the minority. The
following table shows the decoding scheme:
\begin{center}
  \begin{tabular}{|l|l|l|l|}
    \hline
    Received & Likely error & Corrected & Decoded \\\hline
    $000$    & none         & $000$     & $0$     \\
    $100$    & bit 1        & $000$     & $0$     \\
    $010$    & bit 2        & $000$     & $0$     \\
    $001$    & bit 3        & $000$     & $0$     \\
    $011$    & bit 1        & $111$     & $1$     \\
    $101$    & bit 2        & $111$     & $1$     \\
    $110$    & bit 3        & $111$     & $1$     \\
    $111$    & none         & $111$     & $1$     \\\hline
  \end{tabular}
\end{center}

\begin{example}{Decoding the 3-repetition code}{decode-repetition}
  A message was encoded with the 3-repetition code. The receiver
  receives the following data: $001011000100111110$. What was the
  likely message?
\end{example}

\begin{solution}
  We start by dividing the received data into blocks of 3 bits:
  \begin{equation*}
    001~011~000~100~111~110.
  \end{equation*}
  We then decode each block separately, using the majority rule.
  The first block $001$ has a majority of zeros, so the likely error is
  in the third bit and the likely decoding is $0$. The second block $011$
  has a majority of ones, so the likely error is in the first bit and
  the likely decoding is $1$. Continuing this way, we find that the
  likely original message was $01~00~11$.
\end{solution}

Since the 3-repetition code can be used to correct some errors, it is
called an \textbf{error correcting code}%
\index{error correcting code}%
\index{code!error correcting|see{error correcting code}}. This
particular code only works if there are not too many errors; namely,
it can correct at most one error per code block. The main drawback of
the 3-repetition code is that it increases the message length by a
factor of 3. In this section, we will use linear algebra over the
field $\Z_2$ to construct better error correcting codes.

Most error correcting codes do not work by encoding individual
bits. Rather, the codes work by dividing the message into
\textit{message blocks} of length $k$, and encoding each such message
block by a \textit{code block} of length $n$.  This leads us to the
following definition:

\begin{definition}{Code}{code}
  Let $n$ and $k$ be positive integers with $k\leq n$. A
  \textbf{code}%
  \index{code} with \textbf{message length}%
  \index{message length}%
  \index{error correcting code!message length} $k$ and \textbf{block
    length}%
  \index{block length}%
  \index{error correcting code!block length} $n$ is a is a set $C$ of
  $2^k$ different vectors in $\Z_2^n$. We call the elements of $C$ the
  \textbf{code blocks}%
  \index{code block}%
  \index{error correcting code!code block}%
  \index{block!code block}. Since there are $2^k$ different code
  blocks, they can be used to encode \textbf{message blocks}%
  \index{message block}%
  \index{error correcting code!message block}%
  \index{block!message block} of length $k$. We say that the
  \textbf{rate}%
  \index{rate of a code}%
  \index{error correcting code!rate} of the code is $\frac{k}{n}$,
  because for every $n$ bits of encoded data transmitted, $k$
  bits of decoded data are obtained.
\end{definition}

\begin{example}{A simple code}{code-simple}
  Consider the following code with message length $k=2$ and block
  length $n=5$:
  \begin{center}
    \begin{tabular}{|l|l|}
      \hline
      Message block & Code block \\\hline
      $00$ & $00000$ \\
      $01$ & $00111$ \\
      $10$ & $11100$ \\
      $11$ & $11011$ \\\hline
    \end{tabular}
  \end{center}
  \begin{enumialphparenastyle}
    \begin{enumerate}
    \item Encode the message\/ $011110$.
    \item Decode the message\/ $111000011111011$.
    \item What is the rate of this code? Is it better or worse than
      the rate of the 3-repetition code?
    \end{enumerate}
  \end{enumialphparenastyle}
\end{example}

\begin{solution}
  (a) We divide the message into blocks of length $2$:
  $01~11~10$. Then we encode each block separately:
  $00111~11011~11100$. (b) We divide the message into blocks of length
  $5$: $11100~00111~11011$. Then we decode each block separately:
  $10~01~11$. (c) The rate is $\frac{2}{5}=0.4$. It is slightly
  higher, and therefore better, than the rate of the 3-repetition code,
  which is $\frac{1}{3}\approx 0.33$.
\end{solution}

% ----------------------------------------------------------------------
\subsection*{Hamming distance and error correction}

The error correction capabilities of a code depend on a property
called the \textit{Hamming distance} of the code, which we now define.

\begin{definition}{Hamming weight and Hamming distance}{hamming-distance}
  \begin{itemize}
  \item Let $\vect{v}$ be a vector in $\Z_2^n$. The \textbf{Hamming
      weight}%
    \index{Hamming weight}%
    \index{error correcting code!Hamming weight} of\/ $\vect{v}$,
    denoted $\Hw(\vect{v})$, is the number of entries of\/ $\vect{v}$
    that are equal to 1.
  \item Let $\vect{v},\vect{w}$ be two vectors in $\Z_2^n$. The
    \textbf{Hamming distance}%
    \index{Hamming distance}%
    \index{error correcting code!Hamming distance} between $\vect{v}$
    and $\vect{w}$, denoted $\Hd(\vect{v},\vect{w})$, is the number of
    entries where $\vect{v}$ and $\vect{w}$ differ. We can also
    express this as the Hamming weight of\/ $\vect{v}-\vect{w}$, i.e.,
    $\Hd(\vect{v},\vect{w})=\Hw(\vect{v}-\vect{w})$.
  \item Finally, we say that the \textbf{Hamming distance of a code}
    is equal to the smallest Hamming distance between any two code
    blocks.
  \end{itemize}
  A code with block length $n$, message length $k$, and Hamming
  distance $d$ is also called an \textbf{$(n,k,d)$-code}%
  \index{nkd-code@$(n,k,d)$-code}%
  \index{error correcting code!nkd-code@$(n,k,d)$-code}.
\end{definition}

\begin{example}{Hamming distance}{hamming-distance}
  Calculate the Hamming distance between $00111$ and $11100$. What is
  the Hamming distance of 3-repetition code? What is the
  Hamming distance of the code in Example~\ref{exa:code-simple}?
\end{example}

\begin{solution}
  The vectors $00111$ and $11100$ differ in 4 places, so their Hamming
  distance is $\Hd(00111,11100)=4$. This is also equal to the Hamming
  weight of $00111-11100 = 11011$.

  The 3-repetition code has only two code blocks: $000$ and
  $111$. Since their Hamming distance is 3, the Hamming distance of
  the code is also 3.

  To calculate the Hamming distance of the code from
  Example~\ref{exa:code-simple}, we calculate the Hamming distance
  between all pairs of code blocks:
  \begin{equation*}
    \begin{array}{rcl}
      \Hd(00000,00111) &=& 3, \\
      \Hd(00000,11100) &=& 3, \\
      \Hd(00000,11011) &=& 4, \\
      \Hd(00111,11100) &=& 4, \\
      \Hd(00111,11011) &=& 3, \\
      \Hd(11100,11011) &=& 3. \\
    \end{array}
  \end{equation*}
  Since the smallest distance between any two code blocks is $3$, the
  Hamming distance of the code is $3$.
\end{solution}

The significance of a code's Hamming distance is explained by the
following definition and proposition.

\begin{definition}{Error detection and error correction}{detection-correction}
  \begin{itemize}
  \item We say that a code $C$ is \textbf{$m$-error detecting}%
    \index{error correction code!m-error detecting@$m$-error detecting}%
    \index{m-error detecting code@$m$-error detecting code} if for all
    valid code blocks $\vect{v}$, whenever\/ $\vect{w}$ is obtained
    from $\vect{v}$ by introducing up to $m$ bit errors, then
    $\vect{w}$ is not a valid code block.
  \item We say that a code $C$ is \textbf{$m$-error correcting}%
    \index{error correction code!m-error correcting@$m$-error correcting}%
    \index{m-error correcting code@$m$-error correcting code} if for
    all valid code blocks $\vect{v}$, whenever\/ $\vect{w}$ is
    obtained from $\vect{v}$ by introducing up to $m$ bit errors, then
    $\vect{v}$ is the only valid code block within Hamming distance
    $m$ of $\vect{w}$.
  \end{itemize}
\end{definition}

\begin{proposition}{Error detection and error correction}{detection-correction}
  Consider an code with Hamming distance $d$. Then the code is:
  \begin{itemize}
  \item $m$-error detecting if $m\leq d-1$;
  \item $m$-error correcting if $2m\leq d-1$.
  \end{itemize}
\end{proposition}

\begin{proof}
  Assume up to $m$ errors have happened. In other words, let
  $\vect{v}$ be a valid code block, and let $\vect{w}$ be the code
  block obtained from $\vect{v}$ by introducing up to $m$ errors.  To
  prove the first claim, assume $m\leq d-1$.  Then
  $\Hd(\vect{v},\vect{w})\leq m\leq d-1$. Since $d$ is the minimum
  distance between any two valid code blocks, $\vect{w}$ cannot be a
  valid code block. Hence, the errors can be detected.  To prove the
  second claim, assume $2m\leq d-1$. Then
  $\Hd(\vect{v},\vect{w})\leq m$. Assume that there exists another
  valid code block $\vect{u}$ within Hamming distance $m$ of
  $\vect{w}$, i.e., assume $\Hd(\vect{w},\vect{u})\leq m$. Then
  \begin{equation*}
    \Hd(\vect{v},\vect{u}) \leq \Hd(\vect{v},\vect{w}) +
    \Hd(\vect{w},\vect{u}) \leq m+m \leq d-1.
  \end{equation*}
  Therefore, the Hamming distance of $\vect{v}$ and $\vect{u}$ is at
  most $d-1$, contradicting the assumption that the code has Hamming
  distance $d$. Hence, there is no such code block $\vect{u}$, and the
  errors can be corrected.
\end{proof}

\begin{example}{Error detection and error correction}{detection-correction}
  Consider a code with Hamming distance $3$. How many errors per code
  block can the code detect? How many can it correct? Also answer this
  question for Hamming distance $2$, $4$ and $5$.
\end{example}

\begin{solution}
  By Proposition~\ref{prop:detection-correction}, a code with Hamming
  distance $3$ can detect up to $2$ errors and correct up to $1$
  error. The answers for other Hamming distances are summarized in the
  following table:
  \begin{equation*}
    \begin{array}{|c|c|c|}
      \hline
      \mbox{Hamming distance} & \mbox{Errors detected} & \mbox{Errors corrected} \\\hline
      2 & 1 & 0 \\
      3 & 2 & 1 \\
      4 & 3 & 1 \\
      5 & 4 & 2 \\
      \hline
    \end{array}
  \end{equation*}
\end{solution}

\begin{example}{Decoding}{simple-code-decoding}
  The following message has been encoded using the code of
  Example~\ref{exa:code-simple}. It contains some errors, but no more
  than 1 error per code block. Can you decode the message?
  \begin{equation*}
    11101~01000~11011~11111~00011
  \end{equation*}
\end{example}

\begin{solution}
  Since the code of Example~\ref{exa:code-simple} has Hamming distance
  $3$, it can correct up to 1 error per code block, so we are able to
  decode the message uniquely. For each code block, we must find the
  unique valid code block that is within Hamming distance $1$ or less.
  \begin{center}
    \begin{tabular}{|l|l|l|l|}
      \hline
      Received & Error & Corrected & Decoded \\\hline
      $11101$  & bit 5 & $11100$   & $10$    \\
      $01000$  & bit 2 & $00000$   & $00$    \\
      $11011$  & none  & $11011$   & $11$    \\
      $11111$  & bit 3 & $11011$   & $11$    \\
      $00011$  & bit 3 & $00111$   & $01$    \\
      \hline
    \end{tabular}
  \end{center}
  The decoded message is $10~00~11~11~01$.
\end{solution}

We can say that a ``good'' error correcting code is one that has a
high rate (of message length divided by block length) and a large
Hamming distance.

% ----------------------------------------------------------------------
\subsection*{Linear codes}

To construct a code of message length $n$, we need to specify a set of
$2^n$ code blocks. If $n$ is large, it is not really feasible to write
down a list of all the code blocks, and to check all their Hamming
distances by hand. For example, a code of block length $n=10$ requires
$2^{10}=1024$ code blocks, and we need to check more than half a
million Hamming distances. Instead, we will focus on a particular
class of codes that is much easier to describe. These are the linear
codes.

\begin{definition}{Linear code}{linear-code}
  A \textbf{linear code}%
  \index{linear code}%
  \index{error correcting code!linear code} is a subspace of
  $\Z_2^n$. If the subspace is $k$-dimensional, then the code has
  message length $k$ and block length $n$.
\end{definition}

The advantage of a linear code is that to specify the code blocks, we
only need to list $k$ basis elements, rather than all $2^k$ elements
of the code.

\begin{example}{Linear code}{linear-code}
  Show that the code from Example~\ref{exa:code-simple} is
  linear. What is a basis for the code?
\end{example}

\begin{solution}
  Consider all linear combinations of $11100$ and $00111$ (with
  scalars in $\Z_2$):
  \begin{equation*}
    \begin{array}{rcl}
      0(11100) + 0(00111) &=& 00000, \\
      0(11100) + 1(00111) &=& 00111, \\
      1(11100) + 0(00111) &=& 11100, \\
      1(11100) + 1(00111) &=& 11011. \\
    \end{array}
  \end{equation*}
  This shows that the code of Example~\ref{exa:code-simple} is
  $\sspan\set{11100, 00111}$, and hence a subspace of $\Z_2^5$. A
  basis for the code is $\set{11100, 00111}$
\end{solution}

From Section~\ref{sec:null-space}, we know that every subspace of
$\Z_2^n$, and therefore every linear code, is the column space of some
matrix $G$, and also the null space of some matrix $H$. Such matrices
are called a {\em generator matrix} and a {\em check matrix} for the
code, respectively.

\begin{definition}{Generator matrix and check matrix}{generator-check-matrix}
  Consider a linear code $C$ with block length $n$ and message length
  $k$, i.e., a $k$-dimensional subspace of $\Z_2^n$.
  \begin{itemize}
  \item A \textbf{generator matrix}%
    \index{generator matrix}%
    \index{linear code!generator matrix}%
    \index{error correcting code!generator matrix} for the code is an
    $n\times k$-matrix $G$ such that $C$ is the column space of $G$.
  \item A \textbf{check matrix}%
    \index{check matrix}%
    \index{linear code!check matrix}%
    \index{error correcting code!check matrix} for the code is an
    $(n-k)\times n$-matrix $H$ such that $C$ is the null space of $H$.
  \end{itemize}
\end{definition}

\begin{example}{Generator matrix and check matrix}{generator-check-matrix}
  Find a generator matrix and check matrix for the linear code from
  Example~\ref{exa:code-simple}.
\end{example}

\begin{solution}
  We already found in Example~\ref{exa:linear-code} that
  $\set{11100, 00111}$ is a basis for this code. We can obtain a
  generator matrix by using the basis vectors as columns:
  \begin{equation*}
    G = \begin{mymatrix}{cc}
      1 & 0 \\
      1 & 0 \\
      1 & 1 \\
      0 & 1 \\
      0 & 1 \\
    \end{mymatrix}.
  \end{equation*}
  To find a check matrix, assume that $\mat{a,b,c,d,e}$ is a row of
  the check matrix. Then for every code block $\vect{v}\in C$, we must
  have $\mat{a,b,c,d,e}\vect{v}=0$. Since the code blocks are spanned
  by $11100$ and $00111$, it suffices to consider the two equations
  \begin{equation*}
    \begin{mymatrix}{ccccc} a & b & c & d & e \end{mymatrix}
    \begin{mymatrix}{c} 1 \\ 1 \\ 1 \\ 0 \\ 0 \end{mymatrix}
    = 1a+1b+1c+0d+0e = 0
  \end{equation*}
  and
  \begin{equation*}
    \begin{mymatrix}{ccccc} a & b & c & d & e \end{mymatrix}
    \begin{mymatrix}{c} 0 \\ 0 \\ 1 \\ 1 \\ 1 \end{mymatrix}
    = 0a+0b+1c+1d+1e = 0.
  \end{equation*}
  Solving this system of equations, we find that the following is a
  basis for the solution space:
  \begin{equation*}
    \set{
      \begin{mymatrix}{ccccc} 1 & 1 & 0 & 0 & 0 \end{mymatrix},
      \begin{mymatrix}{ccccc} 1 & 0 & 1 & 1 & 0 \end{mymatrix},
      \begin{mymatrix}{ccccc} 1 & 0 & 1 & 0 & 1 \end{mymatrix}
    }
  \end{equation*}
  We can use these basic solutions as the rows of the check matrix:
  \begin{equation*}
    H = \begin{mymatrix}{ccccc}
      1 & 1 & 0 & 0 & 0 \\
      1 & 0 & 1 & 1 & 0 \\
      1 & 0 & 1 & 0 & 1 \\
    \end{mymatrix}
  \end{equation*}
  \vspace{-6ex}\par
\end{solution}

% ----------------------------------------------------------------------
\subsection*{Encoding and the generator matrix}

Let $C$ be a linear code with generator matrix $G$. Since $C$ is the
column space of the generator matrix, there is a simple encoding
method: we can simply multiply the generator matrix $G$ by a message
block to obtain the corresponding code block. In other words, if
$\vect{u}\in\Z_2^k$ is a message block, we can use
$\vect{v}=G\vect{u}\in\Z_2^n$ as the code block.

\begin{example}{Using the generator matrix for encoding}{generator-matrix-encoding}
  Use the generator matrix
  \begin{equation*}
    G = \begin{mymatrix}{cc}
      1 & 0 \\
      1 & 0 \\
      1 & 1 \\
      0 & 1 \\
      0 & 1 \\
    \end{mymatrix}
  \end{equation*}
  to encode the message\/ $01~11~10$.
\end{example}

\begin{solution}
  The message blocks are
  $\vect{u}_1 = \begin{mysmallmatrix}{c} 0 \\ 1 \end{mysmallmatrix}$,
  $\vect{u}_2 = \begin{mysmallmatrix}{c} 1 \\ 1 \end{mysmallmatrix}$,
  and
  $\vect{u}_3 = \begin{mysmallmatrix}{c} 1 \\ 0 \end{mysmallmatrix}$.
  We obtain the corresponding code blocks by multiplication with the
  generator matrix:
  \begin{equation*}
    \vect{v}_1
    = G\vect{u}_1
    = \begin{mymatrix}{c} 0 \\ 0 \\ 1 \\ 1 \\ 1 \end{mymatrix}
    \quad
    \vect{v}_2
    = G\vect{u}_2
    = \begin{mymatrix}{c} 1 \\ 1 \\ 0 \\ 1 \\ 1 \end{mymatrix}
    \quad
    \vect{v}_3
    = G\vect{u}_3
    = \begin{mymatrix}{c} 1 \\ 1 \\ 1 \\ 0 \\ 0 \end{mymatrix}
  \end{equation*}
  Therefore the encoded message is $00111~11011~11100$. Note that
  this is the same answer as in Example~\ref{exa:code-simple}(a).
\end{solution}

% ----------------------------------------------------------------------
\subsection*{Decoding and the check matrix}

Let $C$ be a linear code with check matrix $H$.  By definition, $C$ is
the null space of the check matrix. This means that a vector
$\vect{v}\in\Z_2^n$ is a valid code block (i.e., a code block without
errors) if and only if $H\vect{v}=\vect{0}$. So we can easily use the
check matrix to determine whether a code block contains errors or not.

However, something even better is true. The value of $H\vect{v}$, when
it is not zero, tell us not only {\em that} an error has occurred, but
also {\em which} error has occurred! To see why, consider a code block
$\vect{v}'$ possibly containing errors. Then
$\vect{v}' = \vect{v} + \vect{e}$, where $\vect{v}$ is a valid code
block and $\vect{e}$ is an \textbf{error pattern}%
\index{error pattern}%
\index{error correcting code!error pattern}%
\index{linear code!error pattern}. The error pattern is the vector
$\vect{e}$ that has a $1$ in every component in which an error
occurred, and a $0$ everywhere else.  Then we have
\begin{equation*}
  H\vect{v}' = H(\vect{v} + \vect{e}) = H\vect{v} + H\vect{e} =
  \vect{0} + H\vect{e} = H\vect{e}.
\end{equation*}
Therefore, the value of $H\vect{v}'$ only depends on the error
pattern, and not on $\vect{v}$. The value
$\vect{s} = H\vect{v'} = H\vect{e}$ is called the \textbf{syndrome}%
\index{syndrome}%
\index{error correcting code!syndrome}%
\index{linear code!syndrome} of the error. A code is error correcting,
for a class of error patterns, if and only if each such error pattern
has a different syndrome. In that case, we can simply make a table of
all the syndromes and corresponding error patterns, as an efficient
method for correcting errors. Such a table is called a
\textbf{syndrome table}%
\index{syndrome table}%
\index{error correcting code!syndrome table}%
\index{linear code!syndrome table} and the corresponding decoding
method is called \textbf{syndrome decoding}%
\index{syndrome decoding}%
\index{error correcting code!syndrome decoding}%
\index{linear code!syndrome decoding}.

\begin{example}{Syndrome decoding}{syndrome-decoding}
  Consider the linear code of Example~\ref{exa:simple-code} with check
  matrix
  \begin{equation*}
    H = \begin{mymatrix}{ccccc}
      1 & 1 & 0 & 0 & 0 \\
      1 & 0 & 1 & 1 & 0 \\
      1 & 0 & 1 & 0 & 1 \\
    \end{mymatrix}.
  \end{equation*}
  Make a syndrome table for all single-bit errors.  Then use your
  syndrome table to decode the message
  $11101~01000~11011~11111~00011$.
\end{example}

\begin{solution}
  Since we are only interested in single-bit errors, there are six
  error patterns to consider: $00000$ (no error), $10000$, $01000$,
  $00100$, $00010$, and $00001$. For each of these error patterns
  $\vect{e}$, we compute the corresponding syndrome $H\vect{e}$. For
  example,
  \begin{equation*}
    H \begin{mymatrix}{c} 0 \\ 0 \\ 0 \\ 0 \\ 0 \end{mymatrix}
    = \begin{mymatrix}{c} 0 \\ 0 \\ 0 \end{mymatrix},\quad
    H \begin{mymatrix}{c} 1 \\ 0 \\ 0 \\ 0 \\ 0 \end{mymatrix}
    = \begin{mymatrix}{c} 1 \\ 1 \\ 1 \end{mymatrix},\quad
    H \begin{mymatrix}{c} 0 \\ 1 \\ 0 \\ 0 \\ 0 \end{mymatrix}
    = \begin{mymatrix}{c} 1 \\ 0 \\ 0 \end{mymatrix},
  \end{equation*}
  and so on. We put this information in a table (writing the vectors
  horizontally as usual in this section). This is the syndrome table:
  \begin{equation*}
    \begin{array}{|c|c|}
      \hline
      \mbox{Error pattern $\vect{e}$} & \mbox{Syndrome $H\vect{e}$} \\\hline
      00000 & 000 \\
      10000 & 111 \\
      01000 & 100 \\
      00100 & 011 \\
      00010 & 010 \\
      00001 & 001 \\
      \hline
    \end{array}
  \end{equation*}
  To decode the message $11101~01000~11011~11111~00011$, we first
  calculate the syndrome of each code block by multiplying it by the
  check matrix. The syndrome table then tells us the corresponding
  error pattern, which we can use to correct the code block.

  For example, consider the first code block $11101$. Multiplying by
  $H$, we get the syndrome $001$:
  \begin{equation*}
    H \begin{mymatrix}{c} 1 \\ 1 \\ 1 \\ 0 \\ 1 \end{mymatrix}
    = \begin{mymatrix}{c} 0 \\ 0 \\ 1 \end{mymatrix}
  \end{equation*}
  The syndrome table then tells us that the corresponding error pattern
  is $00001$. Therefore, the corrected code block is $11101 + 00001 =
  11100$. The decoded block is $10$. We proceed in the same way for
  all the code blocks:
  \begin{equation*}
    \begin{array}{|c|c|c|c|c|}
      \hline
      \mbox{Received code block $\vect{v}'$} &
      \mbox{Syndrome $H\vect{v}'$} &
      \mbox{Error pattern $\vect{e}$} &
      \mbox{Corrected code block $\vect{v}=\vect{v}'+\vect{e}$} &
      \mbox{Decoded} \\
      \hline
      11101 & 001 & 00001 & 11100 & 10 \\
      01000 & 100 & 01000 & 00000 & 00 \\
      11011 & 000 & 00000 & 11011 & 11 \\
      11111 & 011 & 00100 & 11011 & 11 \\
      00011 & 011 & 00100 & 00111 & 01 \\
      \hline
    \end{array}
  \end{equation*}
  Therefore, the decoded message is $10~00~11~11~01$. Note that this
  is the same answer we got in Example~\ref{exa:simple-code-decoding}.
  But the syndrome table gives a more systematic method of finding the
  error patterns, which we previously had to do by guessing, or by
  comparing to all possible code blocks.
\end{solution}

\begin{example}{An invalid code block}{invalid-code-block}
  Using the code of the previous example, suppose you have received
  the code block $10010$. Can you decode it?  
\end{example}

\begin{solution}
  The syndrome of this code block is
  \begin{equation*}
    H \begin{mymatrix}{c} 1 \\ 0 \\ 0 \\ 1 \\ 0 \end{mymatrix}
    = \begin{mymatrix}{c} 1 \\ 0 \\ 1 \end{mymatrix}.
  \end{equation*}
  Since the syndrome $101$ is not in our syndrome table, it is not the
  syndrome of any single-bit error. Therefore, the code block $10010$
  must contain more than one error. Since our code is only 1-error
  correcting, this error cannot be corrected. (In fact, there are two
  valid code words within Hamming distance $2$, namely $00000$ and
  $11011$).
\end{solution}

% ----------------------------------------------------------------------
\subsection*{Hamming codes}

The syndrome decoding method immediately gives us the following
theorem:

\begin{theorem}{Check matrices of 1-error correcting codes}{1-error-check}
  Consider a linear code with check matrix $H$. Then the code is
  1-error correcting if and only if all columns of the check matrix
  are non-zero and distinct. 
\end{theorem}

\begin{proof}
  The syndrome of the zero error pattern is always
  $H\vect{0}=\vect{0}$. Let $\vect{e}_i$ be the error pattern
  containing a single bit error in the $i\th$ component. Then its
  syndrome is $H\vect{e}_i$, which is the $i\th$ column of $H$.

  If $H$ has distinct, non-zero columns, then each possible single bit
  error has a different syndrome. Therefore, the errors can be
  corrected.

  Conversely, if $H$ has a column that is zero, or two columns that
  are equal, then the corresponding error patterns have the same
  syndrome, and therefore cannot be corrected.
\end{proof}

This theorem was discovered by Richard Hamming%
\index{Richard Hamming}%
\index{Hamming, Richard} in 1950. Hamming then realized that the best
single-error correcting binary codes can be constructed by letting the
check matrix have {\em all} possible non-zero columns. The resulting
codes are called Hamming codes.

\begin{definition}{Hamming code}{hamming-code}
  Let $r\geq 2$ The $r\th$ \textbf{Hamming code}%
  \index{Hamming code}%
  \index{error correcting code!Hamming code}%
  \index{linear code!Hamming code} is a linear code whose check matrix
  $H$ is an $r\times (2^r-1)$-matrix that has all possible non-zero
  column vectors as its columns.
\end{definition}

From the size of the check matrix, we know that the code length is
$n=2^r-1$. Since the check matrix has rank $r$, its null space has
dimension $k=n-r$. Thus, the $r\th$ Hamming code is an $(n,k,3)$-code,
where $n=2^r-1$ and $k=2^r-1-r$.

It is customary to order the columns of the check matrix so that the
last $r$ columns are the standard basis vectors. In other words, the
check matrix is usually taken to be of the form
\begin{equation*}
  H=\begin{mymatrix}{c|c}A&I\end{mymatrix}
\end{equation*}
where $I$ is the $r\times r$-identity matrix and $A$ is the
$r\times k$-matrix consisting of the remaining columns of $H$. In that
case, we can take the $n\times k$-matrix
\begin{equation*}
  G = \begin{mymatrix}{c} I \\\hline A \end{mymatrix}
\end{equation*}
as the generator matrix, where $I$ is the $k\times k$-identity
matrix. Specifically, for this choice of $G$, we have $HG=0$, ensuring
that the column space of $G$ is contained in the null space of
$H$. Moreover, since $\rank G = k$, the column space of $G$ is
$k$-dimensional, and therefore equal to the null space of $H$, so that
$G$ is indeed a correct generator matrix for the code.

\begin{example}{Hamming code for $r=3$}{hamming-code-3}
  Construct a check matrix and generator matrix for the Hamming code
  with $r=3$. Then encode the message $0101~1110~0111$.
\end{example}

\begin{solution}
  The check matrix must be a $3\times 7$-matrix whose columns are all
  the possible non-zero vectors of length 3. Moreover, we will follow
  the convention of using the standard basis vectors as the last 3
  columns. Such a matrix is
  \begin{equation*}
    H = \begin{mymatrix}{cccc|ccc}
      1 & 1 & 0 & 1 & 1 & 0 & 0 \\
      1 & 0 & 1 & 1 & 0 & 1 & 0 \\
      0 & 1 & 1 & 1 & 0 & 0 & 1 \\
    \end{mymatrix}.
  \end{equation*}
  The corresponding generator matrix $G$ is
  \begin{equation*}
    G = \begin{mymatrix}{cccc}
      1 & 0 & 0 & 0 \\
      0 & 1 & 0 & 0 \\
      0 & 0 & 1 & 0 \\
      0 & 0 & 0 & 1 \\\hline
      1 & 1 & 0 & 1 \\
      1 & 0 & 1 & 1 \\
      0 & 1 & 1 & 1 \\
    \end{mymatrix}.
  \end{equation*}
  To encode the message $0101~1110~0111$, we multiply the generator
  matrix by each code block:
  \begin{equation*}
    G \begin{mymatrix}{c} 0 \\ 1 \\ 0 \\ 1 \end{mymatrix}
    = \begin{mymatrix}{c} 0 \\ 1 \\ 0 \\ 1 \\ 0 \\ 1 \\ 0 \end{mymatrix},
    \quad
    G \begin{mymatrix}{c} 1 \\ 1 \\ 1 \\ 0 \end{mymatrix}
    = \begin{mymatrix}{c} 1 \\ 1 \\ 1 \\ 0 \\ 0 \\ 0 \\ 0 \end{mymatrix},
    \quad
    G \begin{mymatrix}{c} 0 \\ 1 \\ 1 \\ 1 \end{mymatrix}
    = \begin{mymatrix}{c} 0 \\ 1 \\ 1 \\ 1 \\ 0 \\ 0 \\ 1 \end{mymatrix}.
  \end{equation*}
  Thus, the encoded message is $0101010~1110000~0111001$. Note that
  the fact that the top part of the generator matrix $G$ is the
  identity matrix has the pleasant effect that the $k$ first bits of
  each code block are the corresponding message block. This makes
  decoding especially convenient (after any errors have been corrected
  first).
\end{solution}  

\begin{example}{Decoding a Hamming code}{hamming-3-decoding}
  Using the Hamming $(7,4,3)$-code of the previous example, correct
  the errors in the message $1011101~1101001~0001101~1110000$ and
  decode it.
\end{example}

\begin{solution}
  We do not need to make a syndrome table, because the syndromes are
  exactly the columns of the check matrix $H$. More precisely, the
  $i\th$ column of the check matrix is the syndrome of the error
  pattern containing a single-bit error in the $i\th$ bit. 
  We use the check matrix
  \begin{equation*}
    H = \begin{mymatrix}{cccc|ccc}
      1 & 1 & 0 & 1 & 1 & 0 & 0 \\
      1 & 0 & 1 & 1 & 0 & 1 & 0 \\
      0 & 1 & 1 & 1 & 0 & 0 & 1 \\
    \end{mymatrix}
  \end{equation*}
  to calculate the syndrome of each code block:
  \begin{equation*}
    H \begin{mymatrix}{c} 1 \\ 0 \\ 1 \\ 1 \\ 1 \\ 0 \\ 1 \end{mymatrix}
    = \begin{mymatrix}{c} 1 \\ 1 \\ 1 \end{mymatrix},
    \quad
    H \begin{mymatrix}{c} 1 \\ 1 \\ 0 \\ 1 \\ 0 \\ 0 \\ 1 \end{mymatrix}
    = \begin{mymatrix}{c} 1 \\ 0 \\ 1 \end{mymatrix},
    \quad
    H \begin{mymatrix}{c} 0 \\ 0 \\ 0 \\ 1 \\ 1 \\ 0 \\ 1 \end{mymatrix}
    = \begin{mymatrix}{c} 0 \\ 1 \\ 0 \end{mymatrix},
    \quad
    H \begin{mymatrix}{c} 1 \\ 1 \\ 1 \\ 0 \\ 0 \\ 0 \\ 0 \end{mymatrix}
    = \begin{mymatrix}{c} 0 \\ 0 \\ 0 \end{mymatrix}.
  \end{equation*}
  From these syndromes, we can read off the error locations and
  correct the errors:
  \begin{equation*}
    \begin{array}{|c|c|c|c|c|c|}
      \hline
      \mbox{Received code block} &
      \mbox{Syndrome} &
      \mbox{Error position} &
      \mbox{Error pattern} &
      \mbox{Corrected code block} &
      \mbox{Decoded} \\
      \hline
      1011101 & 111 & \mbox{bit 4} & 0001000 & 1010101 & 1010 \\
      1101001 & 101 & \mbox{bit 2} & 0100000 & 1001001 & 1001 \\
      0001101 & 010 & \mbox{bit 6} & 0000010 & 0001111 & 0001 \\
      1110000 & 000 & \mbox{none}  & 0000000 & 1110000 & 1110 \\
      \hline
    \end{array}
  \end{equation*}
  Thus, the decoded message is $1010~1001~0001~1110$.
\end{solution}

\begin{example}{Hamming code for $r=2$}{hamming-code-2}
  What is the Hamming code for $r=2$? Have you seen this code before?
\end{example}

\begin{solution}
  The check matrix and generator matrix for the Hamming code with
  $r=2$ are
  \begin{equation*}
    H = \begin{mymatrix}{c|cc}
      1 & 1 & 0 \\
      1 & 0 & 1 \\
    \end{mymatrix}
    \quad\mbox{and}\quad
    G = \begin{mymatrix}{c} 1 \\\hline 1 \\ 1 \end{mymatrix}.
  \end{equation*}
  The code has block size $n=3$ and message length $k=1$. The encoding
  function is
  \begin{equation*}
    G\begin{mymatrix}{c} 0 \end{mymatrix}
    = \begin{mymatrix}{c} 0 \\ 0 \\ 0 \end{mymatrix},\quad
    G\begin{mymatrix}{c} 1 \end{mymatrix}
    = \begin{mymatrix}{c} 1 \\ 1 \\ 1 \end{mymatrix}.
  \end{equation*}
  This is exactly the 3-repetition code.
\end{solution}

We end this section by remarking that the Hamming codes are very
efficient. The rate of the $r\th$ Hamming code is
\begin{equation*}
  \frac{k}{n}
  ~=~ \frac{2^r-1-r}{2^r-1}
  ~=~ 1 - \frac{r}{2^r-1},
\end{equation*}
which is very close to $1$ when $r$ is large. The following table
lists the block sizes, message sizes, and rates of all Hamming codes
up to $r=8$.
\begin{equation*}
  \begin{array}{|c|c|c|c|}
    \hline
    r & \mbox{Block size $n$} & \mbox{Message size $k$} & \mbox{Rate} \\\hline
    2  & 3    & 1    & 0.333 \\
    3  & 7    & 4    & 0.571 \\
    4  & 15   & 11   & 0.733 \\
    5  & 31   & 26   & 0.839 \\
    6  & 63   & 57   & 0.905 \\
    7  & 127  & 120  & 0.945 \\
    8  & 255  & 247  & 0.969 \\
    \hline
  \end{array}    
\end{equation*}
Of course, since the codes are only 1-error correcting, one cannot
increase the block size indefinitely, or else the probability of
having two or more errors in a block becomes too large. There exist
more sophisticated error correcting codes with larger Hamming
distances, which can correct many errors per code block. You might
learn more about such codes in a course on applied modern algebra.
