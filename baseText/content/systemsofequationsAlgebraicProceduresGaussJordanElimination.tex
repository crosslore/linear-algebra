\section{Gauss-Jordan elimination}

\begin{outcome}
  \begin{enumerate}
  \item[A.] Find the {\rref} of a matrix.
    
  \item[B.] Solve a system of linear equations using Gauss-Jordan elimination.
  \end{enumerate}
\end{outcome}

In the previous section, we saw how to solve a system of equations by
using Gaussian elimination and back substitution. The back
substitution step can be quite confusing and error prone, especially
when there are parameters. For example, in
Example~\ref{exa:twoparametersetofsoln}, we had to substitute $y=s$,
$z=2-t$, and $w=t$ into the equation $x=3-2y+2z-2w$, which required
another simplification step.

In this section, you will learn an alternative procedure called {\em
  Gauss-Jordan elimination} which eliminates the need for back
substitution, at the expense of doing a few additional row operations.
The key to this technique is a special kind of {\ef} called a
{\em {\rref}}.

\begin{definition}{\Rref}{rref}
  An augmented matrix is in \textbf{\rref}\eindex{\rref}\eindex{\ef!reduced} if
  
  \begin{enumerate}
  \item It is in {\ef}.
    
  \item Each leading entry is equal to $1$.
    
  \item All entries above a leading entry are zero.
  \end{enumerate}
\end{definition}

\begin{example}{{\Rref}}{rrefmatrices}
The following augmented matrices are in {\rref}. The leading entries
have been circled for emphasis. Note how all of the leading entries
are equal to $1$, and they have zeros above them.
\begin{equation*}
\begin{mymatrix}{rrrrr|r}
\circled{1} & 2 & 0 & 5 & 0 & 3 \\
0 & 0 & \circled{1} & 2 & 0 & 0 \\
0 & 0 & 0 & 0 & \circled{1} & 1 \\
0 & 0 & 0 & 0 & 0 & 0
\end{mymatrix},\quad\begin{mymatrix}{rrr|r}
\circled{1} & 0 & 0 & 0 \\
0 & \circled{1} & 0 & 0 \\
0 & 0 & \circled{1} & 0 \\
0 & 0 & 0 & \circled{1} \\
0 & 0 & 0 & 0
\end{mymatrix} , \begin{mymatrix}{rrr|r}
\circled{1} & 0 & 0 & 4 \\
0 & \circled{1} & 0 & 3 \\
0 & 0 & \circled{1} & 2 
\end{mymatrix}
\end{equation*}
\end{example}

We can carry every augmented matrix to {\rref} by doing elementary row
operations.

\begin{algorithm}{Gauss-Jordan elimination}{gaussjordan}
  This algorithm provides a method for using row operations to take a
  matrix to its
  {\rref} \eindex{\rref!algorithm}\index{Gauss-Jordan elimination}.
  \begin{enumerate}
  \item First, use Gaussian elimination
    (Algorithm~\ref{algo:gaussianelimination}) to reduce the matrix to
    {\ef}.
  \item Moving from right to left, consider each pivot entry. Without
    changing the row containing the pivot entry, or any rows below it,
    use row operations to create zeros in the column above the pivot
    entry. Finally, divide the row by its pivot entry, to make the
    pivot entry equal to $1$. 
  \end{enumerate}
\end{algorithm}

\begin{example}{Gauss-Jordan elimination}{gaussjordan1}
  Solve the system of equations from
  Example~\ref{exa:systemwithonesoln} using Gauss-Jordan elimination.
  \begin{equation*}
    \begin{array}{r@{~}c@{~}l}
      x+4y+3z &=& 11\\
      2x+10y+7z &=& 27\\
      x+y+2z &=& 5.
    \end{array}
  \end{equation*}
\end{example}

\begin{solution}
  In Example~\ref{exa:systemwithonesoln}, we had already reduced the
  system to {\ef}:
  \begin{equation*}
    \begin{mymatrix}{rrr|r}
      \circled{1} & 4 & 3 & 11 \\
      0 & \circled{2} & 1 & 5 \\
      0 & 0 & \circled{1} & 3
    \end{mymatrix}.
  \end{equation*}
  We reduce it to {\rref} by performing the following row operations:
  \begin{equation*}
    \begin{mymatrix}{rrr|r}
      \circled{1} & 4 & 3 & 11 \\
      0 & \circled{2} & 1 & 5 \\
      0 & 0 & \circled{1} & 3
    \end{mymatrix}
    \stackrel{R_2\leftarrow R_2-R_3}{\stackrel{R_1\leftarrow R_1-3R_3}{\sim}}
    \begin{mymatrix}{rrr|r}
      \circled{1} & 4 & 0 & 2 \\
      0 & \circled{2} & 0 & 2 \\
      0 & 0 & \circled{1} & 3
    \end{mymatrix}
    \stackrel{R_1\leftarrow 2R_2}{\sim}
    \begin{mymatrix}{rrr|r}
      \circled{1} & 0 & 0 & -2 \\
      0 & \circled{2} & 0 & 2 \\
      0 & 0 & \circled{1} & 3
    \end{mymatrix}
    \stackrel{R_2\leftarrow \frac{1}{2}R_2}{\sim}
    \begin{mymatrix}{rrr|r}
      \circled{1} & 0 & 0 & -2 \\
      0 & \circled{1} & 0 & 1 \\
      0 & 0 & \circled{1} & 3
    \end{mymatrix}.
  \end{equation*}
  The resulting matrix is in {\rref}. Note that the final system of
  equations is especially easy to solve, because the three equations
  are $x=-2$, $y=1$, and $z=3$. No back substitution is needed.
\end{solution}

\begin{example}{Gauss-Jordan elimination}{gaussjordan2}
  Solve the system of equations from
  Example~\ref{exa:twoparametersetofsoln} using Gauss-Jordan
  elimination.
  \begin{equation*}
    \begin{array}{r@{~}c@{~}l}
      x +2y  -2z  +2w &=& 3 \\
      x +2y  -z   +3w &=& 5 \\
      x +2y  -3z  +1w &=& 1.
    \end{array}
  \end{equation*}    
\end{example}

\begin{solution}
  In Example~\ref{exa:twoparametersetofsoln}, we had obtained the
  following {\ef}:
  \begin{equation*}
    \begin{mymatrix}{rrrr|r}
      \circled{1} & 2 & -2 & 2 & 3 \\
      0 & 0 & \circled{1} & 1 & 2 \\
      0 & 0 & 0 & 0 & 0
    \end{mymatrix}.
  \end{equation*}
  We reduce this to {\rref} by performing the following additional
  step:
  \begin{equation*}
    \begin{mymatrix}{rrrr|r}
      \circled{1} & 2 & -2 & 2 & 3 \\
      0 & 0 & \circled{1} & 1 & 2 \\
      0 & 0 & 0 & 0 & 0
    \end{mymatrix}
    \stackrel{R1\leftarrow R1+2R2}{\sim}
    \begin{mymatrix}{rrrr|r}
      \circled{1} & 2 & 0 & 4 & 7 \\
      0 & 0 & \circled{1} & 1 & 2 \\
      0 & 0 & 0 & 0 & 0
    \end{mymatrix}.
  \end{equation*}
  The first equation states that $x=7-2y-4w$, and the second equation
  states that $z=2-w$. Using the free variables $y$ and $w$ as
  parameters, we obtain the following general solution:
  \begin{equation*}
    \begin{array}{r@{~}c@{~}l}
      x &=& 7-2y-4w \\
      y &=& y \\
      z &=& 2-w \\
      w &=& w.
    \end{array}
  \end{equation*}
  Note that we did not really have to do back substitution; all we had
  to do is to shift parts of the equations to the right-hand side. If
  the solution looks strange, because it has equations like ``$y=y$''
  in it, keep in mind that this means that $y$ and $w$ are arbitrary
  numbers, i.e., parameters. We can replace $y$ and $w$ by parameters
  $s$ and $t$ on the right-hand side, as before:
  \begin{equation*}
    \begin{array}{r@{~}c@{~}l}
      x &=& 7-2s-4t \\
      y &=& s \\
      z &=& 2-t \\
      w &=& t.
    \end{array}
  \end{equation*}
\end{solution}

Which one is the better procedure to use, Gaussian elimination with
back substitution, or Gauss-Jordan elimination? The answer is: it
depends. In certain applications, it is not necessary to completely
solve a system of equations. Sometimes it is sufficient just to figure
out whether the system is consistent or inconsistent, or whether the
solution is unique or not. In those situations, you already get the
required information from the {\ef} and there is no need to do the
additional steps to reduce the system to {\rref}. Also, in some
situations, Gauss-Jordan elimination can introduce fractions into your
augmented matrix, making the matrix more complicated to work with. In
such cases, it may sometimes be easier to do back substitution. But in
most cases, Gauss-Jordan elimination is simpler to do than back
substitution, and therefore I recommend using the Gauss-Jordan method
most of the time. 
