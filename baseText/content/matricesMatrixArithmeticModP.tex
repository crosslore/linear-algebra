\section{Matrix arithmetic modulo \texorpdfstring{$p$}{p}}

\begin{outcome}
  \begin{enumerate}
  \item Perform matrix operations over the field $\Z_p$.
  \end{enumerate}
\end{outcome}

In Section~\ref{sec:fields}, you learned that most of linear algebra
can be done over scalars from any field $F$, and not just the real
numbers. You also learned that $\Z_p$, the set of integers modulo $p$,
is a field whenever $p$ is a prime number.

Indeed, all of the operations on matrices that we covered in this
chapter make sense over any field: addition, scalar multiplication,
matrix multiplication, inverses, elementary matrices, and the
transpose.

\begin{example}{A matrix product over $\Z_5$}{matrix-product-z5}
  Compute the matrix product $AB$ over the field $\Z_5$, where
  \begin{equation*}
    A = \begin{mymatrix}{ccc}
      1 & 0 & 4 \\
      2 & 3 & 1 \\
    \end{mymatrix}
    \quad\mbox{and}\quad
    B = \begin{mymatrix}{cc}
      3 & 1 \\
      4 & 0 \\
      2 & 2 \\
    \end{mymatrix}.
  \end{equation*}
\end{example}

\begin{solution}
  For example, the $(1,1)$-entry of $AB$ is calculated by multiplying
  the first row of $A$ by the first column of $B$, i.e.,
  \begin{equation*}
    c_{11} ~=~ 1\cdot 3+0\cdot 4 + 4\cdot 2 ~=~ 3 + 0 + 3 ~=~ 1,
  \end{equation*}
  keeping in mind that all arithmetic operations are done in $\Z_5$.
  We repeat the same for the other entries and obtain
  \begin{equation*}
    AB ~=~
    \begin{mymatrix}{ccc}
      1 & 0 & 4 \\
      2 & 3 & 1 \\
    \end{mymatrix}
    \begin{mymatrix}{cc}
      3 & 1 \\
      4 & 0 \\
      2 & 2 \\
    \end{mymatrix}
    ~=~
    \begin{mymatrix}{cc}
      1 & 4 \\
      0 & 4 \\
    \end{mymatrix}.
  \end{equation*}
\end{solution}

\begin{example}{An inverse over $\Z_7$}{matrix-inverse-z7}
  Compute the inverse of the matrix
  \begin{equation*}
    A = \begin{mymatrix}{ccc}
      1 & 4 & 2 \\
      0 & 6 & 2 \\
      5 & 0 & 3 \\
    \end{mymatrix}
  \end{equation*}
  with scalars in the field $\Z_7$.
\end{example}

\begin{solution}
  We use exactly the method of
  Algorithm~\ref{algo:matrix-inversion-algorithm}, i.e., we set up the
  augmented matrix $\mat{A\mid I}$ and reduce it to {\rref}. The only
  thing we have to keep in mind is that all operations are done modulo
  $7$. Also, as usual, instead of dividing by a scalar, we must
  multiply by its inverse.
  \begin{eqnarray*}
    \mat{A\mid I}
    &=&
        \begin{mymatrix}{ccc|ccc}
          1 & 4 & 2  &  1 & 0 & 0 \\
          0 & 6 & 2  &  0 & 1 & 0 \\
          5 & 0 & 3  &  0 & 0 & 1 \\
        \end{mymatrix}
    \\
    &\stackrel{R_3\rowop R_3+2R_1}{\sim}
      &
        \begin{mymatrix}{ccc|ccc}
          1 & 4 & 2  &  1 & 0 & 0 \\
          0 & 6 & 2  &  0 & 1 & 0 \\
          0 & 1 & 0  &  2 & 0 & 1 \\
        \end{mymatrix}
    \\
    &\stackrel{R_2\rowswap R_3}{\sim}
      &
        \begin{mymatrix}{ccc|ccc}
          1 & 4 & 2  &  1 & 0 & 0 \\
          0 & 1 & 0  &  2 & 0 & 1 \\
          0 & 6 & 2  &  0 & 1 & 0 \\
        \end{mymatrix}
    \\
    &\stackrel{R_3\rowop R_3+R_2}{\sim}
      &
        \begin{mymatrix}{ccc|ccc}
          1 & 4 & 2  &  1 & 0 & 0 \\
          0 & 1 & 0  &  2 & 0 & 1 \\
          0 & 0 & 2  &  2 & 1 & 1 \\
        \end{mymatrix}
    \\
    &\stackrel{R_3\rowop 2^{-1}R_3 ~=~ 4R_3}{\sim}
      &
        \begin{mymatrix}{ccc|ccc}
          1 & 4 & 2  &  1 & 0 & 0 \\
          0 & 1 & 0  &  2 & 0 & 1 \\
          0 & 0 & 1  &  1 & 4 & 4 \\
        \end{mymatrix}
    \\
    &\stackrel{R_1\rowop R_1-4R_2}{\sim}
      &
        \begin{mymatrix}{ccc|ccc}
          1 & 0 & 2  &  0 & 0 & 3 \\
          0 & 1 & 0  &  2 & 0 & 1 \\
          0 & 0 & 1  &  1 & 4 & 4 \\
        \end{mymatrix}
    \\
    &\stackrel{R_1\rowop R_1-2R_3}{\sim}
      &
        \begin{mymatrix}{ccc|ccc}
          1 & 0 & 0  &  5 & 6 & 2 \\
          0 & 1 & 0  &  2 & 0 & 1 \\
          0 & 0 & 1  &  1 & 4 & 4 \\
        \end{mymatrix}.
  \end{eqnarray*}
  Therefore, the inverse is
  \begin{equation*}
    A^{-1} =
    \begin{mymatrix}{ccc}
      5 & 6 & 2 \\
      2 & 0 & 1 \\
      1 & 4 & 4 \\
    \end{mymatrix}.
  \end{equation*}
  As usual, we can double-check that we didn't make any mistakes by
  calculating
  \begin{equation*}
    AA^{-1} ~=~
    \begin{mymatrix}{ccc}
      1 & 4 & 2 \\
      0 & 6 & 2 \\
      5 & 0 & 3 \\
    \end{mymatrix}
    \begin{mymatrix}{ccc}
      5 & 6 & 2 \\
      2 & 0 & 1 \\
      1 & 4 & 4 \\
    \end{mymatrix}
    ~=~
    \begin{mymatrix}{ccc}
      1 & 0 & 0 \\
      0 & 1 & 0 \\
      0 & 0 & 1 \\
    \end{mymatrix}
    ~=~ I.
  \end{equation*}
  So indeed, we have calculated the inverse correctly.
\end{solution}
