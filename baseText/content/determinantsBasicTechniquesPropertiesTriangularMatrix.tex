\subsection{The determinant of a triangular matrix}

There is a certain type of matrix for which finding the determinant is
a very simple procedure. 
Consider the following definition. 

\begin{definition}{Triangular matrices}{triangular-matrices}
A matrix $A$ is upper triangular
\index{matrix!upper triangular} if $a_{ij}=0$ whenever $i>j$. Thus the entries of such a
matrix below the main diagonal equal $0$, 
as shown. Here, $\ast$ refers to any nonzero number. 
\index{matrix!lower triangular}
\begin{equation*}
\begin{mymatrix}{cccc}
\ast & \ast & \cdots & \ast \\
0 & \ast & \cdots & \vdots \\
\vdots & \vdots & \ddots & \ast \\
0 & \cdots & 0 & \ast
\end{mymatrix}
\end{equation*}
A lower triangular matrix is defined similarly as a matrix for which all
entries above the
\index{main diagonal}main diagonal are equal to zero.
\end{definition}

The following theorem provides a useful way to calculate the determinant
of a triangular matrix. 

\begin{theorem}{Determinant of a triangular matrix}{determinant-of-triangular-matrix}
Let $A$ be an upper or lower triangular matrix. Then $\det \tup{A} $
is obtained by taking the product of the entries on the main diagonal.
\end{theorem}

The verification of this Theorem can be done by computing the
determinant using Laplace Expansion along the first row or column. 

Consider the following example. 

\begin{example}{Determinant of a triangular matrix}{determinant-of-triangular-matrix}
Let
\begin{equation*}
A=\begin{mymatrix}{rrrr}
1 & 2 & 3 & 77 \\
0 & 2 & 6 & 7 \\
0 & 0 & 3 & 33.7 \\
0 & 0 & 0 & -1
\end{mymatrix}
\end{equation*}
Find $\det \tup{A} .$
\end{example}

\begin{solution} From Theorem \ref{thm:determinant-of-triangular-matrix}, it suffices to take the product of the elements on 
the main diagonal. Thus $\det \tup{A} =1\times 2\times 3\times \tup{
-1} =-6.$ 

Without using Theorem \ref{thm:determinant-of-triangular-matrix}, you could use Laplace Expansion. 
We will expand along the
first column. This gives
\begin{eqnarray*}
\det \tup{A} = 
&&1\begin{absmatrix}{rrr}
2 & 6 & 7 \\
0 & 3 & 33.7 \\
0 & 0 & -1
\end{absmatrix} +0\tup{-1} ^{2+1}\begin{absmatrix}{rrr}
2 & 3 & 77 \\
0 & 3 & 33.7 \\
0 & 0 & -1
\end{absmatrix} + \\
&&0\tup{-1} ^{3+1}\begin{absmatrix}{rrr}
2 & 3 & 77 \\
2 & 6 & 7 \\
0 & 0 & -1
\end{absmatrix} +0\tup{-1} ^{4+1}\begin{absmatrix}{rrr}
2 & 3 & 77 \\
2 & 6 & 7 \\
0 & 3 & 33.7
\end{absmatrix}
\end{eqnarray*}
and the only nonzero term in the expansion is
\begin{equation*}
1\begin{absmatrix}{rrr}
2 & 6 & 7 \\
0 & 3 & 33.7 \\
0 & 0 & -1
\end{absmatrix} 
\end{equation*}
Now find the determinant of this $3 \times 3$ matrix, by expanding along the first column to obtain
\begin{equation*}
\det \tup{A} 
=
1\times \tup{2\times \begin{absmatrix}{rr}
3 & 33.7 \\
0 & -1
\end{absmatrix} +0\tup{-1} ^{2+1}\begin{absmatrix}{rr}
6 & 7 \\
0 & -1
\end{absmatrix} +0\tup{-1} ^{3+1}\begin{absmatrix}{rr}
6 & 7 \\
3 & 33.7
\end{absmatrix} }
\end{equation*}
\begin{equation*}
=1\times 2\times \begin{absmatrix}{rr}
3 & 33.7 \\
0 & -1
\end{absmatrix}
\end{equation*}
Next use Definition \ref{def:two-by-two-determinant} to find the determinant of this $2 \times 2$ matrix, which is
just $3 \times -1  - 0 \times 33.7 = -3$.
Putting all these steps together, we have 
\begin{equation*}
\det \tup{A}
=
1\times 2\times 3\times \tup{-1} =-6
\end{equation*}
which is just the product of the entries down the main diagonal of the
original matrix!
\end{solution}

You can see that while both methods result in the same answer, Theorem \ref{thm:determinant-of-triangular-matrix} provides
a much quicker method. 

In the next section, we explore some important properties of determinants.