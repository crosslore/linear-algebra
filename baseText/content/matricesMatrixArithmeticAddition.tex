\subsection{Addition of matrices}

When adding matrices, all matrices in the sum need have the same size.
For example,
\begin{equation*}
\leftB
\begin{array}{rr}
1 & 2 \\
3 & 4 \\
5 & 2
\end{array}
\rightB 
\end{equation*}
and
\begin{equation*}
\leftB
\begin{array}{rrr}
-1 & 4 & 8\\
2 & 8 & 5
\end{array}
\rightB 
\end{equation*}
cannot be added, as one has size $3 \times 2$ while the other has size $2 \times 3$.

However, the addition
\begin{equation*}
\leftB
\begin{array}{rrr}
4 & 6 & 3\\
5 & 0 & 4\\
11 & -2 & 3
\end{array}
\rightB 
+
\leftB
\begin{array}{rrr}
0 & 5 & 0 \\
4 & -4 & 14 \\
1 & 2 & 6
\end{array}
\rightB
\end{equation*}
is possible.

The formal definition is as follows.

\begin{definition}{Addition of matrices}{additionofmatrices}
Let $A=\leftB a_{ij}\rightB $ and $B=\leftB b_{ij}\rightB $ be two
$m\times n$ matrices. Then $A+B=C$\index{matrix!addition} where $C$ is the $m \times n$
matrix $C=\leftB c_{ij}\rightB$ defined by
\begin{equation*}
c_{ij}=a_{ij}+b_{ij}
\end{equation*}

\end{definition}

This definition tells us that when adding matrices, we simply add corresponding entries of the matrices. 
This is demonstrated in the next example. 

\begin{example}{Addition of matrices of same size}{samesizematrixaddition}
Add the following matrices, if possible.
\begin{equation*}
A = \leftB
\begin{array}{ccc}
1 & 2 & 3 \\
1 & 0 & 4
\end{array}
\rightB,
B = \leftB
\begin{array}{rrr}
5 & 2 & 3 \\
-6 & 2 & 1
\end{array}
\rightB
\end{equation*}
\end{example}

\begin{solution}
Notice that both $A$ and $B$ are of size $2 \times 3$. 
Since $A$ and $B$ are of the same size, the addition is possible. Using Definition \ref{def:additionofmatrices}, 
the addition is done as follows. 
\begin{equation*}
A + B = \leftB
\begin{array}{rrr}
1 & 2 & 3 \\
1 & 0 & 4
\end{array}
\rightB
+
\leftB
\begin{array}{rrr}
5 & 2 & 3 \\
-6 & 2 & 1
\end{array}
\rightB
=\allowbreak 
\leftB
\begin{array}{rrr}
1+5 & 2+2 & 3+3 \\
1+ -6 & 0+2 & 4+1
\end{array}
\rightB
=
\leftB
\begin{array}{rrr}
6 & 4 & 6 \\
-5 & 2 & 5
\end{array}
\rightB
\end{equation*}
\end{solution}

Addition of matrices obeys very much the same properties as normal
addition with numbers. Note that when we write for example $A+B$ then
we assume that both matrices are of equal size so that the operation
is indeed possible.

\begin{proposition}{Properties of matrix addition}{propertiesofaddition}
Let $A,B$ and $C$ be matrices. Then, the following properties\index{matrix!properties of addition} hold. 

\begin{itemize}
\item Commutative Law of Addition
\begin{equation}
A+B=B+A  \label{mat1}
\end{equation}

\item Associative Law of Addition
\begin{equation}
\left( A+B\right) +C=A+\left( B+C\right) \label{mat2}
\end{equation}

\item Existence of an Additive Identity
\begin{equation}
\begin{array}{c}
\mbox{There exists a zero matrix 0 such that}\\
A+0=A  \label{mat3}
\end{array}
\end{equation}

\item Existence of an Additive Inverse
\begin{equation}
\begin{array}{c}
\mbox{There exists a matrix $-A$ such that} \\
A+\left( -A\right) =0 \label{mat4}
\end{array}
\end{equation}
\end{itemize}
\end{proposition}

\begin{proof}
Consider the Commutative Law of Addition given in \ref{mat1}. Let $A,B,C,$ and $D$ be matrices such that $A+B=C$ and 
$B+A=D.$ We want to show that $D=C$. To do so, we will use the definition of matrix addition given in Definition \ref{def:additionofmatrices}.
Now,
\begin{equation*}
c_{ij}=a_{ij}+b_{ij}=b_{ij}+a_{ij}=d_{ij}
\end{equation*}
Therefore, $C=D$ because the $ij^{th}$ entries are the same for all $i$ and $j$. Note that the
conclusion follows from the commutative law of addition of numbers, which says that if $a$ and $b$ are two numbers,
then $a+b = b+a$. 
The proof of the other results are similar, and are left as an exercise.
\end{proof}

We call the zero matrix in \ref{mat3} the \textbf{additive identity}. Similarly, we call the matrix $-A$
in \ref{mat4} the \textbf{additive inverse}. $-A$ is 
defined to equal $\left( -1\right) A = [-a_{ij}].$ In other words, every entry of $A$ is multiplied by $-1$.
In the next section we will study scalar multiplication in more depth 
to understand what is meant by  $\left( -1\right) A.$
