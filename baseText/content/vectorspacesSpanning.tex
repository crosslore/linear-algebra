\section{Spanning Sets}

\begin{outcome}
\begin{enumerate}
\item[A.] Determine if a vector is within a given span.
\end{enumerate}
\end{outcome}

In this section we will examine the concept of spanning introduced earlier in terms of $\mathbb{R}^n$. Here, we will discuss these concepts in terms of abstract vector spaces. 

Consider the following definition. 

\begin{definition}{Subset}{subset}
Let $X$ and $Y$ be two sets. If all elements of $X$ are also elements of $Y$ then we say that $X$ is a \textbf{subset}\index{subset} of $Y$ and we write
\[
X \subseteq Y
\]
\end{definition}

In particular, we often speak of subsets of a vector space, such as $X \subseteq V$. By this we mean that every element in the set $X$ is contained in the vector space $V$. 

\begin{definition}{Linear Combination}{linearcombination}
Let $V$ be a vector space and let $\vect{v}_1, \vect{v}_2, \cdots, \vect{v}_n \subseteq V$. A vector $\vect{v} \in V$ is called a linear combination of the $\vect{v}_i$ if there exist scalars $c_i \in \mathbb{R}$ such that 
\[
\vect{v} = c_1 \vect{v}_1 + c_2 \vect{v}_2 + \cdots + c_n \vect{v}_n
\]
\end{definition}

This definition leads to our next concept of span.

\begin{definition}{Span of Vectors}{span}
Let $\{\vect{v}_{1},\cdots ,\vect{v}_{n}\} \subseteq V$. Then\index{span}
\begin{equation*}
\func{span}\left\{ \vect{v}_{1},\cdots ,\vect{v}_{n}\right\} = 
\left\{ \sum_{i=1}^{n}c_{i}\vect{v}_{i}: c_{i}\in \mathbb{R}
\right\} 
\end{equation*}
\end{definition}

When we say that a vector $\vect{w}$ is in $\func{span}\left\{ \vect{v}_{1},\cdots ,\vect{v}_{n}\right\}$ we mean that $\vect{w}$ can be written as a linear combination of the $\vect{v}_1$. We say that a collection of vectors $\{\vect{v}_{1},\cdots ,\vect{v}_{n}\}$ is a \textbf{spanning set}\index{spanning set} for $V$ if $V = \func{span} \{\vect{v}_{1},\cdots ,\vect{v}_{n}\}$. 

Consider the following example.

\begin{example}{Matrix Span}{matrixspan}
Let $A = \leftB \begin{array}{rr}
1 & 0 \\
0 & 2 
\end{array}\rightB$, $B = \leftB \begin{array}{rr}
0 & 1 \\
1 & 0 
\end{array}\rightB$. 
Determine if $A$ and $B$ are in 
\[
\func{span}\left\{ M_1, M_2 \right\} = \func{span} \left\{ \leftB \begin{array}{rr}
1 & 0 \\
0 & 0 
\end{array}\rightB, \leftB \begin{array}{rr}
0 & 0 \\
0 & 1 
\end{array}\rightB \right\}\] 
\end{example}

\begin{solution}

First consider $A$. We want to see if scalars $s,t$ can be found such that $A = s M_1 + t M_2$. 
\begin{equation*}
\leftB \begin{array}{rr}
1 & 0 \\
0 & 2 
\end{array}\rightB = 
s \leftB \begin{array}{rr}
1 & 0 \\
0 & 0 
\end{array}\rightB + t \leftB \begin{array}{rr}
0 & 0 \\
0 & 1 
\end{array}\rightB
\end{equation*}
The solution to this equation is given by 
\begin{eqnarray*}
1 &=& s \\
2 &=& t
\end{eqnarray*}
and it follows that $A$ is in $\func{span} \left\{ M_1, M_2 \right\}$. 

Now consider $B$. Again we write $B = sM_1 + t M_2$ and see if a solution can be found for $s, t$. 
\begin{equation*}
\leftB \begin{array}{rr}
0 & 1 \\
1 & 0 
\end{array}\rightB = 
s \leftB \begin{array}{rr}
1 & 0 \\
0 & 0 
\end{array}\rightB + t \leftB \begin{array}{rr}
0 & 0 \\
0 & 1 
\end{array}\rightB
\end{equation*}
Clearly no values of $s$ and $t$ can be found such that this equation holds. Therefore $B$ is not in $\func{span} \left\{ M_1, M_2 \right\}$.
\end{solution}

Consider another example. 

\begin{example}{Polynomial Span}{polyspan}
Show that $p(x) = 7x^2 + 4x - 3$ is in $\func{span}\left\{ 4x^2 + x, x^2 -2x + 3 \right\}$. 
\end{example}

\begin{solution}
To show that $p(x)$ is in the given span, we need to show that it can be written as a linear combination of polynomials in the span. Suppose scalars $a, b$ existed such that 
\[
7x^2 +4x - 3= a(4x^2+x) + b (x^2-2x+3) 
\]
If this linear combination were to hold, the following would be true:
\begin{eqnarray*}
4a + b &=& 7 \\
a - 2b &=& 4 \\
3b &=& -3 
\end{eqnarray*}

You can verify that $a = 2, b = -1$ satisfies this system of equations. This means that we can write $p(x)$ as follows:
\[
 7x^2 +4x-3= 2(4x^2+x)  - (x^2-2x+3) 
\]

Hence $p(x)$ is in the given span.
\end{solution}

Consider the following example.

\begin{example}{Spanning Set}{spanningset}
Let $S = \left\{ x^2 + 1, x-2, 2x^2 - x \right\}$. Show that $S$ is a spanning set for $\mathbb{P}_2$, the set of all polynomials of degree at most $2$. 
\end{example}

\begin{solution}
Let $p(x)= ax^2 + bx + c$ be an arbitrary polynomial in $\mathbb{P}_2$. To show that $S$ is a spanning set, it suffices to show that $p(x)$ can be written as a linear combination of the elements of $S$. In other words, can we find $r,s,t$ such that:
\[
p(x) =  ax^2 +bx + c = r(x^2 + 1) + s(x -2) + t(2x^2 - x)
\]

If a solution $r,s,t$ can be found, then this shows that for any such polynomial $p(x)$, it can be written as a linear combination of the above polynomials and $S$ is a spanning set. 

\begin{eqnarray*}
ax^2 +bx + c &=& r(x^2 + 1) + s(x -2) + t(2x^2 - x) \\
&=& rx^2 + r + sx - 2s + 2tx^2 - tx \\
&=& (r+2t)x^2 + (s-t)x +  (r-2s) 
\end{eqnarray*}

For this to be true, the following must hold:
\begin{eqnarray*}
a &=& r+2t \\
b &=& s-t \\
c &=& r-2s
\end{eqnarray*}

To check that a solution exists, set up the augmented matrix and row reduce:
\[
\leftB \begin{array}{rrr|r}
1 & 0 & 2 & a \\
0 & 1 & -1 & b \\
1 & -2 & 0 & c 
\end{array} \rightB \rightarrow \cdots \rightarrow 
\leftB \begin{array}{rrr|c} 
1 & 0 & 0 & \frac{1}{2} a + 2b + \frac{1}{2}c\\
0 & 1 & 0 & \frac{1}{4}a - \frac{1}{4}c \\
0 & 0 & 1 & \frac{1}{4}a - b - \frac{1}{4}c 
\end{array} \rightB
\]

Clearly a solution exists for any choice of $a,b,c$. Hence $S$ is a spanning set for $\mathbb{P}_2$. 
\end{solution}
