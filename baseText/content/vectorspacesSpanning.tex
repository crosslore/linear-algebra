\section{Linear combinations, span, and linear independence}

\begin{outcome}
  \begin{enumerate}
  \item Determine if a vector is within a given span.
  \end{enumerate}
\end{outcome}

We can now revisit many of the concepts first introduced in
Chapter~\ref{cha:vectors-rn} in the context of general vector spaces.
We will look at linear combinations, span, and linear independence in
this section, and at subspaces, bases, and dimension in the next
section.

\begin{definition}{Linear combination}{linear-combination}
  Let $V$ be a vector space over a field $K$. Let
  $\vect{u}_1,\ldots,\vect{u}_n\in V$. A vector
  $\vect{v}\in V$ is called a \textbf{linear combination}%
  \index{linear combination!in a vector space}%
  \index{linear combination!of vectors} of
  $\vect{u}_1,\ldots,\vect{u}_n$ if there exist scalars
  $a_{1},\ldots,a_{n}\in K$ such that
  \begin{equation*}
    \vect{v} = a_1 \vect{u}_1 + \ldots + a_n \vect{u}_n.
  \end{equation*}
\end{definition}

\begin{example}{Linear combination of matrices}{linear-combination-matrix}
  Write the matrix $A=\begin{mymatrix}{rr} 1 & 3 \\ -1 & 2 \end{mymatrix}$
  as a linear combination%
  \index{linear combination!of matrices} of
  \begin{equation*}
    \begin{mymatrix}{rr} 1 & 0 \\ 0 & 1 \end{mymatrix},\quad
    \begin{mymatrix}{rr} 1 & 0 \\ 0 & -1 \end{mymatrix},\quad
    \begin{mymatrix}{rr} 0 & 1 \\ 1 & 0 \end{mymatrix},\quad\mbox{and}\quad
    \begin{mymatrix}{rr} 0 & -1 \\ 1 & 0 \end{mymatrix}.
  \end{equation*}
\end{example}

\begin{solution}
  We must find coefficients $a,b,c,d$ such that
  \begin{equation*}
    \begin{mymatrix}{rr} 1 & 3 \\ -1 & 2 \end{mymatrix}
    ~=~ a \begin{mymatrix}{rr} 1 & 0 \\ 0 & 1 \end{mymatrix}
    + b \begin{mymatrix}{rr} 1 & 0 \\ 0 & -1 \end{mymatrix}
    + c \begin{mymatrix}{rr} 0 & 1 \\ 1 & 0 \end{mymatrix}
    + d \begin{mymatrix}{rr} 0 & -1 \\ 1 & 0 \end{mymatrix},
  \end{equation*}
  or equivalently,
  \begin{equation*}
    \begin{mymatrix}{rr} 1 & 3 \\ -1 & 2 \end{mymatrix}
    ~=~ \begin{mymatrix}{cc} a+b & c-d \\ c+d & a-b \end{mymatrix}.
  \end{equation*}
  This yields a system of four equations in four variables:
  \begin{equation*}
    \begin{array}{r@{~~}c@{~}r}
      a+b &=& 1, \\
      c+d &=& -1, \\
      c-d &=& 3, \\
      a-b &=& 2.
    \end{array}
  \end{equation*}
  We can easily solve the system of equations to find the unique
  solution $a=\frac{3}{2}$, $b=-\frac{1}{2}$, $c=1$, $d=-2$.
  Therefore
  \begin{equation*}
    \begin{mymatrix}{rr} 1 & 3 \\ -1 & 2 \end{mymatrix}
    ~=~ \frac{3}{2} \begin{mymatrix}{rr} 1 & 0 \\ 0 & 1 \end{mymatrix}
    - \frac{1}{2} \begin{mymatrix}{rr} 1 & 0 \\ 0 & -1 \end{mymatrix}
    + 1 \begin{mymatrix}{rr} 0 & 1 \\ 1 & 0 \end{mymatrix}
    - 2 \begin{mymatrix}{rr} 0 & -1 \\ 1 & 0 \end{mymatrix}.
  \end{equation*}
\end{solution}

\begin{example}{Linear combination of polynomials}{linear-combination-polynomials}
  Write the polynomial $p(x) = 7x^2 + 4x - 3$ as a linear combination%
  \index{linear combination!of polynomials} of
  \begin{equation*}
    q_1(x) = x^2,\quad
    q_2(x) = (x+1)^2,\quad\mbox{and}\quad
    q_3(x) = (x+2)^2.
  \end{equation*}
\end{example}

\begin{solution}
  Note that $q_2(x) = (x+1)^2 = x^2 + 2x + 1$ and
  $q_3(x) = (x+2)^2 = x^2 + 4x + 4$. We must find coefficients $a,b,c$
  such that $p(x) = aq_1(x) + bq_2(x) + cq_3(x)$, or equivalently,
  \begin{equation*}
    7x^2 + 4x - 3 ~=~ ax^2 ~+~ b(x^2 + 2x + 1) ~+~ c(x^2 + 4x + 4).
  \end{equation*}
  Collecting equal powers of $x$, we can rewrite this as
  \begin{equation*}
    7x^2 + 4x - 3 ~=~ (a+b+c)x^2 ~+~ (2b+4c)x ~+~ (b+4c).
  \end{equation*}
  Since two polynomials are equal if and only if each corresponding
  coefficient is equal, this yields a system of three equations in
  three variables
  \begin{equation*}
    \begin{array}{r@{~~}c@{~}r}
      a+b+c &=& 7, \\
      2b+4c &=& 4, \\
      b+4c &=& -3.
    \end{array}
  \end{equation*}      
  We can easily solve this system of equations and find that the
  unique solution is $a=\frac{5}{2}$, $b=7$,
  $c=-\frac{5}{2}$. Therefore
  \begin{equation*}
    p(x) ~=~ \frac{5}{2}\,q_1(x) ~+~ 7\,q_2(x) ~-~ \frac{5}{2}\,q_3(x).
  \end{equation*}
\end{solution}

As in Chapter~\ref{cha:vectors-rn}, the span of a set of vectors is
defined as the set of all of its linear combinations. We generalize
the concept of span to consider spans of arbitrary (possibly finite,
possibly infinite) sets of vectors.

\begin{definition}{Span of a set of vectors}{vector-space-span}
  Let $V$ be a vector space over some field $K$, and let $S$ be a set
  of vectors (i.e., a subset of $V$). The \textbf{span}%
  \index{span}%
  \index{vector!span} of $S$ is the set of all linear combinations of
  elements of $S$. In symbols, we have
  \begin{equation*}
    \sspan S
    ~=~ \set{a_1\vect{u}_1+\ldots+a_k\vect{u}_k \mid
      \mbox{
        $\vect{u}_1,\ldots,\vect{u}_k\in S$
        and
        $a_1,\ldots,a_k\in K$
      }}.
  \end{equation*}
\end{definition}

It is important not to misunderstand this definition.  Even when the
set $S$ is infinite, each {\em individual} element
$\vect{v}\in\sspan S$ is a linear combination of only {\em finitely
  many} elements $\vect{u}_1,\ldots,\vect{u}_k$ of $S$.
The definition does not talk about infinite linear combinations
\begin{equation*}
  a_1\vect{u}_1 + a_2\vect{u}_2 + a_3\vect{u}_3 + \ldots
\end{equation*}
Indeed, such infinite sums do not typically exist.  However, different
elements $\vect{v},\vect{w}\in\sspan S$ can be linear combinations of
a different (finite) number of vectors of $S$. For example, it is
possible that $\vect{v}$ is a linear combination of 10 elements of
$S$, and $\vect{w}$ is a linear combination of 100 elements of $S$.

\begin{example}{Spans of sequences}{spans-sequences}
  Consider the vector space $\Seq_K$ of infinite sequences. For every
  $k\in\N$, let $e^k$ be the sequence that has a $1$ in the $k\th$
  component, and is $0$ everywhere else, i.e., 
  \begin{equation*}
    \begin{array}{l}
      e^0 = 1,0,0,0,0,\ldots, \\
      e^1 = 0,1,0,0,0,\ldots, \\
      e^2 = 0,0,1,0,0,\ldots, \\
    \end{array}
  \end{equation*}
  and so on.
  Let $S=\set{e^k \mid k\in\N}$. Which of the following sequences are in
  $\sspan S$?
  \begin{enumialphparenastyle}
    \begin{enumerate}
    \item $f = 1,1,1,0,0,0,0,0,\ldots$ (followed by infinitely many zeros),
    \item $g = 1,2,0,5,0,0,0,0,\ldots$ (followed by infinitely many zeros),
    \item $h = 1,1,1,1,1,1,1,1,\ldots$ (followed by infinitely many ones),
    \item $k = 1,0,1,0,1,0,1,0,\ldots$ (forever alternating between $1$ and $0$).
    \end{enumerate}
  \end{enumialphparenastyle}
\end{example}

\begin{solution}
  \begin{enumialphparenastyle}
    \begin{enumerate}
    \item We have $f\in\sspan S$, because $f = e^0 + e^1 + e^2$.
    \item We have $g\in\sspan S$, because $g = 1e^0 + 2e^1 + 5e^3$.
    \item The sequence $h$ is not in $\sspan S$, because each element
      of $\sspan S$ is, by definition, a linear combinations of {\em
        finitely many} elements of $S$. No linear combinations of
      finitely many $e^k$ can end in infinitely many ones. Note that
      we are not permitted to write an infinite sum such as
      $e^0+e^1+e^2+\ldots$. Such infinite sums are not defined in
      vector spaces.
    \item The sequence $k$ is not in $\sspan S$, for the same reason.
      We would need to add infinitely many sequences of the form $e^k$
      to get a sequence that contains infinitely many non-zero
      elements. However, this is not permitted by the definition of
      span.
    \end{enumerate}
  \end{enumialphparenastyle}
  \vspace{-4ex}
\end{solution}

\begin{example}{Span of polynomials}{span-of-polynomials}
  Let $p(x)=7x^2+4x-3$. Is $p(x)\in\sspan\set{x^2,~ (x+1)^2,~ (x+2)^2}$?
\end{example}

\begin{solution}
  The answer is yes, because we found in
  Example~\ref{exa:linear-combination-polynomials} that
  $p(x) = \frac{5}{2}\,x^2 ~+~ 7\,(x+1)^2 ~-~ \frac{5}{2}\,(x+2)^2$.
\end{solution}

We say that a set of vectors $S$ is a \textbf{spanning set}%
\index{spanning set}%
\index{vector space!spanning set} for $V$ if $V = \sspan S$.

\begin{example}{Spanning set}{spanning-set}
  Let $S = \set{x^2,~ (x+1)^2,~ (x+2)^2 }$. Show that $S$ is a
  spanning set for $\Poly_2$, the vector space of all polynomials of
  degree at most $2$.
\end{example}

\begin{solution}
  This is analogous to Example~\ref{exa:linear-combination-polynomials}.
  Consider an arbitrary element $p(x) = p_2x^2 + p^1x + p_0$ of
  $\Poly_2$. We must show that $p(x)\in\sspan S$, i.e., that there
  exists $a,b,c\in K$ such that
  \begin{equation*}
    p(x) = ax^2 + b(x+1)^2 + c(x+2)^2.
  \end{equation*}
  We can equivalently rewrite this equation as
  \begin{equation*}
    p_2x^2 + p^1x + p_0 ~=~ (a+b+c)x^2 ~+~ (2b+4c)x ~+~ (b+4c),
  \end{equation*}
  which yields the system of equations
  \begin{equation*}
    \begin{array}{r@{~~}c@{~}r}
      a+b+c &=& p_2 \\
      2b+4c &=& p_1 \\
      b+4c &=& p_0
    \end{array}
    \quad\roweq\quad
    \begin{mymatrix}{ccc|c}
      1 & 1 & 1 & p_2 \\
      0 & 2 & 4 & p_1 \\
      0 & 1 & 4 & p_0 \\
    \end{mymatrix}
    \quad\roweq\quad
    \begin{mymatrix}{ccc|c}
      1 & 0 & 0 & p_2-\frac{3}{4}p_1+\frac{1}{2}p_0 \\
      0 & 1 & 0 & p_1-p_0 \\
      0 & 0 & 1 & \frac{1}{2}p_0-\frac{1}{4}p_1 \\
    \end{mymatrix}.
  \end{equation*}
  Since the system has rank 3, it has a solution. Therefore,
  $p(x)\in\sspan S$. Since $p(x)$ was an arbitrary element of
  $\Poly_2$, it follows that $S$ is a spanning set for $\Poly_2$. 
\end{solution}
