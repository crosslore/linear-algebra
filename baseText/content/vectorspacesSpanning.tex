\section{Linear combinations, span, and linear independence}

\begin{outcome}
  \begin{enumerate}
  \item Determine if a vector is within a given span.
  \end{enumerate}
\end{outcome}

We can now revisit many of the concepts first introduced in
Chapter~\ref{cha:vectors-rn} in the context of general vector spaces.
We will look at linear combinations, span, and linear independence in
this section, and at subspaces, bases, and dimension in the next
section.

\begin{definition}{Linear combination}{linear-combination}
  Let $V$ be a vector space over a field $K$. Let
  $\vect{u}_1,\ldots,\vect{u}_n\in V$. A vector
  $\vect{v}\in V$ is called a \textbf{linear combination}%
  \index{linear combination!in a vector space}%
  \index{linear combination!of vectors} of
  $\vect{u}_1,\ldots,\vect{u}_n$ if there exist scalars
  $a_{1},\ldots,a_{n}\in K$ such that
  \begin{equation*}
    \vect{v} = a_1 \vect{u}_1 + \ldots + a_n \vect{u}_n.
  \end{equation*}
\end{definition}

\begin{example}{Linear combination of matrices}{linear-combination-matrix}
  Write the matrix $A=\begin{mymatrix}{rr} 1 & 3 \\ -1 & 2 \end{mymatrix}$
  as a linear combination%
  \index{linear combination!of matrices} of
  \begin{equation*}
    \begin{mymatrix}{rr} 1 & 0 \\ 0 & 1 \end{mymatrix},\quad
    \begin{mymatrix}{rr} 1 & 0 \\ 0 & -1 \end{mymatrix},\quad
    \begin{mymatrix}{rr} 0 & 1 \\ 1 & 0 \end{mymatrix},\quad\mbox{and}\quad
    \begin{mymatrix}{rr} 0 & -1 \\ 1 & 0 \end{mymatrix}.
  \end{equation*}
\end{example}

\begin{solution}
  We must find coefficients $a,b,c,d$ such that
  \begin{equation*}
    \begin{mymatrix}{rr} 1 & 3 \\ -1 & 2 \end{mymatrix}
    ~=~ a \begin{mymatrix}{rr} 1 & 0 \\ 0 & 1 \end{mymatrix}
    + b \begin{mymatrix}{rr} 1 & 0 \\ 0 & -1 \end{mymatrix}
    + c \begin{mymatrix}{rr} 0 & 1 \\ 1 & 0 \end{mymatrix}
    + d \begin{mymatrix}{rr} 0 & -1 \\ 1 & 0 \end{mymatrix},
  \end{equation*}
  or equivalently,
  \begin{equation*}
    \begin{mymatrix}{rr} 1 & 3 \\ -1 & 2 \end{mymatrix}
    ~=~ \begin{mymatrix}{cc} a+b & c-d \\ c+d & a-b \end{mymatrix}.
  \end{equation*}
  This yields a system of four equations in four variables:
  \begin{equation*}
    \begin{array}{r@{~~}c@{~}r}
      a+b &=& 1, \\
      c+d &=& -1, \\
      c-d &=& 3, \\
      a-b &=& 2.
    \end{array}
  \end{equation*}
  We can easily solve the system of equations to find the unique
  solution $a=\frac{3}{2}$, $b=-\frac{1}{2}$, $c=1$, $d=-2$.
  Therefore
  \begin{equation*}
    \begin{mymatrix}{rr} 1 & 3 \\ -1 & 2 \end{mymatrix}
    ~=~ \frac{3}{2} \begin{mymatrix}{rr} 1 & 0 \\ 0 & 1 \end{mymatrix}
    - \frac{1}{2} \begin{mymatrix}{rr} 1 & 0 \\ 0 & -1 \end{mymatrix}
    + 1 \begin{mymatrix}{rr} 0 & 1 \\ 1 & 0 \end{mymatrix}
    - 2 \begin{mymatrix}{rr} 0 & -1 \\ 1 & 0 \end{mymatrix}.
  \end{equation*}
\end{solution}

\begin{example}{Linear combination of polynomials}{linear-combination-polynomials}
  Write the polynomial $p(x) = 7x^2 + 4x - 3$ as a linear combination%
  \index{linear combination!of polynomials} of
  \begin{equation*}
    q_1(x) = x^2,\quad
    q_2(x) = (x+1)^2,\quad\mbox{and}\quad
    q_3(x) = (x+2)^2.
  \end{equation*}
\end{example}

\begin{solution}
  We will show two different methods of solving this problem.
  \begin{itemize}
  \item \textbf{Method 1:} We must find coefficients $a,b,c$ such that
    $p(x) = aq_1(x) + bq_2(x) + cq_3(x)$.  Note that
    $q_2(x) = (x+1)^2 = x^2 + 2x + 1$ and
    $q_3(x) = (x+2)^2 = x^2 + 4x + 4$. Therefore, we must solve the
    equation
    \begin{equation*}
      7x^2 + 4x - 3 ~=~ ax^2 ~+~ b(x^2 + 2x + 1) ~+~ c(x^2 + 4x + 4).
    \end{equation*}
    Collecting equal powers of $x$, we can rewrite this as
    \begin{equation*}
      7x^2 + 4x - 3 ~=~ (a+b+c)x^2 ~+~ (2b+4c)x ~+~ (b+4c).
    \end{equation*}
    Since two polynomials are equal if and only if each corresponding
    coefficient is equal, this yields a system of three equations in
    three variables
    \begin{equation*}
      \begin{array}{r@{~~}c@{~}r}
        a+b+c &=& 7, \\
        2b+4c &=& 4, \\
        b+4c &=& -3.
      \end{array}
    \end{equation*}      
    We can easily solve this system of equations and find that the
    unique solution is $a=\frac{5}{2}$, $b=7$,
    $c=-\frac{5}{2}$. Therefore
    \begin{equation*}
      p(x) ~=~ \frac{5}{2}\,q_1(x) ~+~ 7\,q_2(x) ~-~ \frac{5}{2}\,q_3(x).
    \end{equation*}
  \item\textbf{Method 2:} We must find coefficients $a,b,c$ such that
    $p(x) = aq_1(x) + bq_2(x) + cq_3(x)$. We substitute three
    different values of $x$ into the equation, for example $x=0$,
    $x=-1$, and $x=-2$, to obtain three scalar equations:
    \begin{equation*}
      \begin{array}{l@{~~}l@{~~}l@{\,}l@{~}l@{~}l@{\,}l@{~}l@{~}l@{\,}l@{~}l}
        p(0)  &=& a&q_1(0)  &+& b&q_2(0)  &+& c&q_3(0), \\
        p(-1) &=& a&q_1(-1) &+& b&q_2(-1) &+& c&q_3(-1), \\
        p(-2) &=& a&q_1(-2) &+& b&q_2(-2) &+& c&q_3(-2). \\
      \end{array}
    \end{equation*}
    We can calculate the 12 coefficients: $p(0)=-3$, $p(-1)=0$,
    $p(-2)=17$, $q_1(0)=0$, $q_1(-1)=1$, $q_1(-2)=4$, $q_2(0)=1$,
    $q_2(-1)=0$, $q_2(-2)=1$, $q_3(0)=4$, $q_3(-1)=1$, $q_3(-2)=0$.
    Therefore, we obtain the system of equations
    \begin{equation*}
      \begin{array}{r@{~~}c@{~~}l}
        -3  &=& 0a ~+~ 1b ~+~ 4c, \\
        0   &=& 1a ~+~ 0b ~+~ 1c, \\
        17  &=& 4a ~+~ 1b ~+~ 0c.
      \end{array}
    \end{equation*}
    We can solve this system of equations to find the unique solution
    $a=\frac{5}{2}$, $b=7$, $c=-\frac{5}{2}$. This yields the
    polynomial $\frac{5}{2}\,q_1(x) ~+~ 7\,q_2(x) ~-~
    \frac{5}{2}\,q_3(x)$.

    Unlike in method 1, however, we still have one more step to do: we
    must check whether $p(x)$ is indeed equal to
    $\frac{5}{2}\,q_1(x) ~+~ 7\,q_2(x) ~-~ \frac{5}{2}\,q_3(x)$.
    Because we have only compared the two functions at 3 points $x=0$,
    $x=-1$, and $x=-2$, it is not automatically guaranteed that they
    will be equal for all $x$.
  \end{itemize}
\end{solution}

% ======================================================================
\subsection*{CONTINUE HERE...}

This definition leads to our next concept of span.

\begin{definition}{Span of vectors}{span}
  Let $\set{\vect{v}_{1},\ldots,\vect{v}_{n}} \subseteq V$.
  Then\index{span}
  \begin{equation*} \sspan\set{\vect{v}_{1},\ldots,\vect{v}_{n}} =
    \set{\sum_{i=1}^{n}c_{i}\vect{v}_{i}: c_{i}\in \R }
  \end{equation*}
\end{definition}

When we say that a vector $\vect{w}$ is in
$\sspan\set{\vect{v}_{1},\ldots,\vect{v}_{n}}$ we mean that $\vect{w}$
can be written as a linear combination of the $\vect{v}_1$. We say
that a collection of vectors $\set{\vect{v}_{1},\ldots,\vect{v}_{n}}$
is a \textbf{spanning set}\index{spanning set} for $V$ if $V = \sspan
\set{\vect{v}_{1},\ldots,\vect{v}_{n}}$.

Consider the following example.

\begin{example}{Matrix span}{matrix-span}
  Let $A = \begin{mymatrix}{rr} 1 & 0 \\ 0 & 2 \end{mymatrix}$,
  $B = \begin{mymatrix}{rr} 0 & 1 \\ 1 & 0 \end{mymatrix}$.
  Determine if $A$ and $B$ are in
  \begin{equation*}
    \sspan\set{M_1, M_2 } = \sspan
    \set{\begin{mymatrix}{rr} 1 & 0 \\ 0 & 0 \end{mymatrix},
      \begin{mymatrix}{rr} 0 & 0 \\ 0 & 1 \end{mymatrix} }
  \end{equation*}
\end{example}

\begin{solution}

  First consider $A$. We want to see if scalars $s,t$ can be found
  such that $A = s M_1 + t M_2$.
  \begin{equation*}
    \begin{mymatrix}{rr} 1 & 0 \\ 0 & 2 \end{mymatrix}
    = s \begin{mymatrix}{rr} 1 & 0 \\ 0 & 0 \end{mymatrix}
    + t \begin{mymatrix}{rr} 0 & 0 \\ 0 & 1 \end{mymatrix}
  \end{equation*}
  The solution to this equation is given by
  \begin{eqnarray*}
    1 &=& s \\
    2 &=& t
  \end{eqnarray*}
  and it follows that $A$ is in $\sspan \set{M_1, M_2}$.

  Now consider $B$. Again we write $B = sM_1 + t M_2$ and see if a
  solution can be found for $s, t$.
  \begin{equation*}
    \begin{mymatrix}{rr} 0 & 1 \\ 1 & 0 \end{mymatrix}
    = s \begin{mymatrix}{rr} 1 & 0 \\ 0 & 0 \end{mymatrix}
    + t \begin{mymatrix}{rr} 0 & 0 \\ 0 & 1 \end{mymatrix}
  \end{equation*}
  Clearly no values of $s$ and $t$ can be found such that this
  equation holds. Therefore $B$ is not in $\sspan \set{M_1, M_2 }$.
\end{solution}

Consider another example.

\begin{example}{Polynomial span}{polynomial-span}
  Show that $p(x) =
  7x^2 + 4x - 3$ is in $\sspan\set{4x^2 + x, x^2 -2x + 3 }$.
\end{example}

\begin{solution}
  To show that $p(x)$ is in the given span, we need to
  show that it can be written as a linear combination of polynomials in
  the span. Suppose scalars $a, b$ existed such that
  \begin{equation*}
    7x^2 +4x - 3= a(4x^2+x) + b (x^2-2x+3)
  \end{equation*}
  If this linear combination were to hold, the
  following would be true:
  \begin{eqnarray*}
    4a + b &=& 7 \\
    a - 2b &=& 4 \\
    3b &=& -3
  \end{eqnarray*}

  You can verify that $a = 2, b = -1$ satisfies this system of
  equations. This means that we can write $p(x)$ as follows:
  \begin{equation*}
    7x^2 +4x-3= 2(4x^2+x) - (x^2-2x+3)
  \end{equation*}

  Hence $p(x)$ is in the given span.
\end{solution}

Consider the following example.

\begin{example}{Spanning set}{spanning-set}
  Let $S = \set{x^2 + 1,
    x-2, 2x^2 - x }$. Show that $S$ is a spanning set for $\Poly_2$, the
  set of all polynomials of degree at most $2$.
\end{example}

\begin{solution}
  Let $p(x)= ax^2 + bx + c$ be an arbitrary polynomial
  in $\Poly_2$. To show that $S$ is a spanning set, it suffices to show
  that $p(x)$ can be written as a linear combination of the elements of
  $S$. In other words, can we find $r,s,t$ such that:
  \begin{equation*}
    p(x) = ax^2 +bx + c = r(x^2 + 1) + s(x -2) +
    t(2x^2 - x)
  \end{equation*}

  If a solution $r,s,t$ can be found, then this shows that for any
  such polynomial $p(x)$, it can be written as a linear combination of
  the above polynomials and $S$ is a spanning set.

  \begin{eqnarray*}
    ax^2 +bx + c
    &=& r(x^2 + 1) + s(x -2) + t(2x^2 - x) \\
    &=& rx^2 + r + sx - 2s + 2tx^2 - tx \\
    &=& (r+2t)x^2 + (s-t)x + (r-2s)
  \end{eqnarray*}

  For this to be true, the following must hold:
  \begin{eqnarray*}
    a &=& r+2t \\
    b &=& s-t \\
    c &=& r-2s
  \end{eqnarray*}

  To check that a solution exists, set up the augmented matrix and row
  reduce:
  \begin{equation*}
    \begin{mymatrix}{rrr|r}
      1 & 0 & 2 & a \\
      0 & 1 & -1 & b \\
      1 & -2 & 0 & c
    \end{mymatrix}
    \roweq\ldots\roweq
    \begin{mymatrix}{rrr|c}
      1 & 0 & 0 & \frac{1}{2} a + 2b + \frac{1}{2}c\\
      0 & 1 & 0 & \frac{1}{4}a - \frac{1}{4}c \\
      0 & 0 & 1 & \frac{1}{4}a - b - \frac{1}{4}c
    \end{mymatrix}
  \end{equation*}

  Clearly a solution exists for any choice of $a,b,c$. Hence $S$ is a
  spanning set for $\Poly_2$.
\end{solution}
