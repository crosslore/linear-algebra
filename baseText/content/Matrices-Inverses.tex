\section{Matrix inverses}

\begin{outcome}
  \begin{enumerate}
  \item Determine whether a matrix is invertible, and compute the
    inverse if it exists.
  \item Solve a system of linear equations using matrix algebra.
  \item Prove algebraic properties of matrix inverses.
  \item Determine whether a matrix is a left inverse, right inverse, or
    inverse of another matrix.
  \end{enumerate}
\end{outcome}

\subsection{Definition and uniqueness}

We now define a matrix operation which in some ways plays the role
of division. We cannot divide by a matrix, but we can multiply by the
inverse of a matrix, which is almost as good.

\begin{definition}{The inverse of a matrix}{invertible-matrix}
  Let $A$ and $B$ be $n\times n$-matrices. We say that $B$ is an
  \textbf{inverse}\index{inverse!of a matrix}%
  \index{matrix!inverse} of $A$ if
  \begin{equation*}
    BA=I\quad\mbox{and}\quad
    AB=I.
  \end{equation*}
  If this is the case, we also write $B=A^{-1}$. When a matrix has an
  inverse, it is called \textbf{invertible}\index{matrix!invertible}%
  \index{invertible matrix}.
\end{definition}

\begin{example}{Verifying the inverse of a matrix}{verifying-inverse}
  Let $A=\begin{mymatrix}{rr}
    1 & 1 \\
    1 & 2
  \end{mymatrix}$. Check that $B=\begin{mymatrix}{rr}
    2 & -1 \\
    -1 & 1
  \end{mymatrix}$ is an inverse of $A$.
\end{example}

\begin{solution}
  To check this, multiply
  \begin{equation*}
    AB = \begin{mymatrix}{rr}
      1 & 1 \\
      1 & 2
    \end{mymatrix} \begin{mymatrix}{rr}
      2 & -1 \\
      -1 & 1
    \end{mymatrix} =\allowbreak \begin{mymatrix}{rr}
      1 & 0 \\
      0 & 1
    \end{mymatrix} = I
  \end{equation*}
  and
  \begin{equation*}
    BA = \begin{mymatrix}{rr}
      2 & -1 \\
      -1 & 1
    \end{mymatrix} \begin{mymatrix}{rr}
      1 & 1 \\
      1 & 2
    \end{mymatrix} =\allowbreak \begin{mymatrix}{rr}
      1 & 0 \\
      0 & 1
    \end{mymatrix} = I.
  \end{equation*}
  This shows that $B$ is indeed an inverse of $A$.
\end{solution}

Unlike multiplication of scalars, it can happen that $A\neq 0$ but $A$
does not have an inverse. This is illustrated in the following
example.

\begin{example}{A non-zero matrix with no inverse}{non-invertible-matrix}
  Let $A=\begin{mymatrix}{rr}
    1 & 1 \\
    1 & 1
  \end{mymatrix}$. Show that $A$ is not invertible.
\end{example}

\begin{solution}
  One might think $A$ has an inverse because it does not equal zero.
  However, note that
  \begin{equation*}
    \begin{mymatrix}{rr}
      1 & 1 \\
      1 & 1
    \end{mymatrix} \begin{mymatrix}{r}
      -1 \\
      1
    \end{mymatrix} =\begin{mymatrix}{r}
      0 \\
      0
    \end{mymatrix}.
  \end{equation*}
  If an inverse $A^{-1}$ existed, we would have the following:
  \begin{eqnarray*}
    \begin{mymatrix}{r}
      -1 \\
      1
    \end{mymatrix}
    &=&
        I\begin{mymatrix}{r}
          -1 \\
          1
        \end{mymatrix} \\
    &=&
        (A^{-1}A) \begin{mymatrix}{r}
          -1 \\
          1
        \end{mymatrix} \\
    &=&
        A^{-1}\paren{A\begin{mymatrix}{r}
            -1 \\
            1
          \end{mymatrix}} \\
    &=&
        A^{-1}\paren{\begin{mymatrix}{r}
            0 \\
            0
          \end{mymatrix}} \\
    &=&
        \begin{mymatrix}{r}
          0 \\
          0
        \end{mymatrix}.
\end{eqnarray*}
  This says that
  \begin{equation*}
    \begin{mymatrix}{r}
      -1 \\
      1
    \end{mymatrix}
    =
    \begin{mymatrix}{r}
      0 \\
      0
    \end{mymatrix},
  \end{equation*}
  which is impossible! Therefore, $A$ does not have an inverse.
\end{solution}

Can a matrix have more than one inverse? It turns out that this is not
the case: the following theorem shows that if $A$ has an inverse, then
the inverse is unique. We can therefore speak of ``the'' inverse,
rather than just ``an'' inverse, of $A$.

\begin{theorem}{Uniqueness of inverse}{unique-inverse}
  Suppose $A$ is an $n\times n$-matrix such that both $B$ and $C$ are
  inverses of $A$. Then $B=C$.
\end{theorem}

\begin{proof}
  By assumption, both $B$ and $C$ are inverses of $A$, so we have
  $AB=I$, $BA=I$, $AC=I$, and $CA=I$. Using the associative and unit
  properties of matrix multiplication, we have:
  \begin{equation*}
    B = BI = B(AC) = (BA)C = IC = C.
  \end{equation*}
  Therefore, $B=C$, as desired.
\end{proof}
