\subsection{The transpose}

Another important operation on matrices is that of taking the \textbf{transpose}. For a matrix $A$, we denote the
\textbf{transpose} of $A$ by $A^T$. Before formally defining the transpose, we explore this
operation on the following matrix.
\begin{equation*}
\leftB
\begin{array}{cc}
1 & 4 \\
3 & 1 \\
2 & 6
\end{array}
\rightB ^{T}=\allowbreak \leftB
\begin{array}{ccc}
1 & 3 & 2 \\
4 & 1 & 6
\end{array}
\rightB
\end{equation*}

What happened? The first column became the first row and the second column
became the second row. Thus the $3\times 2$ matrix became a $2\times 3$
matrix. The number $4$ was in the first row and the second column and it
ended up in the second row and first column. 

The definition of the transpose is as follows.

\begin{definition}{The transpose of a matrix}{matrixtranspose}
Let $A$ be an $m\times n$ matrix. Then $A^{T}$, the \textbf{transpose}\index{matrix!transpose}\index{transpose} of $A$,  denotes the $n\times m$
matrix given by 
\begin{equation*}
A^{T} = \leftB a _{ij}\rightB ^{T}= \leftB a_{ji} \rightB
\end{equation*}
\end{definition}

The $\left( i, j  \right)$-entry of $A$ becomes the 
$\left( j,i \right)$-entry of $A^T$. 

Consider the following example.

\begin{example}{The transpose of a matrix}{transposematrix}
Calculate $A^T$ for the following matrix
\begin{equation*}
A = \leftB
\begin{array}{rrr}
1 & 2 & -6 \\
3 & 5 & 4
\end{array}
\rightB
\end{equation*}
\end{example}

\begin{solution}
By Definition \ref{def:matrixtranspose}, we know that for $A = \leftB a_{ij} \rightB$, 
$A^T = \leftB a_{ji} \rightB$. In other words, we switch the row and column
location of each entry. The $\left( 1, 2 \right)$-entry becomes the $\left( 2,1 \right)$-entry.

Thus, 
\begin{equation*}
A^T = 
 \leftB
\begin{array}{rr}
1 & 3 \\
2 & 5 \\
-6 & 4
\end{array}
\rightB 
\end{equation*}

Notice that $A$ is a $2 \times 3$ matrix, while $A^T$ is a $3 \times 2$ matrix. 
\end{solution}

The transpose of a matrix has the following important properties\index{matrix!properties of transpose}.

\begin{lemma}{Properties of the transpose of a matrix}{transposeproperties}
Let $A$ be an $m\times n$ matrix, $B$ an $n\times p$ matrix, and $r$ and $s$ scalars. Then
\begin{enumerate}
\item
$\left(A^{T}\right)^{T} = A$
\item
$\left( AB\right) ^{T}=B^{T}A^{T} $ \label{matrixtranspose1}
\item
$\left( rA+ sB\right) ^{T}=rA^{T}+ sB^{T}$  \label{matrixtranspose2}
\end{enumerate}
\end{lemma}

\begin{proof}
First we prove \ref{matrixtranspose1}. From Definition \ref{def:matrixtranspose},
\begin{equation*}
\left(AB\right)^{T} = \leftB (AB) _{ij} \rightB ^{T}=\leftB (AB)_{ji} \rightB=\sum_{k}a_{jk}b_{ki}= \sum_{k}b_{ki}a_{jk} 
\end{equation*}
\begin{equation*}
= \sum_{k}\leftB b_{ik}\rightB^{T}\leftB
a_{kj}\rightB^{T}=\leftB b_{ij}\rightB ^{T} \leftB a_{ij}\rightB^{T} = B^{T}A^{T} 
\end{equation*}
The proof of Formula \ref{matrixtranspose2} is left as an exercise. 
\end{proof}

The transpose of a matrix is related to other important topics. Consider the following definition.  

\begin{definition}{Symmetric and skew symmetric matrices}{symmetricandskewsymmetric}
An $n\times n$ matrix $A$ is said to be
\textbf{symmetric}\index{matrix!symmetric} if $A=A^{T}.$ It is said to be
\textbf{skew symmetric}\index{matrix!skew symmetric} if $A=-A^{T}.$
\end{definition}

We will explore these definitions in the following examples.

\begin{example}{Symmetric matrices}{symmetricmatrix}
Let
\begin{equation*}
A=\leftB
\begin{array}{rrr}
2 & 1 & 3 \\
1 & 5 & -3 \\
3 & -3 & 7
\end{array}
\rightB 
\end{equation*}
Use Definition \ref{def:symmetricandskewsymmetric} to show that $A$ is symmetric. 
\end{example}

\begin{solution}
By Definition \ref{def:symmetricandskewsymmetric}, we need to show that $A = A^T$. 
Now, using Definition \ref{def:matrixtranspose}, 
\begin{equation*}
A^{T} = \leftB
\begin{array}{rrr}
2 & 1 & 3 \\
1 & 5 & -3 \\
3 & -3 & 7
\end{array}
\rightB
\end{equation*}
Hence, $A = A^{T}$, so $A$ is symmetric.
\end{solution}

\begin{example}{A skew symmetric matrix}{skewsymmetricmatrix}
Let
\begin{equation*}
A=\leftB
\begin{array}{rrr}
0 & 1 & 3 \\
-1 & 0 & 2 \\
-3 & -2 & 0
\end{array}
\rightB 
\end{equation*}
Show that $A$ is skew symmetric.
\end{example}

\begin{solution} By Definition \ref{def:symmetricandskewsymmetric}, 
\begin{equation*}
A^{T} = \leftB
\begin{array}{rrr}
0 & -1 & -3\\
1 &  0 & -2\\
3 &  2 &  0
\end{array}
\rightB 
\end{equation*}

You can see that each entry of $A^T$ is equal to $-1$ times the same entry of $A$. 
Hence, $A^{T} = - A$ and so by Definition \ref{def:symmetricandskewsymmetric}, $A$ is skew symmetric. 
\end{solution}
