\subsection{The transpose}

Another important operation on matrices is that of taking the \textbf{transpose}. For a matrix $A$, we denote the
\textbf{transpose} of $A$ by $A^T$. Before formally defining the transpose, we explore this
operation on the following matrix.
\begin{equation*}
\begin{mymatrix}{cc}
1 & 4 \\
3 & 1 \\
2 & 6
\end{mymatrix} ^{T}=\allowbreak \begin{mymatrix}{ccc}
1 & 3 & 2 \\
4 & 1 & 6
\end{mymatrix}
\end{equation*}

What happened? The first column became the first row and the second column
became the second row. Thus the $3\times 2$ matrix became a $2\times 3$
matrix. The number $4$ was in the first row and the second column and it
ended up in the second row and first column. 

The definition of the transpose is as follows.

\begin{definition}{The transpose of a matrix}{matrix-transpose}
Let $A$ be an $m\times n$ matrix. Then $A^{T}$, the \textbf{transpose}\index{matrix!transpose}\index{transpose} of $A$,  denotes the $n\times m$
matrix given by 
\begin{equation*}
A^{T} = \mat{a _{ij}} ^{T}= \mat{a_{ji} }
\end{equation*}
\end{definition}

The $\tup{i, j  }$-entry of $A$ becomes the 
$\tup{j,i }$-entry of $A^T$. 

Consider the following example.

\begin{example}{The transpose of a matrix}{transpose-matrix}
Calculate $A^T$ for the following matrix
\begin{equation*}
A = \begin{mymatrix}{rrr}
1 & 2 & -6 \\
3 & 5 & 4
\end{mymatrix}
\end{equation*}
\end{example}

\begin{solution}
By Definition \ref{def:matrix-transpose}, we know that for $A = \mat{a_{ij} }$, 
$A^T = \mat{a_{ji} }$. In other words, we switch the row and column
location of each entry. The $\tup{1, 2 }$-entry becomes the $\tup{2,1 }$-entry.

Thus, 
\begin{equation*}
A^T = 
 \begin{mymatrix}{rr}
1 & 3 \\
2 & 5 \\
-6 & 4
\end{mymatrix} 
\end{equation*}

Notice that $A$ is a $2 \times 3$ matrix, while $A^T$ is a $3 \times 2$ matrix. 
\end{solution}

The transpose of a matrix has the following important properties\index{matrix!properties of transpose}.

\begin{lemma}{Properties of the transpose of a matrix}{transpose-properties}
Let $A$ be an $m\times n$ matrix, $B$ an $n\times p$ matrix, and $r$ and $s$ scalars. Then
\begin{enumerate}
\item
$\tup{A^{T}}^{T} = A$
\item
$\tup{AB} ^{T}=B^{T}A^{T} $ \label{matrix-transpose1}
\item
$\tup{rA+ sB} ^{T}=rA^{T}+ sB^{T}$  \label{matrix-transpose2}
\end{enumerate}
\end{lemma}

\begin{proof}
First we prove \ref{matrix-transpose1}. From Definition \ref{def:matrix-transpose},
\begin{equation*}
\tup{AB}^{T} = \mat{(AB) _{ij} } ^{T}=\mat{(AB)_{ji} }=\sum_{k}a_{jk}b_{ki}= \sum_{k}b_{ki}a_{jk} 
\end{equation*}
\begin{equation*}
= \sum_{k}\mat{b_{ik}}^{T}\mat{
a_{kj}}^{T}=\mat{b_{ij}} ^{T} \mat{a_{ij}}^{T} = B^{T}A^{T} 
\end{equation*}
The proof of Formula \ref{matrix-transpose2} is left as an exercise. 
\end{proof}

The transpose of a matrix is related to other important topics. Consider the following definition.  

\begin{definition}{Symmetric and skew symmetric matrices}{symmetric-and-skew-symmetric}
An $n\times n$ matrix $A$ is said to be
\textbf{symmetric}\index{matrix!symmetric} if $A=A^{T}.$ It is said to be
\textbf{skew symmetric}\index{matrix!skew symmetric} if $A=-A^{T}.$
\end{definition}

We will explore these definitions in the following examples.

\begin{example}{Symmetric matrices}{symmetric-matrix}
Let
\begin{equation*}
A=\begin{mymatrix}{rrr}
2 & 1 & 3 \\
1 & 5 & -3 \\
3 & -3 & 7
\end{mymatrix} 
\end{equation*}
Use Definition \ref{def:symmetric-and-skew-symmetric} to show that $A$ is symmetric. 
\end{example}

\begin{solution}
By Definition \ref{def:symmetric-and-skew-symmetric}, we need to show that $A = A^T$. 
Now, using Definition \ref{def:matrix-transpose}, 
\begin{equation*}
A^{T} = \begin{mymatrix}{rrr}
2 & 1 & 3 \\
1 & 5 & -3 \\
3 & -3 & 7
\end{mymatrix}
\end{equation*}
Hence, $A = A^{T}$, so $A$ is symmetric.
\end{solution}

\begin{example}{A skew symmetric matrix}{skew-symmetric-matrix}
Let
\begin{equation*}
A=\begin{mymatrix}{rrr}
0 & 1 & 3 \\
-1 & 0 & 2 \\
-3 & -2 & 0
\end{mymatrix} 
\end{equation*}
Show that $A$ is skew symmetric.
\end{example}

\begin{solution} By Definition \ref{def:symmetric-and-skew-symmetric}, 
\begin{equation*}
A^{T} = \begin{mymatrix}{rrr}
0 & -1 & -3\\
1 &  0 & -2\\
3 &  2 &  0
\end{mymatrix} 
\end{equation*}

You can see that each entry of $A^T$ is equal to $-1$ times the same entry of $A$. 
Hence, $A^{T} = - A$ and so by Definition \ref{def:symmetric-and-skew-symmetric}, $A$ is skew symmetric. 
\end{solution}
