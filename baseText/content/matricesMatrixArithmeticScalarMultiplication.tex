\section{Scalar multiplication}

\begin{outcome}
  \begin{enumerate}
  \item Multiply a matrix by a scalar, and take linear combinations of matrices.
  \item Identify when these operations are not defined.
  \item Apply the algebraic properties of matrix addition and scalar
    multiplication to manipulate an algebraic expression involving
    matrices.
  \end{enumerate}
\end{outcome}

The multiplication of a scalar by a matrix is called the
\textbf{scalar multiplication} of matrices. The new matrix is obtained
by multiplying every entry of the original matrix by the given scalar,
as in the following example.
\begin{equation*}
  3~\begin{mymatrix}{rrrr}
    1 & 2 & 3 & 4 \\
    5 & 2 & 8 & 7 \\
    6 & -9 & 1 & 2
  \end{mymatrix} = \begin{mymatrix}{rrrr}
    3 & 6 & 9 & 12 \\
    15 & 6 & 24 & 21 \\
    18 & -27 & 3 & 6
  \end{mymatrix}.
\end{equation*}
The formal definition of scalar multiplication is as follows.

\begin{definition}{Scalar multiplication of a matrix}{scalar-mult-of-matrices}
  If $k$ is a scalar and $A=\mat{a_{ij}}$ is a matrix, then
  $kA=\mat{ka_{ij}}$\index{matrix!scalar multiplication}%
  \index{scalar multiplication!of a matrix}.
\end{definition}

\begin{example}{Linear combination of matrices}{matrix-linear-combination}
  Find $2A-3B$, where\index{linear combination!of matrices}
  \begin{equation*}
    A = \begin{mymatrix}{rrr}
      1 & 2 & 3 \\
      1 & 0 & 4
    \end{mymatrix}
    \quad\mbox{and}\quad
    B = \begin{mymatrix}{rrr}
      5 & 2 & 3 \\
      -6 & 2 & 1
    \end{mymatrix}
  \end{equation*}
\end{example}

\begin{solution}
  \begin{equation*}
    2A-3B = 
    2 \begin{mymatrix}{rrr}
      1 & 2 & 3 \\
      1 & 0 & 4
    \end{mymatrix}
    - 3 \begin{mymatrix}{rrr}
      5 & 2 & 3 \\
      -6 & 2 & 1
    \end{mymatrix}
    =
    \begin{mymatrix}{rrr}
      2 & 4 & 6 \\
      2 & 0 & 8
    \end{mymatrix}
    - \begin{mymatrix}{rrr}
      15 & 6 & 9 \\
      -18 & 6 & 3
    \end{mymatrix}
    =
    \begin{mymatrix}{rrr}
      -13 & -2 & -3 \\
      20 & -6 & 5
    \end{mymatrix}.
  \end{equation*}
\end{solution}

Scalar multiplication of matrices obeys the same properties as scalar
multiplication of vectors.

\begin{theorem}{Properties of scalar multiplication}{properties-scalar-mult}
  Let $A$ and $B$ be matrices of the same size, and let $k,\ell$ be
  scalars. Then the following properties%
  \index{matrix!properties of scalar multiplication}%
  \index{matrix!scalar multiplication!properties}%
  \index{properties of scalar multiplication!matrix} hold.
  \begin{itemize}
  \item The distributive law over matrix addition
    \begin{equation*}
      k \tup{A+B} = kA + kB.
    \end{equation*}
  \item The distributive law over scalar addition
    \begin{equation*}
      \tup{k + \ell} A = k A + \ell A.
    \end{equation*}
  \item The associative law for scalar multiplication
    \begin{equation*}
      k \tup{\ell A} = \tup{k \ell} A.
    \end{equation*}
  \item The rule for multiplication by $1$
    \begin{equation*}
      1A=A.
    \end{equation*}
  \end{itemize}
\end{theorem}
