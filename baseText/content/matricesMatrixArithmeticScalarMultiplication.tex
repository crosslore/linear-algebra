\subsection{Scalar multiplication of matrices}

Recall that we use the word {\em scalar \em}when referring to numbers. Therefore, {\em scalar multiplication of a matrix \em}is the multiplication of a matrix by a number.  
To illustrate this concept, consider the following example in which a
matrix is multiplied by the scalar $3$.
\begin{equation*}
3\begin{mymatrix}{rrrr}
1 & 2 & 3 & 4 \\
5 & 2 & 8 & 7 \\
6 & -9 & 1 & 2
\end{mymatrix} = \begin{mymatrix}{rrrr}
3 & 6 & 9 & 12 \\
15 & 6 & 24 & 21 \\
18 & -27 & 3 & 6
\end{mymatrix} 
\end{equation*}

The new matrix is obtained by multiplying every entry of the original matrix
by the given scalar. 

The formal definition of scalar multiplication is as follows.

\begin{definition}{Scalar multiplication of matrices}{scalarmultofmatrices}
If $A=\mat{a_{ij}} $ and $k$ is a scalar,
then $kA=\mat{ka_{ij}}$\index{matrix!scalar multiplication}.
\end{definition}

Consider the following example.

\begin{example}{Effect of multiplication by a scalar}{effectofscalarmult}
Find the result of multiplying the following matrix $A$ by $7$.
\begin{equation*}
A=\begin{mymatrix}{rr}
2 & 0 \\
1 & -4
\end{mymatrix}
\end{equation*}
\end{example}

\begin{solution}
By Definition \ref{def:scalarmultofmatrices}, we multiply each element of $A$ by $7$.
Therefore,
\begin{equation*}
7A = 
7\begin{mymatrix}{rr}
2 & 0 \\
1 & -4
\end{mymatrix} =
\begin{mymatrix}{rr}
7(2) & 7(0) \\
7(1) & 7(-4)
\end{mymatrix} =
\begin{mymatrix}{rr}
14 & 0 \\
7 & -28
\end{mymatrix}
\end{equation*}
\end{solution}

Similarly to addition of matrices, there are several properties of scalar multiplication which hold.

\begin{proposition}{Properties of scalar multiplication}{propertiesscalarmult}
Let $A, B$ be matrices, and $k, p$ be scalars. Then, the following properties\index{matrix!properties of scalar multiplication} hold.
\begin{itemize}
\item Distributive Law over Matrix Addition
\begin{equation*}
k \tup{A+B} =k A+ kB  
\end{equation*}

\item Distributive Law over Scalar Addition
\begin{equation*}
\tup{k +p } A= k A+p A
\end{equation*}

\item Associative Law for Scalar Multiplication
\begin{equation*}
k \tup{p A} = \tup{k p } A 
\end{equation*}

\item Rule for Multiplication by $1$
\begin{equation*}
1A=A  
\end{equation*}
\end{itemize}

\end{proposition}

The proof of this proposition is similar to the proof of Proposition \ref{prop:propertiesofaddition} and is left an exercise to the reader.
