\section*{Exercises}

\begin{ex}
  Consider the following functions $T:\R^3\rightarrow \R^2$.
  For each of these functions $T$, show that it is not linear either
  by showing that $T$ does not preserve addition, or by showing that it
  does not preserve scalar multiplication, or by showing that it does
  not preserve the zero vector.
  \begin{enumerate}
  \item $T\begin{mymatrix}{c} x \\ y \\ z \end{mymatrix}
    = \begin{mymatrix}{c}
      x+2y+3z+1 \\
      2y-3x+z
    \end{mymatrix}$.
  \item $T\begin{mymatrix}{c} x \\ y \\ z \end{mymatrix}
    = \begin{mymatrix}{c}
      x+2y^2+3z \\
      2y+3x+z
    \end{mymatrix}$.
  \item $T\begin{mymatrix}{c} x \\ y \\ z \end{mymatrix}
    = \begin{mymatrix}{c}
      \sin x+2y+3z \\
      2y+3x+z
    \end{mymatrix}$.
  \end{enumerate}
  \begin{sol}
    \begin{enumerate}
    \item We have $T\begin{mymatrix}{c} 0 \\ 0 \\ 0 \end{mymatrix}
      = \begin{mymatrix}{c} 1 \\ 0 \end{mymatrix}$, so $T$ does
      not preserve the zero vector.
    \item We have
      $T\begin{mymatrix}{c} 0 \\ 1 \\ 0 \end{mymatrix}
      = \begin{mymatrix}{c} 2 \\ 2 \end{mymatrix}$ and
      $T\begin{mymatrix}{c} 0 \\ 2 \\ 0 \end{mymatrix}
      = \begin{mymatrix}{c} 8 \\ 4 \end{mymatrix}$, so $T$ does not
      preserve scalar multiplication.
    \item We have
      $T\begin{mymatrix}{c} \pi/2 \\ 0 \\ 0 \end{mymatrix}
      = \begin{mymatrix}{c} 1 \\ \pi/2 \end{mymatrix}$ and
      $T\begin{mymatrix}{c} \pi \\ 0 \\ 0 \end{mymatrix}
      = \begin{mymatrix}{c} 0 \\ \pi \end{mymatrix}$, so $T$ does not
      preserve scalar multiplication.
    \end{enumerate}
  \end{sol}
\end{ex}

\begin{ex}
  Consider the following functions $T:\R^3\rightarrow \R^2$. For
  each function $T$, show that $T$ is a linear transformation. Do this
  by showing that $T$ is a matrix transformation, i.e., find a matrix
  $A$ such that $T(\vect{v})=A\vect{v}$.
  \begin{enumerate}
  \item $T\begin{mymatrix}{c} x \\ y \\ z \end{mymatrix}
    = \begin{mymatrix}{c}
      x+2y+3z \\
      2y-3x+z
    \end{mymatrix}$.
  \item $T\begin{mymatrix}{c} x \\ y \\ z \end{mymatrix}
    = \begin{mymatrix}{c}
      7x+2y+z \\
      3x-11y+2z
    \end{mymatrix}$.
  \item $T\begin{mymatrix}{c} x \\ y \\ z \end{mymatrix}
    = \begin{mymatrix}{c}
      3x+2y+z \\
      x+2y+6z
    \end{mymatrix}$.
  \item $T\begin{mymatrix}{c} x \\ y \\ z \end{mymatrix}
    = \begin{mymatrix}{c}
      2y-5x+z \\
      x+y+z
    \end{mymatrix}$.
  \end{enumerate}
\end{ex}

\begin{ex}
  Which of the following functions are linear transformations?
  \begin{enumerate}
  \item The function $f:\Seq_K\to\Seq_K$ defined by
    $f((a_0,a_1,a_2,\ldots)) = (-a_1,-a_3,-a_5,\ldots)$. In words, the
    function $f$ removes the even-numbered elements, and negates the
    odd-numbered elements.
  \item The function $f:\Mat_{2,2}\to\Mat_{2,2}$ defined by $f(A) =
    AB$. Here, $\Mat_{2,2}$ is the vector space of $2\times
    2$-matrices with real entries, and $B$ is a fixed matrix.
  \item The function $f:\Mat_{2,2}\to\Mat_{2,2}$ defined by $f(A) =
    A+B$, where $B$ is a fixed matrix.
  \item The function $f:\Mat_{2,2}\to\Mat_{2,2}$ defined by $f(A) =
    A^T$.
  \end{enumerate}
  \begin{sol}
    Yes, yes, no, yes.
  \end{sol}
\end{ex}

\begin{ex}
  Recall the vector space $\Poly$ of polynomials with coefficients in
  a field $K$. Consider the function $M:\Poly\to\Poly$ defined by
  $M(p(x)) = xp(x)$.
  \begin{enumerate}
  \item Compute $M(x^3)$, $M(2x^2+x)$, and $M(ax^2+bx+c)$.
  \item Show that $M$ is a linear transformation.
  \end{enumerate}
\end{ex}

\begin{ex}
  Recall the vector space $\Poly$ of polynomials with coefficients in
  a field $K$. Consider the function $S:\Poly\to\Poly$ defined by
  $S(p(x)) = p(x+1)$.
  \begin{enumerate}
  \item Compute $S(x^3)$, $S(2x^2+x)$, and $S(ax^2+bx+c)$.
  \item Show that $S$ is a linear transformation.
  \end{enumerate}
\end{ex}

\begin{ex}
  Consider the shift function $\shift:\Seq_K\to\Seq_K$ from
  Example~\ref{exa:shift-unshift}. Find a basis for the solution space
  of each of the following equations:
  \begin{enumerate}
  \item $\shift(a) = a$.
  \item $\shift(a) = -a$.
  \item $\shift(\shift(a)) = \shift(a) + 2a$.
  \end{enumerate}
  \begin{sol}
    \begin{enumerate}
    \item Let $a=(a_0,a_1,a_2,\ldots)$. Then the equation $\shift(a) =
      a$ is equivalent to
      \begin{equation*}
        (a_1,a_2,a_3,\ldots) = (a_0,a_1,a_2,\ldots),
      \end{equation*}
      which translates to the recurrence $a_{n+1}=a_n$. The only free
      variable is $a_0$, and the only solutions are the constant
      sequences. They form a 1-dimensional space with basis
      $\set{(1,1,1,\ldots)}$.
    \item The equation $\shift(a) = -a$ is equivalent to
      \begin{equation*}
        (a_1,a_2,a_3,\ldots) = (-a_0,-a_1,-a_2,\ldots),
      \end{equation*}
      which translates to the recurrence $a_{n+1} = -a_n$. The only
      free variable is $a_0$, and the solutions form a 1-dimensional
      space with basis $\set{(1,-1,1,-1,\ldots)}$.
    \item The equation $\shift(\shift(a)) = \shift(a) + 2a$ is
      equivalent to
      \begin{equation*}
        (a_2,a_3,a_4,\ldots) = (a_1+2a_0,a_2+2a_1,a_3+2a_2,\ldots),
      \end{equation*}
      which translates to the recurrence $a_{n+2} = a_{n+1} + 2a_n$.
      The only free variables are $a_0$ and $a_1$. The most obvious
      basis vectors are obtained by letting $(a_0,a_1)=(1,0)$ and
      $(a_0,a_1)=(0,1)$, giving the sequences
      $(1,0,2,2,6,10,22,\ldots)$ and $(0,1,1,3,5,11,21,\ldots)$. Thus,
      \begin{equation*}
        \set{(1,0,2,2,6,10,22,\ldots), (0,1,1,3,5,11,21,\ldots)}
      \end{equation*}
      is a basis for the solutions. Another basis, which is slightly
      more convenient, is
      \begin{equation*}
        \set{(1,-1,1,-1,1,-1,\ldots), (1,2,4,8,16,32,\ldots)}.
      \end{equation*}
    \end{enumerate}
  \end{sol}
\end{ex}

