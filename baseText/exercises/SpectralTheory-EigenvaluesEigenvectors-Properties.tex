\Opensolutionfile{solutions}[ex]
\section*{Exercises}

\begin{enumialphparenastyle}

\begin{ex}
  Find the algebraic and geometric multiplicity of each eigenvalue of
  the following matrices. Which of the matrices are diagonalizable?
  \begin{equation*}
    (a)\quad
    A = \begin{mymatrix}{rr} 1 & 1 \\ -1 & 3 \end{mymatrix},
    \quad
    (b)\quad
    B = \begin{mymatrix}{rr} -7 & 8 \\ -4 & 5 \end{mymatrix},
  \end{equation*}
  \begin{equation*}
    (c)\quad
    C = \begin{mymatrix}{rrr}
      2 & -1 & -1 \\
      0 & 4 & 2 \\
      0 & -1 & 1 \\
    \end{mymatrix},
    \quad
    (d)\quad
    D = \begin{mymatrix}{rrr}
      3 & 1 & -1 \\
      0 & 2 & 1 \\
      0 & -1 & 4 \\
    \end{mymatrix},
    \quad
    (e)\quad
    E = \begin{mymatrix}{rrr}
      -2 & 0 & 1 \\
      -1 & -1 & 1 \\
      -2 & 1 & 0 \\
    \end{mymatrix}.
  \end{equation*}
  \begin{sol}
    \begin{enumerate}
    \item $\eigenvar=2$ has algebraic multiplicity 2 and geometric
      multiplicity 1. Since the sum of the geometric multiplicities of
      all eigenvalues is $1$, the matrix is not diagonalizable.
    \item $\eigenvar=1$ has algebraic and geometric
      multiplicity 1; $\eigenvar=-3$ has algebraic and geometric
      multiplicity 1. Since the sum of the geometric multiplicities is
      $2$, the matrix is diagonalizable.
    \item $\eigenvar=2$ has algebraic and geometric
      multiplicity 2, $\eigenvar=3$ has algebraic and geometric
      multiplicity 1. Since the sum of the geometric multiplicities is
      $3$, the matrix is diagonalizable.
    \item $\eigenvar=3$ has algebraic multiplicity 3 and geometric
      multiplicity 2. Since the sum of the geometric multiplicities of
      all eigenvalues is $2$, the matrix is not diagonalizable.
    \item $\eigenvar = -1$ has algebraic multiplicity 3 and geometric
      multiplicity 1. Since the sum of the geometric multiplicities of
      all eigenvalues is $1$, the matrix is not diagonalizable.
    \end{enumerate}
  \end{sol}
\end{ex}

\begin{ex}
  Determine which of the following matrices are diagonalizable.
  \begin{enumerate}
  \item $A$ is a $3\times 3$-matrix with eigenvalues $-1$ and $3$. The
    eigenvalue $-1$ has algebraic multiplicity $2$ and geometric
    multiplicity $1$. The eigenvalue $3$ has algebraic and geometric
    multiplicity $1$.
  \item $B$ is a $4\times 4$-matrix with eigenvalues $2$ and $-2$. The
    eigenvalue $2$ has algebraic and geometric multiplicity $1$. The
    eigenvalue $-2$ has algebraic and geometric multiplicity $3$.
  \item $C$ is a $5\times 5$-matrix with eigenvalues $1$ and $3$. The
    eigenvalue $1$ has algebraic and geometric multiplicity $2$, and
    the eigenvalue $3$ has algebraic and geometric multiplicity $1$.
  \end{enumerate}
  \begin{sol}
  \item Not diagonalizable.
  \item Diagonalizable.
  \item Not diagonalizable.
  \end{sol}
\end{ex}

\begin{ex}
  Let
  \begin{equation*}
    A = \begin{mymatrix}{rr}
      5 & 7 \\
      -4 & 3 \\
    \end{mymatrix}.
  \end{equation*}
  Find the characteristic polynomial $p(\eigenvar)$, and compute
  $p(A)$.
\end{ex}

\begin{ex}
  Let
  \begin{equation*}
    A = \begin{mymatrix}{rrr}
      1 & 2 & 0 \\
      0 & 2 & -1 \\
      0 & 1 & 4 \\
    \end{mymatrix}.
  \end{equation*}
  Find the characteristic polynomial $p(\eigenvar)$, and compute
  $p(A)$.
\end{ex}

\begin{ex}
  \begin{enumerate}
  \item Let $A$ be a $2\times 2$-matrix. Prove that $A^2$ is a linear
    combination of $A$ and $I$.
  \item Given an example of a $3\times 3$-matrix $A$ such that $A^2$
    is not a linear combination of $A$ and $I$.
  \end{enumerate}
  \begin{sol}
    \begin{enumerate}
    \item The characteristic polynomial of $A$ is a quadratic
      polynomial, and therefore it is of the form
      $p(\eigenvar) = \eigenvar^2 + b\eigenvar + c$, for some
      $r,s\in\R$. By the Cayley-Hamilton theorem, $p(A)=0$, therefore
      $A^2 = -bA - cI$. This proves that $A^2$ is a linear combination
      of $A$ and $I$.
    \item $A=\begin{mymatrix}{rrr}
        0 & 1 & 0 \\
        0 & 0 & 1 \\
        0 & 0 & 0 \\
      \end{mymatrix}$.
    \end{enumerate}
  \end{sol}
\end{ex}

\end{enumialphparenastyle}
