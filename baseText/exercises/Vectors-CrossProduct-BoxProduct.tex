
\begin{ex}
  Find the volume of the parallelepiped determined by the vectors
  $\begin{mymatrix}{r}
    1 \\
    -7 \\
    -5
  \end{mymatrix}$,
  $\begin{mymatrix}{r}
    1 \\
    -2 \\
    -6
  \end{mymatrix}$, and $\begin{mymatrix}{r}
    3 \\
    2 \\
    3
  \end{mymatrix}$.
  \begin{sol}
    $\paren{\begin{mymatrix}{r}1\\1\\3\end{mymatrix}
      \times \begin{mymatrix}{r}-7\\-2\\2\end{mymatrix}}
    \dotprod \begin{mymatrix}{r}-5\\-6\\3\end{mymatrix} =
    \begin{mymatrix}{r}8\\-23\\5\end{mymatrix}
    \dotprod \begin{mymatrix}{r}-5\\-6\\3\end{mymatrix} =
    113$.
  \end{sol}
\end{ex}

\begin{ex}
  Suppose $\vect{u},\vect{v}$, and $\vect{w}$ are three vectors whose
  components are all integers. Can you conclude the volume of the
  parallelepiped determined from these three vectors will always be an
  integer?
  \begin{sol}
    Yes. It will involve the sum of a product of integers and so it will
    be an integer.
  \end{sol}
\end{ex}

\begin{ex} \label{exer-box-product-zero}
  What does it mean geometrically if the box product of three vectors
  equals zero?
  \begin{sol}
    It means that if you place them so that they all have their tails
    at the same point, the three will lie in the same plane.
  \end{sol}
\end{ex}

\begin{ex}
  Show that
  \begin{equation*}
    (\vect{u}\times \vect{v}) \dotprod \vect{w}=
    (\vect{v}\times \vect{w}) \dotprod \vect{u}=
    (\vect{w}\times \vect{u}) \dotprod \vect{v}.
  \end{equation*}
  % \begin{sol}
  % \end{sol}
\end{ex}

\begin{ex}
  Use the formula from Exercise~\ref{ex:triple-cross-product} to show
  that
  \begin{equation*}
    (\vect{u}\times\vect{v}) \dotprod ((\vect{v}\times\vect{w}) \times
    (\vect{w}\times\vect{z})) =
    ((\vect{v}\times\vect{w})\dotprod\vect{z})\,((\vect{u}\times\vect{v})\dotprod\vect{w}).
  \end{equation*}
  \vspace{-4ex}
  \begin{sol}
    First, we have
    \begin{equation*}
      (\vect{v}\times\vect{w})\times(\vect{w}\times\vect{z})
      ~=~ ((\vect{v}\times\vect{w})\dotprod\vect{z})\,\vect{w}
          - ((\vect{v}\times\vect{w})\dotprod\vect{w})\,\vect{z}.
    \end{equation*}
    Since $\vect{v}\times\vect{w}$ is orthogonal to $\vect{w}$, their
    dot product $(\vect{v}\times\vect{w})\dotprod\vect{w}$ is zero,
    and therefore
    \begin{equation*}
      (\vect{v}\times\vect{w})\times(\vect{w}\times\vect{z})
      ~=~ ((\vect{v}\times\vect{w})\dotprod\vect{z})\,\vect{w}.
    \end{equation*}
    But then
    \begin{equation*}
      (\vect{u}\times\vect{v})\dotprod
      ((\vect{v}\times\vect{w})\times(\vect{w}\times\vect{z}))
      ~=~
      (\vect{u}\times\vect{v})\dotprod(((\vect{v}\times\vect{w})\dotprod\vect{z})\,\vect{w})
      ~=~ ((\vect{v}\times\vect{w})\dotprod\vect{z})\,((\vect{u}\times\vect{v})\dotprod\vect{w}).
    \end{equation*}

  \end{sol}
\end{ex}

\begin{ex}
  Simplify
  $\norm{\vect{u}\times \vect{v}} ^{2}+(\vect{u}\dotprod
    \vect{v}) ^{2}-\norm{\vect{u}} ^{2}\norm{ \vect{v}} ^{2}$.
  \begin{sol}
    We have
    \begin{eqnarray*}
      \norm{\vect{u}\times \vect{v}} ^{2}
      &=&\norm{\vect{u}
          } ^{2}\norm{\vect{v}} ^{2}\sin ^{2}\theta \\
      &=&\norm{\vect{u}} ^{2}\norm{\vect{v}}
          ^{2}(1-\cos ^{2}\theta) \\
      &=&\norm{\vect{u}} ^{2}\norm{\vect{v}}
          ^{2}-\norm{\vect{u}} ^{2}\norm{\vect{v}}
          ^{2}\cos ^{2}\theta \\
      &=&\norm{\vect{u}} ^{2}\norm{\vect{v}}
          ^{2}-(\vect{u}\dotprod \vect{v}) ^{2},
    \end{eqnarray*}
    which implies the expression equals $0$.
  \end{sol}
\end{ex}

\begin{ex}
  This problem uses calculus. For $\vect{u},\vect{v},\vect{w}$
  functions of $t$, prove that the derivative satisfies the following
  product rules:
  \begin{eqnarray*}
    (\vect{u}\times \vect{v}) ^{\prime }
    &=&\vect{u}^{\prime }\times
        \vect{v}+\vect{u}\times \vect{v}^{\prime } \\
    (\vect{u}\dotprod \vect{v}) ^{\prime }
    &=&\vect{u}^{\prime }\dotprod
        \vect{v}+\vect{u}\dotprod \vect{v}^{\prime }
  \end{eqnarray*}
\end{ex}

