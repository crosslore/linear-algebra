\section*{Exercises}


\begin{ex} Suppose you have $\R^{2}$ and the $+$ operation is as
follows:\ 
\begin{equation*}
(a,b) +(c,d) =(a+d,b+c) .
\end{equation*}
Scalar multiplication is defined in the usual way. Is this a vector space?
Explain why or why not. 
%\begin{sol}
%\end{sol}
\end{ex}

\begin{ex} Suppose you have $\R^{2}$ and the $+$ operation is defined as
follows. 
\begin{equation*}
(a,b) +(c,d) =(0,b+d)
\end{equation*}
Scalar multiplication is defined in the usual way. Is this a vector space?
Explain why or why not. 
%\begin{sol}
%\end{sol}
\end{ex}

\begin{ex} Suppose you have $\R^{2}$ and scalar multiplication is defined
as $c(a,b) =(a,cb) $ while vector addition is
defined as usual. Is this a vector space? Explain why or why not.
%\begin{sol}
%\end{sol}
\end{ex}

\begin{ex} Suppose you have $\R^{2}$ and the $+$ operation is defined as
follows. 
\begin{equation*}
(a,b) +(c,d) =(a-c,b-d)
\end{equation*}
Scalar multiplication is same as usual. Is this a vector space? Explain why
or why not. 
%\begin{sol}
%\end{sol}
\end{ex}

\begin{ex} \label{functions}Consider all the functions defined on a nonempty set
which have values in $\R$. Is this a vector space? Explain.
The operations are defined as follows. Here $f,g$ signify functions and $a$
is a scalar. 
\begin{eqnarray*}
(f+g) (x) &=&f(x) +g(x) \\
(af) (x) &=&a(f(x))
\end{eqnarray*}
%\begin{sol}
%\end{sol}
\end{ex}


\begin{ex} Denote by $\R^{\N}$ the set of real valued sequences.
For $\vect{a}\equiv \set{a_{n}} _{n=1}^{\infty },\vect{b}\equiv
\set{b_{n}} _{n=1}^{\infty }$ two of these, define their sum to be
given by 
\begin{equation*}
\vect{a}+\vect{b} =  \set{a_{n}+b_{n}} _{n=1}^{\infty }
\end{equation*}
and define scalar multiplication by 
\begin{equation*}
c\vect{a}=\set{ca_{n}} _{n=1}^{\infty }\text{ where }\vect{a}
=\set{a_{n}} _{n=1}^{\infty }
\end{equation*}
Is this a special case of Problem~\ref{functions}? Is this a vector space?
%\begin{sol}
%\end{sol}
\end{ex}

\begin{ex} Let $\C^{2}$ be the set of ordered pairs of complex numbers.
Define addition and scalar multiplication in the usual way.
\begin{equation*}
(z,w) +(\hat{z},\hat{w}) = (z+\hat{z},w+
\hat{w}) ,\ u(z,w) \equiv (uz,uw)
\end{equation*}
Here the scalars are from $\C$. Show this is a vector space.
%\begin{sol}
%\end{sol}
\end{ex}

\begin{ex} Let $V$ be the set of functions defined on a nonempty set which have
values in a vector space $W$. Is this a vector space? Explain.
%\begin{sol}
%\end{sol}
\end{ex}

\begin{ex} Consider the space of $m\times n$-matrices with operation of addition
and scalar multiplication defined the usual way. That is, if $A,B$ are two $
m\times n$-matrices and $c$ a scalar, 
\begin{equation*}
(A+B) _{ij}=A_{ij}+B_{ij},\ (cA) _{ij}\equiv c(
A_{ij})
\end{equation*}
%\begin{sol}
%\end{sol}
\end{ex}

\begin{ex} Consider the set of $n\times n$ symmetric matrices. That is, $A=A^{T}$.
In other words, $A_{ij}=A_{ji}$. Show that this set of symmetric matrices is
a vector space and a subspace of the vector space of $n\times n$-matrices.
%\begin{sol}
%\end{sol}
\end{ex}

\begin{ex} Consider the set of all vectors in $\R^{2},(x,y) $
such that $x+y\geq 0$. Let the vector space operations be the usual ones. Is
this a vector space? Is it a subspace of $\R^{2}$?
%\begin{sol}
%\end{sol}
\end{ex}

\begin{ex} Consider the vectors in $\R^{2},(x,y) $ such that $xy=0$. Is this a subspace of $\R^{2}$? Is it a vector space? The
addition and scalar multiplication are the usual operations.
%\begin{sol}
%\end{sol}
\end{ex}

\begin{ex} Define the operation of vector addition on $\R^{2}$ by $(
x,y) +(u,v) =(x+u,y+v+1)$. Let scalar
multiplication be the usual operation. Is this a vector space with these
operations? Explain.
%\begin{sol}
%\end{sol}
\end{ex}

\begin{ex} Let the vectors be real numbers. Define vector space operations in the
usual way. That is $x+y$ means to add the two numbers and $xy$ means to
multiply them. Is $\R$ with these operations a vector space? Explain.
%\begin{sol}
%\end{sol}
\end{ex}

\begin{ex} Let the scalars be the rational numbers and let the vectors be
real numbers which are the form $a+b\sqrt{2}$ for $a,b$ rational numbers.
Show that with the usual operations, this is a vector space.
%\begin{sol}
%\end{sol}
\end{ex}

\begin{ex} Let $\Poly_{2}$ be the set of all polynomials of degree 2 or
less. That is, these are of the form $a+bx+cx^{2}$. Addition is defined as 
\begin{equation*}
(a+bx+cx^{2}) +(\hat{a}+\hat{b}x+\hat{c}x^{2})
=(a+\hat{a}) +(b+\hat{b}) x+(c+\hat{c})
x^{2}
\end{equation*}
and scalar multiplication is defined as 
\begin{equation*}
d(a+bx+cx^{2}) =da+dbx+cdx^{2}
\end{equation*}
Show that, with this definition of the vector space operations that $\Poly_{2}$ is a vector space. Now let $V$ denote those polynomials $a+bx+cx^{2}$
such that $a+b+c=0$. Is $V$ a subspace of $\Poly_{2}$? Explain.
%\begin{sol}
%\end{sol}
\end{ex}

\begin{ex} Let $M,N$ be subspaces of a vector space $V$ and consider $M+N$
defined as the set of all $m+n$ where $m\in M$ and $n\in N$. Show that $M+N$
is a subspace of $V$.
%\begin{sol}
%\end{sol}
\end{ex}

\begin{ex} Let $M,N$ be subspaces of a vector space $V$. Then $M\cap N$ consists
of all vectors which are in both $M$ and $N$. Show that $M\cap N$ is a
subspace of $V$.
%\begin{sol}
%\end{sol}
\end{ex}

\begin{ex} Let $M,N$ be subspaces of a vector space $\R^{2}$. Then $N\cup
M$ consists of all vectors which are in either $M$ or $N$. Show that $N\cup
M $ is not necessarily a subspace of $\R^{2}$ by giving an example
where $N\cup M$ fails to be a subspace.
%\begin{sol}
%\end{sol}
\end{ex}


\begin{ex} \label{4-july-prob1}Let $X$ consist of the real valued functions which
are defined on an interval $\mat{a,b}$. For $f,g\in X,\;f+g$ is the
name of the function which satisfies $(f+g) (x)
=f(x) +g(x)$. For $s$ a real number, $
(s f) (x) = s (f(x)
)$. Show this is a vector space. 
\begin{sol}
The axioms of a vector space all hold because they
hold for a vector space. The only thing left to verify is the
assertions about the things which are supposed to exist. $0$ would
be the zero function which sends everything to $0$. This is an additive
identity. Now if $f$ is a function, $-f(x) \equiv (
-f(x))$. Then
\[
(f+(-f)) (x) \equiv f(x)
+(-f) (x) \equiv f(x) +(-f(
x)) =0
\]
Hence $f+-f=\mathbf{0}$. For each $x\in \mat{a,b}$, let $%
f_{x}(x) =1$ and $f_{x}(y) =0$ if $y\neq x$. Then
these vectors are obviously linearly independent.
\end{sol}
\end{ex}

\begin{ex} Consider functions defined on $\set{1,2,\ldots,n} $ having
values in $\R$. Explain how, if $V$ is the set of all such
functions, $V$ can be considered as $\R^{n}$.
\begin{sol}
Let $f(i) $ be the $i\th$ component of a vector $
\vect{x}\in \R^{n}$. Thus a typical element in $\R^{n}$ is $
(f(1) ,\ldots,f(n))$.
\end{sol}
\end{ex}

\begin{ex} Let the vectors be polynomials of degree no more than 3. Show that
with the usual definitions of scalar multiplication and addition wherein,
for $p(x) $ a polynomial, $(a p) (
x) = a p(x) $ and for $p,q$ polynomials $(
p+q) (x) =  p(x) +q(x)$, this
is a vector space.
\begin{sol}
This is just a subspace of the vector space of functions
because it is closed with respect to vector addition and scalar
multiplication. Hence this is a vector space.
\end{sol}
\end{ex}

