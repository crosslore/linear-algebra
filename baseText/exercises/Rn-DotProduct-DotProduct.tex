\begin{enumialphparenastyle}

\begin{ex} Find $\begin{mymatrix}{r}
1 \\
2 \\
3 \\
4
\end{mymatrix} \dotprod \begin{mymatrix}{r}
2 \\
0 \\
1 \\
3
\end{mymatrix}$.
\begin{sol}
$\begin{mymatrix}{r}
1 \\
2 \\
3 \\
4
\end{mymatrix} \dotprod \begin{mymatrix}{r}
2 \\
0 \\
1 \\
3
\end{mymatrix} = 17$.
\end{sol}
\end{ex}

\begin{ex} Use the formula given in Proposition~\ref{prop:dotproduct-angle} to verify the Cauchy Schwarz inequality and
to show that equality occurs if and only if one of the vectors is a scalar
multiple of the other.
\begin{sol}
This formula says that $\vect{u}\dotprod \vect{v}=\norm{
\vect{u}} \norm{\vect{v}} \cos \theta $ where $
\theta $ is the included angle between the two vectors. Thus
\begin{equation*}
\norm{\vect{u}\dotprod \vect{v}} =\norm{\vect{u}}
\norm{\vect{v}} \norm{\cos \theta} \leq
\norm{\vect{u}} \norm{\vect{v}}
\end{equation*}
and equality holds if and only if $\theta =0$ or $\pi$. This means that the
two vectors either point in the same direction or opposite directions. Hence
one is a multiple of the other.
\end{sol}
\end{ex}

\begin{ex} For $\vect{u},\vect{v}$ vectors in $\R^{3}$, define the product, 
$\vect{u}\ast \vect{v} =  u_{1}v_{1}+2u_{2}v_{2}+3u_{3}v_{3}$. Show the axioms
for a dot product all hold for this product. Prove
\begin{equation*}
\norm{\vect{u}\ast \vect{v}} \leq (\vect{u}\ast \vect{u})
^{1/2}(\vect{v}\ast \vect{v}) ^{1/2}.
\end{equation*}
\begin{sol}
This
follows from the Cauchy Schwarz inequality and the proof of Theorem~\ref
{thm:cauchy-schwarz-inequality} which only used the properties of the dot product. Since this new
product has the same properties the Cauchy Schwarz inequality holds for it
as well.
\end{sol}
\end{ex}


\begin{ex} Let $\vect{a},\vect{b}$ be vectors. Show that $(\vect{a}\dotprod \vect{b}) =\frac{1}{4}(\norm{
\vect{a}+\vect{b}} ^{2}-\norm{\vect{a}-\vect{b}} ^{2})$.
%\begin{sol}
%\end{sol}
\end{ex}

\begin{ex} Using the axioms of the dot product, prove the parallelogram identity: 
\begin{equation*}
\norm{\vect{a}+\vect{b}} ^{2}+\norm{\vect{a}-\vect{b}}
^{2}=2\norm{\vect{a}} ^{2}+2\norm{\vect{b}
} ^{2}
\end{equation*}
%\begin{sol}
%\end{sol}
\end{ex}

\end{enumialphparenastyle}
