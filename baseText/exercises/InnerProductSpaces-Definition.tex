\section*{Exercises}

\begin{ex}\label{ex:hilbert-space}.
  In this exercise, we will work out the details of
  Example~\ref{exa:hilbert-space}. We must show that $\Hilb_{\R}$ is a
  vector space. We will do this by showing that it is a subspace of
  $\Seq_{\R}$. Further, we must show that the inner product is
  well-defined. This requires some knowledge of convergent series from
  calculus.
  \begin{enumerate}
  \item Assume $a=(a_0,a_1,\ldots)$ and $b=(b_0,b_1,\ldots)$ are
    square summable sequences. Show that the series
    $a_0b_0 + a_1b_1 + a_2b_2 + \ldots$ converges to a real
    number. Hint: consider the series
    $\abs{a_0b_0} + \abs{a_1b_2} + \ldots$ and use the Cauchy-Schwarz
    inequality and the absolute convergence test.
  \item Using the result of part (a), show that $\Hilb_{\R}$ is a
    subspace of $\Seq_{\R}$. 
  \item Show that the operation $\iprod{a,b} = a_0b_0 + a_1b_1 +
    a_2b_2 + \ldots$ is an inner product on $\Hilb_{\R}$.
  \end{enumerate}
  \begin{sol}
    (a) By assumption, $a$ and $b$ are square summable. Let
    $N=a_0^2 + a_1^2 + \ldots$ and $M=b_0^2 + b_1^2 + \ldots$.
    By the Cauchy-Schwarz inequality, for all $n$, we have
    \begin{equation*}
      \abs{a_0}\abs{b_0} + \ldots + \abs{a_n}\abs{b_n}
      ~\leq~ \sqrt{\abs{a_0}^2 + \ldots + \abs{a_n}^2} \sqrt{\abs{b_0}^2 + \ldots + \abs{b_n}^2}
      ~\leq~ \sqrt{N}\sqrt{M}.
    \end{equation*}
    Therefore the series $\abs{a_0b_0} + \abs{a_1b_1} + \ldots$ is
    bounded. By the absolute convergence test from calculus, it
    follows that the series $a_0b_0 + a_1b_1 + \ldots$ converges.

    (b) It is clear that the zero sequence is square summable, and
    also that a scalar multiple of a square summable sequence is
    square summable. Hence $\Hilb_{\R}$ contains the zero vector and
    is closed under scalar multiplication. To show that it is closed
    under addition, assume $a,b\in\Hilb_{\R}$, and let $c=a+b$. We
    must show that $c$ is square summable. But
    \begin{equation*}
      c_0^2 + c_1^2 + \ldots
      ~=~
      (a_0+b_0)^2 + (a_1+b_1)^2 + \ldots
      ~=~ (a_0^2 + a_1^2 + \ldots) + (b_0^2 + b_1^2 + \ldots)
      + (2a_0b_0 + 2a_1b_1 + \ldots).
    \end{equation*}
    The series $a_0^2 + a_1^2 + \ldots$ and $b_0^2 + b_1^2 + \ldots$
    converge by assumption, and the series $2a_0b_0 + 2a_1b_1 +
    \ldots$ converges by part (a). It follows that $\Hilb_{\R}$ is
    closed under addition.

    (c) Symmetry and linearity follow straightforwardly from
    properties of convergent series. For example,
    \begin{eqnarray*}
      \iprod{a,kb+\ell c}
      &=& a_0(kb_0+\ell c_0) + a_1(kb_1+\ell c_1) + \ldots \\
      &=& k(a_0b_0 + a_1b_1 + \ldots) + \ell(a_0c_0 + a_1c_1 + \ldots) \\
      &=& k\iprod{a,b} + \ell\iprod{a,c}.
    \end{eqnarray*}
    As for the positive definite property, note that
    \begin{equation*}
      \iprod{a,a}
      ~=~ a_0^2 + a_1^2 + \ldots
      ~\geq~ 0.      
    \end{equation*}
    Moreover, since all terms in the series are $\geq 0$, it follows
    that $\iprod{a,a}=0$ if and only if $a_i=0$ for all $i$.
  \end{sol}
\end{ex}

\begin{ex}
  For $\vect{u},\vect{v}$ vectors in $\R^3$, define the product
  $\vect{u}\ast \vect{v} = u_1v_1+2u_2v_2+3u_3v_3$. Show
  that 
  \begin{equation*}
    \abs{\vect{u}\ast \vect{v}} \leq (\vect{u}\ast \vect{u})^{1/2}
    (\vect{v}\ast \vect{v})^{1/2}.
  \end{equation*}
  Hint: first show that the operation
  $\iprod{\vect{u},\vect{v}}=\vect{u}\ast\vect{v}$ is an inner product
  on $\R^3$, then use Proposition~\ref{prop:inner-product-and-norm}.
  \begin{sol}
    The operation $\iprod{\vect{u},\vect{v}}=\vect{u}\ast\vect{v}$ is
    an inner product on $\R^3$. Symmetry and linearity are
    straightforward to check, as is the positive definite
    property. Therefore, the claimed inequality holds by
    Proposition~\ref{prop:cauchy-schwarz-inequality}.
  \end{sol}
\end{ex}

