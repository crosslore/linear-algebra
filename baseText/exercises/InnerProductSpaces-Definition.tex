\section*{Exercises}

\begin{ex}
  For each matrix $A$, determine whether the formula
  $\iprod{\vect{u},\vect{v}} = \vect{u}^T A\vect{v}$ determines an
  inner product on $\R^2$.
  \begin{equation*}
    (a)~ A = \begin{mymatrix}{cc} 1 & 0 \\ 0 & 2 \end{mymatrix},\quad
    (b)~ A = \begin{mymatrix}{cc} 3 & 1 \\ 1 & 3 \end{mymatrix},\quad
    (c)~ A = \begin{mymatrix}{cc} 1 & 1 \\ 0 & 2 \end{mymatrix},\quad
    (d)~ A = \begin{mymatrix}{cc} 1 & 0 \\ 0 & -1 \end{mymatrix},\quad
    (e)~ A = \begin{mymatrix}{cc} 1 & 1 \\ 1 & 1 \end{mymatrix}.
  \end{equation*}
  \begin{sol}
    \begin{enumerate}
    \item Yes, it is an inner product. Since $A=A^T$, symmetry and
      linearity follow as in Example~\ref{exa:rn-with-inner-product}.
      For the positive definite property, note that 
      $\iprod{\vect{u},\vect{u}} = u_1^2 + 2u_2^2\geq 0$, and equality
      holds if and only if $u_1,u_2=0$.
    \item Yes, it is an inner product. Since $A=A^T$, symmetry and
      linearity follow as in Example~\ref{exa:rn-with-inner-product}.
      For the positive definite property, we have
      $\iprod{\vect{u},\vect{u}} = 3u_1^2 + 2u_1u^2 + 3u_2^2 =
      (u_1+u_2)^2 + 2u_1^2 + 2u_2^2 \geq 0$. Equality holds if and
      only if $u_1+u_2$, $u_1$, and $u_2$ are all zero, which is the
      case if and only if $\vect{u}=0$.
    \item No, it is not symmetric. For example,
      $\iprod{
        \begin{mymatrix}{c}1\\0\end{mymatrix},
        \begin{mymatrix}{c}0\\1\end{mymatrix}
      } = 1$
      but
      $\iprod{
        \begin{mymatrix}{c}0\\1\end{mymatrix},
        \begin{mymatrix}{c}1\\0\end{mymatrix}
      } = 0$.
    \item No, it is not positive definite. For example,
      $\iprod{
        \begin{mymatrix}{c}0\\1\end{mymatrix},
        \begin{mymatrix}{c}0\\1\end{mymatrix}
      } = -1$.
    \item No, it is not positive definite. For example,
      $\iprod{
        \begin{mymatrix}{c}1\\-1\end{mymatrix},
        \begin{mymatrix}{c}1\\-1\end{mymatrix}
      } = 0$
      although $\begin{mymatrix}{c}1\\-1\end{mymatrix}\neq \vect{0}$.
    \end{enumerate}
  \end{sol}
\end{ex}

\begin{ex}
  Consider the inner product space $C[0,1]$ as in
  Example~\ref{exa:continuous-interval}. Compute the following inner
  products:
  \begin{equation*}
    (a)~ \iprod{1,x}, \quad
    (b)~ \iprod{x,x^2}, \quad
    (c)~ \iprod{1+x,2+x^2}.
  \end{equation*}
  \begin{sol}
    \begin{enumerate}
    \item $\iprod{1,x} = \int_{0}^{1} 1\cdot x\,dx
      = \int_{0}^{1} x\,dx = \frac{1}{2}$.
    \item $\iprod{x,x^2} = \int_{0}^{1} x\cdot x^2\,dx
      = \int_{0}^{1} x^3\,dx = \frac{1}{4}$.
    \item $\iprod{1+x,2+x^2}
      = \int_{0}^{1} (1+x)(2+x^2)\,dx
      = \int_{0}^{1} 2 + x^2 + 2x + x^3\,dx
      = \frac{43}{12}.$
    \end{enumerate}
  \end{sol}
\end{ex}

\begin{ex}
  Consider the inner product space $C[0,1]$ as in
  Example~\ref{exa:continuous-interval}. Compute the following norms:
  \begin{equation*}
    (a)~ \norm{1}, \quad
    (b)~ \norm{x}, \quad
    (c)~ \norm{x^2+1}.
  \end{equation*}
  \begin{sol}
    \begin{enumerate}
    \item $\norm{1}^2 = \iprod{1,1} = \int_{0}^{1} 1\,dx
      = 1$, therefore $\norm{1}=1$.
    \item $\norm{x}^2 = \iprod{x,x} = \int_{0}^{1} x^2\,dx
      = \frac{1}{3}$, therefore $\norm{x} = \frac{1}{\sqrt{3}}$.
    \item $\norm{x^2+1}^2 = \iprod{x^2+1,x^2+1}
      = \int_{0}^{1} x^4 + 2x^2 + 1\,dx
      = \frac{1}{5} + \frac{2}{3} + 1 = \frac{28}{15}$, therefore
      $\norm{x^2+1} = \sqrt{\frac{28}{15}}$.
    \end{enumerate}
  \end{sol}
\end{ex}

\begin{ex}
  For $\vect{u},\vect{v}$ vectors in $\R^3$, define the product
  $\vect{u}\ast \vect{v} = u_1v_1+2u_2v_2+3u_3v_3$. Show
  that
  \begin{equation*}
    \abs{\vect{u}\ast \vect{v}} \leq (\vect{u}\ast \vect{u})^{1/2}
    (\vect{v}\ast \vect{v})^{1/2}.
  \end{equation*}
  Hint: first show that the operation
  $\iprod{\vect{u},\vect{v}}=\vect{u}\ast\vect{v}$ is an inner product
  on $\R^3$, then use Proposition~\ref{prop:inner-product-and-norm}.
  \begin{sol}
    The operation $\iprod{\vect{u},\vect{v}}=\vect{u}\ast\vect{v}$ is
    an inner product on $\R^3$. Symmetry and linearity are
    straightforward to check, as is the positive definite
    property. Therefore, the claimed inequality holds by
    Proposition~\ref{prop:cauchy-schwarz-inequality}.
  \end{sol}
\end{ex}

\begin{ex}
  In $C[-1,1]$, find (a) the angle between $x$ and $x^2$, (b) the
  angle between $x$ and $x^3$.
  \begin{sol}
    \begin{enumerate}
    \item We have $\iprod{x,x} = \int_{-1}^1 x^2\,dx = \frac{2}{3}$,
      $\iprod{x^2,x^2} = \int_{-1}^1 x^4\,dx = \frac{2}{5}$, and
      $\iprod{x, x^2} = \int_{-1}^1 x^3\,dx = 0$.  Therefore
      \begin{equation*}
        \cos\theta
        ~=~ \frac{\iprod{x,x^2}}{\norm{x}\norm{x^2}}
        ~=~ 0.
      \end{equation*}
      Therefore, the angle $\theta$ is $\pi/2$ radians, or $90$
      degrees. In other words, $x$ and $x^2$ are orthogonal in
      $C[-1,1]$.
    \item We have $\iprod{x,x} = \int_{-1}^1 x^2\,dx = \frac{2}{3}$,
      $\iprod{x^3,x^3} = \int_{-1}^1 x^6\,dx = \frac{2}{7}$, and
      $\iprod{x, x^3} = \int_{-1}^1 x^4\,dx = \frac{2}{5}$.  Therefore
      \begin{equation*}
        \cos\theta
        ~=~ \frac{\iprod{x,x^3}}{\norm{x}\norm{x^3}}
        ~=~ \frac{\frac{2}{5}}{\sqrt{\frac{2}{3}}\sqrt{\frac{2}{7}}}
        ~=~ \frac{\sqrt{21}}{5}.
      \end{equation*}
      The angle $\theta$ is $\cos^{-1}(\frac{\sqrt{21}}{5})$, which is
      approximately $0.4115$ radians or $23.58$ degrees.
    \end{enumerate}
  \end{sol}    
\end{ex}

\begin{ex}\label{ex:hilbert-space}.
  In this exercise, we will work out the details of
  Example~\ref{exa:hilbert-space}. We must show that $\Hilb_{\R}$ is a
  vector space. We will do this by showing that it is a subspace of
  $\Seq_{\R}$. Further, we must show that the inner product is
  well-defined. This requires some knowledge of convergent series from
  calculus.
  \begin{enumerate}
  \item Assume $a=(a_0,a_1,\ldots)$ and $b=(b_0,b_1,\ldots)$ are
    square summable sequences. Show that the series
    $a_0b_0 + a_1b_1 + a_2b_2 + \ldots$ converges to a real
    number. Hint: consider the series
    $\abs{a_0b_0} + \abs{a_1b_2} + \ldots$ and use the Cauchy-Schwarz
    inequality and the absolute convergence test.
  \item Using the result of part (a), show that $\Hilb_{\R}$ is a
    subspace of $\Seq_{\R}$.
  \item Show that the operation $\iprod{a,b} = a_0b_0 + a_1b_1 +
    a_2b_2 + \ldots$ is an inner product on $\Hilb_{\R}$.
  \end{enumerate}
  \begin{sol}
    (a) By assumption, $a$ and $b$ are square summable. Let
    $N=a_0^2 + a_1^2 + \ldots$ and $M=b_0^2 + b_1^2 + \ldots$.
    By the Cauchy-Schwarz inequality, for all $n$, we have
    \begin{equation*}
      \abs{a_0}\abs{b_0} + \ldots + \abs{a_n}\abs{b_n}
      ~\leq~ \sqrt{\abs{a_0}^2 + \ldots + \abs{a_n}^2} \sqrt{\abs{b_0}^2 + \ldots + \abs{b_n}^2}
      ~\leq~ \sqrt{N}\sqrt{M}.
    \end{equation*}
    Therefore the series $\abs{a_0b_0} + \abs{a_1b_1} + \ldots$ is
    bounded. By the absolute convergence test from calculus, it
    follows that the series $a_0b_0 + a_1b_1 + \ldots$ converges.

    (b) It is clear that the zero sequence is square summable, and
    also that a scalar multiple of a square summable sequence is
    square summable. Hence $\Hilb_{\R}$ contains the zero vector and
    is closed under scalar multiplication. To show that it is closed
    under addition, assume $a,b\in\Hilb_{\R}$, and let $c=a+b$. We
    must show that $c$ is square summable. But
    \begin{equation*}
      c_0^2 + c_1^2 + \ldots
      ~=~
      (a_0+b_0)^2 + (a_1+b_1)^2 + \ldots
      ~=~ (a_0^2 + a_1^2 + \ldots) + (b_0^2 + b_1^2 + \ldots)
      + (2a_0b_0 + 2a_1b_1 + \ldots).
    \end{equation*}
    The series $a_0^2 + a_1^2 + \ldots$ and $b_0^2 + b_1^2 + \ldots$
    converge by assumption, and the series $2a_0b_0 + 2a_1b_1 +
    \ldots$ converges by part (a). It follows that $\Hilb_{\R}$ is
    closed under addition.

    (c) Symmetry and linearity follow straightforwardly from
    properties of convergent series. For example,
    \begin{eqnarray*}
      \iprod{a,kb+\ell c}
      &=& a_0(kb_0+\ell c_0) + a_1(kb_1+\ell c_1) + \ldots \\
      &=& k(a_0b_0 + a_1b_1 + \ldots) + \ell(a_0c_0 + a_1c_1 + \ldots) \\
      &=& k\iprod{a,b} + \ell\iprod{a,c}.
    \end{eqnarray*}
    As for the positive definite property, note that
    \begin{equation*}
      \iprod{a,a}
      ~=~ a_0^2 + a_1^2 + \ldots
      ~\geq~ 0.
    \end{equation*}
    Moreover, since all terms in the series are $\geq 0$, it follows
    that $\iprod{a,a}=0$ if and only if $a_i=0$ for all $i$.
  \end{sol}
\end{ex}
