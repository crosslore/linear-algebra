\Opensolutionfile{solutions}[ex]
\section*{Exercises}

\begin{enumialphparenastyle}

\begin{ex} An operation is done to get from the first matrix to the second.
Identify what was done and tell how it will affect the value of the
determinant. 
\begin{equation*}
\begin{mymatrix}{cc}
a & b \\
c & d
\end{mymatrix}  \sim \ldots \sim \begin{mymatrix}{cc}
a & c \\
b & d
\end{mymatrix}
\end{equation*}
\begin{sol}
It does not change the determinant. This was just taking the transpose.
\end{sol}
\end{ex}

\begin{ex} An operation is done to get from the first matrix to the second.
Identify what was done and tell how it will affect the value of the
determinant. 
\begin{equation*}
\begin{mymatrix}{cc}
a & b \\
c & d
\end{mymatrix} \sim \ldots \sim \begin{mymatrix}{cc}
c & d \\
a & b
\end{mymatrix}
\end{equation*}
\begin{sol}
In this case two rows were switched and so the resulting determinant is $-1$
times the first.
\end{sol}
\end{ex}


\begin{ex} An operation is done to get from the first matrix to the second.
Identify what was done and tell how it will affect the value of the
determinant. 
\begin{equation*}
\begin{mymatrix}{cc}
a & b \\
c & d
\end{mymatrix} \sim \ldots \sim \begin{mymatrix}{cc}
a & b \\
a+c & b+d
\end{mymatrix}
\end{equation*}
\begin{sol}
The determinant is unchanged. It was just the first row added to the second.
\end{sol}
\end{ex}


\begin{ex} An operation is done to get from the first matrix to the second.
Identify what was done and tell how it will affect the value of the
determinant. 
\begin{equation*}
\begin{mymatrix}{cc}
a & b \\
c & d
\end{mymatrix} \sim \ldots \sim \begin{mymatrix}{cc}
a & b \\
2c & 2d
\end{mymatrix}
\end{equation*}
\begin{sol}
The second row was multiplied by 2 so the determinant of the result is 2
times the original determinant.
\end{sol}
\end{ex}

\begin{ex} An operation is done to get from the first matrix to the second.
Identify what was done and tell how it will affect the value of the
determinant. 
\begin{equation*}
\begin{mymatrix}{cc}
a & b \\
c & d
\end{mymatrix} \sim \ldots \sim \begin{mymatrix}{cc}
b & a \\
d & c
\end{mymatrix}
\end{equation*}
\begin{sol}
In this case the two columns were switched so the determinant of the second
is $-1$ times the determinant of the first.
\end{sol}
\end{ex}


\begin{ex} Let $A$ be an $r\times r$-matrix and suppose there are $r-1$ rows
(columns) such that all rows (columns) are linear combinations of these $r-1$
rows (columns). Show $\det (A) =0$. 
\begin{sol}
If the determinant is non-zero, then it will remain non-zero with row operations applied to the matrix.
However, by assumption, you can obtain a row of zeros by doing row
operations. Thus the determinant must have been zero after all.
\end{sol}
\end{ex}

\begin{ex} Show $\det (aA) =a^{n}\det (A) $ for an $n \times n $-matrix $A
$ and scalar $a$. 
\begin{sol}
$\det (aA) =\det
(aIA) =\det (aI) \det (A) =a^{n}\det
(A)$. The matrix which has $a$ down the main diagonal has
determinant equal to $a^{n}$.
\end{sol}
\end{ex}


\begin{ex} Construct $2\times 2$-matrices $A$ and $B$ to show that the
$\det A \det B = \det (AB)$. 
\begin{sol}
\begin{equation*}
\det
\paren{\begin{mymatrix}{cc}
1 & 2 \\
3 & 4
\end{mymatrix} \begin{mymatrix}{rr}
-1 & 2 \\
-5 & 6
\end{mymatrix}} = -8
\end{equation*}
\begin{equation*}
\det \begin{mymatrix}{cc}
1 & 2 \\
3 & 4
\end{mymatrix} \det \begin{mymatrix}{rr}
-1 & 2 \\
-5 & 6
\end{mymatrix} = -2 \times 4 = -8
\end{equation*}
\end{sol}
\end{ex}

\begin{ex} Is it true that $\det (A+B) =\det (A) +\det
(B)$? If this is so, explain why. If it is not so,
give a counter example.  
\begin{sol}
This is not true at all. Consider $A=\begin{mymatrix}{cc}
1 & 0 \\
0 & 1
\end{mymatrix} ,B=\begin{mymatrix}{rr}
-1 & 0 \\
0 & -1
\end{mymatrix}$.
\end{sol}
\end{ex}

\begin{ex} An $n\times n$-matrix is called \textbf{nilpotent}
\index{nilpotent} if for some positive integer, $k$ it follows $A^{k}=0$. If
$A$ is a nilpotent matrix and $k$ is the smallest possible integer such that
$A^{k}=0$, what are the possible values of $\det (A)$? 
\begin{sol}
It must
be 0 because $0=\det (0) =\det (A^{k}) =(\det
(A)) ^{k}$.
\end{sol}
\end{ex}

\begin{ex} \label{exer-orthogonal}A matrix is said to be \textbf{orthogonal} \index{matrix!orthogonal} if 
$A^{T}A=I$. Thus the inverse of an orthogonal matrix is just its transpose.
What are the possible values of $\det (A) $ if $A$ is an
orthogonal matrix? 
\begin{sol}
You would need $\det (AA^{T}) =\det
(A) \det (A^{T}) =\det (A) ^{2}=1$ and
so $\det (A) =1$, or $-1$.
\end{sol}
\end{ex}

\begin{ex} Let $A$ and $B$ be two $n\times n$-matrices. $A\sim B$
($A$ is \textbf{similar} to $B$) means there exists an invertible matrix $P$
such that $A=P^{-1}BP$. Show that if $A\sim B$, then 
$\det (A) =\det (B)$. 
\begin{sol}
$\det (A) =\det
(S^{-1}BS) =\det (S^{-1}) \det (B) \det
(S) =\det (B) \det (S^{-1}S) =\det
(B)$.
\end{sol}
\end{ex}

\begin{ex} Tell whether each statement is true or false. If true, provide a proof. If false, provide a counter example. 
\begin{enumerate}
\item If $A$ is a $3\times 3$-matrix with a zero determinant, then one
column must be a multiple of some other column.

\item If any two columns of a square matrix are equal, then the determinant
of the matrix equals zero.

\item For two $n\times n$-matrices $A$ and $B$, $\det (A+B)
=\det (A) +\det (B)$.

\item For an $n\times n$-matrix $A$, $\det (3A) =3\det (
A) $

\item If $A^{-1}$ exists then $\det (A^{-1}) =\det (
A) ^{-1}$.

\item If $B$ is obtained by multiplying a single row of $A$ by $4$ then $%
\det (B) =4\det (A)$.

\item For $A$ an $n\times n$-matrix, $\det (-A) =(
-1) ^{n}\det (A)$.

\item If $A$ is a real $n\times n$-matrix, then $\det (A^{T}A)
\geq 0$.

\item If $A^{k}=0$ for some positive integer $k$, then $\det (
A) =0$.

\item If $AX=0$ for some $X \neq 0$, then $\det (
A) =0$.
\end{enumerate}
\begin{sol}
\begin{enumerate}
\item False. Consider $\begin{mymatrix}{rrr}
1 & 1 & 2 \\
-1 & 5 & 4 \\
0 & 3 & 3
\end{mymatrix} $
\item True.
\item False.
\item False.
\item True.
\item True.
\item True.
\item True.
\item True.
\item True.
\end{enumerate}
\end{sol}
\end{ex}

\end{enumialphparenastyle}
