\section*{Exercises}

\begin{ex}
  Let
  \begin{equation*}
    A = \begin{mymatrix}{cc}
      a & b \\
      c & d \\
    \end{mymatrix}.
  \end{equation*}
  An operation is done to get from $A$ to a matrix $B$. In each case,
  identify which operation was done and explain how it will affect the
  value of the determinant.
  \begin{enumerate}
  \item
    \begin{equation*}
      B = \begin{mymatrix}{cc}
        a & c \\
        b & d \\
      \end{mymatrix}
    \end{equation*}
  \item
    \begin{equation*}
      B = \begin{mymatrix}{cc}
        c & d \\
        a & b \\
      \end{mymatrix}
    \end{equation*}
  \item
    \begin{equation*}
      B = \begin{mymatrix}{cc}
        a   & b   \\
        a+c & b+d \\
      \end{mymatrix}
    \end{equation*}
  \item
    \begin{equation*}
      B = \begin{mymatrix}{cc}
        a  & b  \\
        2c & 2d \\
      \end{mymatrix}
    \end{equation*}
  \item
    \begin{equation*}
      B = \begin{mymatrix}{cc}
        b & a \\
        d & c \\
      \end{mymatrix}
    \end{equation*}
  \end{enumerate}
  \begin{sol}
    \begin{enumerate}
    \item The transpose was taken and $\det(B) = \det(A)$.
    \item Two rows were switched and $\det(B) = -\det(A)$.
    \item The first row was added to the second row and $\det(B) = \det(A)$.
    \item The second row was multiplied by 2 and $\det(B) = 2\det(A)$.
    \item Two columns were switched and $\det(B) = -\det(A)$.
    \end{enumerate}
  \end{sol}
\end{ex}

\begin{ex}
  Let $A$ be an $n\times n$-matrix and suppose there are $n-1$ rows
  such that the remaining row is a linear combinations of these $n-1$
  rows. Show $\det(A) = 0$.
  \begin{sol}
    By assumption, we can obtain a row of zeros by doing row
    operations. Row operations do not change whether the determinant
    is zero, so the determinant must have been zero all along.
  \end{sol}
\end{ex}

\begin{ex}
  \label{ex:determinant3}
  Let $A$ be an $n\times n$-matrix. Show that if $\det(A) \neq 0$ and
  $A\vect{x}=\vect{0}$, then $\vect{x}=\vect{0}$.
  \begin{sol}
    If $\det(A) \neq 0$, then $A^{-1}$ exists. Therefore
    $\vect{x} = A^{-1}A\vect{x} = A^{-1}\vect{0} = \vect{0}$.
  \end{sol}
\end{ex}

\begin{ex}
  Using only Theorems~\ref{thm:determinant-of-triangular-matrix} and
  {\ref{thm:determinant-of-product}}, show that
  $\det(kA) = k^n\det(A)$ for an $n\times n$-matrix $A$ and scalar
  $k$.
  \begin{sol}
    The matrix $kI$ has $k$ down the main diagonal and has determinant
    equal to $k^n$ by
    Theorem~\ref{thm:determinant-of-triangular-matrix}. Using
    Theorem~\ref{thm:determinant-of-product}, it follows that
    $\det(kA) = \det(kIA) = \det(kI) \det(A) = k^n\det(A)$.
  \end{sol}
\end{ex}

\begin{ex}
  Construct two random $2\times 2$-matrices $A$ and $B$ and verify
  that $\det(A)\det(B) = \det(AB)$.
  \begin{sol}
    \begin{equation*}
      \det
      \paren{\begin{mymatrix}{cc}
          1 & 2 \\
          3 & 4
        \end{mymatrix} \begin{mymatrix}{rr}
          -1 & 2 \\
          -5 & 6
        \end{mymatrix}} = -8,
    \end{equation*}
    \begin{equation*}
      \det\paren{\begin{mymatrix}{cc}
        1 & 2 \\
        3 & 4
      \end{mymatrix}}\det\paren{\begin{mymatrix}{rr}
        -1 & 2 \\
        -5 & 6
      \end{mymatrix}} = -2\times 4 = -8.
    \end{equation*}
  \end{sol}
\end{ex}

\begin{ex}
  Is it true that $\det(A+B) = \det(A) + \det(B)$? If this is so,
  explain why. If it is not so, give a counterexample.
  \begin{sol}
    This is not true at all. Consider $A = \begin{mymatrix}{cc}
      1 & 0 \\
      0 & 1
    \end{mymatrix}$ and $B = \begin{mymatrix}{rr}
      -1 & 0 \\
      0 & -1
    \end{mymatrix}$. Then $\det(A)=1$, $\det(B)=1$, and $\det(A+B)=0$.
  \end{sol}
\end{ex}

\begin{ex}
  An $n\times n$-matrix is called \textbf{nilpotent}%
  \index{nilpotent matrix}%
  \index{matrix!nilpotent} if there exists some positive integer $k$
  such that $A^k = 0$. If $A$ is a nilpotent matrix, what are the
  possible values of $\det(A)$?
  \begin{sol}
    Since $A^k=0$, we have $\det(A)^k=\det(A^k)=\det(0)=0$. Therefore,
    it must be the case that $\det(A)=0$.
  \end{sol}
\end{ex}

\begin{ex}
  A square matrix is said to be \textbf{orthogonal}%
  \index{matrix!orthogonal}%
  \index{orthogonal matrix} if $A^TA = I$. Thus the inverse of an
  orthogonal matrix is its transpose. What are the possible values of
  $\det(A)$ if $A$ is an orthogonal matrix?
  \begin{sol}
    If $A$ is orthogonal, we have $\det(A)^2 = \det(A^T)\det(A) =
    \det(A^TA) = \det(I) = 1$. Therefore the only possible values for
    $\det(A)$ are $\pm 1$.
  \end{sol}
\end{ex}

\begin{ex}
  Let $A$ and $B$ be two $n\times n$-matrices. We say that $A$ is
  \textbf{similar}%
  \index{matrix!similar}%
  \index{similar matrices} to $B$, in symbols $A\similar B$, if there
  exists an invertible matrix $P$ such that $A = P^{-1}BP$. Show that
  if $A\similar B$, then $\det(A) = \det(B)$.
  \begin{sol}
    $\det(A) = \det(P^{-1}BP) = \det(P^{-1}) \det(B) \det(P) =
    \frac{1}{\det(P)} \det(B) \det(P) = \det(B)$.
  \end{sol}
\end{ex}

\begin{ex}
  Find the determinant of
  \begin{equation*}
    A = \begin{mymatrix}{ccc}
      1 & 1 & 1 \\
      1 & a & a^2 \\
      1 & b & b^2 \\
    \end{mymatrix}.
  \end{equation*}
  For which values of $a$ and $b$ is this matrix invertible? Hint:
  after you compute the determinant, you can factor out $(a-1)$ and
  $(b-1)$ from it.
  \begin{sol}
    The determinant is
    \begin{equation*}
      \begin{absmatrix}{ccc}
        1 & 1 & 1 \\
        1 & a & a^2 \\
        1 & b & b^2 \\
      \end{absmatrix}
      = \begin{absmatrix}{cc}
        a & a^2 \\
        b & b^2 \\
      \end{absmatrix}
      - \begin{absmatrix}{cc}
        1 & 1 \\
        b & b^2 \\
      \end{absmatrix}
      + \begin{absmatrix}{cc}
        1 & 1 \\
        a & a^2 \\
      \end{absmatrix}
      = ab^2 - ba^2 - b^2 + b + a^2 - a
      = (a-1)(b-1)(b-a).
    \end{equation*}
    Therefore the determinant is $0$ if $a=1$, $b=1$, or $a=b$. In all
    other cases, the determinant is non-zero and the matrix is invertible.
  \end{sol}
\end{ex}

\begin{ex}
  Assume $A$, $B$, and $C$ are $n\times n$-matrices and $ABC$ is
  invertible. Use determinants to show that each of $A,B$, and $C$ is
  invertible.
  \begin{sol}
    This follows because $\det(ABC) = \det(A)\det(B)\det(C)$ and if
    this product is non-zero, then each determinant in the product is
    non-zero. Therefore, each of these matrices is invertible.
  \end{sol}
\end{ex}

\begin{ex}
  Suppose $A$ is an upper triangular matrix. Show that $A^{-1}$ exists
  if and only if all elements of the main diagonal are non-zero. Is it
  true that $A^{-1}$ will also be upper triangular? Explain. Could the
  same be concluded for lower triangular matrices?
  \begin{sol}
    The given condition is what it takes for the determinant to be
    non-zero. Recall that the determinant of an upper triangular
    matrix is just the product of the entries on the main diagonal.
    The inverse will also be upper triangular; this can be seen by
    noting that every invertible upper triangular can be written as a
    product of upper triangular elementary matrices; the inverse of
    each such elementary matrix is upper triangular, and therefore so
    is their product. The analogous statement about lower triangular
    matrices is also true.
  \end{sol}
\end{ex}

\begin{samepage}  % Inhibit page break within the exercise. A student
                  % misread a "false" statement at the top of the page
                  % as a theorem.
\begin{ex}
  Specify whether each statement is true or false. If true, provide a
  proof. If false, provide a counterexample.
  \begin{enumerate}
  \item If $A$ is a $3\times 3$-matrix with determinant zero, then one
    column must be a multiple of some other column.

  \item If any two columns of a square matrix are equal, then the
    determinant of the matrix equals zero.

  \item For two $n\times n$-matrices $A$ and $B$,
    $\det(A+B) = \det(A) + \det(B)$.

  \item For an $n\times n$-matrix $A$, $\det(3A) = 3\det(A)$.

  \item If $A^{-1}$ exists, then $\det(A^{-1}) = \det(A)^{-1}$.

  \item If $B$ is obtained by multiplying a single row of $A$ by $4$,
    then $\det(B) = 4\det(A)$.

  \item For an $n\times n$-matrix $A$, we have
    $\det(-A) = (-1)^n\det(A)$.

  \item If $A$ is a real $n\times n$-matrix, then
    $\det(A^TA) \geq 0$.

  \item If $A^k = 0$ for some positive integer $k$, then
    $\det(A) = 0$.

  \item If $A\vect{x} = 0$ for some $\vect{x} \neq 0$, then
    $\det(A) = 0$.
  \end{enumerate}
  \begin{sol}
    \begin{enumerate}
    \item False. Consider $\begin{mymatrix}{rrr}
        1 & 1 & 0 \\
        1 & 2 & 1 \\
        0 & 1 & 1 \\
      \end{mymatrix}$.
    \item True.
    \item False.
    \item False.
    \item True.
    \item True.
    \item True.
    \item True.
    \item True.
    \item True.
    \end{enumerate}
  \end{sol}
\end{ex}
\end{samepage}

