\Opensolutionfile{solutions}[ex]
\section*{Exercises}

\begin{enumialphparenastyle}

\begin{ex} \label{exerlineartransf}
Consider the following functions which map $\R^{n}$ to $\R^{n}$. 

\begin{enumerate}
\item $T$ multiplies the $j^{th}$ component of $\vect{x}$ by a nonzero
number $b.$

\item $T$ replaces the $i^{th}$ component of $\vect{x}$ with $b$ times the
$j^{th}$ component added to the $i^{th}$ component.

\item $T$ switches the $i^{th}$ and $j^{th}$ components.
\end{enumerate}

Show these functions are linear transformations and describe their matrices $A$ such that $T\left(\vect{x}\right) = A\vect{x}$.
\begin{sol}
\begin{enumerate}
\item The matrix of $T$ is the elementary matrix which multiplies
the $j^{th}$ diagonal entry of the identity matrix by $b$.
\item The matrix of $T$ is the
elementary matrix which takes $b$ times the $j^{th}$ row and adds to the $%
i^{th}$ row.
\item The matrix of $T$ is the elementary matrix which switches the $%
i^{th}$ and the $j^{th}$ rows where the two components are in the $i^{th}$
and $j^{th}$ positions.
\end{enumerate}
\end{sol}
\end{ex}

\begin{ex} You are given a linear transformation $T:\R^{n}\rightarrow
\R^{m}$ and you know that
\begin{equation*}
T\left(A_{i}\right)=B_{i}
\end{equation*}
where $\begin{mymatrix}{ccc}
A_{1} & \cdots & A_{n}
\end{mymatrix} ^{-1}$ exists. Show that the matrix of $T$ is of the form
\begin{equation*}
\begin{mymatrix}{ccc}
B_{1} & \cdots & B_{n}
\end{mymatrix} \begin{mymatrix}{ccc}
A_{1} & \cdots & A_{n}
\end{mymatrix} ^{-1}
\end{equation*}
\begin{sol}
Suppose 
\[
\begin{mymatrix}{c}
\vect{c}_{1}^{T} \\
\vdots \\
\vect{c}_{n}^{T}
\end{mymatrix} =\begin{mymatrix}{ccc}
\vect{a}_{1} & \cdots & \vect{a}_{n}
\end{mymatrix}^{-1}
\]
Thus $\vect{c}_{i}^{T}\vect{a}_{j}=\delta _{ij}$. Therefore,
\begin{eqnarray*}
\begin{mymatrix}{ccc}
\vect{b}_{1} & \cdots & \vect{b}_{n}
\end{mymatrix} \begin{mymatrix}{ccc}
\vect{a}_{1} & \cdots & \vect{a}_{n}
\end{mymatrix} ^{-1}\vect{a}_{i} &=&
\begin{mymatrix}{ccc}
\vect{b}_{1} & \cdots & \vect{b}_{n}
\end{mymatrix} \begin{mymatrix}{c}
\vect{c}_{1}^{T} \\
\vdots \\
\vect{c}_{n}^{T}
\end{mymatrix} \vect{a}_{i} \\
&=&\begin{mymatrix}{ccc}
\vect{b}_{1} & \cdots & \vect{b}_{n}
\end{mymatrix} \vect{e}_{i} \\
&=&\vect{b}_{i}
\end{eqnarray*}
Thus $T\vect{a}_{i}=\begin{mymatrix}{ccc}
\vect{b}_{1} & \cdots & \vect{b}_{n}
\end{mymatrix} \begin{mymatrix}{ccc}
\vect{a}_{1} & \cdots & \vect{a}_{n}
\end{mymatrix} ^{-1}\vect{a}_{i} =  A\vect{a}_{i}.$ If $\vect{x}$ is
arbitrary, then since the matrix $\begin{mymatrix}{ccc}
\vect{a}_{1} & \cdots & \vect{a}_{n}
\end{mymatrix} $ is invertible, there exists a unique $\vect{y}$ such that $
\begin{mymatrix}{ccc}
\vect{a}_{1} & \cdots & \vect{a}_{n}
\end{mymatrix} \vect{y}=\vect{x}$ Hence
\[
T\vect{x}=T\left( \sum_{i=1}^{n}y_{i}\vect{a}_{i}\right)
=\sum_{i=1}^{n}y_{i}T\vect{a}_{i}=\sum_{i=1}^{n}y_{i}A\vect{a}
_{i}=A\left( \sum_{i=1}^{n}y_{i}\vect{a}_{i}\right) =A\vect{x}
\]

\end{sol}
\end{ex}

\begin{ex} Suppose $T$ is a linear transformation such that 
\begin{eqnarray*}
T\begin{mymatrix}{r}
1 \\
2 \\
-6
\end{mymatrix} &=&\begin{mymatrix}{r}
5 \\
1 \\
3
\end{mymatrix} \\
T\begin{mymatrix}{r}
-1 \\
-1 \\
5
\end{mymatrix} &=&\begin{mymatrix}{r}
1 \\
1 \\
5
\end{mymatrix} \\
T\begin{mymatrix}{r}
0 \\
-1 \\
2
\end{mymatrix} &=&\begin{mymatrix}{r}
5 \\
3 \\
-2
\end{mymatrix}
\end{eqnarray*}
Find the matrix of $T$. That is find $A$ such that $T(\vect{x})=A\vect{x}$. \vspace{1mm}
\begin{sol}
\[
\begin{mymatrix}{rrr}
5 & 1 & 5 \\
1 & 1 & 3 \\
3 & 5 & -2
\end{mymatrix} \begin{mymatrix}{ccc}
3 & 2 & 1 \\
2 & 2 & 1 \\
4 & 1 & 1
\end{mymatrix} =\begin{mymatrix}{ccc}
37 & 17 & 11 \\
17 & 7 & 5 \\
11 & 14 & 6
\end{mymatrix}
\]
\end{sol}
\end{ex}

\begin{ex} Suppose $T$ is a linear transformation such that 
\begin{eqnarray*}
T\begin{mymatrix}{r}
1 \\
1 \\
-8
\end{mymatrix} &=&\begin{mymatrix}{r}
1 \\
3 \\
1
\end{mymatrix} \\
T\begin{mymatrix}{r}
-1 \\
0 \\
6
\end{mymatrix} &=&\begin{mymatrix}{r}
2 \\
4 \\
1
\end{mymatrix} \\
T\begin{mymatrix}{r}
0 \\
-1 \\
3
\end{mymatrix} &=&\begin{mymatrix}{r}
6 \\
1 \\
-1
\end{mymatrix}
\end{eqnarray*}
Find the matrix of $T$. That is find $A$ such that $T(\vect{x})=A\vect{x}$. \vspace{1mm}
\begin{sol}
\[
\begin{mymatrix}{rrr}
1 & 2 & 6 \\
3 & 4 & 1 \\
1 & 1 & -1
\end{mymatrix} \begin{mymatrix}{ccc}
6 & 3 & 1 \\
5 & 3 & 1 \\
6 & 2 & 1
\end{mymatrix} =\begin{mymatrix}{ccc}
52 & 21 & 9 \\
44 & 23 & 8 \\
5 & 4 & 1
\end{mymatrix}
\]
\end{sol}
\end{ex}

\begin{ex} Suppose $T$ is a linear transformation such that 
\begin{eqnarray*}
T\begin{mymatrix}{r}
1 \\
3 \\
-7
\end{mymatrix} &=&\begin{mymatrix}{r}
-3 \\
1 \\
3
\end{mymatrix} \\
T\begin{mymatrix}{r}
-1 \\
-2 \\
6
\end{mymatrix} &=&\begin{mymatrix}{r}
1 \\
3 \\
-3
\end{mymatrix} \\
T\begin{mymatrix}{r}
0 \\
-1 \\
2
\end{mymatrix} &=&\begin{mymatrix}{r}
5 \\
3 \\
-3
\end{mymatrix}
\end{eqnarray*}
Find the matrix of $T$. That is find $A$ such that $T(\vect{x})=A\vect{x}$. \vspace{1mm}\vspace{1mm}
\begin{sol}
\[
\begin{mymatrix}{rrr}
-3 & 1 & 5 \\
1 & 3 & 3 \\
3 & -3 & -3
\end{mymatrix} \begin{mymatrix}{ccc}
2 & 2 & 1 \\
1 & 2 & 1 \\
4 & 1 & 1
\end{mymatrix} = \begin{mymatrix}{rrr}
15 & 1 & 3 \\
17 & 11 & 7 \\
-9 & -3 & -3
\end{mymatrix}
\]
\end{sol}
\end{ex}

\begin{ex} Suppose $T$ is a linear transformation such that 
\begin{eqnarray*}
T\begin{mymatrix}{r}
1 \\
1 \\
-7
\end{mymatrix} &=&\begin{mymatrix}{r}
3 \\
3 \\
3
\end{mymatrix} \\
T\begin{mymatrix}{r}
-1 \\
0 \\
6
\end{mymatrix} &=&\begin{mymatrix}{r}
1 \\
2 \\
3
\end{mymatrix} \\
T\begin{mymatrix}{r}
0 \\
-1 \\
2
\end{mymatrix} &=&\begin{mymatrix}{r}
1 \\
3 \\
-1
\end{mymatrix}
\end{eqnarray*}
Find the matrix of $T$. That is find $A$ such that $T(\vect{x})=A\vect{x}$. \vspace{1mm}
\begin{sol}
\[
\begin{mymatrix}{rrr}
3 & 1 & 1 \\
3 & 2 & 3 \\
3 & 3 & -1
\end{mymatrix} \begin{mymatrix}{ccc}
6 & 2 & 1 \\
5 & 2 & 1 \\
6 & 1 & 1
\end{mymatrix} =\allowbreak \begin{mymatrix}{ccc}
29 & 9 & 5 \\
46 & 13 & 8 \\
27 & 11 & 5
\end{mymatrix} 
\]
\end{sol}
\end{ex}

\begin{ex} Suppose $T$ is a linear transformation such that
\begin{eqnarray*}
T\begin{mymatrix}{r}
1 \\
2 \\
-18
\end{mymatrix} &=&\begin{mymatrix}{r}
5 \\
2 \\
5
\end{mymatrix} \\
T\begin{mymatrix}{r}
-1 \\
-1 \\
15
\end{mymatrix} &=&\begin{mymatrix}{r}
3 \\
3 \\
5
\end{mymatrix} \\
T\begin{mymatrix}{r}
0 \\
-1 \\
4
\end{mymatrix} &=&\begin{mymatrix}{r}
2 \\
5 \\
-2
\end{mymatrix}
\end{eqnarray*}
Find the matrix of $T$. That is find $A$ such that $T(\vect{x})=A\vect{x}$. \vspace{1mm}
\begin{sol}
\[
\begin{mymatrix}{rrr}
5 & 3 & 2 \\
2 & 3 & 5 \\
5 & 5 & -2
\end{mymatrix} \begin{mymatrix}{ccc}
11 & 4 & 1 \\
10 & 4 & 1 \\
12 & 3 & 1
\end{mymatrix} =\begin{mymatrix}{ccc}
109 & 38 & 10 \\
112 & 35 & 10 \\
81 & 34 & 8
\end{mymatrix}
\]
\end{sol}
\end{ex}


\begin{ex} Consider the following functions $T:\R^{3}\rightarrow \R^{2}$.
Show that each is a linear transformation and determine for each the matrix $A$ such that 
$T(\vect{x})=A\vect{x}$.

\begin{enumerate}
\item $T\begin{mymatrix}{c}
x \\
y \\
z
\end{mymatrix} =\begin{mymatrix}{c}
x+2y+3z \\
2y-3x+z
\end{mymatrix} $

\item $T\begin{mymatrix}{c}
x \\
y \\
z
\end{mymatrix} =\begin{mymatrix}{c}
7x+2y+z \\
3x-11y+2z
\end{mymatrix} $

\item $T\begin{mymatrix}{c}
x \\
y \\
z
\end{mymatrix} =\begin{mymatrix}{c}
3x+2y+z \\
x+2y+6z
\end{mymatrix} $

\item $T\begin{mymatrix}{c}
x \\
y \\
z
\end{mymatrix} =\begin{mymatrix}{c}
2y-5x+z \\
x+y+z
\end{mymatrix} $
\end{enumerate}
%\begin{sol}
%\end{sol}
\end{ex}

\begin{ex} Consider the following functions $T:\R^{3}\rightarrow \R^{2}.$
Explain why each of these functions $T$ is not linear.

\begin{enumerate}
\item $T\begin{mymatrix}{c}
x \\
y \\
z
\end{mymatrix} =\begin{mymatrix}{c}
x+2y+3z+1 \\
2y-3x+z
\end{mymatrix} $

\item $T\begin{mymatrix}{c}
x \\
y \\
z
\end{mymatrix} =\begin{mymatrix}{c}
x+2y^{2}+3z \\
2y+3x+z
\end{mymatrix} $

\item $T\begin{mymatrix}{c}
x \\
y \\
z
\end{mymatrix} =\begin{mymatrix}{c}
\sin x+2y+3z \\
2y+3x+z
\end{mymatrix} $

\item $T\begin{mymatrix}{c}
x \\
y \\
z
\end{mymatrix} =\begin{mymatrix}{c}
x+2y+3z \\
2y+3x-\ln z
\end{mymatrix} $
\end{enumerate}
%\begin{sol}
%\end{sol}
\end{ex}


\begin{ex} Suppose 
\begin{equation*}
\begin{mymatrix}{ccc}
A_{1} & \cdots & A_{n}
\end{mymatrix} ^{-1}
\end{equation*}
 exists where each $A_{j}\in \R^{n}$ and let
vectors  $\set{B_{1},\cdots ,B_{n}} $ in $\R^{m}$ be given. 
Show that there \textbf{always }exists a linear
transformation $T$ such that $T(A_{i})=B_{i}$.
%\begin{sol}
%\end{sol}
\end{ex}


\begin{ex}  Find the matrix for $T\left(\vect{w} \right) = \func{proj}_{\vect{v}}\left( \vect{w}\right) $
where $\vect{v}=\begin{mymatrix}{rrr}
1 & -2 & 3
\end{mymatrix} ^{T}.$
\begin{sol}
 Recall that $\func{proj}_{\vect{u}}\left( \vect{v}\right) =\frac{\vect{v}\dotprod\vect{u} }{\vectlength \vect{u}\vectlength ^{2}}\vect{u}$ and so the desired matrix
has $i^{th}$ column equal to $\func{proj}_{\vect{u}}\left( \vect{e}_{i}\right) .$ Therefore, the matrix desired is
\[
\frac{1}{14}\begin{mymatrix}{rrr}
1 & -2 & 3 \\
-2 & 4 & -6 \\
3 & -6 & 9
\end{mymatrix}
\]
\end{sol}
\end{ex}

\begin{ex}  Find the matrix for $T\left(\vect{w} \right) = \func{proj}_{\vect{v}}\left( \vect{w}\right) $
where $\vect{v}=\begin{mymatrix}{rrr}
1 & 5 & 3
\end{mymatrix} ^{T}.$
\begin{sol}
\[
\frac{1}{35}\begin{mymatrix}{rrr}
1 & 5 & 3 \\
5 & 25 & 15 \\
3 & 15 & 9
\end{mymatrix}
\]
\end{sol}
\end{ex}

\begin{ex} Find the matrix for $T\left(\vect{w} \right) = \func{proj}_{\vect{v}}\left( \vect{w}\right) $
where $\vect{v}=\begin{mymatrix}{rrr}
1 & 0 & 3
\end{mymatrix} ^{T}.$ 
\begin{sol}
\[
\frac{1}{10}\begin{mymatrix}{ccc}
1 & 0 & 3 \\
0 & 0 & 0 \\
3 & 0 & 9
\end{mymatrix}
\]
\end{sol}
\end{ex}

\begin{ex} Suppose $A$ is a $3\times 3$ skew symmetric matrix such that $A^{T}=-A$. Show
there exists a vector $\vect{v}$ such that for all $\vect{u}\in
\R^{3}$
\begin{equation*}
A\vect{u}=\vect{v} \times \vect{u}
\end{equation*}
\textbf{Hint: }Explain why, since $A$ is skew symmetric it is of the form
\begin{equation*}
A=\begin{mymatrix}{ccc}
0 & -\omega _{3} & \omega _{2} \\
\omega _{3} & 0 & -\omega _{1} \\
-\omega _{2} & \omega _{1} & 0
\end{mymatrix}
\end{equation*}
where the $\omega _{i}$ are numbers. Then consider $\omega _{1}\vect{i}
+\omega _{2}\vect{j}+\omega _{3}\vect{k}$.
%\begin{sol}
%\end{sol}
\end{ex}

\end{enumialphparenastyle}