\section*{Exercises}

\begin{ex}
  Find the vector equation for the line through $(-7,6,0)$ and
  $(-1,1,4)$. Then, find the parametric equations for this
  line.
  % \begin{sol}
  % \end{sol}
\end{ex}

\begin{ex}
  Find parametric equations for the line through the point
  $(7,7,1)$ with direction vector
  $\vect{d} = \begin{mysmallmatrix}{r} 1 \\ 6 \\ 2 \end{mysmallmatrix}$.
  % \begin{sol}
  % \end{sol}
\end{ex}

\begin{ex}
  Parametric equations of the line are
  \begin{equation*}
    \begin{array}{c}
      x = t+2, \\
      y = 6-3t, \\
      z = -t-6.
    \end{array}
  \end{equation*}
  Find a direction vector for the line and a point on the line.
  % \begin{sol}
  % \end{sol}
\end{ex}

\begin{ex}
  The equation of a line in two dimensions is written as $y=x-5$. Find
  a vector equation for this line.
  % \begin{sol}
  % \end{sol}
\end{ex}

\begin{ex}
  Find parametric equations for the line through $(6, 5, -2, 3)$
  and $(5, 1, 2, 1)$.
  % \begin{sol}
  % \end{sol}
\end{ex}

\begin{ex}
  Consider the following vector equation for a line in $\R^3$:
  \begin{equation*}
    \begin{mymatrix}{c} x\\y\\z \end{mymatrix}
    = \begin{mymatrix}{r} 1\\2\\0 \end{mymatrix}
    + t\,\begin{mymatrix}{r} 1\\0\\1 \end{mymatrix}.
  \end{equation*}
  Find a new vector equation for the same line by doing the change of
  parameter $t=2-s$.
  \begin{sol}
    We have
    \begin{equation*}
      \begin{mymatrix}{c} x\\y\\z \end{mymatrix}
      = \begin{mymatrix}{r} 1\\2\\0 \end{mymatrix}
      + (2-s)\,\begin{mymatrix}{c} 1\\0\\1 \end{mymatrix}
      = \begin{mymatrix}{r} 3\\2\\2 \end{mymatrix}
      + s\,\begin{mymatrix}{r} -1\\0\\-1 \end{mymatrix}.
    \end{equation*}
  \end{sol}
\end{ex}

\begin{ex}
  Consider the line given by the following parametric equations:
  \begin{equation*}
    \begin{array}{c}
      x = 2t+2,\\
      y = 5-4t,\\
      z= -t-3.
    \end{array}
  \end{equation*}
  Find symmetric equations for the line.
  % \begin{sol}
  % \end{sol}
\end{ex}

\begin{ex}
  Find the point on the line segment from $P = (-4, 7, 5) $ to
  $Q = (2 , -2 , -3) $ which is $\frac{1}{7}$ of the way from $P$
  to $Q$.
  % \begin{sol}
  % \end{sol}
\end{ex}

\begin{ex} Suppose a triangle in $\R^{n}$ has vertices at $P$, $Q$,
  and $R$.  Consider the lines which are drawn from a vertex to the
  mid point of the opposite side. Show these three lines intersect in
  a point and find the coordinates of this point.
%\begin{sol}
%\end{sol}
\end{ex}

\begin{ex}
  Determine whether the lines
  \begin{equation*}
    \begin{mymatrix}{c} x \\ y \\ z \end{mymatrix}
    = \begin{mymatrix}{c} 1 \\ 1 \\ 2 \end{mymatrix}
    + t \begin{mymatrix}{c} 1 \\ 2 \\ 2 \end{mymatrix}
    \quad\mbox{and}\quad
    \begin{mymatrix}{c} x \\ y \\ z \end{mymatrix}
    = \begin{mymatrix}{c} 1 \\ -1 \\ -4 \end{mymatrix}
    + s \begin{mymatrix}{c} 1 \\ 1 \\ -1 \end{mymatrix}
  \end{equation*}
  intersect. If yes, find the point of intersection.
\end{ex}

\begin{ex}
  Determine whether the lines
  \begin{equation*}
    \begin{mymatrix}{c} x \\ y \\ z \end{mymatrix}
    = \begin{mymatrix}{c} 2 \\ -1 \\ 0 \end{mymatrix}
    + t \begin{mymatrix}{c} 1 \\ 3 \\ 2 \end{mymatrix}
    \quad\mbox{and}\quad
    \begin{mymatrix}{c} x \\ y \\ z \end{mymatrix}
    = \begin{mymatrix}{c} 1 \\ 1 \\ 3 \end{mymatrix}
    + s \begin{mymatrix}{c} 1 \\ 2 \\ 0 \end{mymatrix}
  \end{equation*}
  intersect. If yes, find the point of intersection.
\end{ex}

\begin{ex}
  Find the angle between the two lines
  \begin{equation*}
    \begin{mymatrix}{r} x \\ y \\ z \end{mymatrix}
    = \begin{mymatrix}{r} 3 \\ 0 \\ 1 \end{mymatrix}
    + t \begin{mymatrix}{r} 3 \\ -3 \\ 0 \end{mymatrix}
    \quad\mbox{and}\quad
    \begin{mymatrix}{r} x \\ y \\ z \end{mymatrix}
    = \begin{mymatrix}{r} 3 \\ 0 \\ 1 \end{mymatrix}
    + s \begin{mymatrix}{r} -1 \\ 2 \\ 2 \end{mymatrix}.
  \end{equation*}
%\begin{sol}
%\end{sol}
\end{ex}

\begin{ex} Let $P = (1,2,3)$ be a point in $\R^3$. Let $L$ be the line
  through the point $P_0 = (1, 4, 5)$ with direction vector
  $\vect{d} = \begin{mysmallmatrix}{r} 1 \\ -1 \\
    1 \end{mysmallmatrix}$. Find the shortest distance from $P$ to
  $L$, and find the point $Q$ on $L$ that is closest to $P$.
%\begin{sol}
%\end{sol}
\end{ex}

\begin{ex} Let $P = (0,2,1)$ be a point in $\R^3$. Let $L$ be the line through the points $P_0 = (1, 1, 1)$ and $P_1 = (4, 1, 2)$. Find the shortest distance from $P$ to $L$, and find the point $Q$ on $L$ that is closest to $P$.
%\begin{sol}
%\end{sol}
\end{ex}

\begin{ex}\label{ex:angle-lines}
  When we computed the angle between two lines in
  Example~\ref{exa:angle-between-two-lines}, we calculated two
  different angles and took the smaller of the two. Show that one can
  get the same answer by taking the absolute value of the dot product,
  i.e., by solving
  \begin{equation*}
    \cos\theta =
    \frac{\abs{\vect{u}\dotprod\vect{v}}}{\norm{\vect{u}}\norm{\vect{v}}}.
  \end{equation*}
  \begin{sol}
    From trigonometry, we have the following properties of the cosine
    function:
    \begin{itemize}
    \item $\cos\theta$ is positive for $0\leq\theta\leq\frac{\pi}{2}$,
      and negative for $\frac{\pi}{2}\leq\theta\leq\pi$.
    \item $\cos(\pi-\theta) = -\cos\theta$.
    \end{itemize}
    By the method in Example~\ref{exa:angle-between-two-lines}, we
    calculated $\theta$ such that
    \begin{equation*}
      \cos\theta =
      \frac{\vect{u}\dotprod\vect{v}}{\norm{\vect{u}}\norm{\vect{v}}}.
    \end{equation*}
    If $0\leq\theta\leq\frac{\pi}{2}$, the answer is $\theta$. If
    $\frac{\pi}{2}\leq\theta\leq\pi$, the answer is $\phi=\pi-\theta$.
    But in the last case, the dot product is negative and we have
    \begin{equation*}
      \cos\phi = \cos(\pi-\theta) = -\cos\theta = \frac{-\vect{u}\dotprod\vect{v}~~}{\norm{\vect{u}}\norm{\vect{v}}}
      =
      \frac{\abs{\vect{u}\dotprod\vect{v}}}{\norm{\vect{u}}\norm{\vect{v}}}.
    \end{equation*}
    So in either case we get the correct answer by taking the absolute
    value.
  \end{sol}
\end{ex}

