\Opensolutionfile{solutions}[ex]
\section*{Exercises}

\begin{enumialphparenastyle}

\begin{ex} Give the complete solution to $x^{4}+16=0.$ 
\begin{sol}
 Solution is:
\[
\left( 1-i\right) \sqrt{2},-\left( 1+i\right) \sqrt{2},-\left( 1-i\right)
\sqrt{2},\left( 1+i\right) \sqrt{2}
\]
\end{sol}
\end{ex}

\begin{ex} \label{cuberoots} Find the complex cube roots of $8$.
\begin{sol}
The cube roots are the solutions to $%
z^{3}+8=0$, Solution is: $i\sqrt{3} +1,1-i\sqrt{3},-2$
\end{sol}
\end{ex}

\begin{ex} \label{cuberoots2} Find the four fourth roots of $16$.  
\begin{sol}
The fourth roots are
the solutions to $z^{4}+16=0$, Solution is:
\[
\left( 1-i\right) \sqrt{2},-\left( 1+i\right) \sqrt{2},-\left( 1-i\right)
\sqrt{2},\left( 1+i\right)\sqrt{2}
\]
\end{sol}
\end{ex}

\begin{ex} \label{exercomplex1}De Moivre's theorem says $\left[ r\left( \cos
t+i\sin t\right) \right] ^{n}=r^{n}\left( \cos nt+i\sin nt\right) $ for $n$
a positive integer. Does this formula continue to hold for all integers $n,$
even negative integers$?$ Explain.  
\begin{sol}
Yes, it holds for all integers. First of
all, it clearly holds if $n=0$. Suppose now that $n$ is a negative integer.
Then $-n>0$ and so
\[
\left[ r\left( \cos t+i\sin t\right) \right] ^{n}=\frac{1}{\left[ r\left(
\cos t+i\sin t\right) \right] ^{-n}}=\frac{1}{r^{-n}\left( \cos \left(
-nt\right) +i\sin \left( -nt\right) \right) }
\]
\begin{eqnarray*}
&=&\frac{r^{n}}{\left( \cos \left( nt\right) -i\sin \left( nt\right) \right)
}=\frac{r^{n}\left( \cos \left( nt\right) +i\sin \left( nt\right) \right) }{
\left( \cos \left( nt\right) -i\sin \left( nt\right) \right) \left( \cos
\left( nt\right) +i\sin \left( nt\right) \right) } \\
&=&r^{n}\left( \cos \left( nt\right) +i\sin \left( nt\right) \right)
\end{eqnarray*}
because $\left( \cos \left( nt\right) -i\sin \left( nt\right) \right) \left(
\cos \left( nt\right) +i\sin \left( nt\right) \right) =1.$
\end{sol}
\end{ex}

\begin{ex} Factor $x^{3}+8$ as a product of linear factors. \textbf{Hint:} Use the result of \ref{cuberoots}.
\begin{sol}
Solution
is: $i\sqrt{3}+1,1-i\sqrt{3},-2$ and so this polynomial equals
\[
\left( x+2\right) \left( x-\left( i\sqrt{3}+1\right) \right) \left( x-\left(
1-i\sqrt{3}\right) \right)
\]
\end{sol}
\end{ex}

\begin{ex} Write $x^{3}+27$ in the form $\left( x+3\right) \left(
x^{2}+ax+b\right) $ where $x^{2}+ax+b$ cannot be factored any more using
only real numbers. 
\begin{sol}
$x^{3}+27= \left( x+3\right) \left(
x^{2}-3x+9\right) $
\end{sol}
\end{ex}

\begin{ex} Completely factor $x^{4}+16$ as a product of linear factors. \textbf{Hint:} Use the result of \ref{cuberoots2}. 
\begin{sol}
Solution is:
\[
\left( 1-i\right) \sqrt{2},-\left( 1+i\right) \sqrt{2},-\left( 1-i\right)
\sqrt{2},\left( 1+i\right) \sqrt{2}.
\]
These are just the fourth roots of $-16$. Then to factor, you get
\begin{eqnarray*}
&&\left( x-\left( \left( 1-i\right) \sqrt{2}\right) \right) \left( x-\left(
-\left( 1+i\right) \sqrt{2}\right) \right) \cdot \\
&&\left( x-\left( -\left( 1-i\right) \sqrt{2}\right) \right) \left( x-\left(
\left( 1+i\right) \allowbreak \sqrt{2}\right) \right)
\end{eqnarray*}
\end{sol}
\end{ex}


\begin{ex} Factor $x^{4}+16$ as the product of two quadratic polynomials each of
which cannot be factored further without using complex numbers. 
\begin{sol}
$x^{4}+16=\left( x^{2}-2\sqrt{2}x+4\right) \left( x^{2}+2\sqrt{2}x+4\right) .
$ You can use the information in the preceding problem. Note that $\left(
x-z\right) \left( x-\overline{z}\right) $ has real coefficients.
\end{sol}
\end{ex}

\begin{ex} If $n$ is an integer, is it always true that $\left( \cos \theta
-i\sin \theta \right) ^{n}=\cos \left( n\theta \right) -i\sin \left( n\theta
\right) ?$ Explain.
\begin{sol}
Yes, this is true.
\begin{eqnarray*}
\left( \cos \theta -i\sin \theta \right) ^{n} &=&\left( \cos \left( -\theta
\right) +i\sin \left( -\theta \right) \right) ^{n} \\
&=&\cos \left( -n\theta \right) +i\sin \left( -n\theta \right) \\
&=&\cos \left( n\theta \right) -i\sin \left( n\theta \right)
\end{eqnarray*}
\end{sol}
\end{ex}


\begin{ex} Suppose $p\left( x\right) =a_{n}x^{n}+a_{n-1}x^{n-1}+\cdots
+a_{1}x+a_{0}$ \ is a polynomial and it has $n$ zeros,
\begin{equation*}
z_{1},z_{2},\cdots ,z_{n}
\end{equation*}
listed according to multiplicity. ($z$ is a root of multiplicity $m$ if the
polynomial $f\left( x\right) =\left( x-z\right) ^{m}$ divides $p\left(
x\right) $ but $\left( x-z\right) f\left( x\right) $ does not.) Show that
\begin{equation*}
p\left( x\right) =a_{n}\left( x-z_{1}\right) \left( x-z_{2}\right) \cdots
\left( x-z_{n}\right) 
\end{equation*} 
\begin{sol}
$p\left( x\right) =\left( x-z_{1}\right) q\left( x\right) +r\left( x\right) $
where $r\left( x\right) $ is a nonzero constant or equal to $0$. However, $r\left( z_{1}\right) =0$ and so $r\left( x\right) =0$. Now do to $q\left(
x\right) $ what was done to $p\left( x\right) $ and continue until the
degree of the resulting $q\left( x\right) $ equals $0$. Then you have the
above factorization.
\end{sol}
\end{ex}

\end{enumialphparenastyle}