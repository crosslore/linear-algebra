\Opensolutionfile{solutions}[ex]
\section*{Exercises}

\begin{enumialphparenastyle}

\begin{ex} Give the complete solution to $x^{4}+16=0$. 
\begin{sol}
 Solution is:
\[
\tup{1-i} \sqrt{2},-\tup{1+i} \sqrt{2},-\tup{1-i}
\sqrt{2},\tup{1+i} \sqrt{2}
\]
\end{sol}
\end{ex}

\begin{ex} \label{cube-roots} Find the complex cube roots of $8$.
\begin{sol}
The cube roots are the solutions to $%
z^{3}+8=0$, Solution is: $i\sqrt{3} +1,1-i\sqrt{3},-2$
\end{sol}
\end{ex}

\begin{ex} \label{cube-roots2} Find the four fourth roots of $16$.  
\begin{sol}
The fourth roots are
the solutions to $z^{4}+16=0$, Solution is:
\[
\tup{1-i} \sqrt{2},-\tup{1+i} \sqrt{2},-\tup{1-i}
\sqrt{2},\tup{1+i}\sqrt{2}
\]
\end{sol}
\end{ex}

\begin{ex} \label{exer-complex1}De Moivre's theorem says $\mat{r\tup{\cos
t+i\sin t}} ^{n}=r^{n}\tup{\cos nt+i\sin nt} $ for $n$
a positive integer. Does this formula continue to hold for all integers $n$,
even negative integers$$? Explain.  
\begin{sol}
Yes, it holds for all integers. First of
all, it clearly holds if $n=0$. Suppose now that $n$ is a negative integer.
Then $-n>0$ and so
\[
\mat{r\tup{\cos t+i\sin t}} ^{n}=\frac{1}{\mat{r\tup{
\cos t+i\sin t}} ^{-n}}=\frac{1}{r^{-n}\tup{\cos \tup{
-nt} +i\sin \tup{-nt} } }
\]
\begin{eqnarray*}
&=&\frac{r^{n}}{\tup{\cos \tup{nt} -i\sin \tup{nt} }
}=\frac{r^{n}\tup{\cos \tup{nt} +i\sin \tup{nt} } }{
\tup{\cos \tup{nt} -i\sin \tup{nt} } \tup{\cos
\tup{nt} +i\sin \tup{nt} } } \\
&=&r^{n}\tup{\cos \tup{nt} +i\sin \tup{nt} }
\end{eqnarray*}
because $\tup{\cos \tup{nt} -i\sin \tup{nt} } \tup{
\cos \tup{nt} +i\sin \tup{nt} } =1$.
\end{sol}
\end{ex}

\begin{ex} Factor $x^{3}+8$ as a product of linear factors. \textbf{Hint:} Use the result of {\eqref{cube-roots}}.
\begin{sol}
Solution
is: $i\sqrt{3}+1,1-i\sqrt{3},-2$ and so this polynomial equals
\[
\tup{x+2} \tup{x-\tup{i\sqrt{3}+1} } \tup{x-\tup{
1-i\sqrt{3}} }
\]
\end{sol}
\end{ex}

\begin{ex} Write $x^{3}+27$ in the form $\tup{x+3} \tup{
x^{2}+ax+b} $ where $x^{2}+ax+b$ cannot be factored any more using
only real numbers. 
\begin{sol}
$x^{3}+27= \tup{x+3} \tup{
x^{2}-3x+9} $
\end{sol}
\end{ex}

\begin{ex} Completely factor $x^{4}+16$ as a product of linear factors. \textbf{Hint:} Use the result of {\eqref{cube-roots2}}. 
\begin{sol}
Solution is:
\[
\tup{1-i} \sqrt{2},-\tup{1+i} \sqrt{2},-\tup{1-i}
\sqrt{2},\tup{1+i} \sqrt{2}.
\]
These are just the fourth roots of $-16$. Then to factor, you get
\begin{eqnarray*}
&&\tup{x-\tup{\tup{1-i} \sqrt{2}} } \tup{x-\tup{
-\tup{1+i} \sqrt{2}} } \cdot \\
&&\tup{x-\tup{-\tup{1-i} \sqrt{2}} } \tup{x-\tup{
\tup{1+i} \allowbreak \sqrt{2}} }
\end{eqnarray*}
\end{sol}
\end{ex}


\begin{ex} Factor $x^{4}+16$ as the product of two quadratic polynomials each of
which cannot be factored further without using complex numbers. 
\begin{sol}
$x^{4}+16=\tup{x^{2}-2\sqrt{2}x+4} \tup{x^{2}+2\sqrt{2}x+4} .
$ You can use the information in the preceding problem. Note that $\tup{
x-z} \tup{x-\overline{z}} $ has real coefficients.
\end{sol}
\end{ex}

\begin{ex} If $n$ is an integer, is it always true that $\tup{\cos \theta
-i\sin \theta } ^{n}=\cos \tup{n\theta } -i\sin \tup{n\theta
}$? Explain.
\begin{sol}
Yes, this is true.
\begin{eqnarray*}
\tup{\cos \theta -i\sin \theta } ^{n} &=&\tup{\cos \tup{-\theta
} +i\sin \tup{-\theta } } ^{n} \\
&=&\cos \tup{-n\theta } +i\sin \tup{-n\theta } \\
&=&\cos \tup{n\theta } -i\sin \tup{n\theta }
\end{eqnarray*}
\end{sol}
\end{ex}


\begin{ex} Suppose $p\tup{x} =a_{n}x^{n}+a_{n-1}x^{n-1}+\ldots+a_{1}x+a_{0}$ \ is a polynomial and it has $n$ zeros,
\begin{equation*}
z_{1},z_{2},\ldots,z_{n}
\end{equation*}
listed according to multiplicity. ($z$ is a root of multiplicity $m$ if the
polynomial $f\tup{x} =\tup{x-z} ^{m}$ divides $p\tup{
x} $ but $\tup{x-z} f\tup{x} $ does not.) Show that
\begin{equation*}
p\tup{x} =a_{n}\tup{x-z_{1}} \tup{x-z_{2}} \cdots
\tup{x-z_{n}} 
\end{equation*} 
\begin{sol}
$p\tup{x} =\tup{x-z_{1}} q\tup{x} +r\tup{x} $
where $r\tup{x} $ is a non-zero constant or equal to $0$. However, $r\tup{z_{1}} =0$ and so $r\tup{x} =0$. Now do to $q\tup{
x} $ what was done to $p\tup{x} $ and continue until the
degree of the resulting $q\tup{x} $ equals $0$. Then you have the
above factorization.
\end{sol}
\end{ex}

\end{enumialphparenastyle}
