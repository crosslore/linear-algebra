\begin{enumialphparenastyle}

\begin{ex} Let
\begin{equation*}
A=\begin{mymatrix}{rr}
2 & 1 \\
-1 & 3
\end{mymatrix} 
\end{equation*}
Find $A^{-1}$ if possible. If $A^{-1}$ does not exist, explain why. 
\begin{sol}
$\begin{mymatrix}{rr}
2 & 1 \\
-1 & 3
\end{mymatrix}^{-1}= \begin{mymatrix}{rr}
\vspace{0.05in}\frac{3}{7} & -\vspace{0.05in}\frac{1}{7} \\
\vspace{0.05in}\frac{1}{7} & \vspace{0.05in}\frac{2}{7}
\end{mymatrix}$
\end{sol}
\end{ex}

\begin{ex}Let
\begin{equation*}
A=\begin{mymatrix}{rr}
0 & 1 \\
5 & 3
\end{mymatrix} 
\end{equation*}
Find $A^{-1}$ if possible. If $A^{-1}$ does not exist, explain why. 
\begin{sol}
$\begin{mymatrix}{cc}
0 & 1 \\
5 & 3
\end{mymatrix}^{-1}= \begin{mymatrix}{cc}
-\vspace{0.05in}\frac{3}{5} & \vspace{0.05in}\frac{1}{5} \\
1 & 0
\end{mymatrix}$
\end{sol}
\end{ex}

\begin{ex}Let
\begin{equation*}
A=\begin{mymatrix}{rr}
2 & 1 \\
3 & 0
\end{mymatrix} 
\end{equation*}
Find $A^{-1}$ if possible. If $A^{-1}$ does not exist, explain why.
\begin{sol}
$\begin{mymatrix}{cc}
2 & 1 \\
3 & 0
\end{mymatrix}^{-1}= \begin{mymatrix}{cc}
0 & \vspace{0.05in}\frac{1}{3} \\
1 & -\vspace{0.05in}\frac{2}{3}
\end{mymatrix}$
\end{sol}
\end{ex}

\begin{ex}Let
\begin{equation*}
A=\begin{mymatrix}{rr}
2 & 1 \\
4 & 2
\end{mymatrix} 
\end{equation*}
Find $A^{-1}$ if possible. If $A^{-1}$ does not exist, explain why. 
\begin{sol}
$\begin{mymatrix}{cc}
2 & 1 \\
4 & 2
\end{mymatrix}^{-1}$ does not exist. The {\rref} of this matrix
is $\begin{mymatrix}{cc}
1 & \vspace{0.05in}\frac{1}{2} \\
0 & 0
\end{mymatrix}$
\end{sol}
\end{ex}

\begin{ex}Let $A$ be a $2\times 2$ invertible matrix, with $A=\begin{mymatrix}{cc}
a & b \\
c & d
\end{mymatrix} .$ Find a formula for $A^{-1}$ in terms of $a,b,c,d$.
\begin{sol}
$\begin{mymatrix}{cc}
a & b \\
c & d
\end{mymatrix}^{-1}= \begin{mymatrix}{cc}
\frac{d}{ad-bc} & -\frac{b}{ad-bc} \\
-\frac{c}{ad-bc} & \frac{a}{ad-bc}
\end{mymatrix}$
\end{sol}
\end{ex}

\begin{ex}Let
\begin{equation*}
A=\begin{mymatrix}{rrr}
1 & 2 & 3 \\
2 & 1 & 4 \\
1 & 0 & 2
\end{mymatrix} 
\end{equation*}
Find $A^{-1}$ if possible. If $A^{-1}$ does not exist, explain why.
\begin{sol}
$\begin{mymatrix}{ccc}
1 & 2 & 3 \\
2 & 1 & 4 \\
1 & 0 & 2
\end{mymatrix}^{-1}= \begin{mymatrix}{rrr}
-2 & 4 & -5 \\
0 & 1 & -2 \\
1 & -2 & 3
\end{mymatrix}$
\end{sol}
\end{ex} 

\begin{ex}Let
\begin{equation*}
A=\begin{mymatrix}{rrr}
1 & 0 & 3 \\
2 & 3 & 4 \\
1 & 0 & 2
\end{mymatrix} 
\end{equation*}
Find $A^{-1}$ if possible. If $A^{-1}$ does not exist, explain why. 
\begin{sol}
$\begin{mymatrix}{ccc}
1 & 0 & 3 \\
2 & 3 & 4 \\
1 & 0 & 2
\end{mymatrix}^{-1}= \begin{mymatrix}{rrr}
-2 & 0 & 3 \\
0 & \vspace{0.05in}\frac{1}{3} & -\vspace{0.05in}\frac{2}{3} \\
1 & 0 & -1
\end{mymatrix}$
\end{sol}
\end{ex}

\begin{ex}Let
\begin{equation*}
A=\begin{mymatrix}{rrr}
1 & 2 & 3 \\
2 & 1 & 4 \\
4 & 5 & 10
\end{mymatrix} 
\end{equation*}
Find $A^{-1}$ if possible. If $A^{-1}$ does not exist, explain why. 
\begin{sol}
The {\rref} is 
$\begin{mymatrix}{ccc}
1 & 0 & \vspace{0.05in}\frac{5}{3} \\
0 & 1 & \vspace{0.05in}\frac{2}{3} \\
0 & 0 & 0
\end{mymatrix}$. There is no inverse.
\end{sol}
\end{ex}

\begin{ex}Let
\begin{equation*}
A=\begin{mymatrix}{rrrr}
1 & 2 & 0 & 2 \\
1 & 1 & 2 & 0 \\
2 & 1 & -3 & 2 \\
1 & 2 & 1 & 2
\end{mymatrix}
\end{equation*}
Find $A^{-1}$ if possible. If $A^{-1}$ does not exist, explain why.
\begin{sol}
$\begin{mymatrix}{rrrr}
1 & 2 & 0 & 2 \\
1 & 1 & 2 & 0 \\
2 & 1 & -3 & 2 \\
1 & 2 & 1 & 2
\end{mymatrix}^{-1}= \begin{mymatrix}{rrrr}
-1 & \vspace{0.05in}\frac{1}{2} &  \vspace{0.05in}\frac{1}{2} &  \vspace{0.05in}\frac{1}{2} \\
3 &  \vspace{0.05in}\frac{1}{2} & - \vspace{0.05in}\frac{1}{2} & - \vspace{0.05in}\frac{5}{2} \\
-1 & 0 & 0 & 1 \\
-2 & - \vspace{0.05in}\frac{3}{4} &  \vspace{0.05in}\frac{1}{4} &  \vspace{0.05in}\frac{9}{4}
\end{mymatrix}$
\end{sol}
\end{ex}

\begin{ex}Using the inverse of the matrix, find the solution to the systems:
\begin{enumerate}
\item
\begin{equation*}
\begin{mymatrix}{rr}
2 & 4  \\
1 & 1 
\end{mymatrix} 
\begin{mymatrix}{c}
x \\
y
\end{mymatrix} =\begin{mymatrix}{r}
1 \\
2 
\end{mymatrix}
\end{equation*}

\item
\begin{equation*}
\begin{mymatrix}{rr}
2 & 4 \\
1 & 1 
\end{mymatrix} \begin{mymatrix}{c}
x \\
y 
\end{mymatrix} =\begin{mymatrix}{r}
2 \\
0 
\end{mymatrix} 
\end{equation*}
\end{enumerate}

Now give the solution in terms of $a$ and $b$ to
\[
\begin{mymatrix}{rr}
2 & 4 \\
1 & 1 
\end{mymatrix}
\begin{mymatrix}{c}
x \\
y
\end{mymatrix}
=
\begin{mymatrix}{c}
a \\
b
\end{mymatrix}
\]
%\begin{sol}
%\end{sol}
\end{ex}

\begin{ex}Using the inverse of the matrix, find the solution to the systems: 

\begin{enumerate}
\item
\begin{equation*}
\begin{mymatrix}{rrr}
1 & 0 & 3 \\
2 & 3 & 4 \\
1 & 0 & 2
\end{mymatrix} \begin{mymatrix}{c}
x \\
y \\
z
\end{mymatrix} =\begin{mymatrix}{r}
1 \\
0 \\
1
\end{mymatrix} 
\end{equation*}

\item
\begin{equation*}
\begin{mymatrix}{rrr}
1 & 0 & 3 \\
2 & 3 & 4 \\
1 & 0 & 2
\end{mymatrix} \begin{mymatrix}{c}
x \\
y \\
z
\end{mymatrix} =\begin{mymatrix}{r}
3 \\
-1 \\
-2
\end{mymatrix} 
\end{equation*}
\end{enumerate}

Now give the solution in terms of $a,b,$ and $c$ to the following:
\begin{equation*}
\begin{mymatrix}{rrr}
1 & 0 & 3 \\
2 & 3 & 4 \\
1 & 0 & 2
\end{mymatrix} \begin{mymatrix}{c}
x \\
y \\
z
\end{mymatrix} =\begin{mymatrix}{c}
a \\
b \\
c
\end{mymatrix} 
\end{equation*}

\begin{sol}
\begin{enumerate}
\item $\begin{mymatrix}{c}
x \\
y \\
z
\end{mymatrix} =\begin{mymatrix}{c}
1 \\
-\vspace{0.05in}\frac{2}{3} \\
0
\end{mymatrix}$
\item $\begin{mymatrix}{c}
x \\
y \\
z
\end{mymatrix} = \begin{mymatrix}{r}
-12 \\
1 \\
5
\end{mymatrix}$
\end{enumerate}

$\begin{mymatrix}{c}
x \\
y \\
z
\end{mymatrix} = 
\begin{mymatrix}{c}
3c-2a \\
\frac{1}{3}b-\frac{2}{3}c \\
a-c
\end{mymatrix}$
\end{sol}
\end{ex}

\begin{ex}Show that if $A$ is an $n\times n$ invertible matrix and $X$
is a $n\times 1$ matrix such that $AX=B$ for $B$ an 
$n\times 1$ matrix, then $X=A^{-1}B$. 
\begin{sol}
Multiply both sides of $AX=B$ on the left by $A^{-1}$.
\end{sol}
\end{ex}

\begin{ex}Prove that if $A^{-1}$ exists and $AX=0$ then $X=0$. 
\begin{sol}
Multiply on both sides on the left by $A^{-1}.$ Thus
\[
0=A^{-1}0=A^{-1}\left( AX\right) =\left(
A^{-1}A\right) X=IX = X
\]
\end{sol}
\end{ex}

\begin{ex}\label{exerinverseprod}Show that if $A^{-1}$ exists for an $n\times n$
matrix, then it is unique. That is, if $BA=I$ and $AB=I,$ then $B=A^{-1}.$ 
\begin{sol}
 $A^{-1}=A^{-1}I=A^{-1}\left( AB\right) =\left( A^{-1}A\right) B=IB=B.$
\end{sol}
\end{ex}

\begin{ex}Show that if $A$ is an invertible $n\times n$ matrix, then so is 
$A^{T} $ and $\left( A^{T}\right) ^{-1}=\left( A^{-1}\right) ^{T}.$ 
\begin{sol}
 You need to show that $\left( A^{-1}\right) ^{T}$ acts like the inverse of $A^{T}
$ because from uniqueness in the above problem, this will imply it is the
inverse. From properties of the transpose,
\begin{eqnarray*}
A^{T}\left( A^{-1}\right) ^{T} &=&\left( A^{-1}A\right) ^{T}=I^{T}=I \\
\left( A^{-1}\right) ^{T}A^{T} &=&\left( AA^{-1}\right) ^{T}=I^{T}=I
\end{eqnarray*}
Hence $\left( A^{-1}\right) ^{T}=\left( A^{T}\right) ^{-1}$ and this last
matrix exists.
\end{sol}
\end{ex}

\begin{ex}Show $\left( AB\right) ^{-1}=B^{-1}A^{-1}$ by verifying that 
\begin{equation*}
AB\left(
B^{-1}A^{-1}\right) =I
\end{equation*} and 
\begin{equation*}
B^{-1}A^{-1}\left( AB\right) =I
\end{equation*}
\textbf{Hint:\ }Use Problem \ref{exerinverseprod}.
\begin{sol}
$\left( AB\right)
B^{-1}A^{-1}=A\left( BB^{-1}\right) A^{-1}=AA^{-1}=I$ $B^{-1}A^{-1}\left(
AB\right) =B^{-1}\left( A^{-1}A\right) B=B^{-1}IB=B^{-1}B=I$
\end{sol}
\end{ex}

\begin{ex}Show that $\left( ABC\right) ^{-1}=C^{-1}B^{-1}A^{-1}$ by verifying
that 
\[
\left( ABC\right) \left( C^{-1}B^{-1}A^{-1}\right) =I
\]
and 
\[\left( C^{-1}B^{-1}A^{-1}\right)\left( ABC\right) =I
\] 
\textbf{Hint:\ }Use Problem \ref{exerinverseprod}. 
\begin{sol}
The proof of this exercise follows from the previous one.
\end{sol}
\end{ex}

\begin{ex}If $A$ is invertible, show $\left( A^{2}\right) ^{-1}=\left(
A^{-1}\right) ^{2}.$ \textbf{Hint:\ }Use Problem \ref{exerinverseprod}. 
\begin{sol}
$A^{2}\left( A^{-1}\right) ^{2}=AAA^{-1}A^{-1}=AIA^{-1}=AA^{-1}=I$ $\left(
A^{-1}\right) ^{2}A^{2}=A^{-1}A^{-1}AA=A^{-1}IA=A^{-1}A=I$
\end{sol}
\end{ex}

\begin{ex}If $A$ is invertible, show $\left( A^{-1}\right) ^{-1}=A.$ 
\textbf{Hint:\ }Use Problem \ref{exerinverseprod}. 
\begin{sol}
 $A^{-1}A=AA^{-1}=I$ and so by
uniqueness, $\left( A^{-1}\right) ^{-1}=A$.
\end{sol}
\end{ex}

\end{enumialphparenastyle}
