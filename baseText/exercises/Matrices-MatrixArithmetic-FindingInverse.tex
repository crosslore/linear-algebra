\begin{enumialphparenastyle}

\begin{ex}
  For each of the following matrices, find the inverse if possible. If
  the inverse does not exist, explain why.
  \begin{equation*}
    A=\begin{mymatrix}{rr}
      2 & 1 \\
      -1 & 3
    \end{mymatrix},
    \quad
    B=\begin{mymatrix}{rr}
      0 & 1 \\
      5 & 3
    \end{mymatrix},
    \quad
    C=\begin{mymatrix}{rr}
      2 & 1 \\
      3 & 0
    \end{mymatrix},
    \quad
    D=\begin{mymatrix}{rr}
      2 & 1 \\
      4 & 2
    \end{mymatrix},
    \quad
    E=\begin{mymatrix}{rrr}
      0 & 1 & 2 \\
      1 & 2 & 5 \\
    \end{mymatrix}.
  \end{equation*}
  \begin{sol}
    \def\arraystretch{1.2}
    $A^{-1} =
    \begin{mymatrix}{rr}
      \frac{3}{7} & -\frac{1}{7} \\
      \frac{1}{7} & \frac{2}{7}
    \end{mymatrix}$,
    $B^{-1} = 
    \begin{mymatrix}{cc}
      -\frac{3}{5} & \frac{1}{5} \\
      1 & 0
    \end{mymatrix}$,
    $C^{-1} = 
    \begin{mymatrix}{cc}
      0 & \frac{1}{3} \\
      1 & -\frac{2}{3}
    \end{mymatrix}$.
    $D^{-1}$ does not exist because the {\rref} of $D$ is 
    $\begin{mymatrix}{cc}
      1 & \frac{1}{2} \\
      0 & 0
    \end{mymatrix}$, which has a row of zeros. $E^{-1}$ does not exist
    because $E$ is not a square matrix.
  \end{sol}
\end{ex}

\begin{ex}
  For each of the following matrices, find the inverse if possible. If
  the inverse does not exist, explain why.
  \begin{equation*}
    A=\begin{mymatrix}{rrr}
      1 & 2 & 3 \\
      2 & 1 & 4 \\
      1 & 0 & 2
    \end{mymatrix},
    \quad
    B=\begin{mymatrix}{rrr}
      1 & 0 & 3 \\
      2 & 3 & 4 \\
      1 & 0 & 2
    \end{mymatrix},
    \quad
    C=\begin{mymatrix}{rrr}
      1 & 2 & 3 \\
      2 & 1 & 4 \\
      4 & 5 & 10
    \end{mymatrix},
    \quad
    D=\begin{mymatrix}{rrrr}
      1 & 2 & 0 & 2 \\
      1 & 1 & 2 & 0 \\
      2 & 1 & -3 & 2 \\
      1 & 2 & 1 & 2
    \end{mymatrix}.
  \end{equation*}

  \begin{sol}
    \def\arraystretch{1.2}
    $A^{-1} =
    \begin{mymatrix}{rrr}
      -2 & 4 & -5 \\
      0 & 1 & -2 \\
      1 & -2 & 3
    \end{mymatrix}$,
    $B^{-1} = 
    \begin{mymatrix}{rrr}
      -2 & 0 & 3 \\
      0 & \frac{1}{3} & -\frac{2}{3} \\
      1 & 0 & -1
    \end{mymatrix}$.
    $C^{-1}$ does not exist because the {\rref} of $C$ is 
    $\begin{mymatrix}{ccc}
      1 & 0 & \frac{5}{3} \\
      0 & 1 & \frac{2}{3} \\
      0 & 0 & 0
    \end{mymatrix}$, which has a row of zeros. 
    $D^{-1} =
    \begin{mymatrix}{rrrr}
      -1 & \frac{1}{2} &  \frac{1}{2} &  \frac{1}{2} \\
      3 &  \frac{1}{2} & - \frac{1}{2} & - \frac{5}{2} \\
      -1 & 0 & 0 & 1 \\
      -2 & - \frac{3}{4} &  \frac{1}{4} &  \frac{9}{4}
    \end{mymatrix}$.
  \end{sol}
\end{ex}

\begin{ex}
  Let $A$ be a $2\times 2$ invertible matrix, with
  $A=\begin{mymatrix}{cc}
    a & b \\
    c & d
  \end{mymatrix}$. Find a formula for $A^{-1}$ in terms of $a,b,c,d$.
  \begin{sol}
    $\def\arraystretch{1.3}
    \begin{mymatrix}{cc}
      a & b \\
      c & d
    \end{mymatrix}^{-1}= \begin{mymatrix}{cc}
      \frac{d}{ad-bc} & -\frac{b}{ad-bc} \\
      -\frac{c}{ad-bc} & \frac{a}{ad-bc}
    \end{mymatrix}$.
  \end{sol}
\end{ex}

\begin{ex}
  Using the inverse of the matrix, find the solution to the systems:
  \begin{enumerate}
  \item
    \begin{equation*}
      \begin{mymatrix}{rr}
        2 & 4  \\
        1 & 1 
      \end{mymatrix} 
      \begin{mymatrix}{c}
        x \\
        y
      \end{mymatrix} =\begin{mymatrix}{r}
        1 \\
        2 
      \end{mymatrix},
    \end{equation*}
  \item
    \begin{equation*}
      \begin{mymatrix}{rr}
        2 & 4 \\
        1 & 1 
      \end{mymatrix} \begin{mymatrix}{c}
        x \\
        y 
      \end{mymatrix} =\begin{mymatrix}{r}
        2 \\
        0 
      \end{mymatrix}.
    \end{equation*}
  \end{enumerate}
  Now give the solution in terms of $a$ and $b$ to
  \begin{equation*}
    \begin{mymatrix}{rr}
      2 & 4 \\
      1 & 1 
    \end{mymatrix}
    \begin{mymatrix}{c}
      x \\
      y
    \end{mymatrix}
    =
    \begin{mymatrix}{c}
      a \\
      b
    \end{mymatrix}.
  \end{equation*}
  % \begin{sol}
  % \end{sol}
\end{ex}

\begin{ex}
  Using the inverse of the matrix, find the solution to the systems:
  \begin{enumerate}
  \item
    \begin{equation*}
      \begin{mymatrix}{rrr}
        1 & 0 & 3 \\
        2 & 3 & 4 \\
        1 & 0 & 2
      \end{mymatrix} \begin{mymatrix}{c}
        x \\
        y \\
        z
      \end{mymatrix} =\begin{mymatrix}{r}
        1 \\
        0 \\
        1
      \end{mymatrix},
    \end{equation*}
  \item
    \begin{equation*}
      \begin{mymatrix}{rrr}
        1 & 0 & 3 \\
        2 & 3 & 4 \\
        1 & 0 & 2
      \end{mymatrix} \begin{mymatrix}{c}
        x \\
        y \\
        z
      \end{mymatrix} =\begin{mymatrix}{r}
        3 \\
        -1 \\
        -2
      \end{mymatrix}.
    \end{equation*}
  \end{enumerate}
  Now give the solution in terms of $a,b$, and $c$ to the following:
  \begin{equation*}
    \begin{mymatrix}{rrr}
      1 & 0 & 3 \\
      2 & 3 & 4 \\
      1 & 0 & 2
    \end{mymatrix} \begin{mymatrix}{c}
      x \\
      y \\
      z
    \end{mymatrix} =\begin{mymatrix}{c}
      a \\
      b \\
      c
    \end{mymatrix}.
  \end{equation*}

  \begin{sol}
    \begin{enumerate}
    \item $\def\arraystretch{1.2}
      \begin{mymatrix}{c}
        x \\
        y \\
        z
      \end{mymatrix} =\begin{mymatrix}{r}
        1 \\
        -\frac{2}{3} \\
        0
      \end{mymatrix}$.
    \item $\begin{mymatrix}{c}
        x \\
        y \\
        z
      \end{mymatrix} = \begin{mymatrix}{r}
        -12 \\
        1 \\
        5
      \end{mymatrix}$.
    \end{enumerate}
    $\def\arraystretch{1.2}
    \begin{mymatrix}{c}
      x \\
      y \\
      z
    \end{mymatrix} = 
    \begin{mymatrix}{c}
      3c-2a \\
      \frac{1}{3}b-\frac{2}{3}c \\
      a-c
    \end{mymatrix}$.
  \end{sol}
\end{ex}

\begin{ex}
  Show that if $A$ is an $n\times n$ invertible matrix and $X$ and $B$
  are $n\times 1$-matrices such that $AX=B$, then $X=A^{-1}B$.
  \begin{sol}
    Multiply both sides of $AX=B$ on the left by $A^{-1}$.
  \end{sol}
\end{ex}

\begin{ex}
  Prove that if $A^{-1}$ exists and $AX=0$ then $X=0$.
  \begin{sol}
    Multiply on both sides on the left by $A^{-1}$. Thus
    $0=A^{-1}0=A^{-1}\tup{AX} =\tup{ A^{-1}A} X=IX = X$.
  \end{sol}
\end{ex}

\begin{ex}\label{exer-inverse-prod}
  Show $(AB)^{-1}=B^{-1}A^{-1}$ by verifying that
  \begin{equation*}
    AB(B^{-1}A^{-1}) = I
    \quad\mbox{and}\quad
    B^{-1}A^{-1}(AB) = I.
  \end{equation*}
  \vspace{-4ex}
  \begin{sol}
    $\tup{AB}
    B^{-1}A^{-1}=A\tup{BB^{-1}} A^{-1}=AA^{-1}=I$ and  $B^{-1}A^{-1}\tup{
      AB} =B^{-1}\tup{A^{-1}A} B=B^{-1}IB=B^{-1}B=I$.
  \end{sol}
\end{ex}

\begin{ex}
  Is it possible to have matrices $A$ and $B$ such that $AB=I$, while
  $BA=0$? If it is possible, give an example of such matrices. If it
  is not possible, explain why.
  \begin{sol}
    It is not possible, because in that case, we would have
    $A=IA=(AB)A=A(BA)=A0=0$, and therefore $AB=0$, contradicting $AB=I$.
  \end{sol}
\end{ex}

\begin{ex}
  Show that $\tup{ABC} ^{-1}=C^{-1}B^{-1}A^{-1}$ by verifying that
  \begin{equation*}
    \tup{ABC} \tup{C^{-1}B^{-1}A^{-1}} = I
    \quad\mbox{and}\quad
    \tup{C^{-1}B^{-1}A^{-1}}\tup{ABC} = I.
  \end{equation*} 
\end{ex}

\begin{ex}
  If $A$ is invertible, show that $A^2$ is invertible and
  $(A^{2})^{-1}=(A^{-1})^{2}$.
  \begin{sol}
    This follows from Problem~\ref{exer-inverse-prod} by taking $A=B$.
  \end{sol}
\end{ex}

\begin{ex}If $A$ is invertible, show $(A^{-1})^{-1}=A$.
  Hint: Use the uniqueness of inverses. 
  \begin{sol}
    $A^{-1}A=AA^{-1}=I$ and so $A$ is an inverse of $A^{-1}$. Since
    $(A^{-1})^{-1}$, by definition, is also an inverse of $A^{-1}$, we
    have $(A^{-1})^{-1}=A$ by uniqueness.
  \end{sol}
\end{ex}

\begin{ex}
  Determine whether $B$ is a right inverse, left inverse, both, or
  neither of $A$.
  \begin{enumerate}
  \item
    \begin{equation*}
      A = \begin{mymatrix}{rrr}
        4 & 2 & 1 \\
        2 & 1 & 1 \\
      \end{mymatrix},
      \quad
      B = \begin{mymatrix}{rrr}
        1 & -1 \\
        -1 & 1 \\
        -1 & 2 \\
      \end{mymatrix}.
    \end{equation*}
  \item
    \begin{equation*}
      A = \begin{mymatrix}{rrr}
        1 & 0 \\
        0 & 1 \\
        0 & 2 \\
      \end{mymatrix},
      \quad
      B = \begin{mymatrix}{rrr}
        1 & 2 & -1 \\
        0 & 1 & 0 \\
      \end{mymatrix}.
    \end{equation*}
  \item
    \begin{equation*}
      A = \begin{mymatrix}{rrr}
        1 & 2 \\
        3 & 7 \\
      \end{mymatrix},
      \quad
      B = \begin{mymatrix}{rrr}
        7 & -2 \\
        -3 & 1 \\
      \end{mymatrix}.
    \end{equation*}
  \item
    \begin{equation*}
      A = \begin{mymatrix}{rrr}
        1 & 2 \\
        3 & 1 \\
      \end{mymatrix},
      \quad
      B = \begin{mymatrix}{rrr}
        -1 & 2 \\
        1 & -1 \\
      \end{mymatrix}.
    \end{equation*}
  \end{enumerate}
  \begin{sol}
    (a) $B$ is a right inverse of $A$, (b) $B$ is a left inverse of
    $A$, (c) $B$ is both a right inverse and left inverse of $A$, (d)
    $B$ is neither a right inverse nor a left inverse of $A$.
  \end{sol}
\end{ex}

\begin{ex}
  Show that right inverses are not unique by giving an example of
  matrices $A,B,C$ such that both $B$ and $C$ are right inverses of
  $A$, but $B\neq C$. 
  \begin{sol}
    $A=\begin{mymatrix}{rrr}
      1 & 0 & 0 \\
      0 & 1 & 0 \\
    \end{mymatrix}$,
    $B=\begin{mymatrix}{rr}
      1 & 0 \\
      0 & 1 \\
      0 & 0 \\
    \end{mymatrix}$,
    $C=\begin{mymatrix}{rr}
      1 & 0 \\
      0 & 1 \\
      1 & 1 \\
    \end{mymatrix}$.
  \end{sol}
\end{ex}


\end{enumialphparenastyle}
