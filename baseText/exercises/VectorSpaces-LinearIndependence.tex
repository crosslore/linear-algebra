
\begin{ex}
  For each of the following sets of polynomials, determine whether the
  set is linearly independent. If it is linearly dependent, write one
  polynomial as a linear combination of the other polynomials in the set.
  \begin{enumerate}
  \item $\set{x+1,~ x^2+2,~ x^2-x-3}$.
  \item $\set{x^2+x,~ -2x^2-4x-6,~ 2x-2}$.
  \end{enumerate}
\end{ex}

\begin{ex}
  Determine whether each of the following sets of matrices is linearly
  independent. If it is linearly dependent, write one matrix as a
  linear combination of the other matrices in the set.
  \begin{enumerate}
  \item $\set{
      \begin{mymatrix}{rr}
        1 & 2 \\
        0 & 1
      \end{mymatrix},~
      \begin{mymatrix}{rr}
        -7 & 2 \\
        -2 & -3
      \end{mymatrix},~
      \begin{mymatrix}{rr}
        4 & 0 \\
        1 & 2
      \end{mymatrix}
    }$.
  \item $\set{
      \begin{mymatrix}{rr}
        1 & 0 \\
        0 & 1
      \end{mymatrix},~
      \begin{mymatrix}{rr}
        0 & 1 \\
        0 & 1
      \end{mymatrix},~
      \begin{mymatrix}{rr}
        1 & 0 \\
        1 & 0
      \end{mymatrix},~
      \begin{mymatrix}{rr}
        0 & 0 \\
        1 & 1
      \end{mymatrix}
    }$.
  \end{enumerate}
\end{ex}

\begin{ex}
  Consider polynomials
  \begin{equation*}
    \set{a_ix^3+b_ix^2+c_ix+d_i \mid i=1,2,3,4}.
  \end{equation*}
  Show that this collection of polynomials is linearly independent if
  and only if
  \begin{equation*}
    \begin{mymatrix}{cccc}
      a_1 & b_1 & c_1 & d_1 \\
      a_2 & b_2 & c_2 & d_2 \\
      a_3 & b_3 & c_3 & d_3 \\
      a_4 & b_4 & c_4 & d_4
    \end{mymatrix}
  \end{equation*}
  is an invertible matrix.
  \begin{sol}
    Let $p_i(x)$ denote the $i\th$ of these polynomials. Suppose
    $C_1p_1(x) + \ldots + C_4p_4(x) = 0$. Then collecting terms
    according to the exponent of $x$, we have
    \begin{eqnarray*}
      C_1 a_1 + C_2 a_2 + C_3 a_3 + C_4 a_4 &=& 0, \\
      C_1 b_1 + C_2 b_2 + C_3 b_3 + C_4 b_4 &=& 0, \\
      C_1 c_1 + C_2 c_2 + C_3 c_3 + C_4 c_4 &=& 0, \\
      C_1 d_1 + C_2 d_2 + C_3 d_3 + C_4 d_4 &=& 0.
    \end{eqnarray*}
    The matrix of coefficients is just the transpose of the above
    matrix. There exists a non-trivial solution if and only if the
    determinant of this matrix equals 0.
  \end{sol}
\end{ex}

\begin{ex}
  Assume $\vect{u},\vect{v},\vect{w}$ are linearly independent
  elements of some vector space $V$. Consider the set of vectors
  \begin{equation*}
    \textstyle
    R = \set{2\vect{u} - \vect{w},~
      \vect{w} + \vect{v},~
      3\vect{v} + \frac{1}{2} \vect{u} }.
  \end{equation*}
  Determine whether $R$ is linearly independent.

  \begin{sol}
    To determine whether $R$ is linearly independent, we must solve
    the equation
    \begin{equation*}
      \textstyle
      a(2\vect{u} - \vect{w}) + b(\vect{w} + \vect{v}) + c( 3\vect{v} +
      \frac{1}{2}\vect{u}) ~=~ \vect{0}.
    \end{equation*}
    If the only solution is the trivial solution, the set is linearly
    independent. We rewrite the
    equation as follows.
    \begin{equation*}
      \textstyle
      (2a + \frac{1}{2}c) \vect{u} + (b+3c)\vect{v} + (-a + b) \vect{w} ~=~ \vect{0}.
    \end{equation*}
    Since $\vect{u},\vect{v},\vect{w}$ are linearly independent, the
    coefficients in the last equation must all equal $0$.
    In other words:
    \begin{eqnarray*}
      \textstyle
      2a + \frac{1}{2} c &=& 0, \\
      b + 3c &=& 0, \\
      -a + b &=& 0.
    \end{eqnarray*}
    We solve and find that the unique solution is
    $a=b=c=0$. Therefore, the set $R$ is linearly independent.
  \end{sol}
\end{ex}

