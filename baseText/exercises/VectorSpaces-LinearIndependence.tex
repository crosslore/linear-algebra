\section*{Exercises}


  \begin{ex} Consider the vector space of polynomials of degree at
    most $2$, denoted $\Poly_{2}$. Determine whether the following is
    a basis for $\Poly_{2}$.
\begin{equation*}
\set{x^{2}+x+1,2x^{2}+2x+1,x+1}
\end{equation*}
\textbf{Hint:\ }There is a isomorphism from $\R^{3}$ to
$\Poly_{2}$. It is defined as follows:\
\begin{equation*}
T\vect{e}_{1}=1,T\vect{e}_{2}=x,T\vect{e}_{3}=x^{2}
\end{equation*}
Then extend $T$ linearly. Thus
\begin{equation*}
T\begin{mymatrix}{c}
1 \\
1 \\
1
\end{mymatrix} =x^{2}+x+1,T\begin{mymatrix}{c}
1 \\
2 \\
2
\end{mymatrix} =2x^{2}+2x+1,T\begin{mymatrix}{c}
1 \\
1 \\
0
\end{mymatrix} =1+x
\end{equation*}
It follows that if
\begin{equation*}
\set{\begin{mymatrix}{c}
1 \\
1 \\
1
\end{mymatrix} ,\begin{mymatrix}{c}
1 \\
2 \\
2
\end{mymatrix} ,\begin{mymatrix}{c}
1 \\
1 \\
0
\end{mymatrix} }
\end{equation*}
is a basis for $\R^{3}$, then the polynomials will be a basis for
$\Poly_{2}$ because they will be independent. Recall that an isomorphism
takes a linearly independent set to a linearly independent set. Also, since
$T$ is an isomorphism, it preserves all linear relations.
%\begin{sol}
%\end{sol}
\end{ex}

\begin{ex} Find a basis in $\Poly_{2}$ for the subspace
\begin{equation*}
\sspan\set{1+x+x^{2},1+2x,1+5x-3x^{2}}
\end{equation*}
If the above three vectors do not yield a basis, exhibit one of them as a
linear combination of the others. \textbf{Hint:\ }This is the situation in
which you have a spanning set and you want to cut it down to form a linearly
independent set which is also a spanning set. Use the same isomorphism
above. Since $T$ is an isomorphism, it preserves all linear relations so if
such can be found in $\R^{3}$, the same linear relations will be
present in $\Poly_{2}$.
%\begin{sol}
%\end{sol}
\end{ex}


\begin{ex} Find a basis in $\Poly_{3}$ for the subspace
\begin{equation*}
\sspan\set{
1+x-x^{2}+x^{3},1+2x+3x^{3},-1+3x+5x^{2}+7x^{3},1+6x+4x^{2}+11x^{3}}
\end{equation*}
If the above three vectors do not yield a basis, exhibit one of them as a
linear combination of the others.
%\begin{sol}
%\end{sol}
\end{ex}


\begin{ex} Find a basis in $\Poly_{3}$ for the subspace
\begin{equation*}
\sspan\set{
1+x-x^{2}+x^{3},1+2x+3x^{3},-1+3x+5x^{2}+7x^{3},1+6x+4x^{2}+11x^{3}}
\end{equation*}
If the above three vectors do not yield a basis, exhibit one of them as a
linear combination of the others.
%\begin{sol}
%\end{sol}
\end{ex}


\begin{ex} Find a basis in $\Poly_{3}$ for the subspace
\begin{equation*}
\sspan\set{
x^{3}-2x^{2}+x+2,3x^{3}-x^{2}+2x+2,7x^{3}+x^{2}+4x+2,5x^{3}+3x+2}
\end{equation*}
If the above three vectors do not yield a basis, exhibit one of them as a
linear combination of the others.
%\begin{sol}
%\end{sol}
\end{ex}


\begin{ex} Find a basis in $\Poly_{3}$ for the subspace
\begin{equation*}
\sspan\set{
x^{3}+2x^{2}+x-2,3x^{3}+3x^{2}+2x-2,3x^{3}+x+2,3x^{3}+x+2}
\end{equation*}
If the above three vectors do not yield a basis, exhibit one of them as a
linear combination of the others.
%\begin{sol}
%\end{sol}
\end{ex}


\begin{ex} Find a basis in $\Poly_{3}$ for the subspace
\begin{equation*}
\sspan\set{
x^{3}-5x^{2}+x+5,3x^{3}-4x^{2}+2x+5,5x^{3}+8x^{2}+2x-5,11x^{3}+6x+5}
\end{equation*}
If the above three vectors do not yield a basis, exhibit one of them as a
linear combination of the others.
%\begin{sol}
%\end{sol}
\end{ex}


\begin{ex} Find a basis in $\Poly_{3}$ for the subspace
\begin{equation*}
\sspan\set{
x^{3}-3x^{2}+x+3,3x^{3}-2x^{2}+2x+3,7x^{3}+7x^{2}+3x-3,7x^{3}+4x+3}
\end{equation*}
If the above three vectors do not yield a basis, exhibit one
of them as a linear combination of the others.
%\begin{sol}
%\end{sol}
\end{ex}


\begin{ex} Find a basis in $\Poly_{3}$ for the subspace
\begin{equation*}
\sspan\set{
x^{3}-x^{2}+x+1,3x^{3}+2x+1,4x^{3}+x^{2}+2x+1,3x^{3}+2x-1}
\end{equation*}
If the above three vectors do not yield a basis, exhibit one
of them as a linear combination of the others.
%\begin{sol}
%\end{sol}
\end{ex}


\begin{ex} Find a basis in $\Poly_{3}$ for the subspace
\begin{equation*}
\sspan\set{
x^{3}-x^{2}+x+1,3x^{3}+2x+1,13x^{3}+x^{2}+8x+4,3x^{3}+2x-1}
\end{equation*}
If the above three vectors do not yield a basis, exhibit one
of them as a linear combination of the others.
%\begin{sol}
%\end{sol}
\end{ex}


\begin{ex} Find a basis in $\Poly_{3}$ for the subspace
\begin{equation*}
\sspan\set{
x^{3}-3x^{2}+x+3,3x^{3}-2x^{2}+2x+3,-5x^{3}+5x^{2}-4x-6,7x^{3}+4x-3}
\end{equation*}
If the above three vectors do not yield a basis, exhibit one
of them as a linear combination of the others.
%\begin{sol}
%\end{sol}
\end{ex}


\begin{ex} Find a basis in $\Poly_{3}$ for the subspace
\begin{equation*}
\sspan\set{
x^{3}-2x^{2}+x+2,3x^{3}-x^{2}+2x+2,7x^{3}-x^{2}+4x+4,5x^{3}+3x-2}
\end{equation*}
If the above three vectors do not yield a basis, exhibit one
of them as a linear combination of the others.
%\begin{sol}
%\end{sol}
\end{ex}


\begin{ex} Find a basis in $\Poly_{3}$ for the subspace
\begin{equation*}
\sspan\set{
x^{3}-2x^{2}+x+2,3x^{3}-x^{2}+2x+2,3x^{3}+4x^{2}+x-2,7x^{3}-x^{2}+4x+4\right
}
\end{equation*}
If the above three vectors do not yield a basis, exhibit one
of them as a linear combination of the others.
%\begin{sol}
%\end{sol}
\end{ex}


\begin{ex} Find a basis in $\Poly_{3}$ for the subspace
\begin{equation*}
\sspan\set{
x^{3}-4x^{2}+x+4,3x^{3}-3x^{2}+2x+4,-3x^{3}+3x^{2}-2x-4,-2x^{3}+4x^{2}-2x-4
}
\end{equation*}
If the above three vectors do not yield a basis, exhibit one
of them as a linear combination of the others.
%\begin{sol}
%\end{sol}
\end{ex}


\begin{ex} Find a basis in $\Poly_{3}$ for the subspace
\begin{equation*}
\sspan\set{
x^{3}+2x^{2}+x-2,3x^{3}+3x^{2}+2x-2,5x^{3}+x^{2}+2x+2,10x^{3}+10x^{2}+6x-6
}
\end{equation*}
If the above three vectors do not yield a basis, exhibit one
of them as a linear combination of the others.
%\begin{sol}
%\end{sol}
\end{ex}


\begin{ex} Find a basis in $\Poly_{3}$ for the subspace
\begin{equation*}
\sspan\set{
x^{3}+x^{2}+x-1,3x^{3}+2x^{2}+2x-1,x^{3}+1,4x^{3}+3x^{2}+2x-1}
\end{equation*}
If the above three vectors do not yield a basis, exhibit one
of them as a linear combination of the others.
%\begin{sol}
%\end{sol}
\end{ex}


\begin{ex} Find a basis in $\Poly_{3}$ for the subspace
\begin{equation*}
\sspan\set{
x^{3}-x^{2}+x+1,3x^{3}+2x+1,x^{3}+2x^{2}-1,4x^{3}+x^{2}+2x+1}
\end{equation*}
If the above three vectors do not yield a basis, exhibit one
of them as a linear combination of the others.
%\begin{sol}
%\end{sol}
\end{ex}


\begin{ex} Here are some vectors.
\begin{equation*}
\set{x^{3}+x^{2}-x-1,3x^{3}+2x^{2}+2x-1}
\end{equation*}
If these are linearly independent, extend to a basis for all of $\Poly_{3}$.
%\begin{sol}
%\end{sol}
\end{ex}


\begin{ex} Here are some vectors.
\begin{equation*}
\set{x^{3}-2x^{2}-x+2,3x^{3}-x^{2}+2x+2}
\end{equation*}
If these are linearly independent, extend to a basis for all of $\Poly_{3}$.
%\begin{sol}
%\end{sol}
\end{ex}


\begin{ex} Here are some vectors.
\begin{equation*}
\set{x^{3}-3x^{2}-x+3,3x^{3}-2x^{2}+2x+3}
\end{equation*}
If these are linearly independent, extend to a basis for all of $\Poly_{3}$.
%\begin{sol}
%\end{sol}
\end{ex}


\begin{ex} Here are some vectors.
\begin{equation*}
\set{x^{3}-2x^{2}-3x+2,3x^{3}-x^{2}-6x+2,-8x^{3}+18x+10}
\end{equation*}
If these are linearly independent, extend to a basis for all of $\Poly_{3}$.
%\begin{sol}
%\end{sol}
\end{ex}


\begin{ex} Here are some vectors.
\begin{equation*}
\set{x^{3}-3x^{2}-3x+3,3x^{3}-2x^{2}-6x+3,-8x^{3}+18x+40}
\end{equation*}
If these are linearly independent, extend to a basis for all of $\Poly_{3}$.
%\begin{sol}
%\end{sol}
\end{ex}


\begin{ex} Here are some vectors.
\begin{equation*}
\set{x^{3}-x^{2}+x+1,3x^{3}+2x+1,4x^{3}+2x+2}
\end{equation*}
If these are linearly independent, extend to a basis for all of $\Poly_{3}$.
%\begin{sol}
%\end{sol}
\end{ex}


\begin{ex} Here are some vectors.
\begin{equation*}
\set{x^{3}+x^{2}+2x-1,3x^{3}+2x^{2}+4x-1,7x^{3}+8x+23}
\end{equation*}
If these are linearly independent, extend to a basis for all of $\Poly_{3}$.
%\begin{sol}
%\end{sol}
\end{ex}


\begin{ex} Determine if the following set is linearly independent. If it is linearly dependent, write one vector as a linear combination of the other vectors in the set.
\[
\set{x+1, x^2 + 2, x^2 - x -3 }
\]
%\begin{sol}
%\end{sol}
\end{ex}

\begin{ex} Determine if the following set is linearly independent. If it is linearly dependent, write one vector as a linear combination of the other vectors in the set.
\[
\set{x^2 + x, -2x^2 -4x -6 , 2x - 2 }
\]
%\begin{sol}
%\end{sol}
\end{ex}

\begin{ex} Determine if the following set is linearly independent. If it is linearly dependent, write one vector as a linear combination of the other vectors in the set.
\[
\set{\begin{mymatrix}{rr}
1 & 2 \\
0 & 1
\end{mymatrix}, \begin{mymatrix}{rr}
-7 & 2 \\
-2 & -3
\end{mymatrix}, \begin{mymatrix}{rr}
4 & 0 \\
1 & 2
\end{mymatrix}
 }
\]
%\begin{sol}
%\end{sol}
\end{ex}

\begin{ex} Determine if the following set is linearly independent. If it is linearly dependent, write one vector as a linear combination of the other vectors in the set.
\[
\set{\begin{mymatrix}{rr}
1 & 0 \\
0 & 1
\end{mymatrix}, \begin{mymatrix}{rr}
0 & 1 \\
0 & 1
\end{mymatrix}, \begin{mymatrix}{rr}
1 & 0 \\
1 & 0
\end{mymatrix}, \begin{mymatrix}{rr}
0 & 0 \\
1 & 1
\end{mymatrix}
 }
\]
%\begin{sol}
%\end{sol}
\end{ex}

\begin{ex} If you have $5$ vectors in $\R^{5}$ and the vectors are
linearly independent, can it always be concluded they span $\R^{5}$?
\begin{sol}
Yes. If not, there would exist a vector not in the span. But then
you could add in this vector and obtain a linearly independent set of
vectors with more vectors than a basis.
\end{sol}
\end{ex}

\begin{ex} If you have $6$ vectors in $\R^{5}$, is it possible they are
linearly independent? Explain.
\begin{sol}
No. They can't be.
\end{sol}
\end{ex}

\begin{ex} Let $\Poly_3$ be the polynomials of degree no more than 3. Determine which
of the following are bases for this vector space.

\begin{enumerate}
\item $\set{x+1,x^{3}+x^{2}+2x,x^{2}+x,x^{3}+x^{2}+x} $

\item $\set{x^{3}+1,x^{2}+x,2x^{3}+x^{2},2x^{3}-x^{2}-3x+1} $
\end{enumerate}

\begin{sol}
\begin{enumerate}
\item
\item
Suppose
\[
c_{1}(x^{3}+1) +c_{2}(x^{2}+x) +c_{3}(
2x^{3}+x^{2}) +c_{4}(2x^{3}-x^{2}-3x+1) =0
\]
Then combine the terms according to power of $x$.
\[
(c_{1}+2c_{3}+2c_{4}) x^{3}+(c_{2}+c_{3}-c_{4})
x^{2}+(c_{2}-3c_{4}) x+(c_{1}+c_{4}) =0
\]
Is there a non-zero solution to the system $
\begin{array}{c}
c_{1}+2c_{3}+2c_{4}=0 \\
c_{2}+c_{3}-c_{4}=0 \\
c_{2}-3c_{4}=0 \\
c_{1}+c_{4}=0
\end{array}
$, Solution is:
\[
\mat{c_{1}=0,c_{2}=0,c_{3}=0,c_{4}=0}
\]
Therefore, these are linearly independent.
\end{enumerate}
\end{sol}
\end{ex}

\begin{ex} In the context of the above problem, consider polynomials
\begin{equation*}
\set{a_{i}x^{3}+b_{i}x^{2}+c_{i}x+d_{i},\ i=1,2,3,4}
\end{equation*}
Show that this collection of polynomials is linearly independent on an
interval $\mat{s,t} $ if and only if
\begin{equation*}
\begin{mymatrix}{cccc}
a_{1} & b_{1} & c_{1} & d_{1} \\
a_{2} & b_{2} & c_{2} & d_{2} \\
a_{3} & b_{3} & c_{3} & d_{3} \\
a_{4} & b_{4} & c_{4} & d_{4}
\end{mymatrix}
\end{equation*}
is an invertible matrix.
\begin{sol}
Let $p_{i}(x) $ denote the $i\th$ of
these polynomials. Suppose $\sum_{i}C_{i}p_{i}(x) =0$. Then
collecting terms according to the exponent of $x$, you need to have
\begin{eqnarray*}
C_{1}a_{1}+C_{2}a_{2}+C_{3}a_{3}+C_{4}a_{4} &=&0 \\
C_{1}b_{1}+C_{2}b_{2}+C_{3}b_{3}+C_{4}b_{4} &=&0 \\
C_{1}c_{1}+C_{2}c_{2}+C_{3}c_{3}+C_{4}c_{4} &=&0 \\
C_{1}d_{1}+C_{2}d_{2}+C_{3}d_{3}+C_{4}d_{4} &=&0
\end{eqnarray*}
The matrix of coefficients is just the transpose of the above matrix. There
exists a non-trivial solution if and only if the determinant of this matrix
equals 0.
\end{sol}
\end{ex}

\begin{ex} Let the field of scalars be $\Q$, the rational numbers and let
the vectors be of the form $a+b\sqrt{2}$ where $a,b$ are rational numbers.
Show that this collection of vectors is a vector space with field of scalars
$\Q$ and give a basis for this vector space.
\begin{sol}
When you add two of these you get one and when you multiply one of these by
a scalar, you get another one. A basis is $\set{1,\sqrt{2}}$. By
definition, the span of these gives the collection of vectors. Are they
independent? Say $a+b\sqrt{2}=0$ where $a,b$ are rational numbers. If $a\neq
0$, then $b\sqrt{2}=-a$ which can't happen since $a$ is rational. If $b\neq
0$, then $-a=b\sqrt{2}$ which again can't happen because on the left is a
rational number and on the right is an irrational. Hence both $a,b=0$ and so
this is a basis.
\end{sol}
\end{ex}

\begin{ex} Suppose $V$ is a finite dimensional vector space. Based on the
exchange theorem above, it was shown that any two bases have the same number
of vectors in them. Give a different proof of this fact using the earlier
material in the book. \textbf{Hint: }Suppose $\set{\vect{x}_{1}
,\ldots,\vect{x}_{n}} $ and $\set{\vect{y}_{1},\ldots,\vect{y}
_{m}} $ are two bases with $m<n$. Then define
\begin{equation*}
\phi :\R^{n}\to V,\psi :\R^{m}\to V
\end{equation*}
by
\begin{equation*}
\phi (\vect{a}) = \sum_{k=1}^{n}a_{k}\vect{x}
_{k},\;\psi (\vect{b}) = \sum_{j=1}^{m}b_{j}\vect{y}_{j}
\end{equation*}
Consider the linear transformation, $\psi ^{-1}\circ \phi$. Argue it is a
one to one and onto mapping from $\R^{n}$ to $\R^{m}$. Now
consider a matrix of this linear transformation and its {\rref}.
\begin{sol}
This is obvious because
when you add two of these you get one and when you multiply one of these by
a scalar, you get another one. A basis is $\set{1,\sqrt{2}}$. By
definition, the span of these gives the collection of vectors. Are they
independent? Say $a+b\sqrt{2}=0$ where $a,b$ are rational numbers. If $a\neq
0$, then $b\sqrt{2}=-a$ which can't happen since $a$ is rational. If $b\neq
0$, then $-a=b\sqrt{2}$ which again can't happen because on the left is a
rational number and on the right is an irrational. Hence both $a,b=0$ and so
this is a basis.
\end{sol}
\end{ex}

