\section*{Exercises}

\begin{ex}
  Which of the following sets are subspaces of $\R^3$? Explain.
  \begin{enumerate}
  \item $V_1=\set{\left.\vect{u}=\begin{mymatrix}{c}
          u_1 \\ u_2 \\ u_3
        \end{mymatrix}~\right\vert~ \abs{u_1} \leq 4}$.
  \item $V_2=\set{\left.\vect{u}=\begin{mymatrix}{c}
          u_1 \\ u_2 \\ u_3
        \end{mymatrix}~\right\vert~\text{$u_i\geq 0$ for each $i=1,2,3$}}$.
  \item $V_3=\set{\left.\vect{u}=\begin{mymatrix}{c}
          u_1 \\ u_2 \\ u_3
        \end{mymatrix}~\right\vert~ u_3+u_1=2u_2}$.
  \item $V_4=\set{\left.\vect{u}=\begin{mymatrix}{c}
          u_1 \\ u_2 \\ u_3
        \end{mymatrix}~\right\vert~ u_3\geq u_1}$.
  \item $V_5=\set{\left.\vect{u}=\begin{mymatrix}{c}
          u_1 \\ u_2 \\ u_3
        \end{mymatrix}~\right\vert~ u_3=u_1=0}$.
  \end{enumerate}

  \begin{sol}
    \begin{enumerate}
    \item
      No. We have $\begin{mymatrix}{r}
        1 \\ 0 \\ 0 \\ 0
      \end{mymatrix} \in V_1$ but $10\begin{mymatrix}{r}
        1 \\ 0 \\ 0 \\ 0
      \end{mymatrix} \notin V_1$.
    \item
      This is not a subspace. The vector $\begin{mymatrix}{r}
        1 \\ 1 \\ 1 \\ 1
      \end{mymatrix} $
      is in $V_2$. However, $(-1) \begin{mymatrix}{r}
        1 \\ 1 \\ 1 \\ 1
      \end{mymatrix} $ is not.
    \item This is a subspace. It contains the zero vector and is
      closed with respect to vector addition and scalar
      multiplication.
    \item
      This
      is not a subspace. The vector $\begin{mymatrix}{r}
        0 \\ 0 \\ 1 \\ 0
      \end{mymatrix} $ is in $V_4$. However $(-1) \begin{mymatrix}{r}
        0 \\ 0 \\ 1 \\ 0
      \end{mymatrix}  = \begin{mymatrix}{r}
        0 \\ 0 \\ -1 \\ 0
      \end{mymatrix} $ is not.
    \item This is a subspace. It contains the zero vector and is
      closed with respect to vector addition and scalar
      multiplication.
    \end{enumerate}
  \end{sol}
\end{ex}

\begin{ex}
  Let $\vect{w}\in \R^4$ be a given fixed vector. Let
  \begin{equation*}
    M=\set{\left.\vect{u}
        =\begin{mymatrix}{c}
          u_1 \\ u_2 \\ u_3 \\ u_4
        \end{mymatrix} \in \R^4~\right\vert~ \vect{w}\dotprod \vect{u}
      =0}.
  \end{equation*}
  Is $M$ a subspace of $\R^4$? Explain.
  \begin{sol}
    Yes, this is a subspace because it contains the zero vector and is
    closed with respect to vector addition and scalar
    multiplication. For example, if $\vect{u},\vect{v}\in M$, then
    $\vect{w}\dotprod\vect{u}=0$ and $\vect{w}\dotprod\vect{v}=0$,
    therefore $\vect{w}\dotprod(\vect{u}+\vect{v})=0$, therefore
    $\vect{u}+\vect{v}\in M$.
  \end{sol}
\end{ex}

\begin{ex}
  Let $\vect{w},\vect{v}$ be given vectors in $\R^4$ and define
  \begin{equation*}
    M=\set{\left.\vect{u}=\begin{mymatrix}{c}
        u_1 \\ u_2 \\ u_3 \\ u_4
      \end{mymatrix} \in \R^4
      ~\right\vert~
    \text{$\vect{w}\dotprod \vect{u}=0$ and $\vect{v}\dotprod \vect{u}=0$}}.
  \end{equation*}
  Is $M$ a subspace of $\R^4$? Explain.
  \begin{sol}
    Yes, this is a subspace.
  \end{sol}
\end{ex}

\begin{ex}
  In this exercise, we use scalars from the field $\Z_2$ of integers
  modulo $2$ instead of real numbers (see Section~\ref{sec:fields},
  ``Fields''). Which of the following sets are subspaces of
  $(\Z_2)^3$?
  \begin{enumerate}
  \item $V_1 = \set{
      \begin{mymatrix}{r} 1 \\ 1 \\ 0 \end{mymatrix},
      \begin{mymatrix}{r} 0 \\ 1 \\ 1 \end{mymatrix},
      \begin{mymatrix}{r} 1 \\ 0 \\ 1 \end{mymatrix},
      \begin{mymatrix}{r} 0 \\ 0 \\ 0 \end{mymatrix}
    }$.
  \item $V_2 = \set{
      \begin{mymatrix}{r} 1 \\ 1 \\ 1 \end{mymatrix},
      \begin{mymatrix}{r} 1 \\ 1 \\ 0 \end{mymatrix},
      \begin{mymatrix}{r} 1 \\ 0 \\ 0 \end{mymatrix},
      \begin{mymatrix}{r} 0 \\ 0 \\ 0 \end{mymatrix}
    }$.
  \item $V_3 = \set{
      \begin{mymatrix}{r} 0 \\ 0 \\ 0 \end{mymatrix}
    }$.
  \item $V_3 = \set{
      \begin{mymatrix}{r} 1 \\ 1 \\ 1 \end{mymatrix},
      \begin{mymatrix}{r} 1 \\ 0 \\ 0 \end{mymatrix},
      \begin{mymatrix}{r} 0 \\ 1 \\ 0 \end{mymatrix},
      \begin{mymatrix}{r} 0 \\ 0 \\ 1 \end{mymatrix}
    }$.
  \end{enumerate}
  \begin{sol}
    \begin{enumerate}
    \item $V_1$ is a subspace.
    \item $V_2$ is not a subspace: not closed under addition. For example,
      \begin{equation*} 
        \begin{mymatrix}{r} 1 \\ 1 \\ 1 \end{mymatrix} \in V_2,\quad
        \begin{mymatrix}{r} 1 \\ 1 \\ 0 \end{mymatrix} \in V_2,\quad
        \mbox{but}\quad
        \begin{mymatrix}{r} 1 \\ 1 \\ 1 \end{mymatrix}
        + \begin{mymatrix}{r} 1 \\ 1 \\ 0 \end{mymatrix}
        = \begin{mymatrix}{r} 0 \\ 0 \\ 1 \end{mymatrix} \not\in V_2.
      \end{equation*}
    \item $V_3$ is a subspace.
    \item $V_4$ is not a subspace. For example, it does not contain
      the zero vector.
    \end{enumerate}
  \end{sol}
\end{ex}

\begin{ex}
  Suppose $V, W$ are subspaces of $\R^n$. Let $V\cap W$ be the set
  of all vectors that are in both $V$ and $W$. Show that $V\cap W$
  is also a subspace.
  \begin{sol}
    Because $\vect{0}\in V$ and $\vect{0}\in W$, we have
    $\vect{0}\in V\cap W$. To show that $V\cap W$ is closed under
    addition, assume $\vect{u},\vect{v}\in V\cap W$. Then
    $\vect{u},\vect{v}\in V$, and since $V$ is a subspace we have
    $\vect{u}+\vect{v}\in V$. Similarly $\vect{u},\vect{v}\in W$, and
    since $W$ is a subspace we have $\vect{u}+\vect{v}\in W$. It
    follows that $\vect{u}+\vect{v}$ is in both $V$ and $W$, and
    therefore in $V\cap W$. To show that $V\cap W$ is closed under
    scalar multiplication, assume $\vect{u}\in V\cap W$ and
    $k\in\R$. Then $\vect{u}\in V$, and therefore $k\vect{u}\in
    V$. Similarly $k\vect{u}\in W$, and therefore also
    $k\vect{u}\in V\cap W$.
  \end{sol}
\end{ex}

\begin{ex}
  Let $V$ be a subset of $\R^n$. Show that $V$ is a subspace if and
  only if it is non-empty and the following condition holds: for all
  $\vect{u},\vect{v}\in V$ and all scalars $a,b\in\R$,
  \begin{equation*}
    a\vect{u} + b\vect{v} \in V.
  \end{equation*}
\end{ex}

\begin{ex}
  Let $\vect{u}_1,\ldots,\vect{u}_k$ be vectors in $\R^n$, and let
  $S=\sspan\set{\vect{u}_1,\ldots,\vect{u}_k}$. Show that $S$ is the
  {\em smallest} subspace of $\R^n$ that contains
  $\vect{u}_1,\ldots,\vect{u}_k$.  Specifically, this means you have
  to show: if $V$ is any other subspace of $\R^n$ such that
  $\vect{u}_1,\ldots,\vect{u}_k\in V$, then $S\subseteq V$.
\end{ex}

