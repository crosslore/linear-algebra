\section*{Exercises}

\begin{ex}
  Let $\Poly_3$ be the vector space of polynomials of degree at most
  3. Determine which of the following are bases for this vector space.

  \begin{enumerate}
  \item $\set{x^3+1,~x^2+x,~2x^3+x^2,~2x^3-x^2-3x+1}$.
  \item $\set{x+1,~x^3+x^2+2x,~x^2+x,~x^3+x^2+x}$.
  \end{enumerate}

  \begin{sol}
    \begin{enumerate}
    \item
      Yes. Suppose
      \begin{equation*}
        c_1(x^3+1) + c_2(x^2+x) + c_3(2x^3+x^2) +
        c_4(2x^3-x^2-3x+1) = 0.
      \end{equation*}
      Then collect equal powers of $x$:
      \begin{equation*}
        (c_1+2c_3+2c_4)x^3 + (c_2+c_3-c_4)x^2 +
        (c_2-3c_4)x + (c_1+c_4) = 0.
      \end{equation*}
      Does the system
      \begin{equation*}
        \begin{array}{c}
          c_1 + 2c_3 + 2c_4 = 0 \\
          c_2 + c_3 - c_4 = 0 \\
          c_2 - 3c_4 = 0 \\
          c_1 + c_4 = 0
        \end{array}
      \end{equation*}
      have a non-trivial solution? The only solution is
      \begin{equation*}
        c_1 = 0,\quad
        c_2 = 0,\quad
        c_3 = 0,\quad
        c_4 = 0,
      \end{equation*}
      and therefore, the polynomials are linearly independent. Since
      there are $4$ linearly independent polynomials in a
      $4$-dimensional space, they form a basis.
    \item Yes.
    \end{enumerate}
  \end{sol}
\end{ex}

\begin{ex}
  Determine whether the following is a basis for $\Poly_2$, the
  vector space of polynomials of degree at most $2$.
  \begin{equation*}
    \set{x^2+x+1,~2x^2+2x+1,~x+1}.
  \end{equation*}
\end{ex}

\begin{ex}
  Find a basis for the following subspace of $\Poly_2$:
  \begin{equation*}
    W=\sspan\set{1+x+x^2,~1+2x,~1+5x-3x^2}.
  \end{equation*}
\end{ex}

\begin{ex}
  Find a basis for the following subspace of $\Poly_3$:
  \begin{equation*}
    W = \sspan\set{
      1+x-x^2+x^3,~1+2x+3x^3,~-1+3x+5x^2+7x^3,~1+6x+4x^2+11x^3}.
  \end{equation*}
\end{ex}

\begin{ex}
  Extend the following linearly independent set of polynomials to a
  basis of $\Poly_3$:
  \begin{equation*}
    \set{x^3+x^2-x-1,~3x^3+2x^2+2x-1}.
  \end{equation*}
\end{ex}

\begin{ex}
  Let $V$ be a $5$-dimensional vector space.
  If you have $5$ linearly independent vectors in $V$, can you
  conclude that the vectors span $V$?
  \begin{sol}
    Yes, because the set of $5$ linearly independent vectors can be
    extended to a basis $B$ of $V$. But since $V$ is $5$-dimensional,
    $B$ has only $5$ elements, which must be the original $5$ vectors.
  \end{sol}
\end{ex}

\begin{ex}
  Let $V$ be a $5$-dimensional vector space.  If you have $6$ vectors
  in $V$, is it possible that they are linearly independent? Explain.
  \begin{sol}
    No. Since $V$ has a spanning set of size $5$, the $6$ vectors
    cannot be linearly independent by the Exchange Lemma.
  \end{sol}
\end{ex}

\begin{ex}
  Find a basis for the vector space of symmetric $3\times 3$-matrices%
  \index{matrix!symmetric}%
  \index{symmetric matrix}, i.e., matrices satisfying $A=A^T$.
  What is the dimension of this space?
\end{ex}

\begin{ex}
  Let $W$ be the subspace of $\Poly_3$ (over the field $\R$)
  consisting of all polynomials $p(x)$ that satisfy $p(3)=0$.
  Find a basis for $W$. What is the dimension of $W$?
\end{ex}

\begin{ex}
  Find a basis for
  $U=\set{A\in\Mat_{2,2} ~\left\vert~ A\begin{mysmallmatrix}{rr} 1 &
        0 \\ 1 & -1 \end{mysmallmatrix} = \begin{mysmallmatrix}{rr} 1
        & 1 \\ 0 & -1 \end{mysmallmatrix} A \right.}$.  What is the
  dimension of $U$?
\end{ex}

\begin{ex}
  \begin{enumerate}
  \item Let $k$ be a positive integer, let $W_k\subseteq\Seq_K$ be the
    subspace consisting of all sequences that are periodic%
    \index{periodic sequence}%
    \index{sequence!periodic} with period $k$ (see
    Exercise~\ref{ex:periodic}). Find a basis for $W_k$. What is its
    dimension?
  \item More difficult: find a basis for the infinite-dimensional
    vector space consisting of all periodic sequences of all periods.
  \end{enumerate}
  \begin{sol}
    \begin{enumerate}
    \item In case $k=3$, the following sequences form a basis for
      $W_3$:
      \begin{equation*}
        \begin{array}{l}
          (1,0,0,1,0,0,1,0,0,1,\ldots), \\
          (0,1,0,0,1,0,0,1,0,0,\ldots), \\
          (0,0,1,0,0,1,0,0,1,0,\ldots). \\
        \end{array}
      \end{equation*}
      Therefore, $W_3$ is a 3-dimensional space. For general $k$,
      the situation is analogous and the dimension of $W_k$ is $k$.
    \item Let $U$ be the set of all periodic sequences of all
      periods. We can find an (infinite) spanning set for $U$ by
      taking all the basis vectors for all of the spaces $W_k$:
      \begin{equation*}
        \begin{array}{ll}
          (1,1,1,1,1,1,1,1,1,1,\ldots) & \mbox{(period $1$)},\\[1ex]
          (1,0,1,0,1,0,1,0,1,0,\ldots) & \mbox{(period $2$)}, \\
          (0,1,0,1,0,1,0,1,0,1,\ldots) & \mbox{(period $2$)},\\[1ex]
          (1,0,0,1,0,0,1,0,0,1,\ldots) & \mbox{(period $3$)}, \\
          (0,1,0,0,1,0,0,1,0,0,\ldots) & \mbox{(period $3$)}, \\
          (0,0,1,0,0,1,0,0,1,0,\ldots) & \mbox{(period $3$)},\\[1ex]
          (1,0,0,0,1,0,0,0,1,0,\ldots) & \mbox{(period $4$)}, \\
          (0,1,0,0,0,1,0,0,0,1,\ldots) & \mbox{(period $4$)}, \\
          (0,0,1,0,0,0,1,0,0,0,\ldots) & \mbox{(period $4$)}, \\
          (0,0,0,1,0,0,0,1,0,0,\ldots) & \mbox{(period $4$)}, \\
        \end{array}
      \end{equation*}
      and so on. However, these sequences are not linearly
      independent. For example, we can obtain the sequence
      $(1,0,1,0,1,0,1,0,\ldots)$ of period $2$ as a linear combination
      of two sequences of period $4$, namely
      $(1,0,0,0,1,0,0,0,\ldots)$ and $(0,0,1,0,0,0,1,0,\ldots)$. By
      Proposition~\ref{prop:basis-from-spanning}, we know that it is
      possible to shrink the above spanning set to a basis by removing
      certain sequences. How exactly to do this is an interesting
      question. One way to construct a basis is to keep exactly those
      sequences of period $k$ that start with $\ell$ zeros, where
      $\gcd(\ell,k)=1$. Proving that this really works is an
      interesting project.
    \end{enumerate}
  \end{sol}
\end{ex}

\begin{ex}
  Let $K=\Q$, the field of rational numbers. Consider vectors of the
  form $a+b\sqrt{2}$ where $a,b$ are rational numbers. Show that this
  collection of vectors is a vector space over $\Q$ and give a basis
  for this vector space. What is its dimension?
  \begin{sol}
    When we add two number of the form $a+b\sqrt{2}$, we get another
    number of the same form. When we multiply a number of the form
    $a+b\sqrt{2}$ by a (rational) scalar, we get another number of the
    same form. Also, $0=0+0\sqrt{2}$ is of the required form. The 8
    axioms of a vector space are satisfied because all of them are
    laws of the arithmetic of real numbers. A basis is
    $\set{1,\sqrt{2}}$. By definition, the span of these gives the
    collection of vectors. To prove that they are linearly
    independent, assume $a+b\sqrt{2}=0$, where $a,b$ are rational
    numbers. If $b\neq 0$, then $\sqrt{2}=-\frac{a}{b}$, which cannot
    happen because $\sqrt{2}$ is irrational. If $a\neq 0$, then
    $\frac{1}{\sqrt{2}}=-\frac{b}{a}$, which again cannot happen since
    $\frac{1}{\sqrt{2}}$ is irrational. Hence both $a,b=0$. Therefore,
    $1$ and $\sqrt{2}$ are linearly independent over the rational
    numbers, and form a basis. The dimension of the space is $2$.
  \end{sol}
\end{ex}
