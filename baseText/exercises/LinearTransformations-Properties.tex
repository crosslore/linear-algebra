\Opensolutionfile{solutions}[ex]
\section*{Exercises}

\begin{enumialphparenastyle}

\begin{ex}
  Find the matrix for the linear transformation that reflects every
  vector in $\R^2$ about the $x$-axis and then reflects about the
  $y$-axis.
\end{ex}

\begin{ex}
  Find the matrix for the linear transformation that rotates every
  vector in $\R^{2}$ by an angle of $2\pi/3$ and then reflects
  about the $x$-axis.
  \begin{sol}
    \begin{equation*}
      \begin{mymatrix}{rr}
        1 & 0 \\
        0 & -1
      \end{mymatrix} \begin{mymatrix}{cc}
        \cos \paren{\frac{2\pi }{3}}  & -\sin \paren{\frac{2\pi }{3}}
        \\
        \sin \paren{\frac{2\pi }{3}}  & \cos \paren{\frac{2\pi }{3}}
      \end{mymatrix} = \begin{mymatrix}{cc}
        -\frac{1}{2} & -\frac{1}{2}\sqrt{3} \\
        -\frac{1}{2}\sqrt{3} & \frac{1}{2}
      \end{mymatrix}
    \end{equation*}
  \end{sol}
\end{ex}

\begin{ex}
  Find the matrix for the linear transformation that rotates every
  vector in $\R^{2}$ by an angle of $\pi /6$ and then reflects
  about the $x$-axis followed by a reflection about the $y$-axis.
  \begin{sol}
    \begin{equation*}
      \begin{mymatrix}{rr}
        -1 & 0 \\
        0 & 1
      \end{mymatrix} \begin{mymatrix}{cc}
        \cos \paren{\frac{\pi }{6}}  & -\sin \paren{\frac{\pi }{6}}  \\
        \sin \paren{\frac{\pi }{6}}  & \cos \paren{\frac{\pi }{6}}
      \end{mymatrix} = \begin{mymatrix}{cc}
        -\frac{1}{2}\sqrt{3} & \frac{1}{2} \\
        \frac{1}{2} & \frac{1}{2}\sqrt{3}
      \end{mymatrix}
    \end{equation*}
  \end{sol}
\end{ex}

\begin{ex}
  Find the matrix for the linear transformation that reflects every
  vector in $\R^{2}$ about the $x$-axis and then rotates by an angle
  of $\pi/4$.
  \begin{sol}
    \begin{equation*}
      \begin{mymatrix}{cc}
        \cos \paren{\frac{\pi }{4}}  & -\sin \paren{\frac{\pi }{4}}  \\
        \sin \paren{\frac{\pi }{4}}  & \cos \paren{\frac{\pi }{4}}
      \end{mymatrix} \begin{mymatrix}{rr}
        1 & 0 \\
        0 & -1
      \end{mymatrix} = \begin{mymatrix}{cc}
        \frac{1}{2}\sqrt{2} & \frac{1}{2}\sqrt{2} \\
        \frac{1}{2}\sqrt{2} & -\frac{1}{2}\sqrt{2}
      \end{mymatrix}
    \end{equation*}
  \end{sol}
\end{ex}

\begin{ex}
  Find the matrix of the linear transformation that rotates every
  vector in $\R^{3}$ counterclockwise about the $z$-axis when viewed
  from the positive $z$-axis by an angle of 30 degrees and then
  reflects about the $xy$-plane.
  \begin{sol}
    \begin{equation*}
      \begin{mymatrix}{rrr}
        1 & 0 & 0 \\
        0 & 1 & 0 \\
        0 & 0 & -1
      \end{mymatrix} \begin{mymatrix}{ccc}
        \cos \paren{\frac{\pi }{6}}  & -\sin \paren{\frac{\pi }{6}}  & 0
        \\
        \sin \paren{\frac{\pi }{6}}  & \cos \paren{\frac{\pi }{6}}  & 0
        \\
        0 & 0 & 1
      \end{mymatrix} = \begin{mymatrix}{ccc}
        \frac{1}{2}\sqrt{3} & -\frac{1}{2} & 0 \\
        \frac{1}{2} & \frac{1}{2}\sqrt{3} & 0 \\
        0 & 0 & -1
      \end{mymatrix}
    \end{equation*}
  \end{sol}
\end{ex}

\begin{ex}
  Prove the three properties in
  Proposition~\ref{prop:properties-linear-transformation}, using only
  the definition of a linear transformation (i.e., the fact that it
  preserves addition and scalar multiplication).
  \begin{sol}
    (a) $T(\vect{0}) = T(0\vect{0}) = 0T(\vect{0}) = \vect{0}$.
    (b) $T(-\vect{v}) = T((-1)\vect{v}) = (-1)T(\vect{v}) = -T(\vect{v})$.
    (c) $T(a_1\vect{v}_1 + \ldots + a_k \vect{v}_k)
    = T(a_1\vect{v}_1) + \ldots + T(a_k \vect{v}_k)
    = a_1T(\vect{v}_1) + \ldots + a_kT(\vect{v}_k)$.
  \end{sol}
\end{ex}

\begin{ex}
  Let $T$ be the linear transformation with matrix
  $A = \begin{mymatrix}{rr}
    3 & 1 \\
    -1 & 2
  \end{mymatrix}$ and $S$ the linear transformation with matrix
  $B = \begin{mymatrix}{rr}
    0 & -2 \\
    4 & 2
  \end{mymatrix}$. Find the matrix of $S \circ T$. Compute
  $(S \circ T) (\vect{v})$ for
  $\vect{v} = \begin{mymatrix}{r}
    2 \\
    -1
  \end{mymatrix}$.
  \begin{sol}
    The matrix of $S \circ T$ is given by
    \begin{equation*}
      BA = \begin{mymatrix}{rr}
        0 & -2 \\
        4 & 2
      \end{mymatrix} \begin{mymatrix}{rr}
        3 & 1 \\
        -1 & 2
      \end{mymatrix} = \begin{mymatrix}{rr}
        2 & -4 \\
        10 & 8
      \end{mymatrix}.
    \end{equation*}
    Now,
    \begin{equation*}
      (S \circ T) (\vect{v}) = BA\vect{v}
      =
      \begin{mymatrix}{rr}
        2 & -4 \\
        10 & 8
      \end{mymatrix}
      \begin{mymatrix}{r}
        2 \\
        -1
      \end{mymatrix}
      =
      \begin{mymatrix}{r}
        8 \\
        12
      \end{mymatrix}.
    \end{equation*}
  \end{sol}
\end{ex}


\begin{ex}
  Let $T$ be a linear transformation and suppose
  $T \paren{\begin{mymatrix}{r}
      1 \\
      -4
    \end{mymatrix}} = \begin{mymatrix}{r}
    2 \\
    -3
  \end{mymatrix}$. Suppose $S$ is the linear transformation with
  matrix $B = \begin{mymatrix}{rr}
    1 & 2 \\
    -1 & 3
  \end{mymatrix}$. Find $(S \circ T) (\vect{v})$ for
  $\vect{v} = \begin{mymatrix}{r}
    1 \\
    -4
  \end{mymatrix}$.
  \begin{sol}
    We have
    \begin{equation*}
      (S \circ T) (\vect{v})
      = S(T(\vect{v}))
      = B(T(\vect{v}))
      = \begin{mymatrix}{rr}
        1 & 2 \\
        -1 & 3
      \end{mymatrix}
      \begin{mymatrix}{r}
        2 \\
        -3
      \end{mymatrix}
      = \begin{mymatrix}{r}
        -4 \\
        -11
      \end{mymatrix}.
    \end{equation*}
  \end{sol}
\end{ex}

\begin{ex}
  What is the inverse of a reflection? Rotation? Shearing? Scaling?
  \begin{sol}
    The inverse of a reflection is a reflection, namely, itself. (For
    example, reflecting twice about the $x$-axis returns each vector
    to its original position). The inverse of a rotation is a rotation
    by the same angle in the opposite direction. The inverse of a
    shearing is a shearing in the opposite direction. The inverse of a
    scaling by factor $a$ is a scaling by factor $1/a$.
  \end{sol}
\end{ex}

\begin{ex}
  Let $T$ be a linear transformation with matrix
  $A = \begin{mymatrix}{rr}
    2 & 1 \\
    5 & 2
  \end{mymatrix}$. Find the matrix of $T^{-1}$.
  \begin{sol}
    The matrix of $T^{-1}$ is $A^{-1}$.
    \begin{equation*}
      \begin{mymatrix}{rr}
        2 & 1 \\
        5 & 2
      \end{mymatrix} ^{-1} =
      \begin{mymatrix}{rr}
        -2 & 1 \\
        5 & -2
      \end{mymatrix}
    \end{equation*}
  \end{sol}
\end{ex}

\begin{ex}
  Let $T$ be the linear transformation given by
  $T\paren{\begin{mymatrix}{c} x\\y \end{mymatrix}} =
  \begin{mymatrix}{c} 4x-3y\\2x-2y \end{mymatrix}$.  Find the matrix
  of $T^{-1}$.
  \begin{sol}
    The matrix of $T$ is
    $A = \begin{mymatrix}{rr}
    4 & -3 \\
    2 & -2
  \end{mymatrix}$. The matrix of $T^{-1}$ is $A^{-1} =
  \begin{mymatrix}{cc}
    1 & -3/2 \\
    1 & -2
  \end{mymatrix}$.
  \end{sol}
\end{ex}

\begin{ex} Let $T$ be a linear transformation and suppose $T \paren{\begin{mymatrix}{r}
      1 \\
      2
    \end{mymatrix}} = \begin{mymatrix}{r}
    9 \\
    8
  \end{mymatrix}$ and $T \paren{\begin{mymatrix}{r}
      0 \\
      -1
    \end{mymatrix}} = \begin{mymatrix}{r}
    -4 \\
    -3
  \end{mymatrix}$.
  Find the matrix of $T$ and the matrix of $T^{-1}$.
  % \begin{sol}
  % \end{sol}
\end{ex}

\end{enumialphparenastyle}
