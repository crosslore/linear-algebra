\Opensolutionfile{solutions}[ex]
\section*{Exercises}

\begin{enumialphparenastyle}

\begin{ex} Let $V$ and $W$ be subspaces of $\R^{n}$ and $\R^{m}$
respectively and let $T:V\rightarrow W$ be a linear transformation. Suppose
that $\set{T\vect{v}_{1},\cdots ,T\vect{v}_{r}} $ is linearly
independent. Show that it must be the case that $\set{\vect{v}_{1},\cdots ,
\vect{v}_{r}} $ is also linearly independent.
\begin{sol}
If $\sum_i^r a_i \vect{v}_r =0$, then using linearity properties of $T$ we get 
\[ 0 = T(0) =  T(\sum_i^r a_i \vect{v}_r) = 
\sum_i^r a_i T(\vect{v}_r).\]
Since we assume that  $\set{T\vect{v}_{1},\cdots ,T\vect{v}_{r}} $ is linearly
independent, we must have all $a_i=0$, and therefore we conclude that 
 $\set{\vect{v}_{1},\cdots ,
\vect{v}_{r}} $ is also linearly independent.
\end{sol}
\end{ex}


\begin{ex} Let 
\begin{equation*}
V=\mbox{span}\set{\begin{mymatrix}{c}
1 \\ 
1 \\ 
2 \\ 
0
\end{mymatrix} ,\begin{mymatrix}{c}
0 \\ 
1 \\ 
1 \\ 
1
\end{mymatrix} ,\begin{mymatrix}{c}
1 \\ 
1 \\ 
0 \\ 
1
\end{mymatrix} }
\end{equation*}
Let $T\vect{x}=A\vect{x}$ where $A$ is the matrix 
\begin{equation*}
\begin{mymatrix}{cccc}
1 & 1 & 1 & 1 \\ 
0 & 1 & 1 & 0 \\ 
0 & 1 & 2 & 1 \\ 
1 & 1 & 1 & 2
\end{mymatrix}
\end{equation*}
Give a basis for $\func{im}\tup{T}$.
%\begin{sol}
%\end{sol}
\end{ex}


\begin{ex} Let 
\begin{equation*}
V=\mbox{span}\set{\begin{mymatrix}{c}
1 \\ 
0 \\ 
0 \\ 
1
\end{mymatrix} ,\begin{mymatrix}{c}
1 \\ 
1 \\ 
1 \\ 
1
\end{mymatrix} ,\begin{mymatrix}{c}
1 \\ 
4 \\ 
4 \\ 
1
\end{mymatrix} }
\end{equation*}
Let $T\vect{x}=A\vect{x}$ where $A$ is the matrix 
\begin{equation*}
\begin{mymatrix}{cccc}
1 & 1 & 1 & 1 \\ 
0 & 1 & 1 & 0 \\ 
0 & 1 & 2 & 1 \\ 
1 & 1 & 1 & 2
\end{mymatrix}
\end{equation*}
Find a basis for $\func{im}\tup{T}$. In this case, the original
vectors do not form an independent set.

\begin{sol}
Since the third vector is a linear combinations of the first two, then
the image of the third vector will also be a linear combinations of
the image of the first two.  However the image of the first two
vectors are linearly independent (check!), and hence form a basis of
the image.

Thus a basis for $\func{im}\tup{T} $ is: 

\begin{equation*}
V=\mbox{span}\set{\begin{mymatrix}{c}
2 \\ 
0 \\ 
1 \\ 
3
\end{mymatrix} ,\begin{mymatrix}{c}
4 \\ 
2 \\ 
4 \\ 
5
\end{mymatrix}  }
\end{equation*}

\end{sol}
\end{ex}


\begin{ex} If $\set{\vect{v}_{1},\cdots ,\vect{v}_{r}} $ is linearly
independent and $T$ is a one to one linear transformation, show that $
\set{T\vect{v}_{1},\cdots ,T\vect{v}_{r}} $ is also linearly
independent. Give an example which shows that if $T$ is only linear, it can
happen that, although $\set{\vect{v}_{1},\cdots ,\vect{v}_{r}} $ is
linearly independent, $\set{T\vect{v}_{1},\cdots ,T\vect{v}_{r}} $
is not. In fact, show that it can happen that each of the $T\vect{v}_{j}$
equals 0.
%\begin{sol}
%\end{sol}
\end{ex}


\begin{ex} Let $V$ and $W$ be subspaces of $\R^{n}$ and $\R^{m}$
respectively and let $T:V\rightarrow W$ be a linear transformation. Show
that if $T$ is onto $W$ and if $\set{\vect{v}_{1},\cdots ,\vect{v}
_{r}} $ is a basis for $V$, then $\sspan\set{T\vect{v}
_{1},\cdots ,T\vect{v}_{r}} =W$.
%\begin{sol}
%\end{sol}
\end{ex}


\begin{ex} Define $T:\R^{4}\rightarrow \R^{3}$ as follows. 
\begin{equation*}
T\vect{x}=\begin{mymatrix}{rrrr}
3 & 2 & 1 & 8 \\ 
2 & 2 & -2 & 6 \\ 
1 & 1 & -1 & 3
\end{mymatrix} \vect{x}
\end{equation*}
Find a basis for $\func{im}\tup{T}$. Also find a basis for $\ker
\tup{T}$.
%\begin{sol}
%\end{sol}
\end{ex}


\begin{ex} Define $T:\R^{3}\rightarrow \R^{3}$ as follows. 
\begin{equation*}
T\vect{x}=\begin{mymatrix}{ccc}
1 & 2 & 0 \\ 
1 & 1 & 1 \\ 
0 & 1 & 1
\end{mymatrix} \vect{x}
\end{equation*}
where on the right, it is just matrix multiplication of the vector $\vect{x}$
which is meant. Explain why $T$ is an isomorphism of $\R^{3}$ to $
\R^{3}$.
%\begin{sol}
%\end{sol}
\end{ex}


\begin{ex} Suppose $T:\R^{3}\rightarrow \R^{3}$ is a linear
transformation given by 
\begin{equation*}
T\vect{x}=A\vect{x}
\end{equation*}
where $A$ is a $3\times 3$-matrix. Show that $T$ is an isomorphism if and
only if $A$ is invertible.
%\begin{sol}
%\end{sol}
\end{ex}


\begin{ex} Suppose $T:\R^{n}\rightarrow \R^{m}$ is a linear
transformation given by 
\begin{equation*}
T\vect{x}=A\vect{x}
\end{equation*}
where $A$ is an $m\times n$-matrix. Show that $T$ is never an isomorphism if 
$m\neq n$. In particular, show that if $m>n$, $T$ cannot be onto and if $
m<n, $ then $T$ cannot be one to one.
%\begin{sol}
%\end{sol}
\end{ex}


\begin{ex} Define $T:\R^{2}\rightarrow \R^{3}$ as follows. 
\begin{equation*}
T\vect{x}=\begin{mymatrix}{cc}
1 & 0 \\ 
1 & 1 \\ 
0 & 1
\end{mymatrix} \vect{x}
\end{equation*}
where on the right, it is just matrix multiplication of the vector $\vect{x}$
which is meant. Show that $T$ is one to one. Next let $W=\func{im}\tup{
T}$. Show that $T$ is an isomorphism of $\R^{2}$ and $\func{im
}\tup{T}$.
%\begin{sol}
%\end{sol}
\end{ex}


\begin{ex} In the above problem, find a $2\times 3$-matrix $A$ such that the
restriction of $A$ to $\func{im}\tup{T} $ gives the same result as $
T^{-1}$ on $\func{im}\tup{T}$. \textbf{Hint:\ }You might let $A$ be
such that 
\begin{equation*}
A\begin{mymatrix}{c}
1 \\ 
1 \\ 
0
\end{mymatrix} =\begin{mymatrix}{c}
1 \\ 
0
\end{mymatrix} ,\ A\begin{mymatrix}{c}
0 \\ 
1 \\ 
1
\end{mymatrix} =\begin{mymatrix}{c}
0 \\ 
1
\end{mymatrix}
\end{equation*}
now find another vector $\vect{v}\in \R^{3}$ such that 
\begin{equation*}
\set{\begin{mymatrix}{c}
1 \\ 
1 \\ 
0
\end{mymatrix} ,\begin{mymatrix}{c}
0 \\ 
1 \\ 
1
\end{mymatrix} ,\vect{v}}
\end{equation*}
is a basis. You could pick 
\begin{equation*}
\vect{v}=\begin{mymatrix}{c}
0 \\ 
0 \\ 
1
\end{mymatrix}
\end{equation*}
for example. Explain why this one works or one of your choice works. Then
you could define $A\vect{v}$ to equal some vector in $\R^{2}$.
Explain why there will be more than one such matrix $A$ which will deliver
the inverse isomorphism $T^{-1}$ on $\func{im}\tup{T}$.
%\begin{sol}
%\end{sol}
\end{ex}


\begin{ex} Now let $V$ equal $\sspan\set{\begin{mymatrix}{c}
1 \\ 
0 \\ 
1
\end{mymatrix} ,\begin{mymatrix}{c}
0 \\ 
1 \\ 
1
\end{mymatrix} } $ and let $T:V\rightarrow W$ be a linear transformation
where 
\begin{equation*}
W=\sspan\set{\begin{mymatrix}{c}
1 \\ 
0 \\ 
1 \\ 
0
\end{mymatrix} ,\begin{mymatrix}{c}
0 \\ 
1 \\ 
1 \\ 
1
\end{mymatrix} }
\end{equation*}
$\ $\ and 
\begin{equation*}
T\begin{mymatrix}{c}
1 \\ 
0 \\ 
1
\end{mymatrix} =\begin{mymatrix}{c}
1 \\ 
0 \\ 
1 \\ 
0
\end{mymatrix} ,T\begin{mymatrix}{c}
0 \\ 
1 \\ 
1
\end{mymatrix} =\begin{mymatrix}{c}
0 \\ 
1 \\ 
1 \\ 
1
\end{mymatrix} 
\end{equation*}
Explain why $T$ is an isomorphism. Determine a matrix $A$ which, when
multiplied on the left gives the same result as $T$ on $V$ and a matrix $B$
which delivers $T^{-1}$ on $W$. \textbf{Hint:\ }You need to have 
\begin{equation*}
A\begin{mymatrix}{cc}
1 & 0 \\ 
0 & 1 \\ 
1 & 1
\end{mymatrix} =\begin{mymatrix}{cc}
1 & 0 \\ 
0 & 1 \\ 
1 & 1 \\ 
0 & 1
\end{mymatrix}
\end{equation*}
Now enlarge $\begin{mymatrix}{c}
1 \\ 
0 \\ 
1
\end{mymatrix} ,\begin{mymatrix}{c}
0 \\ 
1 \\ 
1
\end{mymatrix} $ to obtain a basis for $\R^{3}$. You could add in $\begin{mymatrix}{c}
0 \\ 
0 \\ 
1
\end{mymatrix} $ for example, and then pick another vector in $\R^{4}$ and
let $A\begin{mymatrix}{c}
0 \\ 
0 \\ 
1
\end{mymatrix} $ equal this other vector. Then you would have 
\begin{equation*}
A\begin{mymatrix}{ccc}
1 & 0 & 0 \\ 
0 & 1 & 0 \\ 
1 & 1 & 1
\end{mymatrix} =\begin{mymatrix}{ccc}
1 & 0 & 0 \\ 
0 & 1 & 0 \\ 
1 & 1 & 0 \\ 
0 & 1 & 1
\end{mymatrix}
\end{equation*}
This would involve picking for the new vector in $\R^{4}$ the vector 
$\begin{mymatrix}{cccc}
0 & 0 & 0 & 1
\end{mymatrix} ^{T}$. Then you could find $A$. You can do something similar to find
a matrix for $T^{-1}$ denoted as $B$.
%\begin{sol}
%\end{sol}
\end{ex}

\end{enumialphparenastyle}