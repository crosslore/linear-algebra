\section*{Exercises}

\begin{ex}
  Let $B = \set{\begin{mymatrix}{r} 2 \\ -1 \end{mymatrix},
    \begin{mymatrix}{r} 3 \\ 2 \end{mymatrix}}$ be a basis of $\R^2$
  and let $\vect{x} = \begin{mymatrix}{r} 5 \\ -7 \end{mymatrix}$ be a
  vector in $\R^2$. Find $\coord{\vect{x}}_B$.
\end{ex}

\begin{ex}
  Let $B = \set{\begin{mymatrix}{r} 1 \\ -1 \\ 2 \end{mymatrix},
    \begin{mymatrix}{r} 2 \\ 1 \\ 2 \end{mymatrix},
    \begin{mymatrix}{r} -1 \\ 0 \\ 2 \end{mymatrix}}$ be a basis of
  $\R^3$ and let
  $\vect{x} = \begin{mymatrix}{r} 5 \\ -1 \\ 4 \end{mymatrix}$ be a
  vector in $\R^2$. Find $\coord{\vect{x}}_B$.
  \begin{sol}
    $\coord{\vect{x}}_B
    = \begin{mymatrix}{r} 2 \\ 1 \\ -1 \end{mymatrix}$.
  \end{sol}
\end{ex}

\begin{ex}
  Suppose $T:\R^3\to\R^3$ is a linear transformation such that
  \begin{equation*}
    T\begin{mymatrix}{r} 1 \\ 0 \\ 0 \end{mymatrix}
    = \begin{mymatrix}{r} 3 \\ 3 \\ 3 \end{mymatrix},
    \quad
    T\begin{mymatrix}{r} 0 \\ 1 \\ 0 \end{mymatrix}
    = \begin{mymatrix}{r} 1 \\ 2 \\ 3 \end{mymatrix},
    \quad
    T\begin{mymatrix}{r} 0 \\ 0 \\ 1 \end{mymatrix}
    = \begin{mymatrix}{r} 1 \\ 3 \\ -1 \end{mymatrix}.
  \end{equation*}
  Let $E=\set{\vect{e}_1,\vect{e}_2,\vect{e}_3}$ be the standard basis
  of $\R^3$, and let
  \begin{equation*}
    B = \set{\vect{v}_1,\vect{v}_2,\vect{v}_3} = \set{
      \begin{mymatrix}{r} 1 \\ 0 \\ 0 \end{mymatrix},~
      \begin{mymatrix}{r} 1 \\ 1 \\ 0 \end{mymatrix},~
      \begin{mymatrix}{r} 2 \\ 1 \\ 1 \end{mymatrix}
    }
  \end{equation*}
  be another basis.
  \begin{enumerate}
  \item  Find the matrix of $T$ with respect to $E$, i.e.,
    find $\coord{T}_{E,E}$.
  \item Find $\coord{T}_{B,B}$.
  \end{enumerate}
  \begin{sol}
    \begin{enumerate}
      \item The $i\th$ column of $\coord{T}_{E,E}$ is
      $\coord{T(\vect{e}_i)}_E = T(\vect{e}_i)$, so
      \begin{equation*}
        \coord{T}_{E,E} =
        \begin{mymatrix}{ccc}
          3 & 1 & 1 \\
          3 & 2 & 3 \\
          3 & 3 & -1
        \end{mymatrix}.
      \end{equation*}
      \item The $i\th$ column of $\coord{T}_{B,B}$ is
        $\coord{T(\vect{v}_i)}_B$. This requires a calculation:
        \begin{equation*}
          T(\vect{v}_1) = T\begin{mymatrix}{c} 1 \\ 0 \\ 0 \end{mymatrix}
          = \begin{mymatrix}{c} 3 \\ 3 \\ 3 \end{mymatrix} =
          - 3\begin{mymatrix}{c} 1 \\ 0 \\ 0 \end{mymatrix}
          + 0\begin{mymatrix}{c} 1 \\ 1 \\ 0 \end{mymatrix}
          + 3\begin{mymatrix}{c} 2 \\ 1 \\ 1 \end{mymatrix},
          \quad\mbox{therefore}
          \coord{T(\vect{v}_1)}_B =
          \begin{mymatrix}{r} -3 \\ 0 \\ 3 \end{mymatrix}.
        \end{equation*}
        \begin{equation*}
          T(\vect{v}_2) = T\begin{mymatrix}{c} 1 \\ 1 \\ 0 \end{mymatrix}
          = \begin{mymatrix}{c} 4 \\ 5 \\ 6 \end{mymatrix} =
          - 7\begin{mymatrix}{c} 1 \\ 0 \\ 0 \end{mymatrix}
          - 1\begin{mymatrix}{c} 1 \\ 1 \\ 0 \end{mymatrix}
          + 6\begin{mymatrix}{c} 2 \\ 1 \\ 1 \end{mymatrix},
          \quad\mbox{therefore}
          \coord{T(\vect{v}_2)}_B =
          \begin{mymatrix}{r} -7 \\ -1 \\ 6 \end{mymatrix}.
        \end{equation*}
        \begin{equation*}
          T(\vect{v}_3) = T\begin{mymatrix}{c} 2 \\ 1 \\ 1 \end{mymatrix}
          = \begin{mymatrix}{c} 8 \\ 11 \\ 8 \end{mymatrix} =
          - 11\begin{mymatrix}{c} 1 \\ 0 \\ 0 \end{mymatrix}
          + 3\begin{mymatrix}{c} 1 \\ 1 \\ 0 \end{mymatrix}
          + 8\begin{mymatrix}{c} 2 \\ 1 \\ 1 \end{mymatrix},
          \quad\mbox{therefore}
          \coord{T(\vect{v}_3)}_B =
          \begin{mymatrix}{c} -11 \\ 3 \\ 8 \end{mymatrix}.
        \end{equation*}
        We therefore have
        \begin{equation*}
          \coord{T}_{B,B} =
          \begin{mymatrix}{rrc}
            -3 & -7 & -11 \\
            0  & -1 &   3 \\
            3  &  6 &   8 \\
          \end{mymatrix}.
        \end{equation*}
      \end{enumerate}
\end{sol}
\end{ex}

\begin{ex}
  Let $T: \R^2 \to \R^2$ be a linear transformation defined by
  \begin{equation*}
    T \paren{\begin{mymatrix}{c} a \\ b \end{mymatrix}}
    = \begin{mymatrix}{c} a+b \\ a-b \end{mymatrix}.
  \end{equation*}
  Consider the two bases
  \begin{equation*}
    B_1 = \set{\begin{mymatrix}{r} 1 \\ 0 \end{mymatrix},
      \begin{mymatrix}{r} -1 \\ 1 \end{mymatrix}
    }
  \end{equation*}
  and
  \begin{equation*}
    B_2 = \set{\begin{mymatrix}{r} 1 \\ 1 \end{mymatrix},
      \begin{mymatrix}{r} 1 \\ -1 \end{mymatrix}
    }.
  \end{equation*}
  Find the matrix $M_{B_2,B_1}$ of $T$ with respect to the bases $B_1$
  and $B_2$.
  \begin{sol}
    $M_{B_2 B_1} = \begin{mymatrix}{rr}
      1 & -1 \\
      0 & 1
    \end{mymatrix}$.
  \end{sol}
\end{ex}

\begin{ex}
  Let $M=\begin{mymatrix}{rr} 1 & 2 \\ -2 & 1 \end{mymatrix}$, and
  consider the linear transformation $T:\Mat_{2,2}\to\Mat_{2,2}$ given
  by $T(A) = MAM$. Find the matrix of $T$ with respect to the basis
  \begin{equation*}
    B=\set{
      \begin{mymatrix}{cc} 1 & 0 \\ 0 & 0 \end{mymatrix},
      \begin{mymatrix}{cc} 0 & 1 \\ 0 & 0 \end{mymatrix},
      \begin{mymatrix}{cc} 0 & 0 \\ 1 & 0 \end{mymatrix},
      \begin{mymatrix}{cc} 0 & 0 \\ 0 & 1 \end{mymatrix}.
    }
  \end{equation*}
\end{ex}

\begin{ex}
  Consider the linear transformation $T:\Poly_3\to\Poly_3$ given by
  $T(p(x)) = p(x+1)$. Find $\coord{T}_{B,B}$, where $B=\set{1,x,x^2,x^3}$.
\end{ex}

\begin{ex}
  Let $\vect{v}=\begin{mymatrix}{r} 1 \\ -2 \\ 3 \end{mymatrix}$
  and consider the linear function
  $T(\vect{w}) = \proj_{\vect{v}}(\vect{w})$.  Find the matrix of $T$
  with respect to the standard basis of $\R^3$.
  \begin{sol}
    Recall that
    $\proj_{\vect{v}}(\vect{w})
    =\frac{\vect{v}\dotprod\vect{w}}{\norm{\vect{v}}^2}\vect{v}$. The
    desired matrix has $i\th$ column equal to
    $\proj_{\vect{v}}(\vect{e}_i)$. Therefore, the desired matrix is
    \begin{equation*}
      \frac{1}{14}\begin{mymatrix}{rrr}
        1 & -2 & 3 \\
        -2 & 4 & -6 \\
        3 & -6 & 9
      \end{mymatrix}.
    \end{equation*}
  \end{sol}
\end{ex}

\begin{ex}
  Let $\vect{v}=\begin{mymatrix}{r} 1 \\ -2 \\ 3 \end{mymatrix}$
  and consider the linear function
  $T(\vect{w}) = \proj_{\vect{v}}(\vect{w})$.  Find the matrix of $T$
  with respect to the basis
  \begin{equation*}
    B = \set{\vect{v}_1,\vect{v}_2,\vect{v}_3} = \set{
      \begin{mymatrix}{r} 1 \\ -2 \\ 3 \end{mymatrix},~
      \begin{mymatrix}{r} 2 \\ 1 \\ 0 \end{mymatrix},~
      \begin{mymatrix}{r} 3 \\ 0 \\ 1 \end{mymatrix}
    }.
  \end{equation*}
  \begin{sol}
    We have $T(\vect{v}_1) = \vect{v}_1$, $T(\vect{v}_2) = \vect{0}$,
    and $T(\vect{v}_3) = \vect{0}$. Therefore
    \begin{equation*}
      \coord{T}_{B,B} =
      \begin{mymatrix}{rrr}
        1 & 0 & 0 \\
        0 & 0 & 0 \\
        0 & 0 & 0
      \end{mymatrix}.
    \end{equation*}
  \end{sol}
\end{ex}

\begin{ex}
  Suppose that $V$ and $W$ are finite-dimensional vector spaces with
  bases $B$ and $C$, respectively. Let $T:V\to W$ be a linear
  transformation such that
  \begin{equation*}
    T(\vect{v}_i)=\vect{w}_i
  \end{equation*}
  for $i=1,\ldots,n$. Let $M$ be the matrix whose columns are
  $\coord{\vect{v}_1}_B,\ldots,\coord{\vect{v}_n}_B$, and let $N$ be
  the matrix whose columns are
  $\coord{\vect{w}_1}_C,\ldots,\coord{\vect{w}_n}_C$.  Suppose that
  $M$ is invertible. Show that $\coord{T}_{C,B} = NM^{-1}$.
  \begin{sol}
    Since $M$ is invertible, its columns
    $\coord{\vect{v}_1}_B,\ldots,\coord{\vect{v}_n}_B$ are linearly
    independent and span $\R^n$; it follows that
    $\vect{v}_1,\ldots,\vect{v}_n$ is a basis of $V$. To show that
    $\coord{T}_{C,B} = NM^{-1}$, it is sufficient to check that
    $NM^{-1}\coord{\vect{v}_i}_B = \coord{\vect{w}_i}_C$, for all
    $i=1,\ldots,n$. But by assumption, $\coord{\vect{v}_i}_B$ is the
    $i\th$ column of $M$, so that $\coord{\vect{v}_i}_B =
    M\vect{e}_i$, where $\vect{e}_i$ is the $i\th$ basis
    vector. Therefore $M^{-1}\coord{\vect{v}_i}_B  = \vect{e}_i$.
    On the other hand, $N\vect{e}_i$ is the $i\th$ column of $N$,
    i.e., $\coord{\vect{w}_i}_C$. We therefore have
    \begin{equation*}
      NM^{-1}\coord{\vect{v}_i}_B = N\vect{e}_i = \coord{\vect{w}_i}_C,
    \end{equation*}
    as desired.
  \end{sol}
\end{ex}

