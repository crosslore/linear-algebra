\section*{Exercises}

\begin{ex} Consider the following diagram of four circuits.
  \begin{center}
    \scalebox{0.8}{
      \begin{circuitikz}[american, scale=0.8] \draw
        (0,0) to [battery1, v^= $5\volt$~~] (0,4)
        (0,0) to [R = $ 2 \ohm $] (4,0)
        to [R = $ 5 \ohm $] (4,4)
        (0,4) to [R =$ 3 \ohm $] (4,4)
        (6,4) to [battery1, v_= \raisebox{1ex}{$20\volt$}] (4,4)
        (6,4) to [R = $1 \ohm$] (8,4)
        to [R = $1 \ohm$] (8,0)
        (4,0) to [R = $6 \ohm$] (8,0)
        to [R = $3 \ohm$] (8,-4)
        to [R = $2 \ohm$] (4,-4)
        to [R = $1 \ohm$] (4,0)
        (4,-4)to [R = $4 \ohm$] (0,-4)
        (0,0 )to [battery1, v_= $10\volt$~~] (0,-4)
        (2,2) node[scale=4]{$\circlearrowleft$}
        (2,2) node{$I_2$}
        (6,2) node[scale=4]{$\circlearrowleft$}
        (6,2) node{$I_3$}
        (6,-2) node[scale=4]{$\circlearrowleft$}
        (6,-2) node{$I_4$}
        (2,-2) node[scale=4]{$\circlearrowleft$}
        (2,-2) node{$I_1$}
        ;
      \end{circuitikz}
    }
  \end{center}
  The current in amperes in the four circuits is denoted by $I_{1}$,
  $I_{2}$, $I_{3}$, and $I_{4}$. It is understood that a positive
  current means a current flowing in the counterclockwise direction. If
  $I_{k}$ ends up being negative, then it just means the current flows
  in the clockwise direction.  In the above diagram, the top left
  circuit should give the equation
  \begin{equation*}
    2I_{2}-2I_{1}+5I_{2}-5I_{3}+3I_{2}=5.
  \end{equation*}
  Write equations for each of the other three circuits and then give a solution
  to the resulting system of equations.
  \begin{sol}
    The other three equations are
    \begin{eqnarray*}
      4I_{1}+I_{1}-I_{4}+2I_{1}-2I_{2} &=& -10 \\
      6I_{3}-6I_{4}+I_{3}+I_{3}+5I_{3}-5I_{2} &=&-20 \\
      2I_{4}+3I_{4}+6I_{4}-6I_{3}+I_{4}-I_{1} &=&0.
    \end{eqnarray*}
    Then the system is
    \begin{equation*}
      \begin{array}{c}
        2I_{2}-2I_{1}+5I_{2}-5I_{3}+3I_{2}=5 \\
        4I_{1}+I_{1}-I_{4}+2I_{1}-2I_{2}=-10 \\
        6I_{3}-6I_{4}+I_{3}+I_{3}+5I_{3}-5I_{2}=-20 \\
        2I_{4}+3I_{4}+6I_{4}-6I_{3}+I_{4}-I_{1}=0.
      \end{array}
    \end{equation*}
    The solution is:
    \begin{eqnarray*}
      I_{1}&=& -\frac{750}{373} \\
      I_{2}&=& -\frac{1421}{1119} \\
      I_{3}&=& -\frac{3061}{1119} \\
      I_{4}&=& -\frac{1718}{1119}.
    \end{eqnarray*}
  \end{sol}
\end{ex}

\begin{ex} Find $I_{1}$, $I_{2}$, and $I_{3}$, the counterclockwise currents in
  amperes in the three circuits of the following diagram.

  \begin{center}
    \scalebox{0.8}{
      \begin{circuitikz}[american, scale=0.8] \draw
        (0,0) to [battery1, v^= $10\volt$~~] (0,4)
        (0,0) to [R = $ 2 \ohm $] (4,0)
        to [R = $ 5 \ohm $] (4,4)
        (0,4) to [R =$ 3 \ohm $] (4,4)
        (6,4) to [battery1, v_= \raisebox{1ex}{$12\volt$}] (4,4)
        (6,4) to [R = $7 \ohm$] (8,4)
        to [R = $3 \ohm$] (8,0)
        (4,0) to [R = $1 \ohm$] (8,0)
        to [R = $4 \ohm$] (8,-4)
        to [R = $4 \ohm$] (4,-4)
        to [R = $2 \ohm$] (4,0)
        (2,2) node[scale=4]{$\circlearrowleft$}
        (2,2) node{$I_1$}
        (6,2) node[scale=4]{$\circlearrowleft$}
        (6,2) node{$I_2$}
        (6,-2) node[scale=4]{$\circlearrowleft$}
        (6,-2) node{$I_3$}
        ;
      \end{circuitikz}
    }
  \end{center}

  \begin{sol}
    You
    have
    \begin{eqnarray*}
      2I_{1}+5I_{1}+3I_{1}-5I_{2} &=& 10 \\
      I_{2}- I_{3} +3I_{2}+7I_{2}+5I_{2}-5I_{1}  &=&-12 \\
      2I_{3}+4I_{3}+4I_{3}+I_{3}-I_{2} &=& 0
    \end{eqnarray*}
    Simplifying this yields
    \begin{eqnarray*}
      10I_{1}-5I_{2} &=& 10 \\
      -5I_{1} + 16I_{2}- I_{3} &=&-12 \\
      -I_{2} + 11I_{3} &=&0
    \end{eqnarray*}
    The solution is given by
    \begin{equation*}
      I_{1}=\frac{218}{295},I_{2}=-\frac{154}{295},I_{3}=-\frac{14}{295}
    \end{equation*}

  \end{sol}
\end{ex}

