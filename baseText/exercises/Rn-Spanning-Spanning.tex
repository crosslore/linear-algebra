\Opensolutionfile{solutions}[ex]
\section*{Exercises}

\begin{enumialphparenastyle}

\begin{ex} Here are some vectors. 
\begin{equation*}
\begin{mymatrix}{r}
1 \\ 
1 \\ 
-2
\end{mymatrix} ,\begin{mymatrix}{r}
1 \\ 
2 \\ 
-2
\end{mymatrix} ,\begin{mymatrix}{r}
2 \\ 
7 \\ 
-4
\end{mymatrix} ,\begin{mymatrix}{r}
5 \\ 
7 \\ 
-10
\end{mymatrix} ,\begin{mymatrix}{r}
12 \\ 
17 \\ 
-24
\end{mymatrix}
\end{equation*}
Describe the span of these vectors as the span of as few vectors as possible.
%\begin{sol}
%\end{sol}
\end{ex}

\begin{ex} Here are some vectors. 
\begin{equation*}
\begin{mymatrix}{r}
1 \\ 
2 \\ 
-2
\end{mymatrix} ,\begin{mymatrix}{r}
12 \\ 
29 \\ 
-24
\end{mymatrix} ,\begin{mymatrix}{r}
1 \\ 
3 \\ 
-2
\end{mymatrix} ,\begin{mymatrix}{r}
2 \\ 
9 \\ 
-4
\end{mymatrix} ,\begin{mymatrix}{r}
5 \\ 
12 \\ 
-10
\end{mymatrix} ,
\end{equation*}
Describe the span of these vectors as the span of as few vectors as possible.
%\begin{sol}
%\end{sol}
\end{ex}

\begin{ex} Here are some vectors.
\begin{equation*}
\begin{mymatrix}{r}
1 \\ 
2 \\ 
-2
\end{mymatrix} ,\begin{mymatrix}{r}
1 \\ 
3 \\ 
-2
\end{mymatrix} ,\begin{mymatrix}{r}
1 \\ 
-2 \\ 
-2
\end{mymatrix} ,\begin{mymatrix}{r}
-1 \\ 
0 \\ 
2
\end{mymatrix} ,\begin{mymatrix}{r}
1 \\ 
3 \\ 
-1
\end{mymatrix}
\end{equation*}
Describe the span of these vectors as the span of as few vectors as possible.
%\begin{sol}
%\end{sol}
\end{ex}

\begin{ex} Here are some vectors. 
\begin{equation*}
\begin{mymatrix}{r}
1 \\ 
1 \\ 
-2
\end{mymatrix} ,\begin{mymatrix}{r}
1 \\ 
2 \\ 
-2
\end{mymatrix} ,\begin{mymatrix}{r}
1 \\ 
-3 \\ 
-2
\end{mymatrix} ,\begin{mymatrix}{r}
-1 \\ 
1 \\ 
2
\end{mymatrix}
\end{equation*}
Now here is another vector:\ 
\begin{equation*}
\begin{mymatrix}{r}
1 \\ 
2 \\ 
-1
\end{mymatrix} 
\end{equation*}
Is this vector in the span of the first four vectors? If it is, exhibit a
linear combination of the first four vectors which equals this vector, using
as few vectors as possible in the linear combination.
%\begin{sol}
%\end{sol}
\end{ex}

\begin{ex} Here are some vectors. 
\begin{equation*}
\begin{mymatrix}{r}
1 \\ 
1 \\ 
-2
\end{mymatrix} ,\begin{mymatrix}{r}
1 \\ 
2 \\ 
-2
\end{mymatrix} ,\begin{mymatrix}{r}
1 \\ 
-3 \\ 
-2
\end{mymatrix} ,\begin{mymatrix}{r}
-1 \\ 
1 \\ 
2
\end{mymatrix}
\end{equation*}
Now here is another vector:\ 
\begin{equation*}
\begin{mymatrix}{r}
2 \\ 
-3 \\ 
-4
\end{mymatrix} 
\end{equation*}
Is this vector in the span of the first four vectors? If it is, exhibit a
linear combination of the first four vectors which equals this vector, using
as few vectors as possible in the linear combination. 
%\begin{sol}
%\end{sol}
\end{ex}

\begin{ex} Here are some vectors. 
\begin{equation*}
\begin{mymatrix}{r}
1 \\ 
1 \\ 
-2
\end{mymatrix} ,\begin{mymatrix}{r}
1 \\ 
2 \\ 
-2
\end{mymatrix} ,\begin{mymatrix}{r}
1 \\ 
-3 \\ 
-2
\end{mymatrix} ,\begin{mymatrix}{r}
1 \\ 
2 \\ 
-1
\end{mymatrix}
\end{equation*}
Now here is another vector:\ 
\begin{equation*}
\begin{mymatrix}{r}
1 \\ 
9 \\ 
1
\end{mymatrix} 
\end{equation*}
Is this vector in the span of the first four vectors? If it is, exhibit a
linear combination of the first four vectors which equals this vector, using
as few vectors as possible in the linear combination. 
%\begin{sol}
%\end{sol}
\end{ex}

\begin{ex} Here are some vectors. 
\begin{equation*}
\begin{mymatrix}{r}
1 \\ 
-1 \\ 
-2
\end{mymatrix} ,\begin{mymatrix}{r}
1 \\ 
0 \\ 
-2
\end{mymatrix} ,\begin{mymatrix}{r}
1 \\ 
-5 \\ 
-2
\end{mymatrix} ,\begin{mymatrix}{r}
-1 \\ 
5 \\ 
2
\end{mymatrix}
\end{equation*}
Now here is another vector:\ 
\begin{equation*}
\begin{mymatrix}{r}
1 \\ 
1 \\ 
-1
\end{mymatrix} 
\end{equation*}
Is this vector in the span of the first four vectors? If it is, exhibit a
linear combination of the first four vectors which equals this vector, using
as few vectors as possible in the linear combination. 
%\begin{sol}
%\end{sol}
\end{ex}

\begin{ex} Here are some vectors. 
\begin{equation*}
\begin{mymatrix}{r}
1 \\ 
-1 \\ 
-2
\end{mymatrix} ,\begin{mymatrix}{r}
1 \\ 
0 \\ 
-2
\end{mymatrix} ,\begin{mymatrix}{r}
1 \\ 
-5 \\ 
-2
\end{mymatrix} ,\begin{mymatrix}{r}
-1 \\ 
5 \\ 
2
\end{mymatrix}
\end{equation*}
Now here is another vector:\ 
\begin{equation*}
\begin{mymatrix}{r}
1 \\ 
1 \\ 
-1
\end{mymatrix} 
\end{equation*}
Is this vector in the span of the first four vectors? If it is, exhibit a
linear combination of the first four vectors which equals this vector, using
as few vectors as possible in the linear combination. 
%\begin{sol}
%\end{sol}
\end{ex}

\begin{ex} Here are some vectors. 
\begin{equation*}
\begin{mymatrix}{r}
1 \\ 
0 \\ 
-2
\end{mymatrix} ,\begin{mymatrix}{r}
1 \\ 
1 \\ 
-2
\end{mymatrix} ,\begin{mymatrix}{r}
2 \\ 
-2 \\ 
-3
\end{mymatrix} ,\begin{mymatrix}{r}
-1 \\ 
4 \\ 
2
\end{mymatrix}
\end{equation*}
Now here is another vector:\ 
\begin{equation*}
\begin{mymatrix}{r}
-1 \\ 
-4 \\ 
2
\end{mymatrix} 
\end{equation*}
Is this vector in the span of the first four vectors? If it is, exhibit a
linear combination of the first four vectors which equals this vector, using
as few vectors as possible in the linear combination.
%\begin{sol}
%\end{sol}
\end{ex}


\begin{ex} Suppose $\set{\vect{x}_{1},\cdots ,\vect{x}_{k}} $ is a
set of vectors from $\R^{n}$. Show that $\vect{0}$ is in $\func{
span}\set{\vect{x}_{1},\cdots ,\vect{x}_{k}}$.
\begin{sol}
$\sum_{i=1}^{k}0\vect{x}_{k}=\vect{0}$
\end{sol}
\end{ex}

\end{enumialphparenastyle}
