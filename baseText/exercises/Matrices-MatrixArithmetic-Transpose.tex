\section*{Exercises}

\begin{enumialphparenastyle}

\begin{ex} Let $X=\begin{mymatrix}{rrr}
    -1 & -1 & 1
  \end{mymatrix}$ and $Y=\begin{mymatrix}{rrr}
    0 & 1 & 2
  \end{mymatrix}$. Find $X^{T}Y$ and $XY^{T}$ if
  possible. 
  \begin{sol}
    $X^{T}Y = \begin{mymatrix}{rrr}
      0 & -1 & -2 \\
      0 & -1 & -2 \\
      0 & 1 & 2
    \end{mymatrix} , XY^{T} = 1$
  \end{sol}
\end{ex}

\begin{ex} Consider the matrices $
A =\begin{mymatrix}{rr}
1 & 2 \\
3 & 2 \\
1 & -1
\end{mymatrix}, B=\begin{mymatrix}{rrr}
2 & -5 & 2 \\
-3 & 2 & 1
\end{mymatrix}, 
C =\begin{mymatrix}{rr}
1 & 2 \\
5 & 0
\end{mymatrix}, \\ D=\begin{mymatrix}{rr}
-1 & 1 \\
4 & -3
\end{mymatrix}, E=\begin{mymatrix}{r}
1 \\
3
\end{mymatrix}$

Find the following if possible. If it is not possible explain why. 
\begin{enumerate}  
\item $-3A{^T}$
\item $3B - A^{T}$
\item $E^{T}B$
\item $EE^{T}$
\item $B^{T}B$
\item $CA^{T}$
\item $D^{T}BE$
\end{enumerate}

\begin{sol}
\begin{enumerate}
\item $\begin{mymatrix}{rrr}
-3 & -9 & -3 \\
-6 & -6 & 3
\end{mymatrix}$
\item $\begin{mymatrix}{rrr}
5 & -18 & 5 \\
-11 & 4 & 4
\end{mymatrix}$
\item $\begin{mymatrix}{rrr}
-7 & 1 & 5
\end{mymatrix}$
\item $\begin{mymatrix}{rr}
1 & 3 \\
3 & 9
\end{mymatrix}$
\item $\begin{mymatrix}{rrr}
13 & -16 & 1\\
-16 & 29 & -8 \\
1 & -8 & 5
\end{mymatrix}$
\item $\begin{mymatrix}{rrr}
5 & 7 & -1 \\
5 & 15 & 5 
\end{mymatrix}$
\item Not possible.
\end{enumerate}
\end{sol}
\end{ex}

\begin{ex} Let $A$ be an $n\times n$-matrix. Show $A$ equals the sum of a
symmetric and a skew symmetric matrix.  
\textbf{Hint: }Show that $
\frac{1}{2}\tup{A^{T}+A} $ is symmetric and then consider using this
as one of the matrices. 
\begin{sol}
Show that $\frac{1}{2}\tup{A^{T}+A} $ is symmetric and then consider using this
as one of the matrices. $A=\frac{A+A^{T}}{2}+\frac{A-A^{T}}{2}$.
\end{sol}
\end{ex}

\begin{ex} Show that the main diagonal of every skew symmetric matrix consists of only zeros. Recall that the main diagonal consists of every entry of the matrix which is of the form
$a_{ii}$. 
\begin{sol}
If $A$ is symmetric then $A=-A^{T}$. It follows that $a_{ii}=-a_{ii}$ and so each $a_{ii}=0$.
\end{sol}
\end{ex}

\begin{ex} Prove \ref{matrix-transpose2}. That is, show that for an $m \times n$-matrix $A$, an $n \times p$-matrix $B$, and scalars $r, s$, the following holds:
\[
\tup{rA + sB} ^T = rA^{T} + sB^{T}
\]
%\begin{sol}
%\end{sol}
\end{ex}

\begin{ex} \label{exer-Rn3}  Let $A$ be a real $m\times n$-matrix and
let $\vect{u}\in \R^{n}$ and $\vect{v}\in \R^{m}$. Show 
$A\vect{u}\dotprod \vect{v}=\vect{u}\dotprod A^{T}\vect{v}$. 
\textbf{Hint:} Use the definition of matrix
multiplication to do this.
\begin{sol}
$A\vect{x}\dotprod \vect{y}=\sum_{k}\tup{A\vect{x}
} _{k}y_{k}=\sum_{k}\sum_{i}A_{ki}x_{i}y_{k}=\sum_{i}
\sum_{k}A_{ik}^{T}x_{i}y_{k}= \vect{x} \dotprod A^{T}\vect{y} $
\end{sol}
\end{ex}

\begin{ex} Use the result of Problem \ref{exer-Rn3} to verify directly
that $\tup{AB} ^{T}=B^{T}A^{T}$ without making any reference to
subscripts.
\begin{sol}
\begin{eqnarray*}
 AB\vect{x} \dotprod \vect{y} &=& B\vect{x} \dotprod A^{T}\vect{y} \\ 
&=& \vect{x}\dotprod B^{T}A^{T}\vect{y} \\
&=& \vect{x}\dotprod \tup{AB} ^{T} \vect{y}
\end{eqnarray*}
Since this is true for all $\vect{x}$, it follows that, in particular, it
holds for
\[
\vect{x}=B^{T}A^{T}\vect{y}-\tup{AB} ^{T}\vect{y}
\]
and so from the axioms of the dot product,
\[
\tup{B^{T}A^{T}\vect{y}-\tup{AB} ^{T}\vect{y}} \dotprod \tup{B^{T}A^{T}
\vect{y}-\tup{AB} ^{T}\vect{y}} =0
\]
and so $B^{T}A^{T}\vect{y}-\tup{AB} ^{T}\vect{y}=\vect{0}$. However,
this is true for all $\vect{y}$ and so $B^{T}A^{T}-\tup{
AB} ^{T}=0$.
\end{sol}
\end{ex}
 
\end{enumialphparenastyle}
