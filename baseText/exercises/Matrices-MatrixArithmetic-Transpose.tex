\section*{Exercises}

\begin{ex}
  Let $X=\begin{mymatrix}{rrr}
    -1 & -1 & 1
  \end{mymatrix}$ and $Y=\begin{mymatrix}{rrr}
    0 & 1 & 2
  \end{mymatrix}$. Find $X^{T}Y$ and $XY^{T}$ if
  possible.
  \begin{sol}
    $X^{T}Y = \begin{mymatrix}{rrr}
      0 & -1 & -2 \\
      0 & -1 & -2 \\
      0 & 1 & 2
    \end{mymatrix}$, $XY^{T} = 1$.
  \end{sol}
\end{ex}

\begin{ex}
  Consider the matrices
  \begin{equation*}
    A =\begin{mymatrix}{rr}
      1 & 2 \\
      3 & 2 \\
      1 & -1
    \end{mymatrix},
    \quad
    B=\begin{mymatrix}{rrr}
      2 & -5 & 2 \\
      -3 & 2 & 1
    \end{mymatrix},
    \quad
    C =\begin{mymatrix}{rr}
      1 & 2 \\
      5 & 0
    \end{mymatrix},
    \quad
    D=\begin{mymatrix}{rr}
      -1 & 1 \\
      4 & -3
    \end{mymatrix},
    \quad
    E=\begin{mymatrix}{r}
      1 \\
      3
    \end{mymatrix}.
  \end{equation*}
  Find the following if possible. If it is not possible explain why.
  \begin{enumerate}
  \item $-3A{^T}$.
  \item $3B - A^{T}$.
  \item $E^{T}B$.
  \item $EE^{T}$.
  \item $B^{T}B$.
  \item $CA^{T}$.
  \item $D^{T}BE$.
  \end{enumerate}

  \begin{sol}
    \begin{enumerate}
    \item $\begin{mymatrix}{rrr}
        -3 & -9 & -3 \\
        -6 & -6 & 3
      \end{mymatrix}$.
    \item $\begin{mymatrix}{rrr}
        5 & -18 & 5 \\
        -11 & 4 & 4
      \end{mymatrix}$.
    \item $\begin{mymatrix}{rrr}
        -7 & 1 & 5
      \end{mymatrix}$.
    \item $\begin{mymatrix}{rr}
        1 & 3 \\
        3 & 9
      \end{mymatrix}$.
    \item $\begin{mymatrix}{rrr}
        13 & -16 & 1\\
        -16 & 29 & -8 \\
        1 & -8 & 5
      \end{mymatrix}$.
    \item $\begin{mymatrix}{rrr}
        5 & 7 & -1 \\
        5 & 15 & 5
      \end{mymatrix}$.
    \item Not possible because $B$ is a $2\times 3$-matrix and $E$ is
      a $2\times 1$-matrix, cannot multiply $BE$.
    \end{enumerate}
  \end{sol}
\end{ex}

\begin{ex}
  Which of the following matrices are symmetric, antisymmetric, both,
  or neither?
  \begin{equation*}
    A = \begin{mymatrix}{rr}
      0 & 1 \\
      -1 & 0 \\
    \end{mymatrix},
    \quad
    B = \begin{mymatrix}{rr}
      2 & 1 \\
      1 & 3 \\
    \end{mymatrix},
    \quad
    C = \begin{mymatrix}{rr}
      1 & 2 \\
      -2 & 0 \\
    \end{mymatrix},
    \quad
    D = \begin{mymatrix}{rr}
      0 & 0 \\
      0 & 0 \\
    \end{mymatrix}.
  \end{equation*}
  \begin{sol}
    $A$ is antisymmetric, $B$ is symmetric, $C$ is neither, and $D$ is both.
  \end{sol}
\end{ex}

\begin{ex}
  Suppose $A$ is a matrix that is both symmetric and
  antisymmetric. Show that $A=0$.
  \begin{sol}
    We have $A=A^T=-A$. Therefore, each entry $a_{ij}$ of $A$ is equal
    to its own negation. This implies that $a_{ij}=0$, and therefore
    $A$ is the zero matrix.
  \end{sol}
\end{ex}

\begin{ex}
  Let $A$ be an $n\times n$-matrix. Show $A$ equals the sum of a
  symmetric and an antisymmetric matrix.  \textbf{Hint:} Show that
  $\frac{1}{2}(A^{T}+A)$ is symmetric and then consider using
  this as one of the matrices.
  \begin{sol}
    $A=\frac{1}{2}(A+A^{T})+\frac{1}{2}(A-A^{T})$.
  \end{sol}
\end{ex}

\begin{ex}
  Show that the main diagonal of every antisymmetric matrix consists
  of only zeros. Recall that the main diagonal consists of every entry
  of the matrix which is of the form $a_{ii}$.
  \begin{sol}
    If $A$ is antisymmetric then $A=-A^{T}$. It follows that
    $a_{ii}=-a_{ii}$ and so each $a_{ii}=0$.
  \end{sol}
\end{ex}

\begin{ex}
  Show that for $m \times n$-matrices $A,B$ and scalars $r, s$, the
  following holds:
  \begin{equation*}
    (rA + sB) ^T = rA^{T} + sB^{T}.
  \end{equation*}
  \vspace{-4ex}
  \begin{sol}
    This follows from properties {\ref{matrix-transpose-2}} and
    {\ref{matrix-transpose-3}} of
    Lemma~\ref{lem:transpose-properties}.
  \end{sol}
\end{ex}

\begin{ex} \label{exer-Rn3}  Let $A$ be a real $m\times n$-matrix and
  let $\vect{u}\in \R^{n}$ and $\vect{v}\in \R^{m}$. Show
  $(A\vect{u})\dotprod \vect{v}=\vect{u}\dotprod (A^{T}\vect{v})$.
  \begin{sol}
    $(A\vect{u})\dotprod \vect{v} = (A\vect{u})^T\vect{v} =
    (\vect{u}^TA^T)\vect{v}  = \vect{u}^T(A^T\vect{v}) =
    \vect{u}\dotprod (A^T\vect{v})$.
  \end{sol}
\end{ex}

\begin{ex}
  Show that if $A$ is an invertible $n\times n$-matrix, then so is
  $A^{T} $ and $(A^{T})^{-1}=(A^{-1})^{T}$.
  \begin{sol}
    We need to show that $(A^{-1})^{T}$ is the inverse of
    $A^{T}$. From properties of the transpose,
    \begin{eqnarray*}
      A^{T}(A^{-1})^{T} &=& (A^{-1}A)^{T}=I^{T}=I, \\
      (A^{-1})^{T}A^{T} &=& (AA^{-1})^{T}=I^{T}=I.
    \end{eqnarray*}
    Hence $A^{T}$ is invertible and $(A^{T})^{-1}=(A^{-1})^{T}$.
  \end{sol}
\end{ex}

\begin{ex}
  Suppose $A$ is invertible and symmetric. Show that $A^{-1}$ is
  symmetric.
  \begin{sol}
    We have $(A^{-1})^T = (A^T)^{-1} = A^{-1}$, and therefore $A^{-1}$
    is equal to its own transpose, hence symmetric.
  \end{sol}
\end{ex}

