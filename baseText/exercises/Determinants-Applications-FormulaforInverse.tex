\section*{Exercises}

\begin{ex}
  Find the cofactor matrix and the adjugate of each of the following
  matrices.
  \begin{equation*}
    A =
    \begin{mymatrix}{rr}
      1 & 0 \\
      3 & 2 \\
    \end{mymatrix},
    \quad
    B =
    \begin{mymatrix}{rrr}
      1  & 0  & 1  \\
      -1 & 1  & -2 \\
      2  & -1 & 3  \\
    \end{mymatrix},
    \quad
    C =
    \begin{mymatrix}{rrr}
      1 &  1 & 3 \\
      2 &  3 & 1 \\
      1 & -1 & 1 \\
    \end{mymatrix}.
  \end{equation*}
\end{ex}

\begin{ex}
  For each of the following matrices, determine whether it is
  invertible by checking whether the determinant is non-zero. If the
  determinant is non-zero, use the adjugate formula to find the
  inverse.
  \begin{equation*}
    A =
    \begin{mymatrix}{rrr}
      1 & 2 & 3 \\
      0 & 2 & 1 \\
      3 & 1 & 0 \\
    \end{mymatrix},
    \quad
    B =
    \begin{mymatrix}{rrr}
      1 & 2 & 0 \\
      0 & 2 & 1 \\
      3 & 1 & 1 \\
    \end{mymatrix},
    \quad
    C =
    \begin{mymatrix}{rrr}
      1 & 3 & 3 \\
      2 & 4 & 1 \\
      0 & 1 & 1 \\
    \end{mymatrix},
    \quad
    D =
    \begin{mymatrix}{rrr}
      1 & 2 & 3 \\
      0 & 2 & 1 \\
      2 & 6 & 7 \\
    \end{mymatrix},
    \quad
    E =
    \begin{mymatrix}{rrr}
      1 & 0 & 3 \\
      1 & 0 & 1 \\
      3 & 1 & 0 \\
    \end{mymatrix}.
  \end{equation*}

  \begin{sol}
    \begin{enumerate}
    \item $\det(A) = -13$ and so $A$ is invertible. This inverse is
      \begin{eqnarray*}
        \frac{1}{-13}\begin{mymatrix}{rrr}
          \begin{absmatrix}{cc}
            2 & 1 \\
            1 & 0 \\
          \end{absmatrix} & -\begin{absmatrix}{cc}
            0 & 1 \\
            3 & 0 \\
          \end{absmatrix} & \begin{absmatrix}{cc}
            0 & 2 \\
            3 & 1 \\
          \end{absmatrix} \\\\[-1ex]
          -\begin{absmatrix}{cc}
            2 & 3 \\
            1 & 0 \\
          \end{absmatrix} & \begin{absmatrix}{cc}
            1 & 3 \\
            3 & 0 \\
          \end{absmatrix} & -\begin{absmatrix}{cc}
            1 & 2 \\
            3 & 1 \\
          \end{absmatrix} \\\\[-1ex]
          \begin{absmatrix}{cc}
            2 & 3 \\
            2 & 1 \\
          \end{absmatrix} & -\begin{absmatrix}{cc}
            1 & 3 \\
            0 & 1 \\
          \end{absmatrix} & \begin{absmatrix}{cc}
            1 & 2 \\
            0 & 2 \\
          \end{absmatrix}\end{mymatrix}^{T} &=&\frac{1}{-13}\begin{mymatrix}{rrr}
          -1 & 3  & -6 \\
          3  & -9 & 5  \\
          -4 & -1 & 2  \\
        \end{mymatrix}^{T}
        \\
              &=& \def\arraystretch{1.3}
                  \begin{mymatrix}{rrr}
                    \frac{1}{13} & -\frac{3}{13} & \frac{4}{13} \\
                    -\frac{3}{13} & \frac{9}{13} & \frac{1}{13} \\
                    \frac{6}{13} & -\frac{5}{13} & -\frac{2}{13} \\
                  \end{mymatrix}.
      \end{eqnarray*}
    \item $\det(B) = 7$, so $B$ is invertible. The inverse is
      $\displaystyle
      \def\arraystretch{1.3}
      \frac{1}{7}
      \begin{mymatrix}{rrr}
        1 & 3 & -6 \\
        -2 & 1 & 5 \\
        2 & -1 & 2 \\
      \end{mymatrix}^{T} = \begin{mymatrix}{rrr}
        \frac{1}{7} & -\frac{2}{7} & \frac{2}{7} \\
        \frac{3}{7} & \frac{1}{7} & -\frac{1}{7} \\
        -\frac{6}{7} & \frac{5}{7} & \frac{2}{7} \\
      \end{mymatrix}.$
    \item $\det(C) = 3$, so $C$ is invertible. The inverse is
      $\def\arraystretch{1.3}
      \begin{mymatrix}{rrr}
        1 & 0 & -3 \\
        -\frac{2}{3} & \frac{1}{3} & \frac{5}{3} \\
        \frac{2}{3} & -\frac{1}{3} & -\frac{2}{3} \\
      \end{mymatrix}$.
    \item $\det(D) = 0$, so $D$ is not invertible.
    \item $\det(E) = 2$, and so $E$ is invertible. The inverse is
      $\def\arraystretch{1.3}
      \begin{mymatrix}{rrr}
        -\frac{1}{2} & \frac{3}{2} & 0 \\
        \frac{3}{2} & -\frac{9}{2} & 1 \\
        \frac{1}{2} & -\frac{1}{2} & 0 \\
      \end{mymatrix}$.
    \end{enumerate}
  \end{sol}
\end{ex}

\begin{ex}
  Determine whether each of the following matrices is invertible. If
  so, use the adjugate formula to find the inverse. If the inverse
  does not exist, explain why.
  \begin{equation*}
    A = \begin{mymatrix}{rr}
      1 & 1 \\
      1 & 2 \\
    \end{mymatrix},
    \quad
    B = \begin{mymatrix}{rrr}
      1 & 2 & 3 \\
      0 & 2 & 1 \\
      4 & 1 & 1 \\
    \end{mymatrix},
    \quad
    C = \begin{mymatrix}{rrr}
      1 & 2 & 1 \\
      2 & 3 & 0 \\
      0 & 1 & 2 \\
    \end{mymatrix}.
  \end{equation*}
  \begin{sol}
    \begin{enumerate}
    \item $\det(A) = 1$, so $A$ is invertible. The inverse is
      $A^{-1} = \begin{mymatrix}{rr}
        2 & -1 \\
        -1 & 1 \\
      \end{mymatrix}$.
    \item $\det(B) = -15$, so $B$ is invertible. The inverse is
      $\displaystyle B^{-1} = \frac{1}{15}\begin{mymatrix}{rrr}
        -1 & -1 &  4 \\
        -4 & 11 &  1 \\
        8  & -7 & -2 \\
      \end{mymatrix}$.
    \item $\det(C) = 0$, so $C$ is not invertible.
    \end{enumerate}
  \end{sol}
\end{ex}

\begin{ex}
  Use the adjugate formula to find the inverse of the matrix
  \begin{equation*}
    A=\begin{mymatrix}{rrr}
      3 & 0 & 3 \\
      -1 & 2 & -3 \\
      -5 & 4 & -3
    \end{mymatrix}.
  \end{equation*}
  \begin{sol}
    We have
    \begin{equation*}
      \det(A)=36,
      \quad
      \cof(A)
      ~=~ \begin{mymatrix}{rrr}
        6  & 12 & 6   \\
        12 & 6  & -12 \\
        -6 & 6  & 6   \\
      \end{mymatrix},
      \quad
      \adj(A)
      ~=~ \begin{mymatrix}{rrr}
        6  &  12 & -6 \\
        12 &   6 &  6 \\
        6  & -12 &  6 \\
      \end{mymatrix},
    \end{equation*}
    and therefore
    \begin{equation*}
      \def\arraystretch{1.4}
      A^{-1}
      ~=~
      \frac{1}{36}
      \begin{mymatrix}{rrr}
        6  &  12 & -6 \\
        12 &   6 &  6 \\
        6  & -12 &  6 \\
      \end{mymatrix}
      ~=~
      \begin{mymatrix}{rrr}
        \frac{1}{6}  &  \frac{1}{3} & -\frac{1}{6} \\
        \frac{1}{3} &   \frac{1}{6} &  \frac{1}{6} \\
        \frac{1}{6}  & -\frac{1}{3} &  \frac{1}{6} \\
      \end{mymatrix}.
    \end{equation*}
  \end{sol}
\end{ex}

\begin{ex}
  Use the adjugate formula to find the inverse of the matrix
  \begin{equation*}
    A=\begin{mymatrix}{rrr}
      1 &  1 & 0 \\
      3 &  1 & 2 \\
      2 & -2 & 5 \\
    \end{mymatrix}.
  \end{equation*}
  \begin{sol}
    We have
    \begin{equation*}
      \det(A)
      ~=~ 2,
      \quad
      \cof(A)
      ~=~ \begin{mymatrix}{rrr}
        -9 & 11 &  8  \\
        -2 &  2 &  2  \\
        5  & -5 & -4 \\
      \end{mymatrix},
      \quad
      \adj(A)
      ~=~ \begin{mymatrix}{rrr}
        -9 & -2 &  5 \\
        11 &  2 & -5 \\
        8  &  2 & -4 \\
      \end{mymatrix},
    \end{equation*}
    and therefore
    \begin{equation*}
      \def\arraystretch{1.4}
      A^{-1}
      ~=~
      \frac{1}{2}
      \begin{mymatrix}{rrr}
        -9 & -2 &  5 \\
        11 &  2 & -5 \\
        8  &  2 & -4 \\
      \end{mymatrix}
      ~=~
      \begin{mymatrix}{rrr}
        -\frac{9}{2} & -1 & \frac{5}{2} \\
        \frac{11}{2} &  1 & -\frac{5}{2} \\
        4  &  1 & -2 \\
      \end{mymatrix}.
    \end{equation*}
  \end{sol}
\end{ex}

\begin{ex}
  Consider the matrix
  \begin{equation*}
    A =
    \begin{mymatrix}{ccc}
      1 & 0 & 0 \\
      0 & \cos t & -\sin t \\
      0 & \sin t & \cos t \\
    \end{mymatrix}.
  \end{equation*}
  Does there exist a value of $t$ for which this matrix fails to be
  invertible? Explain.
  \begin{sol}
    No. It has non-zero determinant $\det(A)=\cos^2 t+\sin^2 t = 1$ for
    all $t$, so it is invertible for all $t$.
  \end{sol}
\end{ex}

\begin{ex}
  Consider the matrix
  \begin{equation*}
    A =
    \begin{mymatrix}{rrr}
      1 & t & t^{2} \\
      0 & 1 & 2t \\
      t & 0 & 2
    \end{mymatrix}.
  \end{equation*}
  Does there exist a value of $t$ for which this matrix fails to be
  invertible? Explain.
  \begin{sol}
    $\det(A) = \begin{absmatrix}{ccc}
      1 & t & t^{2} \\
      0 & 1 & 2t \\
      t & 0 & 2
    \end{absmatrix} = t^{3}+2$,
    and so $A$ has no inverse when $t=-\sqrt[3]{2}$.
  \end{sol}
\end{ex}

\begin{ex}
  Consider the matrix
  \begin{equation*}
    A =
    \begin{mymatrix}{ccc}
      e^{t} & \cosh t & \sinh t \\
      e^{t} & \sinh t & \cosh t \\
      e^{t} & \cosh t & \sinh t
    \end{mymatrix}.
  \end{equation*}
  Does there exist a value of $t$ for which this matrix fails to be
  invertible? Explain.
  \begin{sol}
    Since the matrix $A$ has two identical rows, we have $\det(A)=0$
    for all $t$. So this matrix is non-invertible for all $t$.
  \end{sol}
\end{ex}

\begin{ex}
  Consider the matrix
  \begin{equation*}
    A =
    \begin{mymatrix}{ccc}
      e^{t} & e^{-t}\cos t & e^{-t}\sin t \\
      e^{t} & -e^{-t}\cos t-e^{-t}\sin t & -e^{-t}\sin t+e^{-t}\cos t \\
      e^{t} & 2e^{-t}\sin t & -2e^{-t}\cos t
    \end{mymatrix}.
  \end{equation*}
  Does there exist a value of $t$ for which this matrix fails to be
  invertible? Explain.
  \begin{sol}
    \begin{equation*}
      \det \begin{mymatrix}{ccc}
        e^{t} & e^{-t}\cos t & e^{-t}\sin t \\
        e^{t} & -e^{-t}\cos t-e^{-t}\sin t & -e^{-t}\sin t+e^{-t}\cos t \\
        e^{t} & 2e^{-t}\sin t & -2e^{-t}\cos t%
      \end{mymatrix} = 5e^{-t} \neq 0
    \end{equation*}
    and so this matrix is always invertible.
  \end{sol}
\end{ex}

\begin{ex}
  Use the adjugate formula to find the inverse of the matrix
  \begin{equation*}
    A=\begin{mymatrix}{ccc}
      e^{t} & 0 & 0 \\
      0 & \cos t & \sin t \\
      0 & \cos t-\sin t & \cos t+\sin t
    \end{mymatrix}.
  \end{equation*}
  \begin{sol}
    \begin{equation*}
      \det \begin{mymatrix}{ccc}
        e^{t} & 0 & 0 \\
        0 & \cos t & \sin t \\
        0 & \cos t-\sin t & \cos t+\sin t
      \end{mymatrix} = e^{t}.
    \end{equation*}
    Hence the inverse is
    \begin{equation*}
      e^{-t}\begin{mymatrix}{ccc}
        1 & 0 & 0 \\
        0 & e^{t}\cos t+e^{t}\sin t & -e^t(\cos t-\sin t) \\
        0 & -e^{t}\sin t & e^{t}\cos t
      \end{mymatrix}^{T} \\
      ~=~ \begin{mymatrix}{ccc}
        e^{-t} & 0 & 0 \\
        0 & \cos t+\sin t  & -\sin t \\
        0 & \sin t-\cos t  & \cos t
      \end{mymatrix}.
    \end{equation*}
    \end{sol}
\end{ex}

\begin{ex}
  Find the inverse, if it exists, of the matrix
  \begin{equation*}
    A =
    \begin{mymatrix}{ccc}
      e^{t} & \cos t & \sin t \\
      e^{t} & -\sin t & \cos t \\
      e^{t} & -\cos t & -\sin t
    \end{mymatrix}.
  \end{equation*}
  \begin{sol}
    \begin{equation*}
      \def\arraystretch{1.3}
      \begin{mymatrix}{ccc}
        e^{t} & \cos t & \sin t \\
        e^{t} & -\sin t & \cos t \\
        e^{t} & -\cos t & -\sin t
      \end{mymatrix}^{-1} \\
      ~=~\begin{mymatrix}{ccc}
        \frac{1}{2}e^{-t} & 0 & \frac{1}{2}e^{-t} \\
        \frac{1}{2}\cos t+\frac{1}{2}\sin t & -\sin t & \frac{1}{2}\sin t-\frac{1}{2}
        \cos t \\
        \frac{1}{2}\sin t-\frac{1}{2}\cos t & \cos t & -\frac{1}{2}\cos t-\frac{1}{2}
        \sin t
      \end{mymatrix}.
    \end{equation*}
  \end{sol}
\end{ex}

