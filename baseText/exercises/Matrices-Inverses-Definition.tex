\begin{ex}
  For each of the following pairs of matrices, determine whether $B$
  is an inverse of $A$.
  \begin{enumerate}
  \item
    \begin{equation*}
      A = \begin{mymatrix}{rr}
        2 & 4 \\
        1 & 3 \\
      \end{mymatrix},
      \quad
      B = \frac{1}{2}\begin{mymatrix}{rr}
        3 & -4 \\
        -1 & 2 \\
      \end{mymatrix}.
    \end{equation*}
  \item
    \begin{equation*}
      A = \begin{mymatrix}{rr}
        1 & -2 \\
        4 & -7 \\
      \end{mymatrix},
      \quad
      B = \begin{mymatrix}{rr}
        -1 & 2 \\
        -4 & 7 \\
      \end{mymatrix}.
    \end{equation*}
  \item
    \begin{equation*}
      A = \begin{mymatrix}{rrr}
        4 & 1 & 3 \\
        2 & 1 & 2 \\
        1 & 0 & 1 \\
      \end{mymatrix},
      \quad
      B = \begin{mymatrix}{rrr}
        1 & -1 & -1 \\
        0 & 1 & -2 \\
        -1 & 1 & 2 \\
      \end{mymatrix}.
    \end{equation*}
  \end{enumerate}
  \begin{sol}
    (a) Yes. (b) No. (c) Yes.
  \end{sol}
\end{ex}

\begin{ex}
  Suppose $AB=AC$ and $A$ is an invertible $n\times n$-matrix. Does it
  follow that $B=C$? Explain why or why not.
  \begin{sol}
    Yes $B=C$. Multiply $AB = AC$ on the left by $A^{-1}$.
  \end{sol}
\end{ex}

\begin{ex}
  Suppose $AB=AC$ and $A$ is a non-invertible $n\times n$-matrix. Does
  it follow that $B=C$? Explain why or why not.
  \begin{sol}
    No. For example, let
    $A=\begin{mymatrix}{cc}1&1\\1&1\end{mymatrix}$,
    $B=\begin{mymatrix}{cc}1&2\\3&4\end{mymatrix}$, and
    $C=\begin{mymatrix}{cc}2&3\\2&3\end{mymatrix}$. Then $AB=AC$ but
    $B\neq C$.
  \end{sol}
\end{ex}
