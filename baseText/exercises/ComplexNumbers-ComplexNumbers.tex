\section*{Exercises}


\begin{ex}
  Let $z=2+7i$ and let $w=3-8i$. Compute $z + w$, $z - 2w$, $zw$, and $\frac{w}{z}$.
  \begin{sol}
    $z + w = 5-i$, $z - 2w = -4 + 23i$, $zw = 62+5i$, and
    $\displaystyle\frac{w}{z} = -\frac{50}{53}-\frac{37}{53}i$.
  \end{sol}
\end{ex}

\begin{ex}
  Let $z = 1 - 4i$. Compute $\overline{z}$, $z^{-1}$, and $\abs{z}$.
\end{ex}

\begin{ex}
  Let $z = 3+5i$ and $w = 2-i$. Compute $\overline{zw}$, $\abs{zw}$,
  and $z^{-1}w$.
\end{ex}

\begin{ex}
  Use the properties of complex numbers to prove that if $z$ is a
  complex number, then there exists a complex number $w$ with
  $\abs{w}=1$ and $wz=\abs{z}$.
  \begin{sol}
    If $z=0$, let $w=1$. If $z\neq 0$, let
    $w =\displaystyle\frac{\abs{z}}{z}$. Note that
    $\displaystyle wz = \frac{\abs{z}}{z}z = \abs{z}$ and
    $\displaystyle \abs{w} = \frac{\abs{z}}{\abs{z}} = 1$.
  \end{sol}
\end{ex}

\begin{ex}
  I claim that $1=-1$. Here is why.
  \begin{equation*}
    -1=i^{2}=\sqrt{-1}\sqrt{-1}=\sqrt{(-1) ^{2}}=\sqrt{1}=1.
  \end{equation*}
  What is wrong with this argument?
  \begin{sol}
    The problem is that there is no single $\sqrt{-1}$. In the complex
    numbers, $-1$ has two square roots, namely $i$ and $-i$. Since
    each complex number has two square roots, and generally neither of
    them is positive or even real, the notation $\sqrt{z}$ does not
    have a fixed meaning in the complex numbers. Therefore, the
    equation $\sqrt{z}\sqrt{w} = \sqrt{zw}$ cannot be used. At best,
    we could maybe say $\sqrt{z}\sqrt{w} = \pm\sqrt{zw}$.
  \end{sol}
\end{ex}

