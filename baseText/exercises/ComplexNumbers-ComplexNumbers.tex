\Opensolutionfile{solutions}[ex]
\section*{Exercises}

\begin{enumialphparenastyle}

\begin{ex} Let $z=2+7i$ and let $w=3-8i$. Compute the following.

\begin{enumerate}
\item $z + w$

\item $z - 2w$

\item $zw$

\item $\frac{w}{z}$
\end{enumerate}
\begin{sol}
\begin{enumerate}
\item $z + w = 5-i$
\item $z - 2w = -4 + 23i$
\item $zw = 62+5i$ 
\item $\frac{w}{z} = -\frac{50}{53}-\frac{37}{53}i$
\end{enumerate}
\end{sol}
\end{ex}

\begin{ex} Let $z = 1 - 4i$. Compute the following.

\begin{enumerate}

\item $\overline{z}$

\item $z^{-1}$

\item $\abs{z}$
\end{enumerate}
%\begin{sol}
%\end{sol}
\end{ex}

\begin{ex} Let $z = 3 +5i$ and $w = 2 - i$. Compute the following.

\begin{enumerate}
\item $\overline{zw}$

\item $\abs{zw}$

\item $z^{-1}w$

\end{enumerate}
%\begin{sol}
%\end{sol}
\end{ex}

\begin{ex} If $z$ is a complex number, show there exists a complex
number $w$ with $\abs{w }=1$ and $wz=\abs{z}$. 
\begin{sol}
If $z=0$, let $w =1$. If $z\neq 0$, let $w =\displaystyle\frac{\overline{z}}{\abs{z}}$
\end{sol}
\end{ex}

\begin{ex} \label{exer-complex2}If $z,w$ are complex
\index{complex numbers!conjugate of a product} numbers prove $
\overline{zw}=\overline{z} \; \overline{w}$ and then show by induction that 
$\overline{z_{1}\cdots z_{m}}=\overline{z_{1}}\cdots \overline{z_{m}}$. Also
verify that $\overline{\sum_{k=1}^{m}z_{k}}=\sum_{k=1}^{m}\overline{z_{k}}$.
In words this says the conjugate of a product equals the product of the
conjugates and the conjugate of a sum equals the sum of the conjugates. 
\begin{sol}
\[
\overline{(a+bi) (c+di) }=\overline{ac-bd+(
ad+bc)i }=(ac-bd) -(ad+bc)i (
a-bi) (c-di) =ac-bd-(ad+bc)i 
\]
 which is the
same thing. Thus it holds for a product of two complex numbers. Now suppose
 you have that it is true for the product of $n$ complex numbers. Then
\[
\overline{z_{1}\cdots z_{n+1}}=\overline{z_{1}\cdots z_{n}}\ \overline{
z_{n+1}}
\]
and now, by induction this equals
\[
\overline{z_{1}}\cdots \overline{z_{n}}\ \overline{z_{n+1}}
\]
As to sums, this is even easier.
\[
\overline{\sum_{j=1}^{n}(x_{j}+iy_{j}) }=\overline{%
\sum_{j=1}^{n}x_{j}+i\sum_{j=1}^{n}y_{j}}
\]
\[
=\sum_{j=1}^{n}x_{j}-i\sum_{j=1}^{n}y_{j}=\sum_{j=1}^{n}x_{j}-iy_{j}=
\sum_{j=1}^{n}\overline{(x_{j}+iy_{j}) }.
\]
\end{sol}
\end{ex}

\begin{ex} \label{15-july-prob2}Suppose $p(x)
=a_{n}x^{n}+a_{n-1}x^{n-1}+\ldots +a_{1}x+a_{0}$ where all the $a_{k}$ are
real numbers. Suppose also that $p(z) =0$ for some $z\in \C$. Show it follows that $p(\overline{z}) =0$ also. 
\begin{sol}
If $p(z) =0$, then you have
\[
\overline{p(z) }=0=\overline{a_{n}z^{n}+a_{n-1}z^{n-1}+\ldots+a_{1}z+a_{0}}
\]
\[
=\overline{a_{n}z^{n}}+\overline{a_{n-1}z^{n-1}}+\ldots +\overline{a_{1}z}+
\overline{a_{0}}
\]
\[
=\overline{a_{n}}\ \overline{z}^{n}+\overline{a_{n-1}}\ \overline{z}
^{n-1}+\ldots +\overline{a_{1}}\ \overline{z}+\overline{a_{0}}
\]
\[
=a_{n}\overline{z}^{n}+a_{n-1}\overline{z}^{n-1}+\ldots +a_{1}\overline{z}
+a_{0}
\]
\[
=p(\overline{z})
\]
\end{sol}
\end{ex}

\begin{ex} I claim that $1=-1$. Here is why.
\begin{equation*}
-1=i^{2}=\sqrt{-1}\sqrt{-1}=\sqrt{(-1) ^{2}}=\sqrt{1}=1
\end{equation*}
This is clearly a remarkable result but is there something wrong with it? If
so, what is wrong? 
\begin{sol}
The problem is that there is no single $\sqrt{-1}$.
\end{sol}
\end{ex}

\end{enumialphparenastyle}
