\begin{ex}
  Find $\proj_{\vect{v}}(\vect{w}) $ where
  $\vect{w}=\begin{mymatrix}{r}
    1 \\
    0 \\
    -2
  \end{mymatrix}$ and $\vect{v}=\begin{mymatrix}{r}
    1 \\
    2 \\
    3
  \end{mymatrix}$.
  \begin{sol}
    $\displaystyle
    \def\arraystretch{1.2}
    \frac{\vect{v}\dotprod \vect{w}}{\vect{v}\dotprod
      \vect{v}}\,\vect{v}=\frac{-5}{14}\begin{mymatrix}{r}
      1 \\
      2 \\
      3
    \end{mymatrix} =\begin{mymatrix}{c}
      -\frac{5}{14} \\
      -\frac{5}{7} \\
      -\frac{15}{14}
    \end{mymatrix}$.
  \end{sol}
\end{ex}

\begin{ex}
  Find $\proj_{\vect{v}}(\vect{w})$ where
  $\vect{w}=\begin{mymatrix}{r}
    1 \\
    2 \\
    -2
  \end{mymatrix} $ and $\vect{v}=\begin{mymatrix}{r}
    1 \\
    0 \\
    3
  \end{mymatrix}$.
  \begin{sol}
    $\displaystyle
    \def\arraystretch{1.2}
    \frac{\vect{v}\dotprod \vect{w}}{\vect{v}\dotprod
      \vect{v}}\,\vect{v}=\frac{-5}{10}\begin{mymatrix}{r}
      1 \\
      0 \\
      3
    \end{mymatrix} =\begin{mymatrix}{r}
      -\frac{1}{2} \\
      0~ \\
      -\frac{3}{2}
    \end{mymatrix}$.
  \end{sol}
\end{ex}

\begin{ex}
  Find $\proj_{\vect{v}}(\vect{w}) $ where
  $\vect{w}=\begin{mymatrix}{r}
    1 \\
    2 \\
    -2 \\
    1
  \end{mymatrix}$ and $\vect{v}=\begin{mymatrix}{r}
    1 \\
    2 \\
    3 \\
    0
  \end{mymatrix}$.
  \begin{sol}
    $\displaystyle
    \def\arraystretch{1.2}
    \frac{\vect{v}\dotprod \vect{w}}{\vect{v}\dotprod
      \vect{v}}\,\vect{v}=\frac{\begin{mymatrix}{cccc} 1 & 2 & 3 & 0
      \end{mymatrix}^T \dotprod \begin{mymatrix}{cccc}
        1 & 2 & -2 & 1
      \end{mymatrix}^T}{1+4+9}\begin{mymatrix}{r}
      1 \\
      2 \\
      3 \\
      0
    \end{mymatrix}
    =\begin{mymatrix}{r}
      -\frac{1}{14} \\
      -\frac{2}{14} \\
      -\frac{3}{14} \\
      0~
    \end{mymatrix}$.
  \end{sol}
\end{ex}

\begin{ex}
  Find $\comp_{\vect{v}}(\vect{w}) $ where
  $\vect{w}=\begin{mymatrix}{r}
    1 \\
    1 \\
    2
  \end{mymatrix} $ and $\vect{v}=\begin{mymatrix}{r}
    0 \\
    3 \\
    3
  \end{mymatrix}$.
\end{ex}

\begin{ex}
  Find $\comp_{\vect{v}}(\vect{w}) $ where
  $\vect{w}=\begin{mymatrix}{r}
    1 \\
    -1 \\
    0
  \end{mymatrix} $ and $\vect{v}=\begin{mymatrix}{r}
    1 \\
    1 \\
    2
  \end{mymatrix}$.
\end{ex}

\begin{ex}
  Does it make sense to speak of $\proj_{\vect{0}}(\vect{w})$?
  \begin{sol}
    No, it does not. The vector $\vect{0}$ has no direction. The
    formula for $\proj_{\vect{0}}(\vect{w})$ doesn't make sense
    either.
  \end{sol}
\end{ex}

\begin{ex}
  Decompose the vector $\vect{v}$ into $\vect{v}=\vect{a}+\vect{b}$
  where $\vect{a}$ is parallel to $\vect{u}$ and $\vect{b}$ is
  orthogonal to $\vect{u}$.
  \begin{equation*}
    \vect{v} = \begin{mymatrix}{r} 3\\2\\-5 \end{mymatrix},\quad
    \vect{u} = \begin{mymatrix}{r} 1\\-1\\2 \end{mymatrix}.
  \end{equation*}

  \begin{sol}
    \begin{equation*}
      \vect{a} ~=~ \begin{mymatrix}{c} -3/2\\3/2\\-3 \end{mymatrix},\quad
      \vect{b} ~=~ \begin{mymatrix}{c} 9/2\\1/2\\-2 \end{mymatrix}.
    \end{equation*}
  \end{sol}
\end{ex}

\begin{ex}
  Prove the Cauchy Schwarz inequality in $\R^{n}$ as follows.  For
  $\vect{u},\vect{v}$ vectors, consider
  \begin{equation*}
    (\vect{u}-
    \proj_{\vect{v}}(\vect{u})) \dotprod (\vect{u}-
    \proj_{\vect{v}}(\vect{u})) \geq 0
  \end{equation*}
  Simplify using the axioms of the dot product and then put in the
  formula for the projection. Notice that this expression equals $0$
  and you get equality in the Cauchy Schwarz inequality if and only if
  $\vect{u}=\proj_{\vect{v}}(\vect{u})$. What is the geometric
  meaning of $\vect{u}=\proj_{\vect{v}}(\vect{u})$?
  \begin{sol}
    \begin{equation*}
      \paren{\vect{u}-\frac{\vect{v}\dotprod\vect{u}}{\norm{\vect{v}}^{2}}\,\vect{v}}
      \dotprod
      \paren{\vect{u}-\frac{\vect{v}\dotprod\vect{u}}{\norm{\vect{v}}^{2}}\,\vect{v}} 
      = \norm{\vect{u}}^{2} - 2(\vect{v}\dotprod \vect{u})^{2}\frac{1}{\norm{\vect{v}}^{2}}+(\vect{v}\dotprod \vect{u})^{2}\frac{1}{ \norm{\vect{v}}^{2}}
      \geq 0,
    \end{equation*}
    and so
    \begin{equation*}
      \norm{\vect{u}} ^{2}\norm{\vect{v}}
      ^{2}\geq (\vect{v}\dotprod \vect{u}) ^{2}.
    \end{equation*}
    We get equality exactly when
    $\displaystyle\vect{u}=\proj_{\vect{v}}(\vect{u}) =
    \frac{\vect{v}\dotprod \vect{u}}{\norm{\vect{v}} ^{2}}\,\vect{v}$,
    or in other words, when $\vect{u}$ is a multiple of $\vect{v}$.
  \end{sol}
\end{ex}

\begin{ex}\label{perp-linear}
  Let $\vect{v},\vect{w}$ $\vect{u}$ be vectors. Show that
  $(\vect{w}+\vect{u})_{\perp}=\vect{w}_{\perp}+\vect{u}_{\perp}$,
  where $\vect{w}_{\perp}=\vect{w}-\proj_{\vect{v}}(\vect{w})$.
  \begin{sol}
    \begin{eqnarray*}
      \vect{w}-\proj_{\vect{v}}(\vect{w})+\vect{u}-\proj_{\vect{v}}(\vect{u})
      &=& \vect{w}+\vect{u}-(\proj_{\vect{v}}(\vect{w})+\proj_{\vect{v}}(\vect{u})) \\
      &=&\vect{w}+\vect{u}-\proj_{\vect{v}}(\vect{w}+\vect{u}).
    \end{eqnarray*}
    This follows because
    \begin{eqnarray*}
      \proj_{\vect{v}}(\vect{w})+\proj_{\vect{v}}(\vect{u})
      &=& \frac{\vect{v}\dotprod\vect{w}}{\norm{\vect{v}}^{2}}\vect{v}
          + \frac{\vect{v}\dotprod\vect{u}}{\norm{\vect{v}}^{2}}\vect{v} \\
      &=& \frac{\vect{v}\dotprod(\vect{w}+\vect{u})}{\norm{\vect{v}}^{2}}\vect{v} \\
      &=& \proj_{\vect{v}}(\vect{w}+\vect{u}).
    \end{eqnarray*}
  \end{sol}
\end{ex}

\begin{ex}
  Show that
  \begin{equation*}
    \vect{u}\dotprod(\vect{v}-\proj_{\vect{u}}(\vect{v}))=0
  \end{equation*}
  and conclude every vector in $\R^{n}$ can be written as the sum of
  two vectors, one which is orthogonal and one which is parallel to
  the given vector.
  \begin{sol}
    $\displaystyle\vect{u}\dotprod(\vect{v}-\proj_{\vect{u}}(\vect{v}))
    = \vect{u} \dotprod \vect{v}-\frac{\vect{u}\cdot\vect{v}}{\norm{\vect{u}}^{2}}\vect{u}\dotprod\vect{u}
    = \vect{u} \dotprod \vect{v}
    - \vect{u} \dotprod \vect{v}
    =0$. Therefore, we can write
    $\vect{v} = (\vect{v} - \proj_{\vect{u}}(\vect{v})) +
    \proj_{\vect{u}}(\vect{v})$. The first is orthogonal to
    $\vect{u}$ and the second is a multiple of $\vect{u}$ so is
    parallel to $\vect{u}$.
  \end{sol}
\end{ex}

