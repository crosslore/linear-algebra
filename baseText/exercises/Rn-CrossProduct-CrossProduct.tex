\begin{enumialphparenastyle}

\begin{ex} Show that if $\vect{a}\times \vect{u}=\vect{0}$ for any unit vector $\vect{u}
$, then $\vect{a}=\vect{0}$.
\begin{sol}
If $\vect{a}\neq \vect{0}$, then the condition says that $\vectlength \vect{a}\times \vect{u}\vectlength = \vectlength \vect{a} \vectlength \sin \theta =0$ for all angles $\theta $. Hence $\vect{a} = \vect{0}$ after all.
\end{sol}
\end{ex}

\begin{ex} Find the area of the triangle determined by the three points, $\left(
1,2,3\right) ,\left( 4,2,0\right) $ and $\left( -3,2,1\right) .$
\begin{sol}
$\leftB \begin{array}{r}
3 \\
0 \\
-3
\end{array}
\rightB \times \leftB \begin{array}{r}
 -4 \\
0 \\
-2
\end{array}
\rightB =\leftB
\begin{array}{r}
0 \\
18 \\
0
\end{array}
\rightB .$ So the area is $9.$
\end{sol}
\end{ex}

\begin{ex} Find the area of the triangle determined by the three points, $\left(
1,0,3\right) ,\left( 4,1,0\right) $ and $\left( -3,1,1\right) .$
\begin{sol}
 $\leftB \begin{array}{r}
3 \\
1 \\
-3
\end{array}
\rightB \times \leftB
\begin{array}{r}
 -4 \\
1 \\
-2
\end{array}
\rightB =\leftB
\begin{array}{c}
1 \\
18 \\
7
\end{array}
\rightB$. The area is given by 
\[
\frac{1}{2}\sqrt{1+\left( 18\right) ^{2}+49}=\frac{1}{2}\sqrt{374}
\]
\end{sol}
\end{ex}

\begin{ex} Find the area of the triangle determined by the three points, $\left(
1,2,3\right) ,\left( 2,3,4\right) $ and $\left( 3,4,5\right) .$ Did
something interesting happen here? What does it mean geometrically?
\begin{sol}
$\leftB
\begin{array}{rrr}
1 & 1 & 1
\end{array}
\rightB \times \leftB
\begin{array}{rrr}
 2 & 2 & 2
\end{array}
\rightB =\leftB
\begin{array}{ccc}
0 & 0 & 0
\end{array}
\rightB $.  The area is 0. It means the three points are on the same line.
\end{sol}
\end{ex}

\begin{ex} Find the area of the parallelogram determined by the vectors $\leftB
\begin{array}{r}
1 \\
2 \\
3
\end{array}
\rightB $, $\leftB
\begin{array}{r}
3 \\
-2 \\
1
\end{array}
\rightB .$
\begin{sol}
$\leftB \begin{array}{r}
1 \\
2 \\
3
\end{array}
\rightB \times
\leftB
\begin{array}{r}
3 \\
-2 \\
1
\end{array}
\rightB =\leftB \begin{array}{r}
8 \\
8 \\
-8
\end{array}
\rightB .$ The area is $8\sqrt{3}$
\end{sol}
\end{ex}


\begin{ex} Find the area of the parallelogram determined by the vectors
$\leftB
\begin{array}{r}
1 \\
0 \\
3
\end{array}
\rightB $, $\leftB
\begin{array}{r}
4 \\
-2 \\
1
\end{array}
\rightB .$
\begin{sol}
$\leftB \begin{array}{r}
1 \\
0 \\
3
\end{array}
\rightB \times
\leftB
\begin{array}{r}
4 \\
-2 \\
1
\end{array}
\rightB =\leftB
\begin{array}{r}
6 \\
11 \\
-2
\end{array}
\rightB .$ The area is $\sqrt{36+121+4}= \sqrt{161}$
\end{sol}
\end{ex}


\begin{ex} Is $\vect{u}\times \left( \vect{v}\times \vect{w}\right) =\left( 
\vect{u}\times \vect{v}\right) \times \vect{w}$? What is the meaning of 
$\vect{u}\times \vect{v}\times \vect{w}?$ Explain. 
\textbf{Hint: }Try $\left( \vect{i}\times \vect{j}
\right)\times \vect{k}$.
\begin{sol}
 $\left( \vect{i}\times \vect{j}\right) \times
\vect{j}=\vect{k}\times \vect{j}=-\vect{i}.$ However, $\vect{i}\times \left( \vect{j}\times \vect{j}
\right) =\vect{0}$ and so the cross product is not associative.
\end{sol}
\end{ex}

\begin{ex} Verify directly that the coordinate description of the cross product, 
$\vect{u}\times \vect{v}$ has the property that it is perpendicular to both 
$\vect{u}$ and $\vect{v}$. Then show by direct computation that this
coordinate description satisfies
\begin{align*}
\vectlength \vect{u}\times \vect{v}\vectlength ^{2}& =\vectlength \vect{u}
\vectlength ^{2}\vectlength \vect{v}\vectlength ^{2}-\left( \vect{u}\dotprod \vect{v}\right) ^{2} \\
& =\vectlength \vect{u}\vectlength ^{2}\vectlength \vect{v}\vectlength
^{2}\left( 1-\cos ^{2}\left( \theta \right) \right)
\end{align*}
where $\theta $ is the angle included between the two vectors. Explain why 
$\vectlength \vect{u}\times \vect{v}\vectlength $ has the correct magnitude.
\begin{sol}
Verify directly from the coordinate description of the cross product that the right hand rule applies to the vectors $\vect{i},\vect{j},\vect{k}.$ Next verify that the
distributive law holds for the coordinate description of the cross product.
This gives another way to approach the cross product. First define it in
terms of coordinates and then get the geometric properties from this.
However, this approach does not yield the right hand rule property very
easily. From the coordinate description,
\[
\vect{a}\times \vect{b}\cdot \vect{a}=\varepsilon _{ijk}a_{j}b_{k}a_{i}=-\varepsilon
_{jik}a_{j}b_{k}a_{i}=-\varepsilon _{jik}b_{k}a_{i}a_{j}=-\vect{a}\times
\vect{b}\cdot \vect{a}
\]
and so $\vect{a}\times \vect{b}$ is perpendicular to $\vect{a}$. Similarly, $
\vect{a}\times \vect{b}$ is perpendicular to $\vect{b}$. Now we need that 
\[
\vectlength \vect{a}\times \vect{b}\vectlength ^{2}=\vectlength \vect{a}
\vectlength ^{2}\vectlength \vect{b}\vectlength ^{2}\left( 1-\cos
^{2}\theta \right) =\vectlength \vect{a}\vectlength ^{2}\vectlength \vect{b
}\vectlength ^{2}\sin ^{2}\theta
\]
and so $\vectlength \vect{a}\times \vect{b}\vectlength =\vectlength \vect{a}
\vectlength \vectlength \vect{b}\vectlength \sin \theta ,$ the area of the
parallelogram determined by $\vect{a},\vect{b}$. Only the right hand rule is a
little problematic. However, you can see right away from the component
definition that the right hand rule holds for each of the standard unit
vectors. Thus $\vect{i}\times \vect{j}=\vect{k}$ etc.
\[
\left\vert
\begin{array}{ccc}
\vect{i} & \vect{j} & \vect{k} \\
1 & 0 & 0 \\
0 & 1 & 0
\end{array}
\right\vert =\vect{k}
\]
\end{sol}
\end{ex}

\begin{ex} Suppose $A$ is a $3\times 3$ skew symmetric matrix such that $A^{T}=-A$. Show
there exists a vector $\vect{\Omega }$ such that for all $\vect{u}\in
\mathbb{R}^{3}$
\begin{equation*}
A\vect{u}=\vect{\Omega} \times \vect{u}
\end{equation*}
\textbf{Hint: }Explain why, since $A$ is skew symmetric it is of the form
\begin{equation*}
A=\leftB
\begin{array}{ccc}
0 & -\omega _{3} & \omega _{2} \\
\omega _{3} & 0 & -\omega _{1} \\
-\omega _{2} & \omega _{1} & 0
\end{array}
\rightB
\end{equation*}
where the $\omega _{i}$ are numbers. Then consider $\omega _{1}\vect{i}
+\omega _{2}\vect{j}+\omega _{3}\vect{k}$.
%\begin{sol}
%\end{sol}
\end{ex}

\end{enumialphparenastyle}