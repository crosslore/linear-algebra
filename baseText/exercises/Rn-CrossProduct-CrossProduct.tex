
\begin{ex}
  Show that if $\vect{a}\times \vect{u}=\vect{0}$ for every unit
  vector $\vect{u}$, then $\vect{a}=\vect{0}$.
  \begin{sol}
    If $\vect{a}\neq \vect{0}$, then the condition says that
    $\norm{\vect{a}\times \vect{u}} = \norm{\vect{a}} \sin \theta =0$
    for all angles $\theta$. Hence $\vect{a} = \vect{0}$ after all.
  \end{sol}
\end{ex}

\begin{ex}
  Find the area of the triangle determined by the three points,
  $(1,2,3) ,(4,2,0) $ and $(-3,2,1)$.
  \begin{sol}
    $\begin{mymatrix}{r}
      3 \\
      0 \\
      -3
    \end{mymatrix} \times \begin{mymatrix}{r}
      -4 \\
      0 \\
      -2
    \end{mymatrix} =\begin{mymatrix}{r}
      0 \\
      18 \\
      0
    \end{mymatrix}$. So the area is $9$.
  \end{sol}
\end{ex}

\begin{ex}
  Find the area of the triangle determined by the three points,
  $(1,0,3) ,(4,1,0) $ and $(-3,1,1)$.
  \begin{sol}
    $\begin{mymatrix}{r}
      3 \\
      1 \\
      -3
    \end{mymatrix} \times \begin{mymatrix}{r}
      -4 \\
      1 \\
      -2
    \end{mymatrix} =\begin{mymatrix}{c}
      1 \\
      18 \\
      7
    \end{mymatrix}$. The area is given by
    $\displaystyle\frac{1}{2}\sqrt{1+18^{2}+49}=\frac{1}{2}\sqrt{374}$.
  \end{sol}
\end{ex}

\begin{ex}
  Find the area of the triangle determined by the three points,
  $(1,2,3) ,(2,3,4) $ and $(3,4,5)$. Did something
  interesting happen here? What does it mean geometrically?
  \begin{sol}
    $\begin{mymatrix}{rrr}
      1 & 1 & 1
    \end{mymatrix} \times \begin{mymatrix}{rrr}
      2 & 2 & 2
    \end{mymatrix} =\begin{mymatrix}{ccc}
      0 & 0 & 0
    \end{mymatrix}$.  The area is 0. It means the three points are on
    the same line.
  \end{sol}
\end{ex}

\begin{ex}
  Find the area of the parallelogram determined by the vectors
  $\begin{mymatrix}{r}
    1 \\
    2 \\
    3
  \end{mymatrix}$, $\begin{mymatrix}{r}
    3 \\
    -2 \\
    1
  \end{mymatrix}$.
  \begin{sol}
    $\begin{mymatrix}{r}
      1 \\
      2 \\
      3
    \end{mymatrix} \times
    \begin{mymatrix}{r}
      3 \\
      -2 \\
      1
    \end{mymatrix} =\begin{mymatrix}{r}
      8 \\
      8 \\
      -8
    \end{mymatrix}$. The area is $8\sqrt{3}$.
  \end{sol}
\end{ex}


\begin{ex}
  Find the area of the parallelogram determined by the vectors
  $\begin{mymatrix}{r}
    1 \\
    0 \\
    3
  \end{mymatrix}$, $\begin{mymatrix}{r}
    4 \\
    -2 \\
    1
  \end{mymatrix}$.
  \begin{sol}
    $\begin{mymatrix}{r}
      1 \\
      0 \\
      3
    \end{mymatrix} \times
    \begin{mymatrix}{r}
      4 \\
      -2 \\
      1
    \end{mymatrix} =\begin{mymatrix}{r}
      6 \\
      11 \\
      -2
    \end{mymatrix}$. The area is $\sqrt{36+121+4}= \sqrt{161}$.
  \end{sol}
\end{ex}


\begin{ex}
  Is
  $\vect{u}\times (\vect{v}\times\vect{w}) = (\vect{u}\times
    \vect{v})\times\vect{w}$? What is the meaning of
  $\vect{u}\times \vect{v}\times\vect{w}$? Explain.  \textbf{Hint:}
  Try $(\vect{i}\times\vect{j})\times\vect{j}$.
  \begin{sol}
    $(\vect{i}\times\vect{j})\times\vect{j} = \vect{k}\times
    \vect{j}=-\vect{i}$. However,
    $\vect{i}\times(\vect{j}\times\vect{j}) = \vect{0}$ and so
    the cross product is not associative. The expression
    $\vect{u}\times\vect{v}\times\vect{w}$ has no meaning.
  \end{sol}
\end{ex}

\begin{ex}
  Verify directly that the coordinate description of the cross
  product, $\vect{u}\times \vect{v}$ has the property that it is
  perpendicular to both $\vect{u}$ and $\vect{v}$. Then show by direct
  computation that this coordinate description satisfies
  \begin{align*}
    \norm{\vect{u}\times \vect{v}} ^{2}
    & =\norm{\vect{u}
      } ^{2}\norm{\vect{v}} ^{2}-(\vect{u}\dotprod \vect{v}) ^{2} \\
    & =\norm{\vect{u}} ^{2}\norm{\vect{v}}
      ^{2}(1-\cos ^{2}(\theta)),
  \end{align*}
  where $\theta$ is the angle included between the two
  vectors. Explain why $\norm{\vect{u}\times\vect{v}}$ has the
  correct magnitude.
\end{ex}

\begin{ex}\label{ex:triple-cross-product}
  Prove the following formula by direct calculation:
  $\vect{u}\times(\vect{v}\times\vect{w}) ~=~
  (\vect{u}\dotprod\vect{w})\,\vect{v} -
  (\vect{u}\dotprod\vect{v})\,\vect{w}$.
  \begin{sol}
    Let $\vect{u}=\mat{u_1,u_2,u_3}^T$,
    $\vect{v}=\mat{v_1,v_2,v_3}^T$, and
    $\vect{w}=\mat{w_1,w_2,w_3}^T$.
    Then
    \begin{eqnarray*}
      \vect{u}\times(\vect{v}\times\vect{w})
      &=&
      \begin{mymatrix}{c}u_1\\u_2\\u_3\end{mymatrix}
      \times
      \begin{mymatrix}{c}
        v_2w_3 - v_3w_2 \\
        v_3w_1 - v_1w_3 \\
        v_1w_2 - v_2w_1 \\
      \end{mymatrix}
      ~=~
      \begin{mymatrix}{c}
        u_2(v_1w_2 - v_2w_1) - u_3(v_3w_1 - v_1w_3) \\
        u_3(v_2w_3 - v_3w_2) - u_1(v_1w_2 - v_2w_1) \\
        u_1(v_3w_1 - v_1w_3) - u_2(v_2w_3 - v_3w_2) \\
      \end{mymatrix}\\
      &=&
      \begin{mymatrix}{c}
        u_2v_1w_2 - u_2v_2w_1 - u_3v_3w_1 + u_3v_1w_3 \\
        u_3v_2w_3 - u_3v_3w_2 - u_1v_1w_2 + u_1v_2w_1 \\
        u_1v_3w_1 - u_1v_1w_3 - u_2v_2w_3 + u_2v_3w_2 \\
      \end{mymatrix}
    \end{eqnarray*}
    and
    \begin{eqnarray*}
      (\vect{u}\dotprod\vect{w})\,\vect{v} -
      (\vect{u}\dotprod\vect{v})\,\vect{w}
      &=& (u_1w_1+u_2w_2+u_3w_3)
          \begin{mymatrix}{c} v_1\\v_2\\v_3 \end{mymatrix}
      - (u_1v_1+u_2v_2+u_3v_3)
      \begin{mymatrix}{c} w_1\\w_2\\w_3 \end{mymatrix}
      \\
      &=& \begin{mymatrix}{c}
        u_1w_1v_1+u_2w_2v_1+u_3w_3v_1 - u_1v_1w_1-u_2v_2w_1-u_3v_3w_1 \\
        u_1w_1v_2+u_2w_2v_2+u_3w_3v_2 - u_1v_1w_2-u_2v_2w_2-u_3v_3w_2 \\
        u_1w_1v_3+u_2w_2v_3+u_3w_3v_3 - u_1v_1w_3-u_2v_2w_3-u_3v_3w_3 \\
      \end{mymatrix}.
    \end{eqnarray*}
    We can see by careful inspection that these are equal.
  \end{sol}
\end{ex}

\begin{ex}
  Use the formula from Exercise~\ref{ex:triple-cross-product} to prove
  that the cross product satisfies the so-called \textbf{Jacobi
    identity}%
  \index{Jacobi identity}\index{cross product!Jacobi identity}:
  \begin{equation*}
    \vect{u}\times(\vect{v}\times\vect{w})
    + \vect{v}\times(\vect{w}\times\vect{u})
    + \vect{w}\times(\vect{u}\times\vect{v})
    = \vect{0}.
  \end{equation*}
  \begin{sol}
    \begin{equation*}
      \begin{array}{l}
        \vect{u}\times(\vect{v}\times\vect{w}) 
        ~+~ \vect{v}\times(\vect{w}\times\vect{u}) 
        ~+~ \vect{w}\times(\vect{u}\times\vect{v}) \\
        {} = ((\vect{u}\dotprod\vect{w})\,\vect{v}
        - (\vect{u}\dotprod\vect{v})\,\vect{w})
        ~+~ ((\vect{v}\dotprod\vect{u})\,\vect{w}
        - (\vect{v}\dotprod\vect{w})\,\vect{u})
        ~+~ ((\vect{w}\dotprod\vect{v})\,\vect{u}
        - (\vect{w}\dotprod\vect{u})\,\vect{v}) \\
        {} = ((\vect{u}\dotprod\vect{w})\,\vect{v}
        - (\vect{w}\dotprod\vect{u})\,\vect{v})
        ~+~ ((\vect{v}\dotprod\vect{u})\,\vect{w}
        - (\vect{u}\dotprod\vect{v})\,\vect{w})
        ~+~ ((\vect{w}\dotprod\vect{v})\,\vect{u}
        - (\vect{v}\dotprod\vect{w})\,\vect{u}) \\
        {} = \vect{0}.
      \end{array}
    \end{equation*}
  \end{sol}
  \vspace{-4ex}
\end{ex}

