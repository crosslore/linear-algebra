\section*{Exercises}

\begin{ex}
  Let $T:\Poly_2 \to \R$ be a linear transformation such that
  \begin{equation*}
    T(x^2)=1,
    \quad
    T(x^2+x)=5,
    \quad\mbox{and}\quad
    T(x^2+x+1)=-1.
  \end{equation*}
  Find $T(ax^2+bx+c)$.
  \begin{sol}
    By linearity we have
    $T(x^2)=1$, $T(x) = T(x^2+x - x^2)= T(x^2+x) - T(x^2)= 5-1=5$, and
    $T(1) = T(x^2+x+1 -(x^2+x))=T(x^2+x+1) -T(x^2+x))= -1-5=-6$.
    Thus $T(ax^2+bx+c) = aT(x^2) + bT(x) + cT(1) = a+5b-6c$.
  \end{sol}
\end{ex}

\begin{ex}
  Let vectors $\vect{v}_1,\ldots,\vect{v}_n\in\R^n$ and
  $\vect{w}_1,\ldots,\vect{w}_n\in\R^m$ be given. Let $A$ be the
  matrix whose columns are $\vect{v}_1,\ldots,\vect{v}_n$, and assume
  that $A^{-1}$ exists. Show that there exists a linear transformation
  $T$ such that $T(\vect{v}_i)=\vect{w}_i$ for $i=1,\ldots,n$.
  \begin{sol}
    The matrix $A$ is invertible if and only if its rank is $n$, which
    means that $\vect{v}_1,\ldots,\vect{v}_n$ are linearly independent
    and therefore a basis of $\R^n$. The existence of $T$ then follows
    from Theorem~\ref{thm:transformation-basis}.
  \end{sol}
\end{ex}

