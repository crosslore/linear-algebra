\section*{Exercises}


\begin{ex}
  Which of the following vector functions are linear transformations?
  \begin{equation*}
    T_1\paren{\begin{mymatrix}{c} x \\ y \end{mymatrix}}
    = \begin{mymatrix}{c} 2x+y \\ x-2y \\ -x-y \end{mymatrix},\quad
    T_2\paren{\begin{mymatrix}{c} x \\ y \\ z \end{mymatrix}}
    = \begin{mymatrix}{c} x+y^2 \\ (x+y)z \\ 0 \end{mymatrix},\quad
    T_3\paren{\begin{mymatrix}{c} x \\ y \\ z \end{mymatrix}}
    = \begin{mymatrix}{c} 0 \\ 0 \end{mymatrix}.
  \end{equation*}
  \begin{sol}
    $T_1$ and $T_3$ are linear, and $T_2$ is not. The transformation
    $T_3$ is called the \textbf{zero transformation}%
    \index{zero transformation}%
    \index{linear transformation!zero transformation}.
  \end{sol}
\end{ex}

\begin{ex} Consider the following functions $T:\R^{3}\rightarrow \R^{2}$.
  Explain why each of these functions $T$ is not linear.

  \begin{enumerate}
  \item $T\begin{mymatrix}{c}
      x \\
      y \\
      z
    \end{mymatrix} =\begin{mymatrix}{c}
      x+2y+3z+1 \\
      2y-3x+z
    \end{mymatrix} $

  \item $T\begin{mymatrix}{c}
      x \\
      y \\
      z
    \end{mymatrix} =\begin{mymatrix}{c}
      x+2y^{2}+3z \\
      2y+3x+z
    \end{mymatrix} $

  \item $T\begin{mymatrix}{c}
      x \\
      y \\
      z
    \end{mymatrix} =\begin{mymatrix}{c}
      \sin x+2y+3z \\
      2y+3x+z
    \end{mymatrix} $

  \item $T\begin{mymatrix}{c}
      x \\
      y \\
      z
    \end{mymatrix} =\begin{mymatrix}{c}
      x+2y+3z \\
      2y+3x-\ln z
    \end{mymatrix} $
  \end{enumerate}
  % \begin{sol}
  % \end{sol}
\end{ex}

\begin{ex}
  Let $A$ be an $m\times n$-matrix. Show the vector function
  $T:\R^{n}\to \R^{m}$ defined by $T(\vect{v})=A\vect{v}$ is a
  linear transformation.
  \begin{sol}
    We have
    $T(a\vect{v}+b\vect{w}) = A(a\vect{v}+b\vect{w}) = a(A\vect{v}) +
    b(A\vect{w}) = aT(\vect{v}) + bT(\vect{w})$ by properties of
    matrix multiplication. Therefore, $T$ is linear by
    Proposition~\ref{prop:linear-transformation-alternative}.
  \end{sol}
\end{ex}

\begin{ex}
  Let $\vect{u}\in\R^n$ be a fixed vector.  Show that the function $T$
  defined by
  $T(\vect{v}) = \vect{v}-\proj_{\vect{u}}(\vect{v})$ is a
  linear transformation.
  \begin{sol}
    We have
    \begin{eqnarray*}
      T_{\vect{u}}(a\vect{v}+b\vect{w})
      &=& a\vect{v}+b\vect{w}
          -\frac{(a\vect{v}+b\vect{w})\dotprod\vect{u}}{\norm{\vect{u}}^{2}}
          \vect{u} \\
      &=& a\vect{v}
          -a\frac{(\vect{v}\dotprod\vect{u})}{\norm{\vect{u}}^{2}}\vect{u}
          +b\vect{w}
          -b\frac{(\vect{w}\dotprod\vect{u})}{\norm{\vect{u}}^{2}}\vect{u} \\
      &=& aT_{\vect{u}}(\vect{v}) +bT_{\vect{u}}(\vect{w}).
    \end{eqnarray*}
    Therefore, $T$ is linear.
  \end{sol}
\end{ex}

\begin{ex}
  Let $\vect{u}\in\R^n$ be a fixed non-zero vector. The function $T$
  defined by $T(\vect{v})=\vect{u}+\vect{v}$ has the effect of
  translating all vectors by adding $\vect{u}$. Show this is not a
  linear transformation.
  \begin{sol}
    If $T$ were a linear transformation, it should satisfy
    $T(\vect{0}) = \vect{0}$, but it does not. Also
    $T(\vect{v}+\vect{w}) \neq T\vect{v}+T\vect{w}$.
  \end{sol}
\end{ex}

