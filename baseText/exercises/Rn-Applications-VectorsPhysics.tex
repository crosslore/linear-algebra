\begin{enumialphparenastyle}

\begin{ex} The wind blows from the South at $20\textrm{km}/\textrm{h}$ and an
airplane which flies at $600\textrm{km}/\textrm{h}$ in still air is heading
East. Find the velocity of the airplane and its location after
two hours. \vspace{1mm}
%\begin{sol}
%\end{sol}
\end{ex}

\begin{ex} The wind blows from the West at $30\textrm{km}/\textrm{h}$ and an
airplane which flies at $400\textrm{km}/\textrm{h}$ in still air is heading
North East. Find the velocity of the airplane and its position
after two hours. \vspace{1mm}
%\begin{sol}
%\end{sol}
\end{ex}

\begin{ex} The wind blows from the North at $10\textrm{km}/\textrm{h}$. An
airplane which flies at $300\textrm{km}/\textrm{h}$ in still air is supposed
to go to the point whose coordinates are at $\tup{100, 100 }. $ In what direction should the airplane fly? \vspace{1mm}
%\begin{sol}
%\end{sol}
\end{ex}

\begin{ex} Three forces act on an object. Two are $\begin{mymatrix}{r}
3 \\
-1 \\
-1
\end{mymatrix} $ and $\begin{mymatrix}{r}
1 \\
-3 \\
4
\end{mymatrix} $ Newtons. Find the third force if the object is not to move.
\vspace{1mm}
%\begin{sol}
%\end{sol}
\end{ex}

\begin{ex} Three forces act on an object. Two are $\begin{mymatrix}{r}
6 \\
-3 \\
3
\end{mymatrix} $ and $\begin{mymatrix}{r}
2 \\
1 \\
3
\end{mymatrix} $ Newtons. Find the third force if the total force on the object is
to be $\begin{mymatrix}{r}
7 \\
1 \\
3
\end{mymatrix} $. \vspace{1mm}
%\begin{sol}
%\end{sol}
\end{ex}

\begin{ex} A river flows West at the rate of $b$ kilometers per hour. A boat can move
at the rate of $8\textrm{km}/\textrm{h}$. Find the smallest value of $b$ such that
it is not possible for the boat to proceed directly across the river.
\vspace{1mm}
%\begin{sol}
%\end{sol}
\end{ex}

\begin{ex} The wind blows from West to East at a speed of $50\textrm{km}/\textrm{h}$ and
an airplane which travels at $400\textrm{km}/\textrm{h}$ in still air is heading
North West. What is the velocity of the airplane relative to the ground?
What is the component of this velocity in the direction North? \vspace{1mm}
\begin{sol}
The velocity is the sum of two vectors. $50\vect{i}+\frac{
300}{\sqrt{2}} \tup{\vect{i}+\vect{j}} =\tup{50+\frac{300}{\sqrt{2}}
} \vect{i}+ \frac{300}{\sqrt{2}}\vect{j}$. The component in the
direction of North is then $\frac{300}{\sqrt{2}}= 150\sqrt{2}$
and the velocity relative to the ground is
\[
\tup{50+\frac{300}{\sqrt{2}}} \vect{i}+\frac{300}{\sqrt{2}}\vect{j}
\]
\end{sol}
\end{ex}

\begin{ex} The wind blows from West to East at a speed of $60\textrm{km}/\textrm{h}$ and
an airplane can travel travels at $100\textrm{km}/\textrm{h}$ in still air. How
many degrees West of North should the airplane head in order to travel
exactly North? \vspace{1mm}
%\begin{sol}
%\end{sol}
\end{ex}


\begin{ex} The wind blows from West to East at a speed of $50\textrm{km}/\textrm{h}$ and
an airplane which travels at $400\textrm{km}/\textrm{h}$ in still air heading
somewhat West of North so that, with the wind, it is flying due North. It
uses $30.0$ gallons of gas every hour. If it has to travel $600.0$ kilometers due
North, how much gas will it use in flying to its destination? \vspace{1mm}
%\begin{sol}
%\end{sol}
\end{ex}

\begin{ex} An airplane is flying due north at $150.0\textrm{km}/\textrm{h}$ but it is
not actually going due North because there is a wind which is pushing the
airplane due east at $40.0\textrm{km}/\textrm{h}$. After one hour, the plane starts
flying $30^{\circ }$ East of North. Assuming the plane starts at $\tup{
0,0} $, where is it after $2$ hours? Let North be the direction of the
positive $y$ axis and let East be the direction of the positive $x$ axis.
\vspace{1mm}
\begin{sol}
 Velocity of plane for the first hour: $\begin{mymatrix}{cc} 
0 & 150
\end{mymatrix}  + \begin{mymatrix}{cc}
40 & 0
\end{mymatrix} =\begin{mymatrix}{cc}
40 & 150
\end{mymatrix} $. After one hour it is at $\tup{40,150} $. Next the
velocity of the plane is $150\begin{mymatrix}{cc}
 \frac{1}{2} & \frac{\sqrt{3}}{2}
\end{mymatrix}
+\begin{mymatrix}{cc}
 40 & 0
\end{mymatrix} $ in kilometers per hour. After two hours it is then at 
$\tup{40,150} + 150\begin{mymatrix}{cc}
 \frac{1}{2} & \frac{\sqrt{3}}{2}
\end{mymatrix}
+\begin{mymatrix}{cc}
 40 & 0
\end{mymatrix}  =  \begin{mymatrix}{cc}
155 & 75\sqrt{3}+150
\end{mymatrix} = \begin{mymatrix}{cc}
155.0 & 279.\,\allowbreak 9
\end{mymatrix} $
\end{sol}
\end{ex}

\begin{ex} 
City A is located at the origin $\tup{0,0 }$ while city B is located at $\tup{300,500 } $ where distances are in kilometers. An airplane flies at $250\textrm{km}/\textrm{h}$ in still air. This airplane wants to fly from city A to city
B but the wind is blowing in the direction of the positive $y$ axis at a
speed of $50\textrm{km}/\textrm{h}$. Find a unit vector such that if the plane heads
in this direction, it will end up at city B having flown the shortest
possible distance. How long will it take to get there? \vspace{1mm}
\begin{sol}
Wind: $\begin{mymatrix}{cc}
0 & 50
\end{mymatrix} $. Direction it needs to travel: $\tup{3,5 } \frac{1}{\sqrt{34}}$. Then you need $250 \begin{mymatrix}{cc}
 a & b
\end{mymatrix} + \begin{mymatrix}{cc}
0 & 50
\end{mymatrix} $ to
have this direction where $\begin{mymatrix}{cc}
 a & b
\end{mymatrix} $ is an appropriate unit
vector. Thus you need
\begin{eqnarray*}
a^{2}+b^{2} &=&1 \\ 
& & \\
\frac{250b+50}{250a} &=&\frac{5}{3}
\end{eqnarray*}
Thus $a=\frac{3}{5},b=\frac{4}{5}$. The velocity of the plane relative
to the ground is $\begin{mymatrix}{cc}
150 & 250
\end{mymatrix} $. The speed of the plane relative to the ground is given by 
\[
\sqrt{\tup{150} ^{2}+\tup{250} ^{2}}=
291.55 \textrm{km}/\textrm{h}
\]
It has to go a distance of $\sqrt{\tup{300} ^{2}+\tup{500}
^{2}}=\allowbreak 583.\,\allowbreak 10$ kilometers. Therefore, it takes
\[
\frac{\allowbreak 583.\,\allowbreak 1}{\allowbreak 291.\,\allowbreak 55}=2
\text{ hours}
\]
\end{sol}
\end{ex}

\begin{ex} A certain river is one half kilometer wide with a current flowing at $2\textrm{km}/\textrm{h}$ from East to West. A man swims directly toward the
opposite shore from the South bank of the river at a speed of $3$ kilometers
per hour. How far down the river does he find himself when he has swam
across? How far does he end up travelling? \vspace{1mm}
\begin{sol}
Water:$\begin{mymatrix}{rr}
-2 & 0
\end{mymatrix} $ 
Swimmer:$\begin{mymatrix}{rr}
0 & 3
\end{mymatrix} $ 
Speed relative to earth: $\begin{mymatrix}{rr}
 -2 & 3
\end{mymatrix} $. 
It takes him $1/6$ of an hour to get across. Therefore, he ends up travelling $\frac{1}{6}\sqrt{4+9}= \frac{1}{6}\sqrt{13}$ kilometers. He ends up $1/3$ kilometer
down stream.
\end{sol}
\end{ex}

\begin{ex} A certain river is one half kilometer wide with a current flowing at $2\textrm{km}/\textrm{h}$ from East to West. A man can swim at $3\textrm{km}/\textrm{h}$ in
still water. In what direction should he swim in order to travel directly
across the river? What would the answer to this problem be if the river
flowed at $3\textrm{km}/\textrm{h}$ and the man could swim only at the rate of $2\textrm{km}/\textrm{h}$? \vspace{1mm}
\begin{sol}
Man:\ $3\begin{mymatrix}{rr}
 a & b
\end{mymatrix} $ 
Water:\ $\begin{mymatrix}{rr}
-2 & 0
\end{mymatrix} $
Then you need $3a=2$ and so $a=2/3$ and hence $b=\sqrt{5}/3$. The vector is
then $\begin{mymatrix}{cc}
 \frac{2}{3} & \frac{\sqrt{5}}{3}
\end{mymatrix} $. 

In the second case, he
could not do it. You would need to have a unit vector $\begin{mymatrix}{rr} a & b
\end{mymatrix} $
such that $2a=3$ which is not possible. 
\end{sol}
\end{ex}


\begin{ex} Three forces are applied to a point which does not move. Two of the
forces are $2 \vect{i}+2 \vect{j} -6 \vect{k}$ Newtons and $8 \vect{i}+ 8 \vect{j}+ 3 \vect{k}$ Newtons. Find
the third force. \vspace{1mm}
%\begin{sol}
%\end{sol}
\end{ex}

\begin{ex} The total force acting on an object is to be $4 \vect{i}+
 2 \vect{j} -3 \vect{k}$ Newtons. A force
of $-3 \vect{i} -1 \vect{j}+ 8
\vect{k}$ Newtons is being applied. What other force should be applied to
achieve the desired total force? \vspace{1mm}
%\begin{sol}
%\end{sol}
\end{ex}

\begin{ex} A bird flies from its nest $8$ km in the direction $\frac{5}{6}\pi $
north of east where it stops to rest on a tree. It then flies $1$ km in the
direction due southeast and lands atop a telephone pole. Place an $xy$
coordinate system so that the origin is the bird's nest, and the positive 
$x$ axis points east and the positive $y$ axis points north. Find the
displacement vector from the nest to the telephone pole. \vspace{1mm}
%\begin{sol}
%\end{sol}
\end{ex}


\begin{ex} If $\vect{F}$ is a force and $\vect{D}$ is a vector, show 
$\func{proj}_{\vect{D}}\tup{\vect{F}} =\tup{\norm{\vect{F}
} \cos \theta } \vect{u}$ where $\vect{u}$ is the unit
vector in the direction of $\vect{D}$, where $\vect{u}=\vect{D}/\norm{\vect{D}
} $ and $\theta $ is the included angle between the two vectors, 
$\vect{F}$ and $\vect{D}$. $\norm{\vect{F}} \cos \theta $
is sometimes called the component
\index{component of a force} of the force, $\vect{F}$ in the direction, $\vect{D}$.
\begin{sol}
$\func{proj}_{\vect{D}}\tup{\vect{F}} = \frac{\vect{F}\dotprod
\vect{D}}{\norm{\vect{D}} }\frac{\vect{D}}{\norm{\vect{D}
} }=\tup{\norm{\vect{F}} \cos \theta }
\frac{\vect{D}}{\norm{\vect{D}} }=\tup{\norm{
\vect{F}} \cos \theta } \vect{u}$
\end{sol}
\end{ex}

\end{enumialphparenastyle}
