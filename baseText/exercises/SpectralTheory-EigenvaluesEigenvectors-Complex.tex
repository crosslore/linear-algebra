\Opensolutionfile{solutions}[ex]
\section*{Exercises}

\begin{enumialphparenastyle}

\begin{ex}
  Find the (real or complex) eigenvalues and eigenvectors of the
  following matrices. Diagonalize each matrix if possible.
  \begin{equation*}
    A = \begin{mymatrix}{rr}
      2 & 1 \\
      -1 & 2 \\
    \end{mymatrix},
    \quad
    B = \begin{mymatrix}{rr}
      -1 & -2 \\
      1 &  1 \\
    \end{mymatrix},
    \quad
    C = \begin{mymatrix}{rrr}
      2   & -1  &  3  \\
      0   &  1  &  1  \\
      -1  &  1  &  0  \\
    \end{mymatrix}
    \quad
    D = \begin{mymatrix}{rrr}
      1  &  1 & 1 \\
      -1 &  2 & 0 \\
      1  & -1 & 1 \\
    \end{mymatrix}
  \end{equation*}
  \begin{sol}
    \begin{enumerate}
    \item The characteristic polynomial is
      $\eigenvar^2 -4\eigenvar + 5$, the eigenvalues are
      $\eigenvar_1 = 2+i$ and $\eigenvar_2 = 2-i$, and the
      corresponding basic eigenvectors are
      $\vect{v}_1 = \mat{1, i\,}^T$ and
      $\vect{v}_2 = \mat{1, -i\,}^T$, respectively. The matrix $A$ is
      diagonalizable as $A=PDP^{-1}$, where
      \begin{equation*}
        P = \begin{mymatrix}{cc}
          1 & 1 \\
          i & -i \\
        \end{mymatrix}
        \quad\mbox{and}\quad
        D = \begin{mymatrix}{cc}
          2+i & 0 \\
          0 & 2-i \\
        \end{mymatrix}.
      \end{equation*}
    \item The characteristic polynomial is $\eigenvar^2 + 1$, the
      eigenvalues are $\eigenvar_1=i$ and $\eigenvar_2=-i$, and the
      corresponding basic eigenvectors are
      $\vect{v}_1 = \mat{1-i, -1}^T$ and
      $\vect{v}_2 = \mat{1+i, -1}^T$, respectively. Therefore
      $B=PDP^{-1}$, where
      \begin{equation*}
        P = \begin{mymatrix}{cc}
          1-i & 1+i \\
          -1 & -1 \\
        \end{mymatrix}
        \quad\mbox{and}\quad
        D = \begin{mymatrix}{cc}
          i & 0 \\
          0 & -i \\
        \end{mymatrix}.
      \end{equation*}
    \item The characteristic polynomial is
      $x^2 -2x + 2 - (x^3 -2x^2 + 2x)$
      $-\eigenvar^3 - \eigenvar^2 + 2 =
      (1-\eigenvar)(\eigenvar^2-2\eigenvar+2)$.  The eigenvalues are
      $\eigenvar_1 = 1$, $\eigenvar_2 = 1+i$, and $\eigenvar_3 =
      1-i$. The corresponding basic eigenvectors are
      $\vect{v}_1 = \mat{1, 1, 0}^T$,
      $\vect{v}_2 = \mat{-1-2i, -i, 1}^T$,
      and $\vect{v}_2 = \mat{1+2i, i, 1}^T$,
      respectively. Therefore
      $C=PDP^{-1}$, where
      \begin{equation*}
        P = \begin{mymatrix}{ccc}
          1 & -1-2i & 1+2i \\
          1 &  -i   &  i   \\
          0 &   1   &  1   \\
        \end{mymatrix}
        \quad\mbox{and}\quad
        D = \begin{mymatrix}{ccc}
          1 &  0  &  0  \\
          0 & 1+i &  0  \\
          0 &  0  & 1-i \\
        \end{mymatrix}.
      \end{equation*}
    \item The characteristic polynomial is
      $-\eigenvar^3 +4\eigenvar^2 -5\eigenvar + 2 =
      -(\eigenvar-1)(\eigenvar-1)(\eigenvar-2)$. Therefore the
      eigenvalues are $\eigenvar=1$ with algebraic multiplicity 2 and
      $\eigenvar=2$ with algebraic multiplicity 1. The eigenspace for
      $\eigenvar=1$ is 1-dimensional; it is spanned by
      $\mat{1,1,-1}^T$. Since the geometric multiplicity of the
      eigenvalue $1$ is less than its algebraic multiplicity, the
      matrix is not diagonalizable, not even over the complex numbers.
    \end{enumerate}
  \end{sol}
\end{ex}

\end{enumialphparenastyle}
