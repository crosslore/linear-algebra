
\begin{ex}
  Find $\cos \theta$ where $\theta$ is the angle between the vectors
  \begin{equation*}
    \vect{u}
    =
    \begin{mymatrix}{r}
      3 \\
      -1 \\
      -1
    \end{mymatrix},
    \vect{v}
    =
    \begin{mymatrix}{r}
      1 \\
      4 \\
      2
    \end{mymatrix}
  \end{equation*}
  \begin{sol}
    $\displaystyle\cos\theta=\frac{\begin{mymatrix}{ccc}
        3 & -1 & -1
      \end{mymatrix}^T \dotprod
      \begin{mymatrix}{ccc}
        1 & 4 & 2
      \end{mymatrix}^T}
    {\sqrt{9+1+1}\sqrt{1+16+4}}= \frac{-3}{\sqrt{11}\sqrt{21}}$.
  \end{sol}
\end{ex}

\begin{ex}
  Find $\cos \theta$ where $\theta$ is the angle between the vectors
  \begin{equation*}
    \vect{u}
    =
    \begin{mymatrix}{r}
      1 \\
      -2 \\
      1
    \end{mymatrix},
    \vect{v}
    =
    \begin{mymatrix}{r}
      1 \\
      2 \\
      -7
    \end{mymatrix}
  \end{equation*}
  \begin{sol}
    $\displaystyle\cos\theta =
    \frac{-10}{\sqrt{1+4+1}\sqrt{1+4+49}}$.
  \end{sol}
\end{ex}

\begin{ex}
  Use the formula given in Proposition~\ref{prop:dot-product-angle} to
  verify the Cauchy Schwarz inequality and to show that equality
  occurs if and only if one of the vectors is a scalar multiple of the
  other.
  \begin{sol}
    This formula says that
    $\vect{u}\dotprod \vect{v}=\norm{\vect{u}} \norm{\vect{v}} \cos
    \theta$ where $ \theta $ is the included angle between the two
    vectors. Thus
    \begin{equation*}
      \norm{\vect{u}\dotprod \vect{v}} =\norm{\vect{u}}
      \norm{\vect{v}} \norm{\cos \theta} \leq
      \norm{\vect{u}} \norm{\vect{v}}
    \end{equation*}
    and equality holds if and only if $\theta =0$ or $\pi$. This means
    that the two vectors either point in the same direction or
    opposite directions. Hence one is a multiple of the other.
  \end{sol}
\end{ex}
