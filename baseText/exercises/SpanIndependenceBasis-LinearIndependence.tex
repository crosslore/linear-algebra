\section*{Exercises}

\begin{ex}
  Which of the following vectors are redundant? If there are redundant
  vectors, write each of them as a linear combination of previous
  vectors.
  \begin{equation*}
    \vect{u}_1 = \begin{mymatrix}{r} 1 \\ 0 \\ 1 \end{mymatrix},\quad
    \vect{u}_2 = \begin{mymatrix}{r} 2 \\ 0 \\ 2 \end{mymatrix},\quad
    \vect{u}_3 = \begin{mymatrix}{r} 1 \\ 2 \\ 1 \end{mymatrix},\quad
    \vect{u}_4 = \begin{mymatrix}{r} 1 \\ 6 \\ 1 \end{mymatrix}.
  \end{equation*}
  \begin{sol}
    $\vect{u}_2$ and $\vect{u}_4$ are redundant. We have
    $\vect{u}_2=2\vect{u}_1$ and $\vect{u}_4=3\vect{u}_3-2\vect{u}_1$.
  \end{sol}
\end{ex}

\begin{ex}
  Which of the following vectors are redundant? If there are redundant
  vectors, write each of them as a linear combination of previous
  vectors.
  \begin{equation*}
    \vect{u}_1 = \begin{mymatrix}{r} -1 \\ -2 \\ 2 \\ 3 \end{mymatrix},\quad
    \vect{u}_2 = \begin{mymatrix}{r} -3 \\ -4 \\ 3 \\ 3 \end{mymatrix},\quad
    \vect{u}_3 = \begin{mymatrix}{r} 0 \\ -1 \\ 4 \\ 3 \end{mymatrix},\quad
    \vect{u}_4 = \begin{mymatrix}{r} 0 \\  2 \\ -3 \\ -6 \end{mymatrix}.
  \end{equation*}
  % \begin{sol}
  % \end{sol}
\end{ex}

\begin{ex}
  Use the method of
  Theorem~\ref{thm:characterization-linear-independence} to determine
  whether the following vectors are linearly independent. If they are
  linearly dependent, find a non-trivial linear combination of the
  vectors that is equal to $\vect{0}$.
  \begin{equation*}
    \vect{u}_1 = \begin{mymatrix}{r} 2 \\ 3 \\ 2 \end{mymatrix},\quad
    \vect{u}_2 = \begin{mymatrix}{r} -1 \\ 0 \\ 2 \end{mymatrix},\quad
    \vect{u}_3 = \begin{mymatrix}{r} -3 \\ -4 \\ -2 \end{mymatrix},\quad
    \vect{u}_4 = \begin{mymatrix}{r} 5 \\ 6 \\ 2 \end{mymatrix}.
  \end{equation*}
  \begin{sol}
    The vectors are linearly dependent. We have
    $\vect{u}_2+3\vect{u}_3+2\vect{u}_4=\vect{0}$.
  \end{sol}
\end{ex}

\begin{ex}
  Use the method of
  Theorem~\ref{thm:characterization-linear-independence} to determine
  whether the following vectors are linearly independent. If they are
  linearly dependent, find a non-trivial linear combination of the
  vectors that is equal to $\vect{0}$.
  \begin{equation*}
    \vect{u}_1 = \begin{mymatrix}{r} 1 \\ -1 \\ 0 \\ 1 \end{mymatrix},\quad
    \vect{u}_2 = \begin{mymatrix}{r} 1 \\ 6 \\ 7 \\ 1 \end{mymatrix},\quad
    \vect{u}_3 = \begin{mymatrix}{r} 3 \\ 5 \\ 8 \\ 3 \end{mymatrix},\quad
    \vect{u}_4 = \begin{mymatrix}{r} 1 \\ 0 \\ 1 \\ 1 \end{mymatrix}.
  \end{equation*}
  % \begin{sol}
  % \end{sol}
\end{ex}

\begin{ex}
  Are the following vectors linearly independent? If not, write one of
  them as a linear combination of the others.
  \begin{equation*}
    \vect{u} = \begin{mymatrix}{r} 1 \\ 3 \\  1 \end{mymatrix},\quad
    \vect{v} = \begin{mymatrix}{r} 1 \\ 4 \\  2 \end{mymatrix},\quad
    \vect{w} = \begin{mymatrix}{r} 1 \\ 1 \\ -1 \end{mymatrix}.
  \end{equation*}
  \begin{sol}
    The vectors are linearly dependent. We have
    $\vect{w} = 3\vect{u}-2\vect{v}$.
  \end{sol}
\end{ex}

\begin{ex}
  Find a linearly independent set of vectors that has the same span as
  the given vectors.
  \begin{equation*}
    \vect{u}_1 = \begin{mymatrix}{r} 2 \\  0 \\  3 \end{mymatrix},\quad
    \vect{u}_2 = \begin{mymatrix}{r} 1 \\  3 \\  5 \end{mymatrix},\quad
    \vect{u}_3 = \begin{mymatrix}{r} 3 \\  3 \\  8 \end{mymatrix},\quad
    \vect{u}_4 = \begin{mymatrix}{r} 3 \\ -3 \\  1 \end{mymatrix}.
  \end{equation*}
  \begin{sol}
    $
    \begin{mymatrix}{rrrr}
      2 & 1 & 3 & 3 \\
      0 & 3 & 3 & -3 \\
      3 & 5 & 8 & 1 \\
    \end{mymatrix}
    \roweq
    \begin{mymatrix}{rrrr}
      1 & 4 & 5 & -2 \\
      0 & 1 & 1 & -1 \\
      0 & 0 & 0 &  0 \\
    \end{mymatrix}
    $.
    Linearly independent subset: $\set{\vect{u}_1,\vect{u}_2}$.
  \end{sol}
\end{ex}

\begin{ex}
  Find a linearly independent set of vectors that has the same span as
  the given vectors.
  \begin{equation*}
    \vect{u}_1 = \begin{mymatrix}{r} 1 \\ 3 \\  3 \\ 1 \end{mymatrix},\quad
    \vect{u}_2 = \begin{mymatrix}{r} 2 \\ 6 \\  6 \\ 2 \end{mymatrix},\quad
    \vect{u}_3 = \begin{mymatrix}{r} 1 \\ 0 \\ -3 \\ 1 \end{mymatrix},\quad
    \vect{u}_4 = \begin{mymatrix}{r} 1 \\ 2 \\  1 \\ 1 \end{mymatrix}.
  \end{equation*}
  % \begin{sol}
  % \end{sol}
\end{ex}

\begin{ex}
  Here are some vectors in $\R^{4}$.
  \begin{equation*}
    \vect{u}_1 = \begin{mymatrix}{r} 1 \\ 1 \\ -1 \\ 1 \end{mymatrix},\quad
    \vect{u}_2 = \begin{mymatrix}{r} 1 \\ 2 \\ -1 \\ 1 \end{mymatrix},\quad
    \vect{u}_3 = \begin{mymatrix}{r} 1 \\ -2 \\ -1 \\ 1 \end{mymatrix},\quad
    \vect{u}_4 = \begin{mymatrix}{r} 1 \\ 2 \\ 0 \\ 1 \end{mymatrix},\quad
    \vect{u}_5 = \begin{mymatrix}{r} 1 \\ -1 \\ -1 \\ 1 \end{mymatrix}.
  \end{equation*}
  Explain why these vectors can't possibly be linearly
  independent. Then obtain a linearly independent subset of these
  vectors that has the same span as these vectors.
  % \begin{sol}
  % \end{sol}
\end{ex}

\begin{ex}
  Here are some vectors in $\R^{4}$.
  \begin{equation*}
    \vect{u}_1 = \begin{mymatrix}{r} 1 \\ -1 \\ -1 \\ 1 \end{mymatrix},\quad
    \vect{u}_2 = \begin{mymatrix}{r} -3 \\ 3 \\ 3 \\ -3 \end{mymatrix},\quad
    \vect{u}_3 = \begin{mymatrix}{r} 1 \\ 0 \\ -1 \\ 1 \end{mymatrix},\quad
    \vect{u}_4 = \begin{mymatrix}{r} 2 \\ -9 \\ -2 \\ 2 \end{mymatrix},\quad
    \vect{u}_5 = \begin{mymatrix}{r} 1 \\ 0 \\ 0 \\ 1 \end{mymatrix}.
  \end{equation*}
  Explain why these vectors can't possibly be linearly
  independent. Then find a non-trivial linear combination of these
  vectors that equals $\vect{0}$.
  % \begin{sol}
  % \end{sol}
\end{ex}

\begin{ex}
  Here are some vectors.
  \begin{equation*}
    \vect{u}_1 = \begin{mymatrix}{r} 1 \\ 1 \\ -2 \end{mymatrix},\quad
    \vect{u}_2 = \begin{mymatrix}{r} 2 \\ 2 \\ -4 \end{mymatrix},\quad
    \vect{u}_3 = \begin{mymatrix}{r} 2 \\ 7 \\ -4 \end{mymatrix},\quad
    \vect{u}_4 = \begin{mymatrix}{r} 5 \\ 7 \\ -10 \end{mymatrix},\quad
    \vect{u}_5 = \begin{mymatrix}{r} 12 \\ 17 \\ -24 \end{mymatrix}.
  \end{equation*}
  Describe the span of these vectors as the span of as few vectors as
  possible.
  \begin{sol}
    \begin{equation*}
      \begin{mymatrix}{rrrrr}
        1 & 2 & 2 & 5 & 12 \\
        1 & 2 & 7 & 7 & 17 \\
        -2 & -4 & -4 & -10 & -24 \\
      \end{mymatrix}
      \roweq
      \begin{mymatrix}{rrrrr}
        1 & 1 & 2 & 5 & 12 \\
        0 & 0 & 5 & 2 & 5 \\
        0 & 0 & 0 & 0 & 0 \\
      \end{mymatrix}.
    \end{equation*}
    Linearly independent subset: $\set{\vect{u}_1,\vect{u}_3}$. Since
    the rank is $2$, this is the smallest possible.
  \end{sol}
\end{ex}

\begin{ex}
  Here are some vectors.
  \begin{equation*}
    \vect{u}_1 = \begin{mymatrix}{r} 1 \\ 2 \\ -2 \end{mymatrix},\quad
    \vect{u}_2 = \begin{mymatrix}{r} 1 \\ 3 \\ -2 \end{mymatrix},\quad
    \vect{u}_3 = \begin{mymatrix}{r} 1 \\ -2 \\ -2 \end{mymatrix},\quad
    \vect{u}_4 = \begin{mymatrix}{r} -1 \\ 0 \\ 2 \end{mymatrix},\quad
    \vect{u}_5 = \begin{mymatrix}{r} 1 \\ 3 \\ -1 \end{mymatrix}.
  \end{equation*}
  Describe the span of these vectors as the span of as few vectors as
  possible.
  % \begin{sol}
  % \end{sol}
\end{ex}

\begin{ex}
  In this exercise, we use scalars from the field $\Z_3$ of integers
  modulo $3$ instead of real numbers (see Section~\ref{sec:fields},
  ``Fields'').  Use the extended casting-out algorithm to determine
  which of the following vectors are redundant. If there are redundant
  vectors, write each of them as a linear combination of previous
  vectors.
  \begin{equation*}
    \vect{u}_1 = \begin{mymatrix}{r} 2 \\ 1 \\ 0 \end{mymatrix},\quad
    \vect{u}_2 = \begin{mymatrix}{r} 1 \\ 1 \\ 2 \end{mymatrix},\quad
    \vect{u}_3 = \begin{mymatrix}{r} 2 \\ 0 \\ 2 \end{mymatrix},\quad
    \vect{u}_4 = \begin{mymatrix}{r} 0 \\ 1 \\ 1 \end{mymatrix}.
  \end{equation*}
  \begin{sol}
    We write the vectors as the columns of a matrix and reduce to
    {\rref} over $\Z_3$:
    \begin{equation*}
      \begin{mymatrix}{rrrr}
        2 & 1 & 2 & 0 \\
        1 & 1 & 0 & 1 \\
        0 & 2 & 2 & 1 \\
      \end{mymatrix}
      \quad\roweq\ldots\roweq\quad
      \begin{mymatrix}{rrrr}
        1 & 0 & 2 & 2 \\
        0 & 1 & 1 & 2 \\
        0 & 0 & 0 & 0 \\
      \end{mymatrix}.
    \end{equation*}
    $\vect{u}_3$ and $\vect{u}_4$ are redundant. We have
    $\vect{u}_3=2\vect{u}_1+1\vect{u}_2$ and
    $\vect{u}_4=2\vect{u}_1+2\vect{u}_2$.
  \end{sol}
\end{ex}

\begin{ex}
  Let $\vect{u},\vect{v},\vect{w}$ be linearly independent vectors in
  $\R^n$. Are the vectors $\vect{u}+\vect{v}$, $2\vect{u}+\vect{w}$,
  and $\vect{w}-2\vect{v}$ linearly independent?
  \begin{sol}
    From
    $a(\vect{u}+\vect{v}) + b(2\vect{u}+\vect{w}) +
    c(\vect{w}-2\vect{v})=\vect{0}$ we get
    $(a+2b)\vect{u} + (a-2c)\vect{v} + (b+c)\vect{w}=\vect{0}$.  Since
    $\vect{u},\vect{v},\vect{w}$ are linearly independent, this last
    system has only the trivial solution, so $a+2b=0$, $a-2c=0$, and
    $b+c=0$. However, these three equations have non-trivial
    solutions, for example $(a,b,c)=(2,-1,1)$. So the vectors
    $\vect{u}+\vect{v}$, $2\vect{u}+\vect{w}$, and
    $\vect{w}-2\vect{v}$ are linearly dependent.
  \end{sol}
\end{ex}

\begin{ex}
  Let $\vect{u},\vect{v},\vect{w}$ be linearly independent vectors in
  $\R^n$. Are the vectors $\vect{u}+\vect{v}$, $\vect{u}+\vect{w}$,
  and $\vect{w}+\vect{v}$ linearly independent?
  \begin{sol}
    From
    $a(\vect{u}+\vect{v}) + b(\vect{u}+\vect{w}) +
    c(\vect{w}+\vect{v})=\vect{0}$ we get
    $(a+b)\vect{u} + (a+c)\vect{v} + (b+c)\vect{w}=\vect{0}$.  Since
    $\vect{u},\vect{v},\vect{w}$ are linearly independent, this last
    system has only the trivial solution, so $a+b=0$, $a+c=0$, and
    $b+c=0$. Solving, we find the unique solution
    $(a,b,c)=(0,0,0)$. So the vectors $\vect{u}+\vect{v}$,
    $\vect{u}+\vect{w}$, and $\vect{w}+\vect{v}$ are linearly
    independent.
  \end{sol}
\end{ex}

\begin{ex}
  Suppose $A$ is an $m\times n$-matrix and
  $\set{\vect{w} _{1},\ldots,\vect{w}_{k}}$ is a linearly independent
  set of vectors in $\R^{m}$. Now suppose
  $A\vect{z}_{i}=\vect{w}_{i}$. Show
  $\set{ \vect{z}_{1},\ldots,\vect{z}_{k}}$ is also linearly
  independent.
  \begin{sol}
    Assume
    $a_{1}\vect{z}_{1}+\ldots+a_{k}\vect{z}_{k}=\vect{0}$. Multiplying
    both sides of the equation by $A$, we get
    \begin{equation*}
      a_{1}A\vect{z}_{1}+\ldots+a_{k}A\vect{z}_{k}=
      a_{1}\vect{w}_{1}+\ldots+a_{k}\vect{w}_{k}=\vect{0}.
    \end{equation*}
    Since the $\vect{w}_{i}$ are linearly independent, it follows that
    each $a_{i}=0$. Therefore the $\vect{z}_{i}$ are linearly
    independent as well.
  \end{sol}
\end{ex}

