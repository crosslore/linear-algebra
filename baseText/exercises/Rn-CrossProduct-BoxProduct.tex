\begin{enumialphparenastyle}

\begin{ex} Find the volume of the parallelepiped determined by the vectors
$\begin{mymatrix}{r}
1 \\
-7 \\
-5
\end{mymatrix} $, \\
 $\begin{mymatrix}{r}
1 \\
-2 \\
-6
\end{mymatrix}$, and $\begin{mymatrix}{r}
3 \\
2 \\
3
\end{mymatrix}$.
\begin{sol}
 $\left\vert
\begin{array}{rrr}
1 & -7 & -5 \\
1 & -2 & -6 \\
3 & 2 & 3
\end{array}
\right\vert = 113$
\end{sol}
\end{ex}

\begin{ex} Suppose $\vect{u},\vect{v}$, and $\vect{w}$ are three vectors whose
components are all integers. Can you conclude the volume of the
parallelepiped determined from these three vectors will always be an integer?
\begin{sol}
Yes. It will involve the sum of product of integers and so it will
be an integer.
\end{sol}
\end{ex}

\begin{ex} \label{exerboxproductzero} What does it mean geometrically if the box
product of three vectors gives zero?
\begin{sol}
It means that if you place them so that
they all have their tails at the same point, the three will lie in the same
plane.
\end{sol}
\end{ex}

\begin{ex} Using Problem \ref{exerboxproductzero}, find an equation of a plane
containing the two position vectors, $\vect{p}$ and $\vect{q}$ and the
point $0$. 
\textbf{Hint: }If $\left( x,y,z\right) $ is a point on
this plane, the volume of the parallelepiped determined by $\left(
x,y,z\right) $ and the vectors $\vect{p},\vect{q}$ equals 0.
\begin{sol}
$\vect{x}\dotprod \left( \vect{a}\times \vect{b}\right) =0$
\end{sol}
\end{ex}

\begin{ex} Find the normal vector to the plane going through the points $P=(1,2,3)$, $Q=(-2,1,8)$ and $R=(2,2,2)$.
%\begin{sol}
%\end{sol}
\end{ex}

\begin{ex} Using the notion of the box product yielding either plus or minus the
volume of the parallelepiped determined by the given three vectors, show
that
\begin{equation*}
\left( \vect{u}\times \vect{v}\right) \dotprod \vect{w}=\vect{u}\dotprod \left(
\vect{v}\times \vect{w}\right)
\end{equation*}
In other words, the dot and the cross can be switched as long as the order
of the vectors remains the same. \textbf{Hint:\ }There are two ways to do
this, by the coordinate description of the dot and cross product and by
geometric reasoning. 
%\begin{sol}
%\end{sol}
\end{ex}

\begin{ex} Simplify $\left( \vect{u}\times \vect{v}\right) \dotprod \left( \left( 
\vect{v}\times \vect{w}\right) \times \left( \vect{w}\times \vect{z}\right)\right) .$
\begin{sol}
Here $\left[ \vect{v},\vect{w},\vect{z}\right]$ denotes the box product. Consider the cross product term. From the above,
\begin{eqnarray*}
\left( \vect{v}\times \vect{w}\right) \times \left( \vect{w}\times \vect{z}\right) &=& 
\left[ \vect{v},\vect{w},\vect{z}\right] \vect{w}-\left[ \vect{w},\vect{w},\vect{z}\right] \vect{v} \\
&=&\left[ \vect{v},\vect{w},\vect{z}\right] \vect{w}
\end{eqnarray*}
Thus it reduces to
\[
\left( \vect{u}\times \vect{v}\right) \dotprod \left[ \vect{v},\vect{w},\vect{z}\right] \vect{w}=\left[ \vect{v},\vect{w},\vect{z}\right] \left[ \vect{u},\vect{v},\vect{w}\right]
\]
\end{sol}
\end{ex}

\begin{ex} Simplify $\vectlength \vect{u}\times \vect{v}\vectlength ^{2}+\left( 
\vect{u}\dotprod \vect{v}\right) ^{2}-\vectlength \vect{u}\vectlength ^{2}\vectlength
\vect{v}\vectlength ^{2}.$
\begin{sol}
\begin{eqnarray*}
\vectlength \vect{u}\times \vect{v}\vectlength ^{2} &=&\varepsilon
_{ijk}u_{j}v_{k}\varepsilon _{irs}u_{r}v_{s}=\left( \delta _{jr}\delta
_{ks}-\delta _{kr}\delta _{js}\right) u_{r}v_{s}u_{j}v_{k} \\
&=&u_{j}v_{k}u_{j}v_{k}-u_{k}v_{j}u_{j}v_{k}=\vectlength \vect{u}
\vectlength ^{2}\vectlength \vect{v}\vectlength ^{2}-\left( \vect{u}\dotprod \vect{v}\right) ^{2}
\end{eqnarray*}
It follows that the expression reduces to $0$. You can also do the following.
\begin{eqnarray*}
\vectlength \vect{u}\times \vect{v}\vectlength ^{2} &=&\vectlength \vect{u}
\vectlength ^{2}\vectlength \vect{v}\vectlength ^{2}\sin ^{2}\theta \\
&=&\vectlength \vect{u}\vectlength ^{2}\vectlength \vect{v}\vectlength
^{2}\left( 1-\cos ^{2}\theta \right) \\
&=&\vectlength \vect{u}\vectlength ^{2}\vectlength \vect{v}\vectlength
^{2}-\vectlength \vect{u}\vectlength ^{2}\vectlength \vect{v}\vectlength
^{2}\cos ^{2}\theta \\
&=&\vectlength \vect{u}\vectlength ^{2}\vectlength \vect{v}\vectlength
^{2}-\left( \vect{u}\dotprod \vect{v}\right) ^{2}
\end{eqnarray*}
which implies the expression equals $0$.
\end{sol}
\end{ex}

\begin{ex} For $\vect{u},\vect{v},\vect{w}$ functions of $t,$ prove the following product rules:
\begin{eqnarray*}
\left( \vect{u}\times \vect{v}\right) ^{\prime } &=&\vect{u}^{\prime }\times
\vect{v}+\vect{u}\times \vect{v}^{\prime } \\
\left( \vect{u}\dotprod \vect{v}\right) ^{\prime } &=&\vect{u}^{\prime }\dotprod
\vect{v}+\vect{u}\dotprod \vect{v}^{\prime }
\end{eqnarray*}
\begin{sol}
We will show it using the summation convention and permutation symbol.
\begin{eqnarray*}
\left( \left( \vect{u}\times \vect{v}\right) ^{\prime }\right) _{i} &=
&\left( \left( \vect{u}\times \vect{v}\right) _{i}\right) ^{\prime }=\left(
\varepsilon _{ijk}u_{j}v_{k}\right) ^{\prime } \\
&=&\varepsilon _{ijk}u_{j}^{\prime }v_{k}+\varepsilon
_{ijk}u_{k}v_{k}^{\prime }=\left( \vect{u}^{\prime } \times
\vect{v}+\vect{u}\times \vect{v}^{\prime }\right) _{i}
\end{eqnarray*}
and so $\left( \vect{u}\times \vect{v}\right) ^{\prime }=\vect{u}^{\prime }
\times \vect{v}+\vect{u}\times \vect{v}^{\prime }.$ 
\end{sol}
\end{ex}

\end{enumialphparenastyle}