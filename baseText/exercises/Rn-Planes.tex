\section*{Exercises}

\begin{enumialphparenastyle}

% ----------------------------------------------------------------------

\begin{ex}
  Find vector and parametric equations for the plane through the
  points $P = (0,1,1)$, $Q = (-1,2,1)$, and $R = (1,1,2)$.
\end{ex}

\begin{ex}
  Consider the following vector equation for a plane in $\R^4$:
  \begin{equation*}
    \begin{mymatrix}{c} x\\y\\z\\w \end{mymatrix}
    = \begin{mymatrix}{r} 1\\2\\0\\0 \end{mymatrix}
    + t\,\begin{mymatrix}{r} 1\\0\\0\\1 \end{mymatrix}
    + s\,\begin{mymatrix}{r} -1\\-1\\1\\0 \end{mymatrix}.
  \end{equation*}
  Find a new vector equation for the same plane by doing the change of
  parameters%
  \index{plane!change of parameters}\index{change of parameters!plane}
  $t=1-r_1$, $s=r_1+r_2$.
  \begin{sol}
    We have
    \begin{equation*}
      \begin{mymatrix}{c} x\\y\\z\\w \end{mymatrix}
      = \begin{mymatrix}{r} 1\\2\\0\\0 \end{mymatrix}
      + (1-r_1)\,\begin{mymatrix}{r} 1\\0\\0\\1 \end{mymatrix}
      + (r_1+r_2)\,\begin{mymatrix}{r} -1\\-1\\1\\0 \end{mymatrix}
      = \begin{mymatrix}{r} 2\\2\\0\\1 \end{mymatrix}
      + r_1\,\begin{mymatrix}{r} -2\\-1\\1\\-1 \end{mymatrix}
      + r_2\,\begin{mymatrix}{r} -1\\-1\\1\\0 \end{mymatrix}.
    \end{equation*}
  \end{sol}
\end{ex}

\begin{ex}
  Determine which of the following points lie on the plane through the
  points $P = (2,6,1)$, $R = (1,4,1)$, and $Q = (1,2,-1)$.
  \begin{enumerate}
  \item $S_1=(1,2,4)$.
  \item $S_2=(1,5,2)$.
  \item $S_3=(0,0,0)$.
  \end{enumerate}
\end{ex}

\begin{ex}
  Use cross products to find the normal vector to the plane going
  through the points $P=(1,2,3)$, $Q=(-2,1,8)$ and $R=(2,2,2)$.
\end{ex}

\begin{ex}
  Find normal and general equations of the plane through the point
  $P=(1,1,2)$ and orthogonal to $\vect{n}=\mat{1,0,-1}^T$.
\end{ex}

\begin{ex}
  Find normal and general equations for the plane through the points
  $P = (0,1,3)$, $Q=(2,-1,0)$, and $R=(1,2,2)$.
\end{ex}

\begin{ex}
  Find a vector equation for the plane $2x+y-z=1$.
\end{ex}

\begin{ex}
  The chapter mentions that the normal equation and general equation
  of a plane only work in $\R^3$, and not in general $\R^n$. Why does
  the equation $ax+by+cz+dw=e$ not describe a plane in $\R^4$?
  \begin{sol}
    The general solution of $ax+by+cz+dw=e$ involves at least three
    parameters.  But the vector equation of a plane only has two
    parameters, therefore $ax+by+cz+dw=e$ does not describe a
    plane. (It describes a 3-dimensional so-called {\em
      hyperplane}\index{hyperplane}\index{plane!hyperplane} inside
    $\R^4$).
  \end{sol}
\end{ex}

\begin{ex}
  Find the intersection between the planes $x+3y+4z=3$ and $2x+5y-z=2$.
  Is the intersection a line, a plane, or empty?
\end{ex}

\begin{ex}
Find the intersection of the line
  \begin{equation*}
    \begin{mymatrix}{r} x \\ y \\ z \end{mymatrix}
    = \begin{mymatrix}{r} 1 \\ 1 \\ 1 \end{mymatrix}
    + t \begin{mymatrix}{r} 2 \\ -1 \\ 2 \end{mymatrix}
  \end{equation*}
  and the plane $x+3y+z = 6$.
  Is the intersection a point, a line, or empty?
\end{ex}

\begin{ex}
  Find the angle between the planes $x+y=5$ and $2x+y-z=4$.
\end{ex}

\begin{ex}
  Find the angle between the line
  \begin{equation*}
    \begin{mymatrix}{r} x \\ y \\ z \end{mymatrix}
    = \begin{mymatrix}{r} 0 \\ 3 \\ 7 \end{mymatrix}
    + t \begin{mymatrix}{r} 1 \\ 1 \\ 4 \end{mymatrix}
  \end{equation*}
  and the plane $4x+7y+4z = 6$.
\end{ex}

\begin{ex}
  In Example~\ref{exa:angle-line-plane}, we calculated the angle
  $\theta$ between a line and a plane by calculating the angle $\phi$
  between the direction vector of the line and the normal vector of
  the plane according to the formula
  \begin{equation*}
    \cos\phi =
    \frac{\vect{n}\dotprod\vect{d}}{\norm{\vect{n}}\norm{\vect{d}}}
  \end{equation*}
  and then taking $\theta = \frac{\pi}{2}-\phi$.
  \begin{enumerate}
  \item Explain what happens when the dot product is negative. How
    should we adjust the formula to ensure that $\theta$ is always
    between $0$ and $\frac{\pi}{2}$?
  \item Show that one can get the answer in a single step with the
    formula
    \begin{equation*}
      \sin\theta =
      \frac{\abs{\vect{n}\dotprod\vect{d}}}{\norm{\vect{n}}\norm{\vect{d}}}.
    \end{equation*}
  \end{enumerate}
  \begin{sol}
    (a) If the dot product is negative, $\phi$ will be greater than
    $\frac{\pi}{2}$, and therefore $\theta$ will end up being
    negative. We can fix this, similarly to
    Exercise~\ref{ex:angle-lines}, by taking the absolute value of the
    dot product, i.e., by solving
    \begin{equation}\label{eqn:ex-angle-line-plane}
      \cos\phi =
      \frac{\abs{\vect{n}\dotprod\vect{d}}}{\norm{\vect{n}}\norm{\vect{d}}}
    \end{equation}
    (b) From trigonometry, we know that $\sin\theta =
    \sin(\frac{\pi}{2}-\phi) = \cos\phi$. Together with
    {\eqref{eqn:ex-angle-line-plane}}, this gives the desired formula.
  \end{sol}
\end{ex}

\begin{ex}
  Find the shortest distance from the point $P = (1,1,-1)$ to the plane
  given by $x + 2y + 2z = 6$, and find the point $Q$ on the plane
  that is closest to $P$.
\end{ex}

\begin{ex}
  Use Exercise~\ref{exer-box-product-zero} to find an equation of a
  plane containing the two vectors $\vect{p}$ and $\vect{q}$ and the
  point $0$. \textbf{Hint:} If $(x,y,z)$ is a point in this
  plane, the volume of the parallelepiped determined by $(x,y,z)$
  and the vectors $\vect{p}$, $\vect{q}$ equals 0.
  \begin{sol}
    $\vect{x}\dotprod (\vect{a}\times \vect{b}) =0$.
  \end{sol}
\end{ex}

\end{enumialphparenastyle}
