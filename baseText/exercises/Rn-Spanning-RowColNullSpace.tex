\section*{Exercises}


\begin{ex}
  Determine the rank and nullity and find a basis of the column
  space, row space, and null space of each of the following matrices.
  \begin{enumerate}
  \item
    \begin{equation*}
      A = \begin{mymatrix}{rrrrrr}
        1 & 3 & 2 \\ 
        3 & 9 & 6 \\ 
        1 & 3 & 2 \\
      \end{mymatrix} 
    \end{equation*}
  \item
    \begin{equation*}
      B = \begin{mymatrix}{rrrrrr}
        1 & 3 & 0 & 2 \\
        3 & 9 & 1 & 7 \\
        1 & 3 & 1 & 3 \\
      \end{mymatrix} 
    \end{equation*}
  \item
    \begin{equation*}
      C = \begin{mymatrix}{rrrrrr}
        1 & 0 & 3 \\
        3 & 1 & 10 \\
        1 & 1 & 4 \\
        1 & -1 & 2 \\
      \end{mymatrix}
    \end{equation*}
  \item
    \begin{equation*}
      D = \begin{mymatrix}{rrrrr}
        0 & 0 & -1 & 0 & 1 \\ 
        1 & 2 & 3 & -2 & -18 \\ 
        1 & 2 & 2 & -1 & -11 \\ 
        -1 & -2 & -2 & 1 & 11
      \end{mymatrix}
    \end{equation*}
  \item
    \begin{equation*}
      E = \begin{mymatrix}{rrrr}
        1 & 0 & 3 & 0 \\ 
        3 & 1 & 10 & 0 \\ 
        -1 & 1 & -2 & 1 \\ 
        1 & -1 & 2 & -2
      \end{mymatrix}
    \end{equation*}
  \end{enumerate}
\end{ex}

\begin{ex}
  Find $\nullspace(A)$ for the following matrices. 
  \begin{enumerate}
  \item 
    \begin{equation*}
      A = \begin{mymatrix}{rr}
        2 & 3 \\
        4 & 6 
      \end{mymatrix}
    \end{equation*}
  \item
    \begin{equation*}
      A = \begin{mymatrix}{rrr}
        1 & 0 & -1 \\
        -1 & 1 & 3 \\
        3 & 2 & 1 
      \end{mymatrix}
    \end{equation*}
  \item 
    \begin{equation*}
      A = \begin{mymatrix}{rrr}
        2 & 4 & 0 \\
        3 & 6 & -2 \\
        1 & 2 & -2
      \end{mymatrix}
    \end{equation*}
  \item 
    \begin{equation*}
      A = \begin{mymatrix}{rrrr}
        2 & -1 & 3 & 5 \\
        2 & 0 & 1 & 2 \\
        6 & 4 & -5 & -6 \\
        0 & 2 & -4 & -6 
      \end{mymatrix}
    \end{equation*}
  \end{enumerate}
\end{ex}

\begin{ex}
  Show that if $A$ is an $m\times n$-matrix, then $\nullspace(A)$
  is a subspace of $\R^n$.
  \begin{sol}
    Clearly $\vect{0}\in\nullspace(A)$, since $A\vect{0}=\vect{0}$.
    If $\vect{x},\vect{y}\in\nullspace(A)$, then
    $A(\vect{x}+\vect{y}) = A\vect{x}+A\vect{y} = \vect{0}
    +\vect{0}=\vect{0}$, and therefore
    $\vect{x}+\vect{y}\in\nullspace(A)$. Similarly, if
    $\vect{x}\in\nullspace(A)$ and $k$ is a scalar, then
    $A(k\vect{x})=k(A\vect{x})=k\vect{0}=\vect{0}$, so
    $k\vect{x}\in\nullspace(A)$. So $\nullspace(A)$ contains
    $\vect{0}$, and is closed under addition and scalar
    multiplication. It is therefore a subspace of $\R^n$. 
  \end{sol}
\end{ex}

\begin{ex}\label{ex:column-space-is-image}
  Let $A$ be an $m\times n$-matrix. Show that $\col(A) =
  \set{A\vect{u} \mid \vect{u}\in\R^n}$. 
  \begin{sol}
    Let
    \begin{equation*}
      A = \begin{mymatrix}{ccc}
        a_{11} & \cdots & a_{1n} \\
        \vdots & \ddots & \vdots \\
        a_{m1} & \cdots & a_{mn} \\
      \end{mymatrix}.
    \end{equation*}
    Since $\col(A)$ is the span of the columns of $A$, we have
    $\vect{v}\in\col(A)$ if and only if there exists scalars
    $u_1,\ldots,u_n$ such that
    \begin{equation*}
      \vect{v}
      = u_1 \begin{mymatrix}{c} a_{11} \\ \vdots \\ a_{m1} \end{mymatrix}
      + \ldots
      + u_n \begin{mymatrix}{c} a_{1n} \\ \vdots \\ a_{mn} \end{mymatrix}.
    \end{equation*}
    But this equation is equivalent to
    \begin{equation*}
      \vect{v} =
      \begin{mymatrix}{ccc}
        a_{11} & \cdots & a_{1n} \\
        \vdots & \ddots & \vdots \\
        a_{m1} & \cdots & a_{mn} \\
      \end{mymatrix}
      \begin{mymatrix}{c} u_1 \\ \vdots \\ u_n \end{mymatrix}
      = A\vect{u}.
    \end{equation*}
    Therefore, $\vect{v}\in\col(A)$ if and only if $\vect{v}$ is of
    the form $A\vect{u}$, for some $\vect{u}\in\R^n$. In other words,
    $\col(A) = \set{A\vect{u}\mid\vect{u}\in\R^n}$.
  \end{sol}
\end{ex}

\begin{ex}
  Show that $\rank(A)=\rank(A^T)$.
  \begin{sol}
    The row space of $A$ is the same as the column space of $A^T$
    (except that it uses row vectors instead of column
    vectors). Therefore, $\rank(A) = \dim(\row(A)) = \dim(\col(A^T)) =
    \rank(A^T)$.
  \end{sol}
\end{ex}

\begin{ex}
  For invertible matrices $B$ and $C$ of appropriate size, show that
  $\rank(A) = \rank(BA) = \rank(AC)$.
  \begin{sol}
    From the theory of elementary matrices, we know that $B$ can be
    written as a product of elementary matrices $B=E_1\cdots E_k$.  It
    follows that $BA = E_1\cdots E_kA$. Since each elementary matrix
    corresponds to an elementary row operation, it follows that $BA$
    and $A$ are row equivalent. Therefore, $BA$ and $A$ have the same
    row space. It follows that
    $\rank(BA) = \dim(\row(BA)) = \dim(\row(A)) = \rank(A)$. This
    proves the first claim. To show the claim about $AC$, first note
    that by the above argument, $\rank(C^TA^T) = \rank(A^T)$, because
    $C^T$ is invertible. Then $\rank(AC)=\rank(A)$ follows by taking
    the transpose.
  \end{sol}
\end{ex}

\begin{ex}
  Suppose $A$ is an $m\times n$-matrix and $B$ is an $n\times p$-matrix.
  Show that 
  \begin{equation*}
    \nullity(AB) \leq \nullity(A) + \nullity(B).
  \end{equation*}
  \textbf{Hint:} Consider the subspace $\col(B)\cap\nullspace(A)$ and
  suppose a basis for this subspace is
  $\set{\vect{w}_{1},\ldots,\vect{w}_{k}}$. Let
  $\set{\vect{z}_{1},\ldots,\vect{z}_{k}}$ be such that
  $B\vect{z}_{i}=\vect{w}_{i}$. Now suppose
  $\set{\vect{u}_{1},\ldots,\vect{u}_{r}}$ is a basis for
  $\nullspace(B)$, and argue that
  $\nullspace(AB) \subseteq \sspan\set{\vect{u}_{1},\ldots,
    \vect{u}_{r},\vect{z}_{1},\ldots,\vect{z}_{k}}$.
  \begin{sol}
    Let $\set{\vect{w}_{1},\ldots,\vect{w}_{k}}$ be a basis of
    $\col(B)\cap\nullspace(A)$. Then for each $i$, we have
    $\vect{w}_i\in\col(B)$, and therefore by
    Exercise~\ref{ex:column-space-is-image}, we can find $\vect{z}_i$
    such that $\vect{w}_i = B\vect{z}_{i}$. Also let
    $\set{\vect{u}_{1},\ldots,\vect{u}_{r}} $ be a basis for
    $\nullspace(B)$. Now assume $\vect{x}\in\nullspace(AB)$. Then
    $AB\vect{x}=\vect{0}$, and therefore
    $B\vect{x} \in \nullspace(A)\cap \col(B)$. We therefore have
    \begin{equation*}
      B\vect{x} = c_1\vect{w}_{1}+\ldots+c_k\vect{w}_{k}
      = B(c_1\vect{z}_{1}+\ldots+c_k\vect{z}_{k}).
    \end{equation*}
    This implies
    \begin{equation*}
      \vect{x}-(c_1\vect{z}_{1}+\ldots+c_k\vect{z}_{k})\in \nullspace(B)
    \end{equation*}
    and so it is of the form
    \begin{equation*}
      \vect{x}-(c_1\vect{z}_{1}+\ldots+c_k\vect{z}_{k}) =
      d_1\vect{u}_1 + \ldots + d_r\vect{u}_r. 
    \end{equation*}
    It follows that
    \begin{equation*}
      \vect{x}\in \sspan\set{\vect{z}_{1},\ldots,\vect{z}_{k},
        \vect{u}_{1},\ldots,\vect{u}_{r}}.
    \end{equation*}
    Since we have shown that every element of $\nullspace(AB)$ is in
    the span of these $k+r$ vectors, it follows that 
    \begin{eqnarray*}
      \dim(\nullspace(AB))
      &\leq & k+r \\
      &=& \dim(\col(B)\cap\nullspace(A)) + \dim(\nullspace(B)) \\
      &\leq & \dim(\nullspace(A)) + \dim(\nullspace(B)),
    \end{eqnarray*}
    and therefore $\nullity(AB) \leq \nullity(A) + \nullity(B)$.
  \end{sol}
\end{ex}

