\section*{Exercises}

\begin{enumialphparenastyle}

\begin{ex}
\label{Chapter-1Q8}Consider the following augmented matrix in which $\ast $ denotes an
arbitrary number and $\blacksquare $ denotes a non-zero number. Determine
whether the given augmented matrix is consistent. If consistent, is the
solution unique? 
\begin{equation*}
\begin{mymatrix}{ccccc|c}
\blacksquare & \ast & \ast & \ast & \ast & \ast \\
0 & \blacksquare & \ast & \ast & 0 & \ast \\
0 & 0 & \blacksquare & \ast & \ast & \ast \\
0 & 0 & 0 & 0 & \blacksquare & \ast
\end{mymatrix}
\end{equation*}
\begin{sol}
The solution exists but is not unique.
\end{sol}
\end{ex}

\begin{ex}
Consider the following augmented matrix in which $\ast $ denotes an arbitrary
number and $\blacksquare $ denotes a non-zero number. Determine whether the
given augmented matrix is consistent. If consistent, is the solution unique?
\begin{equation*}
\begin{mymatrix}{ccc|c}
\blacksquare & \ast & \ast & \ast \\
0 & \blacksquare & \ast & \ast \\
0 & 0 & \blacksquare & \ast
\end{mymatrix}
\end{equation*}
\begin{sol}
A solution exists and is unique.
\end{sol}
\end{ex}


\begin{ex}
Consider the following augmented matrix in which $\ast $ denotes an arbitrary
number and $\blacksquare $ denotes a non-zero number. Determine whether the
given augmented matrix is consistent. If consistent, is the solution unique? 
\begin{equation*}
\begin{mymatrix}{ccccc|c}
\blacksquare & \ast & \ast & \ast & \ast & \ast \\
0 & \blacksquare & 0 & \ast & 0 & \ast \\
0 & 0 & 0 & \blacksquare & \ast & \ast \\
0 & 0 & 0 & 0 & \blacksquare & \ast
\end{mymatrix}
\end{equation*}
%\begin{sol}
%\end{sol}
\end{ex}

\begin{ex}
Consider the following augmented matrix in which $\ast $ denotes an arbitrary
number and $\blacksquare $ denotes a non-zero number. Determine whether the
given augmented matrix is consistent. If consistent, is the solution unique?
\begin{equation*}
\begin{mymatrix}{ccccc|c}
\blacksquare & \ast & \ast & \ast & \ast & \ast \\
0 & \blacksquare & \ast & \ast & 0 & \ast \\
0 & 0 & 0 & 0 & \blacksquare & 0 \\
0 & 0 & 0 & 0 & \ast & \blacksquare
\end{mymatrix}
\end{equation*}
\begin{sol}
There might be a solution. If so, there are infinitely many.
\end{sol}
\end{ex}

\begin{ex} 
Suppose a system of equations has fewer equations than variables. Will such a system necessarily be consistent? If so, explain why and if not, give an example
which is not consistent.
\begin{sol}
No. Consider $x+y+z=2$ and $x+y+z=1$.
\end{sol}
\end{ex}

\begin{ex}
If a system of equations has more equations than variables, can it
have a solution? If so, give an example and if not, explain why not.
\begin{sol}
These can
have a solution. For example, $x+y=1,2x+2y=2,3x+3y=3$ even has an infinite
set of solutions.
\end{sol}
\end{ex}

\begin{ex}
Find $h$ such that
\begin{equation*}
\begin{mymatrix}{rr|r}
2 & h & 4 \\
3 & 6 & 7
\end{mymatrix}
\end{equation*}
is the augmented matrix of an \textit{inconsistent} system.
\begin{sol}
$h=4$
\end{sol}
\end{ex}

\begin{ex}
Find $h$ such that
\begin{equation*}
\begin{mymatrix}{rr|r}
1 & h & 3 \\
2 & 4 & 6
\end{mymatrix}
\end{equation*}
is the augmented matrix of a \textit{consistent} system. 
\begin{sol}
 Any $h$ will work.
\end{sol}
\end{ex}

\begin{ex}
Find $h$ such that
\begin{equation*}
\begin{mymatrix}{rr|r}
1 & 1 & 4 \\
3 & h & 12
\end{mymatrix}
\end{equation*}
is the augmented matrix of a \textit{consistent} system. 
\begin{sol}
 Any $h$ will work.
\end{sol}
\end{ex}


\begin{ex}
Choose $h$ and $k$ such that the augmented matrix shown has each of the following: 
\begin{enumerate}
\item one solution
\item no solution
\item infinitely many solutions
\end{enumerate}
\begin{equation*}
\begin{mymatrix}{rr|r}
1 & h & 2 \\
2 & 4 & k
\end{mymatrix} 
\end{equation*}
\begin{sol}
If $h\neq 2$ there will be a unique solution for any $k$. If $h=2$ and $%
k\neq 4$, there are no solutions. If $h=2$ and $k=4$, then there are
infinitely many solutions.
\end{sol}
\end{ex}


\begin{ex}
Choose $h$ and $k$ such that the augmented matrix shown has each of the following: 
\begin{enumerate}
\item one solution
\item no solution
\item infinitely many solutions
\end{enumerate}
\begin{equation*}
\begin{mymatrix}{rr|r}
1 & 2 & 2 \\
2 & h & k
\end{mymatrix} 
\end{equation*}
\begin{sol}
If $h\neq 4$, then there is exactly one solution. If $h=4$ and $k\neq 4$,
then there are no solutions. If $h=4$ and $k=4$, then there are infinitely
many solutions.
\end{sol}
\end{ex}


\begin{ex}
Determine if the system is consistent. If so, is the solution unique?
\begin{equation*}
\begin{array}{c}
x+2y+z-w=2 \\
x-y+z+w=1 \\
2x+y-z=1 \\
4x+2y+z=5
\end{array}
\end{equation*}
\begin{sol}
There is no solution. The system is inconsistent. You can see this from the
augmented matrix. $\begin{mymatrix}{rrrr|r}
1 & 2 & 1 & -1 & 2 \\
1 & -1 & 1 & 1 & 1 \\
2 & 1 & -1 & 0 & 1 \\
4 & 2 & 1 & 0 & 5
\end{mymatrix}$, {\ef}: $\begin{mymatrix}{rrrr|r}
1 & 2 & 1 & -1 & 2 \\
0 & -3 & 0 & 2 & -1 \\
0 & 0 & -3 & 0 & -2 \\
0 & 0 & 0 & 0 & 1
\end{mymatrix}$.
\end{sol}
\end{ex}

\begin{ex}
Determine if the system is consistent. If so, is the solution unique? 
\begin{equation*}
\begin{array}{c}
x+2y+z-w=2 \\
x-y+z+w=0 \\
2x+y-z=1 \\
4x+2y+z=3
\end{array}
\end{equation*}
\begin{sol}
Solution is: $\mat{w=\frac{3}{2}y-1,x=\frac{2}{3}-\frac{1}{2}y,z=\frac{1}{3
}} $
\end{sol}
\end{ex}

\begin{ex} Determine which matrices are in {\ef}. 

\begin{enumerate}
\item $\begin{mymatrix}{rrrr}
1 & 1 & 2 & 0 \\
0 & 0 & 3 & 2 \\
0 & 0 & 0 & 0
\end{mymatrix} $

\item $\begin{mymatrix}{rrr}
1 & 0 & 0 \\
2 & 1 & 7
\end{mymatrix} $

\item $\begin{mymatrix}{rrrrrr}
0 & 1 & 0 & 0 & 5 \\
0 & 0 & 1 & 0 & 4 \\
0 & 0 & 0 & 1 & 3
\end{mymatrix} $
\end{enumerate}
\begin{sol}
\begin{enumerate}
\item This one is.
\item This one is not.
\item This one is.
\end{enumerate}
\end{sol}
\end{ex}

\begin{ex}\label{ex:rr-ef}
Row reduce each of the following matrices to {\ef}.
\begin{equation*}
(a)~
\begin{mymatrix}{rrrr}
2 & -1 & 3 & -1 \\
1 & 0 & 2 & 1 \\
1 & -1 & 1 & -2
\end{mymatrix}
\quad
(b)~
\begin{mymatrix}{rrrr}
0 & 0 & -1 & -1 \\
1 & 1 & 1 & 0 \\
1 & 1 & 0 & -1
\end{mymatrix}
\quad
(c)~
\begin{mymatrix}{rrrr}
3 & -6 & -7 & -8 \\
1 & -2 & -2 & -2 \\
1 & -2 & -3 & -4
\end{mymatrix}
\end{equation*}
\begin{equation*}
(d)~
\begin{mymatrix}{rrrr}
2 & 4 & 5 & 15 \\
1 & 2 & 3 & 9 \\
1 & 2 & 2 & 6
\end{mymatrix}
\quad
(e)~
\begin{mymatrix}{rrrr}
4 & -1 & 7 & 10 \\
1 & 0 & 3 & 3 \\
1 & -1 & -2 & 1
\end{mymatrix}
\quad
(f)~
\begin{mymatrix}{rrrr}
3 & 5 & -4 & 2 \\
1 & 2 & -1 & 1 \\
1 & 1 & -2 & 0
\end{mymatrix}
\end{equation*}
\begin{equation*}
(g)~
\begin{mymatrix}{rrrr}
-2 & 3 & -8 & 7 \\
1 & -2 & 5 & -5 \\
1 & -3 & 7 & -8
\end{mymatrix}
\end{equation*}
%\begin{sol}
%\end{sol}
\end{ex}
 
\begin{ex}
Find the general solution of the system whose augmented matrix is 
\begin{equation*}
(a)~
\begin{mymatrix}{rrr|r}
1 & 2 & 0 & 2 \\
1 & 3 & 4 & 2 \\
1 & 0 & 2 & 1
\end{mymatrix} 
\quad
(b)~
\begin{mymatrix}{rrr|r}
1 & 2 & 0 & 2 \\
2 & 0 & 1 & 1 \\
3 & 2 & 1 & 3
\end{mymatrix} 
\quad
(c)~
\begin{mymatrix}{rrr|r}
1 & 1 & 0 & 1 \\
1 & 0 & 4 & 2
\end{mymatrix} 
\end{equation*}
\begin{equation*}
(d)~
\begin{mymatrix}{rrrrr|r}
1 & 0 & 2 & 1 & 1 & 2 \\
0 & 1 & 0 & 1 & 2 & 1 \\
1 & 2 & 0 & 0 & 1 & 3 \\
1 & 0 & 1 & 0 & 2 & 2
\end{mymatrix} 
\quad
(e)~
\begin{mymatrix}{rrrrr|r}
1 & 0 & 2 & 1 & 1 & 2 \\
0 & 1 & 0 & 1 & 2 & 1 \\
0 & 2 & 0 & 0 & 1 & 3 \\
1 & -1 & 2 & 2 & 2 & 0
\end{mymatrix} 
\end{equation*}

\begin{sol}
(b) The {\ef} is $\begin{mymatrix}{rrr|r}
1 & 0 & \vspace{0.05in}\frac{1}{2} & \vspace{0.05in}\frac{1}{2} \\
0 & 1 & -\vspace{0.05in}\frac{1}{4} & \vspace{0.05in}\frac{3}{4} \\
0 & 0 & 0 & 0
\end{mymatrix}$. Therefore, the solution is of the form $z=t,y=\frac{3}{4}+t\tup{
\frac{1}{4}} ,x=\frac{1}{2}-\frac{1}{2}t$ where $t\in \R$.

(c) The {\ef} is $\begin{mymatrix}{rrr|r}
1 & 0 & 4 & 2 \\
0 & 1 & -4 & -1
\end{mymatrix} $ and so the solution is $z=t,y=-1+4t,x=2-4t$.

(d) The {\ef} is $\begin{mymatrix}{rrrrr|r}
1 & 0 & 0 & 0 & 9 & 3 \\
0 & 1 & 0 & 0 & -4 & 0 \\
0 & 0 & 1 & 0 & -7 & -1 \\
0 & 0 & 0 & 1 & 6 & 1
\end{mymatrix} $ and so $x_{5}=t,x_{4}=1-6t,x_{3}=-1+7t,x_{2}=4t,x_{1}=3-9t$.

(e) The {\ef} is $\begin{mymatrix}{rrrrr|r}
1 & 0 & 2 & 0 & -\vspace{0.05in}\frac{1}{2} & \vspace{0.05in}\frac{5}{2} \\
0 & 1 & 0 & 0 & \vspace{0.05in}\frac{1}{2} & \vspace{0.05in}\frac{3}{2} \\
0 & 0 & 0 & 1 & \vspace{0.05in}\frac{3}{2} & -\vspace{0.05in}\frac{1}{2} \\
0 & 0 & 0 & 0 & 0 & 0
\end{mymatrix}$. Therefore, let $x_{5}=t,x_{3}=s$. Then the other variables are
given by $x_{4}=-\frac{1}{2}-\frac{3}{2}t,x_{2}=\frac{3}{2}-t\frac{1}{2}
,,x_{1}=\frac{5}{2}+\frac{1}{2}t-2s$.
\end{sol}
\end{ex}

\begin{ex} Solve the system of equations $7x+14y+15z=22,
$ $2x+4y+3z=5$, and $3x+6y+10z=13$.
\begin{sol}
Solution is: $\mat{x=1-2t,z=1,y=t} $
\end{sol}
\end{ex}

\begin{ex} Solve the system of equations $3x-y+4z=6$, 
$y+8z=0$, and $-2x+y=-4$.
\begin{sol}
Solution is: $\mat{x=2-4t,y=-8t,z=t} $
\end{sol}
\end{ex}

\begin{ex} Solve the system of equations $9x-2y+4z=-17,
$ $13x-3y+6z=-25$, and $-2x-z=3$.
\begin{sol}
 Solution is: $\mat{x=-1,y=2,z=-1} $
\end{sol}
\end{ex}

\begin{ex} Solve the system of equations
$65x+84y+16z=546$, $81x+105y+20z=682$, and $84x+110y+21z=713$.
\begin{sol}
Solution is:
$\mat{x=2,y=4,z=5} $
\end{sol}
\end{ex}

\begin{ex} Solve the system of equations 
$8x+2y+3z=-3,8x+3y+3z=-1$, and $4x+y+3z=-9$.
\begin{sol}
Solution is: $\mat{x=1,y=2,z=-5} $
\end{sol}
\end{ex}

\begin{ex} Suppose a system of equations has fewer equations than variables and
you have found a solution to this system of equations. Is it possible that
your solution is the only one?\ Explain.
\begin{sol}
No. Consider $x+y+z=2$ and $x+y+z=1$.
\end{sol}
\end{ex}

\begin{ex} Suppose a system of linear equations has an augmented
  matrix with 2 rows and 4 columns and the last column is a pivot
  column. Could the system of linear equations be consistent? Explain.
  \begin{sol}
    No. This would lead to $0=1$.
  \end{sol}
\end{ex}

\begin{ex} Suppose the coefficient matrix of a system of $n$ equations with $n$
variables has the property that every column is a pivot column. Does it
follow that the system of equations must have a solution? If so, must the
solution be unique? Explain. 
\begin{sol}
Yes. It has a unique solution.
\end{sol}
\end{ex}

\begin{ex} Suppose there is a unique solution to a system of linear equations.
What must be true of the pivot columns in the augmented matrix?
\begin{sol}
The last column must not be a pivot column. The remaining columns must each be pivot
columns.
\end{sol}
\end{ex}


\begin{ex} The steady state temperature, $u$, of a plate solves Laplace's
equation, $\Delta u=0$. One way to approximate the solution is to divide the plate into a square mesh and require the temperature
at each node to equal the average of the temperature at the four adjacent
nodes. In the following picture, the numbers represent the observed
temperature at the indicated nodes. Find the temperature at
the interior nodes, indicated by $x,y,z$, and $w$. One of the equations is 
$z=\vspace{0.05in}\frac{1}{4}\tup{10+0+w+x}$. 

\begin{picture}(1,60)
 \put(100,0){\begin{picture}(1,1)
 %\put(0,0){\line(1,0){90}}
 \setlength{\unitlength}{.6pt}
 \put(0,30){\line(1,0){90}}
 \put(0,60){\line(1,0){90}}
 \put(30,0){\line(0,1){90}}
 \put(60,0){\line(0,1){90}}
 \put(60,0){\circle*{3}}
 \put(30,0){\circle*{3}}
 \put(0,60){\circle*{3}}
 \put(0,30){\circle*{3}}
 \put(30,30){\circle*{3}}
  \put(30,60){\circle*{3}}
   \put(60,30){\circle*{3}}
    \put(60,60){\circle*{3}}
     \put(60,90){\circle*{3}}
      \put(30,90){\circle*{3}}
   \put(90,30){\circle*{3}}
    \put(90,60){\circle*{3}}
     \put(63,-4){$10$}
 \put(33,-4){$10$}
 \put(-10,60){$20$}
 \put(-10,30){$20$}
 \put(34,34){$x$}
  \put(34,64){$y$}
   \put(64,34){$z$}
    \put(64,64){$w$}
     \put(64,90){$30$}
      \put(34,90){$30$}
   \put(94,30){$0$}
    \put(94,60){$0$}
 \end{picture}}
 \end{picture}

\begin{sol}
You need $
\begin{array}{c}
\frac{1}{4}\tup{20+30+w+x} -y=0 \\
\frac{1}{4}\tup{y+30+0+z} -w=0 \\
\frac{1}{4}\tup{20+y+z+10} -x=0 \\
\frac{1}{4}\tup{x+w+0+10} -z=0
\end{array}
$, Solution is: $\mat{w=15,x=15,y=20,z=10}$.
\end{sol}
\end{ex}

\begin{ex} Find the rank of the following matrix.
\begin{equation*}
\begin{mymatrix}{rrrr}
4 & -16 & -1 & -5 \\
1 & -4 & 0 & -1 \\
1 & -4 & -1 & -2
\end{mymatrix}
\end{equation*}
%\begin{sol}
%\end{sol}
\end{ex}

\begin{ex} Find the rank of the following matrix.
\begin{equation*}
\begin{mymatrix}{rrrr}
3 & 6 & 5 & 12 \\
1 & 2 & 2 & 5 \\
1 & 2 & 1 & 2
\end{mymatrix}
\end{equation*}
%\begin{sol}
%\end{sol}
\end{ex}

\begin{ex} Find the rank of the following matrix.
\begin{equation*}
\begin{mymatrix}{rrrrr}
0 & 0 & -1 & 0 & 3 \\
1 & 4 & 1 & 0 & -8 \\
1 & 4 & 0 & 1 & 2 \\
-1 & -4 & 0 & -1 & -2
\end{mymatrix}
\end{equation*}
%\begin{sol}
%\end{sol}
\end{ex}

\begin{ex} Find the rank of the following matrix.
\begin{equation*}
\begin{mymatrix}{rrrr}
4 & -4 & 3 & -9 \\
1 & -1 & 1 & -2 \\
1 & -1 & 0 & -3
\end{mymatrix}
\end{equation*}
%\begin{sol}
%\end{sol}
\end{ex}

\begin{ex} Find the rank of the following matrix.
\begin{equation*}
\begin{mymatrix}{rrrrr}
2 & 0 & 1 & 0 & 1 \\
1 & 0 & 1 & 0 & 0 \\
1 & 0 & 0 & 1 & 7 \\
1 & 0 & 0 & 1 & 7
\end{mymatrix}
\end{equation*}
%\begin{sol}
%\end{sol}
\end{ex}

\begin{ex} Find the rank of the following matrix.
\begin{equation*}
\begin{mymatrix}{rrr}
4 & 15 & 29 \\
1 & 4 & 8 \\
1 & 3 & 5 \\
3 & 9 & 15
\end{mymatrix}
\end{equation*}
%\begin{sol}
%\end{sol}
\end{ex}

\begin{ex} Find the rank of the following matrix. 
\begin{equation*}
\begin{mymatrix}{rrrrr}
0 & 0 & -1 & 0 & 1 \\
1 & 2 & 3 & -2 & -18 \\
1 & 2 & 2 & -1 & -11 \\
-1 & -2 & -2 & 1 & 11
\end{mymatrix}
\end{equation*}
%\begin{sol}
%\end{sol}
\end{ex}

\begin{ex} Find the rank of the following matrix.
\begin{equation*}
\begin{mymatrix}{rrrrr}
1 & -2 & 0 & 3 & 11 \\
1 & -2 & 0 & 4 & 15 \\
1 & -2 & 0 & 3 & 11 \\
0 & 0 & 0 & 0 & 0
\end{mymatrix}
\end{equation*}
%\begin{sol}
%\end{sol}
\end{ex}

\begin{ex} Find the rank of the following matrix.
\begin{equation*}
\begin{mymatrix}{rrr}
-2 & -3 & -2 \\
1 & 1 & 1 \\
1 & 0 & 1 \\
-3 & 0 & -3
\end{mymatrix}
\end{equation*}
%\begin{sol}
%\end{sol}
\end{ex}

\begin{ex} Find the rank of the following matrix.
\begin{equation*}
\begin{mymatrix}{rrrrr}
4 & 4 & 20 & -1 & 17 \\
1 & 1 & 5 & 0 & 5 \\
1 & 1 & 5 & -1 & 2 \\
3 & 3 & 15 & -3 & 6
\end{mymatrix}
\end{equation*}
%\begin{sol}
%\end{sol}
\end{ex}

\begin{ex} Find the rank of the following matrix.
\begin{equation*}
\begin{mymatrix}{rrrrr}
-1 & 3 & 4 & -3 & 8 \\
1 & -3 & -4 & 2 & -5 \\
1 & -3 & -4 & 1 & -2 \\
-2 & 6 & 8 & -2 & 4
\end{mymatrix}
\end{equation*}
%\begin{sol}
%\end{sol}
\end{ex}

\begin{ex}
  Suppose $A$ is an $m\times n$-matrix. Explain why the rank of $A$ is
  always no larger than $\min \tup{m,n}$.
  \begin{sol}
    The rank is the number of pivot entries in the {\ef}. There is at
    most one pivot entry in each row and column. Therefore, the rank
    cannot be larger than the number of rows or the number of columns;
    in other words, the rank is at most $\min\tup{m,n}$.
  \end{sol}
\end{ex}

\begin{ex}
  State whether each of the following sets of data are
  possible for a system of equations. If possible, describe the
  solution set.  That is, indicate whether there exists a unique
  solution, no solution or infinitely many solutions. Here, $A$ is
  the coefficient matrix, and $\mat{A\mid B}$ denotes the
  augmented matrix of the system.
  
  \begin{enumerate}
  \item $A$ is a $5\times 6$-matrix, $\limfunc{rank}\tup{A} =4$ and 
    $\limfunc{rank}\mat{A\mid B} =4$. 
    
  \item $A$ is a $3\times 4$-matrix, $\limfunc{rank}\tup{A} =3$ and 
    $\limfunc{rank}\mat{A\mid B} =2$.
    
  \item $A$ is a $4\times 2$-matrix, $\limfunc{rank}\tup{A} =4$ and 
    $\limfunc{rank}\mat{A\mid B} =4$. 
    
  \item $A$ is a $5\times 5$-matrix, $\limfunc{rank}\tup{A} =4$ and 
    $\limfunc{rank}\mat{A\mid B} =5$. 
    
  \item $A$ is a $4\times 2$-matrix, $\limfunc{rank}\tup{A} =2$ and 
    $\limfunc{rank}\mat{A\mid B} =2$.
  \end{enumerate}
  
  \begin{sol}
    \begin{enumerate}
    \item This says $B$ is in the span of four of the columns. Thus the columns are not independent. Infinite solution set.
    \item This surely can't happen. If you add in another column, the rank does not get smaller.
    \item This says $B$ is in the span of the columns and the columns must be
      independent. You can't have the rank equal 4 if you only have two columns.
    \item This says $B$ is not in the span of the columns. In this case, there is no solution to the system of equations represented by the augmented matrix.
    \item In this case, there is a
      unique solution since the columns of $A$ are independent.
    \end{enumerate}
  \end{sol}
\end{ex}

\begin{ex} Consider the system $-5x+2y-z=0$ and $-5x-2y-z=0$. Both equations
equal zero and so $-5x+2y-z=-5x-2y-z$ which is equivalent to $y=0$. Does it follow that $x$
and $z$ can equal anything?  Notice that when $x=1$, $z=-4$, and $y=0$ are plugged in
to the equations, the equations do not equal $0$. Why?
\begin{sol}
These are not legitimate row
operations. They do not preserve the solution set of the system.
\end{sol}
\end{ex}

\end{enumialphparenastyle}
