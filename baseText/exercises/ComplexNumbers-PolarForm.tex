\Opensolutionfile{solutions}[ex]
\section*{Exercises}

\begin{enumialphparenastyle}

\begin{ex} Let $z = 3 + 3i$ be a complex number written in standard form. Convert $z$ to polar form, and write it in the form $z = re^{i\theta}$.
%\begin{sol}
%\end{sol}
\end{ex}

\begin{ex} Let $z = 2i$ be a complex number written in standard form. Convert $z$ to polar form, and write it in the form $z = re^{i\theta}$.
%\begin{sol}
%\end{sol}
\end{ex}

\begin{ex} Let $z = 4e^{\frac{2\pi}{3}i}$ be a complex number written in polar form. Convert $z$ to standard form, and write it in the form $z = a+bi$.
%\begin{sol}
%\end{sol}
\end{ex}

\begin{ex} Let $z = -1e^{\frac{\pi}{6}i}$ be a complex number written in polar form. Convert $z$ to standard form, and write it in the form $z = a+bi$.
%\begin{sol}
%\end{sol}
\end{ex}

\begin{ex} If $z$ and $w$ are two complex numbers and the polar form of $z$
involves the angle $\theta $ while the polar form of $w$ involves the angle 
$\phi$, show that in the polar form for $zw$ the angle involved is $\theta
+\phi$. 
\begin{sol}
 You have $z=\abs{z}(\cos
\theta +i\sin \theta) $ and $w=\abs{w}(\cos
\phi +i\sin \phi)$. Then when you multiply these, you get
\begin{eqnarray*}
&&\abs{z}\abs{w}(\cos \theta +i\sin
\theta) (\cos \phi +i\sin \phi) \\
&=&\abs{z}\abs{w}(\cos \theta \cos
\phi -\sin \theta \sin \phi +i(\cos \theta \sin \phi +\cos \phi \sin
\theta)) \\
&=&\abs{z}\abs{w}(\cos (\theta
+\phi) +i\sin (\theta +\phi))
\end{eqnarray*}
\end{sol}
\end{ex}

\end{enumialphparenastyle}