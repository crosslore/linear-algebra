\section*{Exercises}

\begin{ex}
  Consider the set $\R^2$ with the following non-standard addition
  operation $\oplus$:
  \begin{equation*}
    (a,b) \oplus (c,d) = (a+d,b+c).
  \end{equation*}
  Scalar multiplication is defined in the usual way. Is this a vector space?
  Explain why or why not.
  % \begin{sol}
  % \end{sol}
\end{ex}

\begin{ex}
  Consider $\R^{2}$ with the following non-standard addition operation
  $\oplus$:
  \begin{equation*}
    (a,b) \oplus (c,d) = (0,b+d).
  \end{equation*}
  Scalar multiplication is defined in the usual way. Is this a vector space?
  Explain why or why not.
  % \begin{sol}
  % \end{sol}
\end{ex}

\begin{ex}
  Consider $\R^{2}$ with the following non-standard scalar multiplication:
  \begin{equation*}
    c\odot(a,b) = (a,cb).
  \end{equation*}
  Vector addition is defined as usual. Is this a vector space? Explain
  why or why not.
  % \begin{sol}
  % \end{sol}
\end{ex}

\begin{ex}
  Consider $\R^{2}$ with the following non-standard addition operation
  $\oplus$:
  \begin{equation*}
    (a,b) \oplus (c,d) = (a-c,b-d).
  \end{equation*}
  Scalar multiplication is defined as usual. Is this a vector space?
  Explain why or why not.
  % \begin{sol}
  % \end{sol}
\end{ex}

\begin{ex}
  Prove that the set $\Seq_K$ from
  Example~\ref{exa:vector-space-sequences} is a vector space.
  Hint: this is a special case of
  Example~\ref{exa:vector-space-function}, if you realize that a
  sequence $(a_i)_{i\in\N}$ is the same thing as a function $a:\N\to K$.
\end{ex}

\begin{ex}
  Prove that the set $\Poly$ from
  Example~\ref{exa:vector-space-polynomials} is a vector space.
\end{ex}

\begin{ex}
  Let $V$ be the set of functions defined on a set $X$ that have
  values in a vector space $W$. Is this a vector space? Explain.
  % \begin{sol}
  % \end{sol}
\end{ex}

\begin{ex}
  Consider the set $\R^{2}$ with the following non-standard operations
  of addition and scalar multiplication:
  \begin{equation*}
    \begin{array}{rcl}
      (a,b) \oplus (c,d) &=& (a+c-1,\, b+d-1), \\
      k \odot (c,d) &=& (kc + (1-k),\, kd + (1-k)). \\
    \end{array}
  \end{equation*}
  Show that $\R^2$ is a vector space with these operations. Hint: the
  zero vector is not $(0,0)$, but $(1,1)$.
  % \begin{sol}
  % \end{sol}
\end{ex}

\begin{ex}
  Consider the set $\R$ of real numbers. Addition of real numbers is
  defined in the usual way, and scalar multiplication is just
  multiplication of one real number by another. In other words, $x+y$
  means to add the two numbers and $xy$ means to multiply them. Show
  that $\R$, with these operations, is a real vector space.
  % \begin{sol}
  % \end{sol}
\end{ex}

\begin{ex}
  Let $K=\Q$ be the field of rational numbers, and let $V$ be the set
  of real numbers of the form $a+b\sqrt{2}$, where $a$ and $b$ are
  rational numbers. Show that with the usual operations, $V$ is a
  $\Q$-vector space.
  % \begin{sol}
  % \end{sol}
\end{ex}

\begin{ex}
  Let $K=\Q$ be the field of rational numbers, and let $V=\R$ be the
  set of real numbers. Show that with the usual operations, $V$ is a
  $\Q$-vector space.
  % \begin{sol}
  % \end{sol}
\end{ex}

\begin{ex}
  Let $\Poly_{3}$ be the set of all polynomials of degree 3 or
  less. That is, these are of the form $ax^3+bx^2+cx+d$. Addition and
  scalar multiplication of polynomials are defined as usual.  Show
  that $\Poly_{3}$ is a vector space.
  % \begin{sol}
  % \end{sol}
\end{ex}

\begin{ex}
  Let $X=\set{1,2,\ldots,n}$, and consider the space $\Func_{X,\R}$ of
  real-valued functions defined on $X$. Explain how $\Func_{X,\R}$ can be
  considered as $\R^{n}$.
  \begin{sol}
    Let $f(i)$ be the $i\th$ component of a vector
    $\vect{x}\in\R^{n}$. Thus a typical element in $\R^{n}$ is
    $ (f(1),\ldots,f(n))$.
  \end{sol}
\end{ex}

\begin{ex}
  Prove the cancellation law from
  Proposition~\ref{prop:vector-space-elementary}.
\end{ex}
