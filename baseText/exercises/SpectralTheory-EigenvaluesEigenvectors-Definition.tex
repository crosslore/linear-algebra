\section*{Exercises}


\begin{ex}
  Consider the matrix
  \begin{equation*}
    A = \begin{mymatrix}{rrr}
      1 & -2 & -2 \\
      2 & -3 & -2 \\
      -2 & 2 &  1 \\
    \end{mymatrix}.
  \end{equation*}
  Which of the following vectors are eigenvectors of $A$? Find the
  corresponding eigenvalues.
  \begin{equation*}
    \vect{v}_1 = \begin{mymatrix}{r} 1 \\  1 \\ -1 \end{mymatrix},\quad
    \vect{v}_2 = \begin{mymatrix}{r} 1 \\  1 \\  0 \end{mymatrix},\quad
    \vect{v}_3 = \begin{mymatrix}{r} 2 \\  2 \\ -1 \end{mymatrix},\quad
    \vect{v}_4 = \begin{mymatrix}{r} 0 \\ -1 \\  1 \end{mymatrix}.
  \end{equation*}
\end{ex}

\begin{ex}
  Let
  \begin{equation*}
    A = \begin{mymatrix}{rrr}
      1  &  0 & 0 \\
      -5 & -1 & 5 \\
      -3 &  0 & 4 \\
    \end{mymatrix}.
  \end{equation*}
  Find the eigenvectors corresponding to the eigenvalue $\eigenvar=4$.
\end{ex}

\begin{ex}
  Let
  \begin{equation*}
    A = \begin{mymatrix}{rrr}
      7 &  -4 &   8 \\
      -1 &  4 &  -2 \\
      -2 &  2 &  -1 \\
    \end{mymatrix}.
  \end{equation*}
  Find the eigenvectors corresponding to the eigenvalue $\eigenvar=3$.
\end{ex}

\begin{ex}
  Let
  \begin{equation*}
    A = \begin{mymatrix}{rrr}
      4 &   0 &   3 \\
      -3 &   1 &  -3 \\
      0 &   0 &   1 \\
    \end{mymatrix}.
  \end{equation*}
  This matrix has eigenvalues $\eigenvar=1$ and $\eigenvar=4$. Find a
  basis for each eigenspace.
\end{ex}

\begin{ex}
  Let
  \begin{equation*}
    A = \begin{mymatrix}{rrr}
      2 &   4 &  -4 \\
      -1 &  6 &  -9 \\
      0 &   0 &  -3 \\
    \end{mymatrix}.
  \end{equation*}
  This matrix has eigenvalues $\eigenvar=3$ and $\eigenvar=4$. Find a
  basis for each eigenspace.
\end{ex}

\begin{ex}
  Suppose $A$ is a $3\times 3$-matrix with eigenvalues
  $\eigenvar_1=1$, $\eigenvar_2=0$, and $\eigenvar_3=2$ and
  corresponding eigenvectors
  \begin{equation*}
    \vect{v}_1 = \begin{mymatrix}{r}
      -1 \\
      -2 \\
      -2
    \end{mymatrix},
    \quad
    \vect{v}_2 = \begin{mymatrix}{r}
      1 \\
      1 \\
      1
    \end{mymatrix},
    \quad\mbox{and}\quad
    \vect{v}_3 = \begin{mymatrix}{r}
      -1 \\
      -4 \\
      -3
    \end{mymatrix}.
  \end{equation*}
  (By ``corresponding'', we mean that $\vect{v}_1$ corresponds to
  $\eigenvar_1$, $\vect{v}_2$ corresponds to $\eigenvar_2$, and so
  on).  Find
  \begin{equation*}
    A\begin{mymatrix}{r}
      3 \\
      -4 \\
      3
    \end{mymatrix}.
  \end{equation*}
  % \begin{sol}
  % \end{sol}
\end{ex}

\begin{ex}
  Let $A$ be an $n\times n$-matrix, and assume $\eigenvar$ is an
  eigenvalue of $A$. Show that $\eigenvar^2$ is an eigenvalue of
  $A^2$.
  \begin{sol}
    If $\vect{v}$ is an eigenvector corresponding to the eigenvalue
    $\eigenvar$, then $A^{2}\vect{v}=A(A\vect{v}) =
    A(\eigenvar\vect{v}) = \eigenvar(A\vect{v}) =
    \eigenvar^2\vect{v}$. Therefore, $\vect{v}$ is an eigenvector of
    $A^2$ with eigenvalue $\eigenvar^2$.
  \end{sol}
\end{ex}

\begin{ex}
  Let $A$ be an invertible $n\times n$-matrix, and assume $\eigenvar$
  is an eigenvalue of $A$. Show that $\eigenvar\neq 0$ and that
  $\eigenvar^{-1}$ is an eigenvalue of $A^{-1}$.
  \begin{sol}
    We have
    $\eigenvar A^{-1}\vect{v} = A^{-1}\eigenvar\vect{v} =
    A^{-1}A\vect{v} = \vect{v}$. Since $\vect{v}\neq 0$, this implies
    $\eigenvar\neq 0$. Moreover, it implies
    $A^{-1}\vect{v} = \eigenvar^{-1}\vect{v}$. Thus, $\eigenvar^{-1}$
    is an eigenvalue of $A^{-1}$.
  \end{sol}
\end{ex}

\begin{ex}
  If $A$ is an $n\times n$-matrix and $c$ is a non-zero constant,
  compare the eigenvalues of $A$ and $cA$.
  \begin{sol}
    Say $A\vect{v}=\eigenvar \vect{v}$. Then
    $ cA\vect{v}=c\eigenvar \vect{v}$ and so the eigenvalues of $cA$ are
    just $ c\eigenvar $ where $\eigenvar $ is an eigenvalue of $A$.
  \end{sol}
\end{ex}

\begin{ex}
  Let $A,B$ be invertible $n\times n$-matrices which commute. That is,
  $AB=BA$. Suppose $\vect{v}$ is an eigenvector of $B$. Show that then
  $A\vect{v}$ must also be an eigenvector for $B$.
  \begin{sol}
    Suppose $\vect{v}$ is an eigenvector of $B$, i.e.,
    $B\vect{v}=\eigenvar \vect{v}$. Then
    $BA\vect{v}=AB\vect{v} =A\eigenvar \vect{v}=\eigenvar A\vect{v}$,
    and therefore $A\vect{v}$ is an eigenvector of $B$.
  \end{sol}
\end{ex}

\begin{ex}
  Suppose $A$ is an $n\times n$-matrix and it satisfies $A^{m}=A$ for
  some $m$ a positive integer larger than 1. Show that if $\eigenvar $
  is an eigenvalue of $A$ then $\eigenvar$ equals either $0$, $1$, or
  $-1$.
  \begin{sol}
    Let $\vect{v}$ be the eigenvector. Then
    $A^{m}\vect{v}=\eigenvar^{m}\vect{v}$ and
    $A^{m}\vect{v}=A\vect{v}=\eigenvar\vect{v}$. Therefore
    $\eigenvar^{m}=\eigenvar$. Hence if $\eigenvar \neq 0$, we must
    have $\eigenvar^{m-1}=1$, which implies that $\eigenvar=\pm 1$.
  \end{sol}
\end{ex}

\begin{ex}
  Show that if $A\vect{v}=\eigenvar \vect{v}$ and
  $A\vect{w}=\eigenvar \vect{w}$, then whenever $k,p$ are scalars,
  \begin{equation*}
    A(k\vect{v}+p\vect{w}) =\eigenvar (k\vect{v}+p\vect{w})
  \end{equation*}
  Does this imply that $k\vect{v}+p\vect{w}$ is an eigenvector? Explain.
  \begin{sol}
    The formula follows from properties of matrix
    multiplication. However, this vector might not be an eigenvector
    because it might equal $0$ and eigenvectors cannot equal $0$.
  \end{sol}
\end{ex}

