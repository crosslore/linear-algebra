\begin{enumialphparenastyle}

\begin{ex} If $A$ is an invertible $n\times n$ matrix, compare the eigenvalues of
$A$ and $A^{-1}$. More generally, for $m$ an arbitrary integer, compare the
eigenvalues of $A$ and $A^{m}$. \vspace{1mm}
\begin{sol}
$A^{m}X=\lambda ^{m}X$ for
any integer. In the case of $-1,A^{-1}\lambda X=AA^{-1}X=X$
so $A^{-1}X =\lambda ^{-1}X$. Thus the eigenvalues of $A^{-1}$ are just $\lambda ^{-1}$ where $\lambda $ is an eigenvalue of $A$.
\end{sol}
\end{ex} 

\begin{ex} If $A$ is an $n\times n$ matrix and $c$ is a nonzero constant, compare
the eigenvalues of $A$ and $cA$. \vspace{1mm} 
\begin{sol}
Say $AX=\lambda X.$ Then $
cAX=c\lambda X$ and so the eigenvalues of $cA$ are just $
c\lambda $ where $\lambda $ is an eigenvalue of $A$.
\end{sol}
\end{ex}

\begin{ex} Let $A,B$ be invertible $n\times n$ matrices which commute. That is, $AB=BA$. Suppose $X$ is an eigenvector of $B$. Show that then 
$AX$ must also be an eigenvector for $B$. \vspace{1mm} 
\begin{sol}
 $BAX=ABX
=A\lambda X=\lambda AX$. Here it is assumed that $BX=\lambda X$.
\end{sol}
\end{ex}

\begin{ex} Suppose $A$ is an $n\times n$ matrix and it satisfies $A^{m}=A$ for
some $m$ a positive integer larger than 1. Show that if $\lambda $ is an
eigenvalue of $A$ then $\left\vert \lambda \right\vert $ equals either 0 or $
1$. \vspace{1mm}
\begin{sol}
Let $X$ be the eigenvector. Then $A^{m}X=\lambda ^{m}
X,A^{m}X=AX=\lambda X$ and so
\[
\lambda ^{m}=\lambda
\]
Hence if $\lambda \neq 0,$ then
\[
\lambda ^{m-1}=1
\]
and so $\left\vert \lambda \right\vert =1.$
\end{sol}
\end{ex}

\begin{ex} Show that if $AX=\lambda X$ and $AY=\lambda Y$, then whenever $k,p$ are scalars,
\begin{equation*}
A\left( kX+pY\right) =\lambda \left( kX+pY\right) 
\end{equation*}
Does this imply that $kX+pY$ is an eigenvector? Explain.
\vspace{1mm} 
\begin{sol}
The formula follows from properties of matrix multiplications. However,
this vector might not be an eigenvector because it might equal $0$
and eigenvectors cannot equal $0$. 
\end{sol}
\end{ex}

\begin{ex} Suppose $A$ is a $3\times 3$ matrix and the following information is
available. 
\begin{eqnarray*}
A\leftB
\begin{array}{r}
0 \\
-1 \\
-1
\end{array}
\rightB &=&0\leftB
\begin{array}{r}
0 \\
-1 \\
-1
\end{array}
\rightB \\
A\leftB
\begin{array}{r}
1 \\
1 \\
1
\end{array}
\rightB &=&-2\leftB
\begin{array}{r}
1 \\
1 \\
1
\end{array}
\rightB \\
A\leftB
\begin{array}{r}
-2 \\
-3 \\
-2
\end{array}
\rightB &=&-2\leftB
\begin{array}{r}
-2 \\
-3 \\
-2
\end{array}
\rightB
\end{eqnarray*}
Find $A\leftB
\begin{array}{r}
1 \\
-4 \\
3
\end{array}
\rightB. $
%\begin{sol}
%\end{sol}
\end{ex}

\begin{ex} Suppose $A$ is a $3\times 3$ matrix and the following information is
available.
\begin{eqnarray*}
A\leftB
\begin{array}{r}
-1 \\
-2 \\
-2
\end{array}
\rightB &=&1\leftB
\begin{array}{r}
-1 \\
-2 \\
-2
\end{array}
\rightB \\
A\leftB
\begin{array}{r}
1 \\
1 \\
1
\end{array}
\rightB &=& 0\leftB
\begin{array}{r}
1 \\
1 \\
1
\end{array}
\rightB \\
A\leftB
\begin{array}{r}
-1 \\
-4 \\
-3
\end{array}
\rightB &=&2\leftB
\begin{array}{r}
-1 \\
-4 \\
-3
\end{array}
\rightB
\end{eqnarray*}
Find $A\leftB
\begin{array}{r}
3 \\
-4 \\
3
\end{array}
\rightB. $
%\begin{sol}
%\end{sol}
\end{ex}

\begin{ex} Suppose $A$ is a $3\times 3$ matrix and the following information is
available.
\begin{eqnarray*}
A\leftB
\begin{array}{r}
0 \\
-1 \\
-1
\end{array}
\rightB &=&2\leftB
\begin{array}{r}
0 \\
-1 \\
-1
\end{array}
\rightB \\
A\leftB
\begin{array}{r}
1 \\
1 \\
1
\end{array}
\rightB &=& 1\leftB
\begin{array}{r}
1 \\
1 \\
1
\end{array}
\rightB \\
A\leftB
\begin{array}{r}
-3 \\
-5 \\
-4
\end{array}
\rightB &=&-3\leftB
\begin{array}{r}
-3 \\
-5 \\
-4
\end{array}
\rightB
\end{eqnarray*}
Find $A\leftB
\begin{array}{r}
2 \\
-3 \\
3
\end{array}
\rightB. $ \vspace{1mm}
%\begin{sol}
%\end{sol}
\end{ex}

\end{enumialphparenastyle}